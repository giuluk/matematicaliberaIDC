%%%%%%%%%%%%%%%%%%%%%%%%%%%%%%%%%%%%%%%%%%%%%%%%%%%%%%%%%%%%%%%%%%%% 
%        Matematica Libera
%
% Copyright 2022-26 Daniele Zambelli
%
% This work may be distributed and/or modified under the
% conditions of the LaTeX Project Public License, either version 1.3
% of this license or (at your option) any later version.
% The latest version of this license is in
%   http://www.latex-project.org/lppl.txt
% and version 1.3 or later is part of all distributions of LaTeX
% version 2005/12/01 or later.
%
% This work has the LPPL maintenance status `maintained'.
% 
% The Current Maintainer of this work is 
% Daniele Zambelli - daniele.zambelli@gmail.com
%
% This work is part of ``Matematica Libera'' project: 
%
%  https://bitbucket.org/zambu/matematicalibera/src/master/
%
%%%%%%%%%%%%%%%%%%%%%%%%%%%%%%%%%%%%%%%%%%%%%%%%%%%%%%%%%%%%%%%%%%%%

%========================
% Definizione delle directory
%========================
% \newcommand{\basedir}{matematicadolce/}
\newcommand{\depdir}{deposito/}
\newcommand{\magdir}{\depdir magazzino/}
\newcommand{\matdir}{\depdir materiali/}

%========================
% Preambolo
%========================
%------------------------
% Classe del documento  e pacchetti

\documentclass[a4paper,10pt]{\magdir matlibera}
\usepackage{\magdir matliberamat}
\usepackage{\magdir matliberagraf}
% \usepackage{makeidx}
% \usepackage{lipsum}
\makeindex
%------------------------
% Variabili del progetto

\input{\magdir variabili}

%------------------------
% Variabili per questo volume

\newcommand{\editore}{}
\newcommand{\tipo}{Modulo}
\newcommand{\numero}{3}
\newcommand{\titolo}{Calcolo letterale 2}
% \newcommand{\titolodritto}{\tipo \numero: \titolo}
% \newcommand{\pdftitolo}{Calcolo letterale}
\newcommand{\docvers}{\texttt{1.0.0}}
% \newcommand{\edizione}{2023}
% \newcommand{\Edizione}{Edizione}
%\newcommand{\inclusivita}{}%
% {versione accessibile}{versione ciechi}
\newcommand{\anno}{2023}
\newcommand{\mese}{settembre}
\newcommand{\oggi}{\mese\ \anno}
\newcommand{\mcisbn}{}

%------------------------
% Con o senza soluzioni nelle tabelle

\newif\ifsolall % va messo nella classe???
\solallfalse    % Non tutte le soluzioni (tabelle vuote)
% \solalltrue     % tutte le soluzioni (tabelle gia compilate)

%========================
% Documento
%========================
\begin{document}
\frontmatter
% Copyright (c) 2023 Daniele Zambelli - daniele.zambelli@gmail.com
% --------------------------------
% Variabili che devono essere istanziate e che vengono usate da 
% frontespizio.
% \newcommand{\serie}{Matematica Libera}
% \newcommand{\tipo}{Modulo} %{Volume}
% \newcommand{\numero}{12}
% \newcommand{\titolo}{Calcolo infinitesimale}
% \newcommand{\inclusivita}{}%{(versione accessibile)}{(versione ciechi)}
% \newcommand{\descr}{Testo a moduli \protect\\ [1em]
%                     per la Scuola Secondaria di $II$ grado}
% \newcommand{\editore}{Pippo}
% \newcommand{\anno}{2023}
% --------------------------------

% Variabili che devono essere istanziate ma che non vengono usate da 
% frontespizio.
% \newcommand{\titolodritto}{\volume \numero: Calcolo infinitesimale}
% \newcommand{\docvers}{\texttt{0.9.9}}
% \newcommand{\mcisbn}{}

\pdfbookmark{frontespizio}{frontespizio}
\thispagestyle{empty}

\newlength{\drop}
\drop=0.15\textheight
\vspace{\drop}
% {\flushleft %\centering
\fontsize{26pt}{0in}%
\selectfont \textbf{\serie}

\vspace{.5em}
\Large \descr
% }

\vspace{1\drop}

\begin{flushright}
\fontsize{18pt}{0pt}%
\selectfont \textbf{\tipo{} \numero}\\[1em]
\fontsize{26pt}{0pt}%
\selectfont \textbf{\titolo}\\[.8em]
% \fontsize{18pt}{0pt}%
%\LARGE \textit{\inclusivita}
% \selectfont \textit{\inclusivita}
\end{flushright}

\vspace*{\fill}

\begin{flushright} % \begin{center}
\Large 
\editore{} -- \textbf{\anno} --
\end{flushright} % \end{center}

\normalsize
\newpage

\input{\matdir 00_intestazioni/les/colophon}
% \newpage
\setcounter{tocdepth}{2}     % Profondità indice
% Rimuove la voce 'indice' dall'indice
\begin{KeepFromToc}
  \tableofcontents
\end{KeepFromToc}

% Decommentare per la prefazione
\chapter{Prefazione}
% \addcontentsline{toc}{chapter}{Prefazione}
% \markboth{Prefazione}{Prefazione}

I contenuti che presentiamo traggono spunto dai materiali pubblicati nel 
corso Matematica Dolce, volumi terzo e quinto. Quei materiali sono stati 
ampiamente revisionati nel contenuto, nella forma con cui vengono 
presentati, nella struttura e nel formato, anche in seguito a un grande 
lavoro di ristrutturazione e semplificazione degli strumenti \LaTeX per 
la produzione del testo.

Più in dettaglio:

\paragraph {Forma}

\begin{itemize} [nosep]
\item 
Avendo usato strumenti \LaTeX diversi la grafica e la pagina 
appaiono diversi.
\item 
Nei margini incominciano ad apparire rimandi a video lezioni.
\item 
I capitoli iniziano con una breve sintesi del contenuto.
\item 
Alla fine di ogni testo è presente un indice analitico.
\end{itemize}

\paragraph {Struttura}~

Il materiale verrà presentato sia 
suddiviso nei 5 anni della scuola di secondo grado sia 
in moduli per argomento.

\paragraph {Formato}~

Oltre al formato pdf orientato alla stampa, il materiale verrà anche 
trasformato nel formato html  con le formule in mathml e mathjax per 
permetterne una più semplice fruizione da parte di chi ha difficoltà a 
usare carta o pdf.

\paragraph {Caratteristiche}~

I cambiamenti apportati giustificano un cambio di nome, 
grati alla ``vecchia'' esperienza di \emph{Matematica Dolce} 
poniamo ora l'accento su un altro aspetto di questo progetto che si 
chiamerà \textbf{Matematica Libera}.

Rimangono inalterate le principali caratteristiche del progetto 
originario:\\[.5em]
{\Large \centering
libertà \quad collaborazione \quad evoluzione \quad polimorfismo \quad
accessibilità.
% \begin{enumerate} [nosep]
% \item libertà,
% \item collaborazione,
% \item evoluzione,
% \item polimorfismo,
% \item accessibilità.
% \end{enumerate}
}

\paragraph {Dove trovarlo}~

L'intero progetto e le ultime versioni dei testi prodotti saranno 
scaricabili da 
\href{https://bitbucket.org/zambu/matematicalibera/downloads/}
{bitbucket.org/zambu/matematicalibera/downloads/}.

\paragraph {Contatti}~

Se questo progetto ti interessa faccelo sapere: 
\href{mailto:daniele.zambelli@gmail.com}{daniele.zambelli@gmail.com}

\vspace{2em}
Un ringraziamento a tutti quelli che useranno e diffonderanno questo 
materiale e
\dots

\dots buon divertimento con la matematica!
\begin{flushright}
Bruno Stecca, Daniele Zambelli
\end{flushright}
% \setcounter{tocdepth}{2}
% \cleardoublepage

%\newpage
%\input{ml05CalcoloLetterale_pref.tex}
% \cleardoublepage

% --------------------------------
% Capitoli
% --------------------------------
\mainmatter
% ..................................................
% Modulo 3: Calcolo letterale 2
\capitolo{\matdir 03_Polinomi/02_Divisibilita/poldiv01/}
         {divisibilita_scomposizione}
\capitolo{\matdir 03_Polinomi/03_Frazioni_algebriche/frazalg01/}
         {frazioni_algebriche}

% %..................................................
% % Azzeramento numerazione capitoli
\renewcommand{\thechapter}{\Alph{chapter}}
\setcounter{chapter}{0}

% Generazione indici
\ifdefined\HCode 
.
\else
\cleardoublepage
 \phantomsection
\addcontentsline{toc}{chapter}{\indexname}
 \printindex
\fi
\end{document}

