%%%%%%%%%%%%%%%%%%%%%%%%%%%%%%%%%%%%%%%%%%%%%%%%%%%%%%%%%%%%%%%%%%%%
%%        Matematica dolce
%%%%%%%%%%%%%%%%%%%%%%%%%%%%%%%%%%%%%%%%%%%%%%%%%%%%%%%%%%%%%%%%%%%%
%% Copyright 2016-2020 Daniele Zambelli
%------------------------------
%% intestazioni.tex
%------------------------------
%
% This work may be distributed and/or modified under the
% conditions of the LaTeX Project Public License, either version 1.3
% of this license or (at your option) any later version.
% The latest version of this license is in
%   http://www.latex-project.org/lppl.txt
% and version 1.3 or later is part of all distributions of LaTeX
% version 2005/12/01 or later.
%
% This work has the LPPL maintenance status `maintained'.
% 
% The Current Maintainer of this work is 
% Dimitrios Vrettos - d.vrettos@gmail.com
%
% This work consists of the files:
%  -  algebra1_light_1ed.tex (this file)
%  -  Makefile
%  -  README
%  -  the part of code of the files under chap/, img/  and lbr/ directories
%%%%%%%%%%%%%%%%%%%%%%%%%%%%%%%%%%%%%%%%%%%%%%%%%%%%%%%%%%%%%%%%%%%%

%========================
% Classe del documento
%========================

% % tratti da Internet
% \ifx\pdfoutput\undefined                  % compilazione htlatex
% \documentclass{memoir}
% % \usepackage{graphicx}
% % \DeclareGraphicsExtensions{.png, .gif, .jpg} % dà errore
% \newcommand{\href}[2]{\Link[#1]{}{} #2 \EndLink}
% \newcommand{\hypertarget}[2]{\Link[]{}{#1} #2 \EndLink}
% % \newcommand{\hyperlink}[2]{\Link[]{#1}{} #2 \EndLink} % dà errore
% \else                                     % compilazione pdflatex
% \documentclass[10pt,a4paper,openright]{\magdir matsweetmem}
% % \usepackage[pdftex]{graphicx}
% % \DeclareGraphicsExtensions{.pdf,.png,.jpg} % dà errore
% \RequirePackage[colorlinks,hyperindex,pagebackref]{hyperref}
% % \usepackage[colorlinks, hypertexnames=false]{hyperref} % originale di m.d.
% \fi
% 
% \ifdefined\HCode
%    \def\pgfsysdriver{pgfsys-dvisvgm4ht.def}
% \fi 

% fusione mia
\ifdefined\HCode                          % compilazione htlatex
  \documentclass{memoir}
%   \documentclass{\magdir matsweetmem}
  \def\pgfsysdriver{pgfsys-dvisvgm4ht.def}
  % \usepackage{graphicx}
  % \DeclareGraphicsExtensions{.png, .gif, .jpg} % dà errore
  \newcommand{\href}[2]{\Link[#1]{}{} #2 \EndLink}
  \newcommand{\hypertarget}[2]{\Link[]{}{#1} #2 \EndLink}
  % \newcommand{\hyperlink}[2]{\Link[]{#1}{} #2 \EndLink} % dà errore
\else                                     % compilazione pdflatex
  \documentclass[a4paper]{\magdir matsweetmem}
%   \documentclass[10pt,a4paper,openright]{\magdir matsweetmem}
  % \usepackage[pdftex]{graphicx}
  % \DeclareGraphicsExtensions{.pdf,.png,.jpg} % dà errore
  \RequirePackage[colorlinks,hyperindex,pagebackref]{hyperref}
  % \usepackage[colorlinks, hypertexnames=false]{hyperref} %originale di m.d.
  %========================
  % Per l'accessibilità
  %========================
  % Il seguente pacchetto dà errore se chiamato con make4ht:
  \usepackage[accsupp]{axessibility}  % 349      Undefined control sequence.
  % \usepackage{microtype}
  % \DisableLigatures{encoding = *, family = *}

%   \usepackage{fourier}  % in goniometria è definito \wideparen
  % % % buono \widearc, ma è incompatibile con \overrightarrow
  % % Ho estratto da fourier.sty le seguenti 4 righe e sembra funzionare!
  % \DeclareFontEncoding{FMX}{}{}
  % \DeclareFontSubstitution{FMX}{futm}{m}{n}
  % \DeclareSymbolFont{largesymbols}{FMX}{futm}{m}{n}
\fi

%--------------------- original
% \documentclass[10pt,a4paper,openright]{\magdir matsweetmem}
% \documentclass{\magdir matsweetmem}
% Per la versione in scala di grigio commentare la precedente e
% decommentare la riga seguente
% \documentclass[10pt,a4paper,openright,gray]{\magdir matsweetmem}
%---------------------

% \usepackage{framed}     % Per fare delle prove
% \usepackage{xcolor}     % Per fare delle prove
% \usepackage{lipsum}     % Per fare delle prove
% % \usepackage{tcolorbox}     % Per fare delle prove
% \usepackage{blindtext}%    % Per fare delle prove
% \usepackage[theorems,breakable]{tcolorbox}%    % Per fare delle prove


%========================
%  Lingua & codifica
%========================
\PassOptionsToPackage{monochrome}{xcolor}
% \usepackage[monochrome]{color}
\usepackage[T1]{fontenc} 
\usepackage{textcomp} 	
\usepackage[utf8]{inputenc}
\usepackage[italian]{babel} 
\usepackage{icomma} % per la virgola come separatore decimale in modo math
\usepackage{quoting}        % per i testi citati
\quotingsetup{font=small}
%========================
% Font
%========================
\renewcommand{\rmdefault}{ppl}
\usepackage{mathpazo}
%========================
% Tipografia
%========================
\usepackage[bindingoffset=6mm]{geometry}
\usepackage{multicol}
\usepackage{multirow}
\usepackage{indentfirst}
\usepackage{emptypage}
\usepackage{nonumonpart}
% \usepackage[inline]{enumitem}  % Command \enumerate* already defined.
% TODO: eliminare la definizione di \enumerate* per poter usare le
% inline liste
\usepackage{enumitem} % usare questo anche per le liste compatte 
%                           \begin{enumerate} [nosep] o [noitemsep]
% \usepackage{paralist} % va in conflitto con il pacchetto precedente
\usepackage{tabto}
\usepackage{microtype}
%========================
% Tabelle aggiunto da claudio
%========================
\usepackage{threeparttable}
%========================
% Matematica I
%========================
\usepackage[scaled=.95]{helvet}
% % \usepackage{eulervm} % eulervm va in conflitto con fourier per: 
% % \overrightarrow{} 
% % dov'è che ho usato questo pacchetto? per fare che cosa?
% \usepackage{fourier}  % in goniometria è definito \wideparen
% % % % buono \widearc, ma è incompatibile con \overrightarrow
% % % Ho estratto da fourier.sty le seguenti 4 righe e sembra funzionare!
% % \DeclareFontEncoding{FMX}{}{}
% % \DeclareFontSubstitution{FMX}{futm}{m}{n}
% % \DeclareSymbolFont{largesymbols}{FMX}{futm}{m}{n}
\DeclareMathAccent{\widearc}{\mathord}{largesymbols}{216}% per \widearc 
\usepackage[e]{esvect} % per i vettori
\usepackage{ushort} % per lesottolineature semplici o doppie
\usepackage{amsmath}
\usepackage{amssymb}
\usepackage{amsthm}
\usepackage{cancel}
% \usepackage[]{units} Ridefinito \unit mi sembra più semplice!
\usepackage[np,noaddmissingzero,autolanguage]{numprint}
\usepackage{mathtools}
\usepackage[normalem]{ulem} % sottolineature:
%   \uline, \uuline, \uwave, \sout, \xout, \dashuline, \dotuline
%   \renewcommand{\ULdepth}{1.8pt}
%\usepackage{eurosym}  % per il simbolo €


% \RequirePackage{tcolorbox}      % Per fare delle prove sui theorem
\usepackage{tcolorbox}      % Per fare delle prove sui theorem
\tcbuselibrary{most, theorems}    % Per fare delle prove sui theorem
\tcbuselibrary{listingsutf8}

%========================
% Grafica
%========================
% \usepackage[pdftex]{graphicx}
% \usepackage{tcolorbox}      % Per fare delle prove sui theorem
% \tcbuselibrary{theorems}    % Per fare delle prove sui theorem
% \usepackage{cleveref}       % Per fare delle prove sui theorem
\usepackage{rotating}
\usepackage{shadow}
\usepackage{fancybox} 
% \usepackage{empheq} inutile: sostituito da \boxed
\usepackage{framed}
\usepackage{wrapfig}
%========================
% pgf & TikZ
%========================
\usepackage{tikz}
\usepackage{pgfplots}
\usepgfplotslibrary{patchplots}
\pgfplotsset{compat=1.8}
% \usepackage{tkz-euclide}            % dà errore con make4ht
\usepackage{tkz-fct}                % dà errore con make4ht
% \usepackage{circuitikz}
%\usetikzlibrary{circuits}            % dà errore con make4ht
\usepackage{tikz-qtree} % aggiunto da daniele
% \usetkzobj{all}
% \usepackage{qtree} % per funzionitopologia

%========================
% Librerie TikZ
%========================
\usetikzlibrary{
  arrows,%
  arrows.meta,
  through,
  automata,%
  backgrounds,%
  calc,%
  decorations.markings,%
  decorations.shapes,%
  decorations.text,% 
  decorations.pathreplacing,%
  fit,%
  matrix,%
  mindmap,%
  patterns,%
  positioning,%
  intersections,%aggiunto da claudio
  shapes,%
  shapes.geometric,
%  spline % linea curva per una lista di punti (differenziazione_grafici)
}
%========================
% Simboli
%========================
\usepackage{marvosym}
\usepackage{eurosym}
\usepackage{pifont}
% \usepackage{hiero}
%========================
% Matematica II               ????????????????????????????
%========================
\usepackage{\magdir matsweet}
%\usepackage{\magdir shadethm}
%========================
% Collegamenti
%========================
% \usepackage[colorlinks, hypertexnames=false]{hyperref} già nell'if iniziale
\usepackage{bookmark}
%========================
% Per il coding
%========================
\usepackage{listings} 
\lstset{basicstyle=\small, language=Python, showstringspaces=false}
\lstset{frame=trbl, frameround=ftff}

%========================
% Per la geometria
%========================
% \usepackage{textgreek}
%========================
% Personalizzazioni
%========================
% %=============================%
% VARIABILI MATEMATICA LIBERA %
%=============================%

% -----------------------------
% Nome del progetto

\newcommand{\serie}{Matematica Libera}
\newcommand{\descr}{Testo per la Scuola Secondaria di II grado}
\newcommand{\titolodritto}{\tipo \numero: \titolo}

% -----------------------------
%% Nomi di autori, collaboratori, etc

\newcommand{\coord}{Daniele~Zambelli}

\newcommand{\autori}{
% Leonardo~Aldegheri,
% Elisabetta~Campana, 
% Luciana~Formenti, 
% Carlotta~Gualtieri,
% Michele~Perini,
% Maria~Antonietta~Pollini, 
% Diego~Rigo,
% Nicola~Sansonetto, 
% Andrea~Sellaroli,
Bruno~Stecca, 
Daniele~Zambelli}

\newcommand{\colab}{
% Alberto~Bicego,
% Alessandro~Canevaro, 
% Alberto~Filippini%
}

\newcommand{\texcol}{
% Claudio~Carboncini, 
% Silvia~Cibola, 
% Tiziana~Manca,
% Michele~Perini,
% Andrea~Sellaroli,
Claudio~Beccari, 
Daniele~Zambelli%
}
%%%%%%% EOF

% \graphicspath{{img/}}
\setsecnumdepth{subsection} 
\maxtocdepth{subsection}
\setlength{\cftpartnumwidth}{2.25em}
% \setlength{\shadeboxsep}{5pt}            % danno errore con make4ht
% \setlength{\shadeboxrule}{.4pt} 
% \setlength{\shadedtextwidth}{\textwidth}
% \addtolength{\shadedtextwidth}{-2\shadeboxsep}
% \addtolength{\shadedtextwidth}{-2\shadeboxrule}
% \setlength{\shadeleftshift}{0pt}
% \setlength{\shaderightshift}{0pt}
\linespread{1.05}
\captionnamefont{\small\scshape}
\captiontitlefont{\small}
\newcommand{\mail}[1]{\href{mailto:#1}{\texttt{#1}}}
\definecolor{grigio80}{gray}{0.8}
\definecolor{grigio70}{gray}{0.7}
\hypersetup{%
  pdffitwindow=true,%
  linkcolor=RoyalBlue,%	
%   linkcolor=Black,%	
  linktocpage=true,%
  filecolor=black,%
  urlcolor=RoyalBlue,%
%   urlcolor=Black,%
  plainpages=false,%
  pdftitle={\pdftitolo, \edizione \tipo},%
  pdfauthor=Dimitrios Vrettos,%
  pdfdisplaydoctitle=true%
}
\bookmarksetup{startatroot}
%======================== o l'una o l'altra
% Personalizzazione per matematica dolce
% per: postulato, proposizione, parte, capitolo, numnameref, tggg
% inacessibleblock
% \input{lbr/definizioni}
%========================
% Caratteri sans serif
%========================
% \renewcommand{\familydefault}{\sfdefault}


% ------------------------------------
% Peccato che non funzioni:
\usepackage{fancyhdr}
%--------------------------------------------
% Altra soluzione trovata guitex e modificata
\pagestyle{fancy}
\fancyhf{}
% % \setlength{\headheight}{15pt}      % altezza dell'head
% % \fancyfoot[C]{\thepage}            % numero in basso centrato
% \renewcommand{\chaptermark}[1]{\markboth{\thechapter. #1}{}} % con numero
% \renewcommand{\sectionmark}[1]{\markright{\thesection\ #1}}
% % \renewcommand{\chaptermark}[1]{\markboth{#1}{}}            % senza numero
% % \renewcommand{\sectionmark}[1]{\markright{#1}}
% % --- Dispari destra, Pari sinistra con numeri di pagina in minuscolo.
% \fancyhead[RO]{{\footnotesize \nouppercase{\rightmark} 
%                \qquad \oldstylenums{\thepage}}}
% \fancyhead[LE]{{\footnotesize \oldstylenums{\thepage}
%                \qquad \nouppercase{\leftmark}}}
% % % --- Dispari sinistra, Pari destra
% % \fancyhead[RO, LE]{\thepage}
% % \fancyhead[LO]{\nouppercase{\slshape\rightmark}} 
% % \fancyhead[RE]{\nouppercase{\slshape\leftmark}}
% % % --- Centrato
% % \fancyhead[RO, LE]{\thepage}
% % \fancyhead[CO]{\nouppercase{\slshape\rightmark}} 
% % \fancyhead[CE]{\nouppercase{\slshape\leftmark}}
% % % --- Lunghezza della linea di separazione
% % \renewcommand{\headrulewidth}{0pt} % senza linea di separazione
% % \fancyheadoffset[RE,LO]{-0.1\textwidth} % lunghezza della testatina
