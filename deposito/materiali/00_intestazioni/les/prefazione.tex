\chapter{Prefazione}
% \addcontentsline{toc}{chapter}{Prefazione}
% \markboth{Prefazione}{Prefazione}

I contenuti che presentiamo traggono spunto dai materiali pubblicati nel 
corso Matematica Dolce, volumi terzo e quinto. Quei materiali sono stati 
ampiamente revisionati nel contenuto, nella forma con cui vengono 
presentati, nella struttura e nel formato, anche in seguito a un grande 
lavoro di ristrutturazione e semplificazione degli strumenti \LaTeX per 
la produzione del testo.

Più in dettaglio:

\paragraph {Forma}

\begin{itemize} [nosep]
\item 
Avendo usato strumenti \LaTeX diversi la grafica e la pagina 
appaiono diversi.
\item 
Nei margini incominciano ad apparire rimandi a video lezioni.
\item 
I capitoli iniziano con una breve sintesi del contenuto.
\item 
Alla fine di ogni testo è presente un indice analitico.
\end{itemize}

\paragraph {Struttura}~

Il materiale verrà presentato sia 
suddiviso nei 5 anni della scuola di secondo grado sia 
in moduli per argomento.

\paragraph {Formato}~

Oltre al formato pdf orientato alla stampa, il materiale verrà anche 
trasformato nel formato html  con le formule in mathml e mathjax per 
permetterne una più semplice fruizione da parte di chi ha difficoltà a 
usare carta o pdf.

\paragraph {Caratteristiche}~

I cambiamenti apportati giustificano un cambio di nome, 
grati alla ``vecchia'' esperienza di \emph{Matematica Dolce} 
poniamo ora l'accento su un altro aspetto di questo progetto che si 
chiamerà \textbf{Matematica Libera}.

Rimangono inalterate le principali caratteristiche del progetto 
originario:\\[.5em]
{\Large \centering
libertà \quad collaborazione \quad evoluzione \quad polimorfismo \quad
accessibilità.
% \begin{enumerate} [nosep]
% \item libertà,
% \item collaborazione,
% \item evoluzione,
% \item polimorfismo,
% \item accessibilità.
% \end{enumerate}
}

\paragraph {Dove trovarlo}~

L'intero progetto e le ultime versioni dei testi prodotti saranno 
scaricabili da 
\href{https://bitbucket.org/zambu/matematicalibera/downloads/}
{bitbucket.org/zambu/matematicalibera/downloads/}.

\paragraph {Contatti}~

Se questo progetto ti interessa faccelo sapere: 
\href{mailto:daniele.zambelli@gmail.com}{daniele.zambelli@gmail.com}

\vspace{2em}
Un ringraziamento a tutti quelli che useranno e diffonderanno questo 
materiale e
\dots

\dots buon divertimento con la matematica!
\begin{flushright}
Bruno Stecca, Daniele Zambelli
\end{flushright}
% \setcounter{tocdepth}{2}
% \cleardoublepage
