% (c) 2012 -2014 Dimitrios Vrettos - d.vrettos@gmail.com
% (c) 2014 Claudio Carboncini - claudio.carboncini@gmail.com
% (c) 2014 Daniele Zambelli - daniele.zambelli@gmail.com

\section{Esercizi}

\subsection{Esercizi dei singoli paragrafi}

%\subsubsection*{1.4 - Operazioni con i numeri naturali}
\subsubsection*{\numnameref{sec:nat_operazioni}}

\immagine*{Testo alternativo: si evidenzia che, se non diversamente 
specificato, all'interno dell'ambiente matematico al posto dei 
``puntini'' per indicare una parte vuota (e che dovrà successivamente 
essere riempita),
è inserita la lettera ``p'' come abbreviazione di ``puntini''.
Ciò viene fatto in quanto quando si converte da HTML a LAMBDA, i
``puntini'' non vengono rappresentati e al loro posto non
compare nulla.}{}

\begin{esercizio}
% \label{ese:1.1}
Dimostra la seguente affermazione:
% \begin{htmulticols}{3}
 \begin{enumeratees}
\item La somma di due numeri naturali è un numero naturale.
%  \item sottrazione;
%  \item moltiplicazione;
%  \item divisione;
%  \item potenza;
%  \item radice quadrata.
 \end{enumeratees}
% \end{htmulticols}
\end{esercizio}

\begin{esercizio}
% \label{ese:1.1}
Rappresenta con grafi e con un linguaggio di programmazione le seguenti
funzioni:
\begin{htmulticols}{3}
 \begin{enumeratees}
 \item addizione; \qquad
 \item sottrazione; \qquad
 \item moltiplicazione; \qquad
 \item divisione; \qquad
 \item potenza; \qquad
 \item radice quadrata.
 \end{enumeratees}
\end{htmulticols}
\end{esercizio}

\begin{esercizio}
% \label{ese:1.1}
Rappresenta con grafi le seguenti espressioni:
\begin{htmulticols}{3}
 \begin{enumeratees}
 \item \(57 + 62\) \qquad
 \item \(26 - 7\) \qquad
 \item \(25 \cdot 5\) \qquad
 \item \(48 : 3\) \qquad
 \item \(4^3\) \qquad
 \item \(\sqrt{49}\) \qquad
 \end{enumeratees}
\end{htmulticols}
\end{esercizio}

\begin{esercizio}
% \label{ese:1.1}
Rispondi alle seguenti domande:
 \begin{enumeratees}
 \item Esiste il numero naturale che aggiunto a~3 dà come somma~6?
 \item Esiste il numero naturale che aggiunto a~12 dà come somma~7?
 \item Esiste il numero naturale che moltiplicato per~4 dà come prodotto~12?
 \item Esiste il numero naturale che moltiplicato per~5 dà come prodotto~11?
 \end{enumeratees}
\end{esercizio}

\begin{esercizio}
% \label{ese:1.2}
 Inserisci il numero naturale mancante, se esiste:
\begin{htmulticols}{4}
\begin{enumeratees}
 \item \(7-\ldots =1\) \qquad
 \immagine*{Testo alternativo: \(7-p=1\)}{}
 \item\(3-3=\ldots~\) \qquad
 \immagine*{Testo alternativo: \(3-3=p\)}{}
 \item\(5-6=\ldots~\) \qquad
 \immagine*{Testo alternativo: \(5-6=p\)}{}
 \item \(3-\ldots =9\) \qquad
 \immagine*{Testo alternativo: \(3-p=9\)}{}
 \item\(15:5=\ldots~\) \qquad
 \immagine*{Testo alternativo: \(15:5=p\)}{}
 \item\(18:\ldots =3\) \qquad
 \immagine*{Testo alternativo: \(18:p=3\)}{}
 \item \(\ldots:4=5\) \qquad
 \immagine*{Testo alternativo: \(p:4=5\)}{}
 \item\(12:9=\ldots~\) 
 \immagine*{Testo alternativo: \(12:9=p\)}{}
\end{enumeratees}
\end{htmulticols}
\end{esercizio}

\begin{esercizio}
% \label{ese:1.3}
 Vero o falso?
\begin{htmulticols}{3}
\TabPositions{2.2cm}
\begin{enumeratees}
 \item \(5:0=0\) \tab \verofalso
 \item \(0:5=0\) \tab \verofalso
 \item \(5:5=0\) \tab \verofalso
 \item \(1:0=1\) \tab \verofalso
 \item \(0:1=0\) \tab \verofalso
 \item \(0:0=0\) \tab \verofalso
 \item \(1:1=1\) \tab \verofalso
 \item \(1:5=1\) \tab \verofalso
 \item \(4:0=0\) \tab \verofalso
\end{enumeratees}
\end{htmulticols}
\end{esercizio}

\begin{esercizio}
% \label{ese:1.4}
 Se è vero che~\(p=n\times m\), quali affermazioni sono vere?
\begin{htmulticols}{2}
\TabPositions{3.5cm}
\begin{enumeratees}
 \item \(p\) è multiplo di~\(n\) \tab \verofalso
 \item \(p\) è multiplo di~\(m\) \tab \verofalso
 \item \(m\) è multiplo di~\(p\) \tab \verofalso
 \item \(m\) è multiplo di~\(n\) \tab \verofalso
 \item \(p\) è divisibile per~\(m\) \tab \verofalso
 \item \(m\) è divisibile per~\(n\) \tab \verofalso
 \item \(p\) è divisore di~\(m\) \tab \verofalso
 \item \(n\) è multiplo di~\(m\) \tab \verofalso
\end{enumeratees}
\end{htmulticols}
\end{esercizio}

\begin{esercizio}
% \label{ese:1.5}
 Quali delle seguenti affermazioni sono vere?

\begin{htmulticols}{2}
\TabPositions{3.5cm}
 \begin{enumeratees}
 \item 6 è un divisore di~3 \tab \verofalso
 \item 3 è un divisore di~6 \tab \verofalso
 \item 8 è un multiplo di~2 \tab \verofalso
 \item 5 è divisibile per~10 \tab \verofalso
 \end{enumeratees}
\end{htmulticols}
\end{esercizio}

\pagebreak %---------------------------------------------

\begin{esercizio}
% \label{ese:1.6}
 Esegui le seguenti operazioni:
\begin{htmulticols}{3}
 \begin{enumeratees}
 \item \(18\divint~3=\ldots\)
 \item \(18\bmod~3=\ldots\)
 \item \(20\divint~3=\ldots\)
 \item \(20\bmod~3=\ldots\)
 \item \(185\divint~7=\ldots\)
 \item \(185\bmod~7=\ldots\)
 \item \(97\divint~5=\ldots\)
 \item \(97\bmod~5=\ldots\)
 \item \(240\divint~12=\ldots\)
 \item \(240\bmod~12=\ldots\)
 \item \(700\divint~8=\ldots\)
 \item \(700\bmod~8=\ldots\)
 \end{enumeratees}
\end{htmulticols}
\end{esercizio}

% \newpage

\begin{esercizio}
% \label{ese:1.7}
 Esegui le seguenti divisioni con numeri a più cifre, senza usare la 
calcolatrice
\begin{htmulticols}{4}
 \begin{enumeratees}
 \item \(311:22\)
 \item \(429:37\)
 \item \(512:31\)
 \item \(629:43\)
 \item \(755:53\)
 \item \(894:61\)
 \item \(968:45\)
 \item \(991:13\)
 \item \(1232:123\)
 \item \(2324:107\)
 \item \(3435:201\)
 \item \(4457:96\)
 \item \(5567:297\)
 \item \(6743:311\)
 \item \(7879:201\)
 \item \(8967:44\)
 \item \(13455:198\)
 \item \(22334:212\)
 \item \(45647:721\)
 \item \(67649:128\)
 \end{enumeratees}
\end{htmulticols}
\end{esercizio}

%\subsubsection*{1.5 - Proprietà delle operazioni}
%\subsubsection*{\numnameref{sec:nat_proprieta}}

\begin{esercizio}
% \label{ese:1.8}
Stabilisci se le seguenti uguaglianze sono vere o false indicando la
proprietà utilizzata:
\TabPositions{5.2cm}
 \begin{enumeratees}
 \item \(33:11=11:33\)	\tab  proprietà\dotfill\enspace\verofalso
 \item \(108-72:9=(108-72):9\)	\tab  proprietà\dotfill\enspace\verofalso
 \item \(8-4=4-8\)	\tab  proprietà\dotfill\enspace\verofalso
 \item \(35\cdot 10=10\cdot 35\)	\tab  
proprietà\dotfill\enspace\verofalso
 \item \(9\cdot(2+3)=9\cdot3+9\cdot2\)	\tab  
proprietà\dotfill\enspace\verofalso
 \item \(80-52+36=(20-13-9)\cdot 4\)	\tab  
proprietà\dotfill\enspace\verofalso
 \item \((28-7):7=28:7-7:7\)	\tab  proprietà\dotfill\enspace\verofalso
 \item \((8\cdot 1):2=8:2\)	\tab  proprietà\dotfill\enspace\verofalso
 \item \((8-2)+3=8-(2+3)\)	\tab  proprietà\dotfill\enspace\verofalso
 \item \((13+11)+4=13+(11+4)\)	\tab  proprietà\dotfill\enspace\verofalso
 \end{enumeratees}
\end{esercizio}

\begin{esercizio}
% \label{ese:1.9}
Data la seguente operazione tra i numeri naturali 
\(a\circ b=2\cdot a +3\cdot 
b\), verifica se è:
 \begin{enumeratees}
 \item commutativa, cioè se~\(a\circ b=b\circ a\)
 \item associativa, cioè se~\(a\circ (b\circ c)=(a\circ b)\circ c\)
 \item 0 è elemento neutro
 \end{enumeratees}
\end{esercizio}

%\subsubsection*{1.6 - Potenza}
\subsubsection*{\numnameref{subsec:nat_potenza}}

\begin{esercizio}
% \label{ese:1.10}
Inserisci i numeri mancanti:
 \begin{htmulticols}{2}
 \begin{enumeratees}
 \item \(3^1\cdot3^2\cdot3^3=3^{\ldots+\ldots+\ldots}=3^{\ldots}\)
 \immagine*{Testo alternativo: \(3^1\cdot3^2\cdot3^3=3^(p+p+p)=3^p\)}{}
 \item \(3^4:3^2=3^{\ldots-\ldots}=3^{\ldots}\)
 \immagine*{Testo alternativo: \(3^4:3^2=3^{p-p}=3^p\)}{}
 \item \((3:7)^5=3^{\ldots}:7^{\ldots}\)
 \immagine*{Testo alternativo: \((3:7)^5=3^p:7^p\)}{}
 \item \(6^3:5^3=(6:5)^{\ldots}\)
 \immagine*{Testo alternativo: \(6^3:5^3=(6:5)^p\)}{}
 \item \(7^3\cdot5^3\cdot2^3=(7\cdot 5 \cdot 2)^{\ldots}\)
 \immagine*{Testo alternativo: \(7^3\cdot5^3\cdot2^3=(7\cdot 5 \cdot 2)^p\)}{}
 \item \((2^6)^2=2^{\ldots\cdot\ldots}=2^{\ldots}\)
 \immagine*{Testo alternativo: \((2^6)^2=2^{p \cdot p}=2^p\)}{}
 \item \((18^6):(9^6)=(\ldots\ldots)^{\ldots}=2^{\ldots}\)
 \immagine*{Testo alternativo: \((18^6):(9^6)=(p p)^{p}=2^p\)}{}
 \item \((5^6\cdot5^4)^4:[(5^2)^3]^6=\ldots\ldots\ldots=5^{\ldots}\)
 \immagine*{Testo alternativo: \((5^6\cdot5^4)^4:[(5^2)^3]^6=p p p=5^p\)}{}
 \end{enumeratees}
 \end{htmulticols}
\end{esercizio}

% \paragraph{\ref{ese:1.11}}
% a)~\(6^6\),\quad b)\(5^4\),\quad c)~1,\quad d)~\(6^3\)
% 
\pagebreak %-------------------------------------------------
% \begin{esercizio}[*]
\begin{esercizio}
% \label{ese:1.11}
Calcola applicando le proprietà delle potenze:
 \begin{htmulticols}{2}
 \begin{enumeratees}
 \item \(2^5\cdot2^3:2^2\cdot3^6\) \sol{6^6}
 \item \((5^2)^3:5^3\cdot5\) \sol{5^4}
 \item \([(2^1)^4\cdot 3^4]^2:6^5\cdot6^0\) \sol{6^3}
%  \end{enumeratees}
%  \end{htmulticols}
% \end{esercizio}
% 
% % \pagebreak %-------------------------------------------------
% 
% \begin{esercizio}
% % \label{ese:1.12}
% Calcola:
%  \begin{htmulticols}{2}
%  \begin{enumeratees}
 \item \(2^2\cdot(2^3+2^2)\) \sol{48}
 \item \([(3^6:3^4)^2\cdot3^2]^1\) \sol{3^6}
 \item \(5^4 : (3^2+4^2)\) \sol{25}
 \item \(3^4\cdot(3^4+4^2-2^2)^0:3^3\) \sol{3}
 \item \(\{[(2^3)^2:2^3]^3:2^5\}\) \sol{2^4}
 \end{enumeratees}
 \end{htmulticols}
\end{esercizio}

\begin{esercizio}
% \label{ese:1.13}
 Completa, applicando le proprietà delle potenze:
\begin{htmulticols}{3}
 \begin{enumeratees}
 \item \(7^4\cdot7^{\ldots}=7^5\)
 \immagine*{Testo alternativo: \(7^4\cdot7^p=7^5\)}{}
 \item \(3^9\cdot5^9=(\ldots\ldots)^9\)
 \immagine*{Testo alternativo: \(3^9\cdot5^9=(p p)^9\)}{}
 \item \(5^{15}:5^{\ldots}=5^5\)
 \immagine*{Testo alternativo: \(5^{15}:5^p=5^5\)}{}
 \item \((\ldots\ldots)^6\cdot5^6=15^6\)
 \immagine*{Testo alternativo: \((p p)^6\cdot5^6=15^6\)}{}
 \item \(8^4:2^4=2^{\ldots}\)
 \immagine*{Testo alternativo: \(8^4:2^4=2^p\)}{}
 \item \((18^5:6^5)^2=3^{\ldots}\)
 \immagine*{Testo alternativo: \((18^5:6^5)^2=3^p\)}{}
 \item \(20^7:20^0=20^{\ldots}\)
 \immagine*{Testo alternativo: \(20^7:20^0=20^p\)}{}
 \item \((\ldots^3)^4=1\)
 \immagine*{Testo alternativo: \((p^3)^4=1\)}{}
 \item \((7^3) \cdot 7^{\ldots}=7^{14}\)
 \immagine*{Testo alternativo: \item \((7^3) \cdot 7^p=7^{14}\)}{}
 \end{enumeratees}
\end{htmulticols}
\end{esercizio}

\begin{esercizio}
% \label{ese:1.14}
 Il risultato di~\(3^5+5^3\) è: \hfill
 \fbox{A}~368 ~~~~ \fbox{B}~\((3+5)^5\) ~~~~ 
\fbox{C}~\(15+15\) ~~~~ \fbox{D}~\(8^8\)
 \end{esercizio}

\begin{esercizio}
% \label{ese:1.15}
 Il risultato di~\((73+27)^2\) è: \hfill
 \fbox{A}~200 ~~~~ \fbox{B}~\(73^2+27^2\) 
~~ \fbox{C}~\(10^4\) ~~~~\fbox{D}~\(1000\)
\end{esercizio}

% \newpage

%\subsubsection*{1.11 - Espressioni numeriche}
\subsubsection*{\numnameref{sec:nat_espressioni}}

\begin{esercizio}
% \label{ese:1.16}
Esegui le seguenti operazioni rispettando la precedenza algebrica
 \begin{htmulticols}{4}
 \begin{enumeratees}
 \item \(15+7-2\)
 \item \(16-4+2\)
 \item \(18-8-4\)
 \item \(16\times 2-2\)
 \item \(12-2\times 2\)
 \item \(10-5\times 2\)
 \item \(20\times~4:5\)
 \item \(16:4\times~2\)
 \item \(2+2^2+3\)
 \item \(4\times 2^3+1\)
 \item \(2^4:2-4\)
 \item \((1+2)^3-2^3\)
 \item \((3^2)^3-3^2\)
 \item \(2^4+2^3\)
 \item \(2^3\times3^2\)
 \item \(3^3:3^2\times3^2\)
 \end{enumeratees}
 \end{htmulticols}
\end{esercizio}

%\usepackage{hyperref}

Le espressioni che seguono sono state elaborate a partire da quelle che si 
possono trovare all'indirizzo:
\href{http://www.ubimath.org/potenze}{www.ubimath.org/potenze} - 
Ringrazio Ubaldo Pernigo per la competenza e disponibilità.

% \begin{esercizio} % \label{ese:1.17}
% \(2^3+2^2\cdot5-2\cdot2^2+14:2\) \sol{27}
% \end{esercizio}
% \begin{esercizio} % \label{ese:1.17}
% \(3^3:3+6^2:3+2^3\cdot2-14:2\cdot5-2^0\) \sol{1}
% \end{esercizio}
% \begin{esercizio} % \label{ese:1.17}
% \(3^2+2^3-3\cdot2+4^2:2-8\) \sol{11}
% \end{esercizio}
% \begin{esercizio} % \label{ese:1.17}
% \(2^3+5^2-4^2+2^2-20:2-5^0\) \sol{10}
% \end{esercizio}
% \begin{esercizio} % \label{ese:1.17}
% \(3^3:9+2^4:4-3\cdot1^5\) \sol{4}
% \end{esercizio}
% \begin{esercizio} % \label{ese:1.17}
% \(1^5+(2^2+2^4)\cdot5-5^2\cdot2^2\) \sol{1}
% \end{esercizio}
% \begin{esercizio} % \label{ese:1.17}
% \(2\cdot3:[3^3-2^2\cdot5+2^3-36:2^2]\) \sol{1}
% \end{esercizio}
% \begin{esercizio} % \label{ese:1.17}
% \([(1^2+5^1:5-2)^2\cdot(2^2\cdot2^2)^2+2\cdot(2\cdot2^3):2^3]^2:
%   (1^2\cdot2^2)\) \sol{4}
% \end{esercizio}
% \begin{esercizio} % \label{ese:1.17}
% \([(5^2-24)^3\cdot8^2-(4^2\cdot2)]:2^3\) \sol{4}
% \end{esercizio}
% \begin{esercizio} % \label{ese:1.17}
% \((7-5)^2+(2^3-2^2-2)^3-5\cdot2\) \sol{2}
% \end{esercizio}
\begin{esercizio} % \label{ese:1.17}
\(2^2+3^2\cdot5^2-3\cdot2^4+7\cdot5^2-2^3\cdot5^2-2^2\cdot3^3\) \sol{48}
\end{esercizio}
\begin{esercizio} % \label{ese:1.17}
\(2^2\cdot[(2^2\cdot3:3+5\cdot2^2):(2\cdot3)+1^3]\) \sol{20}
\end{esercizio}
\begin{esercizio} % \label{ese:1.17}
\(10^1+(2+11-3^2)^2-(2^2+4^2+6)\) \sol{0}
\end{esercizio}
\begin{esercizio} % \label{ese:1.17}
\(2^1+3^2+4^2-5^2-4^0\) \sol{1}
\end{esercizio}
\begin{esercizio} % \label{ese:1.17}
\(24:(3\cdot2^2)+2^2\cdot(3^2+3^0-2^3)\) \sol{10}
\end{esercizio}
\begin{esercizio} % \label{ese:1.17}
\(5+2\cdot[5+2\cdot(2^2+5):3-3^2]-2\cdot3\) \sol{3}
\end{esercizio}
\begin{esercizio} % \label{ese:1.17}
\((5^2+3^2-1):3+(3^3+1):7\) \sol{15}
\end{esercizio}
\begin{esercizio} % \label{ese:1.17}
\((3\cdot4+2^3\cdot2+7\cdot6):10\cdot3-2^2\cdot5\) \sol{1}
\end{esercizio}
\begin{esercizio} % \label{ese:1.17}
\(3^2+4^2+2\cdot3+(7+2):9+(27-2):5\) \sol{37}
\end{esercizio}
\begin{esercizio} % \label{ese:1.17}
\((3^3+3^2+3^1+3^0-10):6+6^2:6\) \sol{11}
\end{esercizio}
\begin{esercizio} % \label{ese:1.17}
\(\{[(2^6-2^5-2^4-2^3):2^2+1]^3\cdot2-24\}^2+3\) \sol{903}
\end{esercizio}
\begin{esercizio} % \label{ese:1.17}
\(\{16:(6^2-10\cdot2)+[(7\cdot3+3^3\cdot3-2)^2:10^3]:(7^2-11\cdot4)-2\}^5\)
 \sol{1}
\end{esercizio}
\begin{esercizio} % \label{ese:1.17}
\([(2^2\cdot2^5):(2\cdot2^3)]^2\) \sol{64}
\end{esercizio}
\begin{esercizio} % \label{ese:1.17}
\((2^2\cdot2)^2:(5\cdot2^2-2^2)+[7^2:(5^2-3^2\cdot2)+13^3:13^2]:2^2+(7^4\cdot7
^2)^0-3^2\) \sol{1}
\end{esercizio}
\begin{esercizio} % \label{ese:1.17}
\([13^6\cdot(13^5:13)]^2:[13^{13}:(13^2\cdot13^3)^2]^6\) \sol{169}
\end{esercizio}
\begin{esercizio} % \label{ese:1.17}
\((3\cdot5-2^2\cdot2)\cdot3^2+3^3\cdot2^2-7\cdot3^2\) \sol{108}
\end{esercizio}
\begin{esercizio} % \label{ese:1.17}
\([(3^4)^3:3^{10}]^5:3^9+(5^4)^3:5^{10}-2^2\cdot7^1\) \sol{0}
\end{esercizio}
\begin{esercizio} % \label{ese:1.17}
\([(7^4\cdot2^4\cdot9^4):(7^2\cdot2^2\cdot9^2)]^4:(504^8:4^8)\) \sol{1}
\end{esercizio}
\begin{esercizio} % \label{ese:1.17}
\((13\cdot3^3-2^6\cdot5)^2:31+[(6-5)^6+(2^2+3^2-2^1)]:(2^4:2^2)\) \sol{34}
\end{esercizio}
\begin{esercizio} % \label{ese:1.17}
\((2^4-5^2:5\cdot3):1+(2\cdot3\cdot6-2^2\cdot3^2)+2^2\cdot3^2:
[2^3\cdot3+2^2\cdot3\cdot(2^3-7)]\) \sol{2}
\end{esercizio}
\begin{esercizio} % \label{ese:1.17}
\(25:5+(8^2-15\cdot3-2^3)-27:(4^2+3-10)\) \sol{13}
\end{esercizio}
\begin{esercizio} % \label{ese:1.17}
\(\{[(2^6\cdot2^4:2^8):2^2+1]^3:2^2\}^0\) \sol{1}
\end{esercizio}
\begin{esercizio} % \label{ese:1.17}
\([(5^2)^3\cdot5^4]:[5^4\cdot(5^2)^2]\) \sol{25}
\end{esercizio}
\begin{esercizio} % \label{ese:1.17}
\([(3^2\cdot3^4)\cdot(3^2\cdot3)]^2:3^{16}\) \sol{9}
\end{esercizio}
% \begin{esercizio} % \label{ese:1.17}
% \((5^2\cdot5)^4:(5^9\cdot5^2)\) \sol{5}
% \end{esercizio}
\begin{esercizio} % \label{ese:1.17}
\(1^3+(2^2)^3:(5-4+1)^4+[7^2:(5^2-3^2\cdot2)+13^4:13^3]:2^2+1^5\) \sol{11}
\end{esercizio}
\begin{esercizio} % \label{ese:1.17}
\(2^2+\{[7\cdot(5^3:5^2\cdot3^0+5^1)+(3^5:3^2+3)]:(5^4:5^2)-2^2\}-
[2^3\cdot5:(2\cdot5)]^3:2^4\) \sol{0}
\end{esercizio}
% \begin{esercizio} % \label{ese:1.17}
% 
% \(4+\{[7\cdot(5\cdot3^3:3^3+5)+(3^3+3)]:5^2-2^2\}-
% [(2^3\cdot3^2-2^6)\cdot5:10]^3:2^4\) \sol{0}
% \end{esercizio}
% \begin{esercizio} % \label{ese:1.17}
% \(6+\{[8-(2\cdot3^2-4^2)^2]^3:2^4+6\}^4:10^3-3^2\) \sol{7}
% \end{esercizio}

%\subsubsection*{1.11 - Espressioni con un buco}
\subsubsection*{\numnameref{sec:nat_espressioni_buco}}

%\usepackage{hyperref}

Le espressioni che seguono sono state elaborate a partire da quelle che si 
possono trovare all'indirizzo:
\href{http://www.ubimath.org/potenze}{www.ubimath.org/potenze} - 
Ringrazio Ubaldo Pernigo per la competenza e disponibilità.

% \begin{esercizio} % \label{ese:1.17}
% \(0^5:9+4^2+3^3-{\dots}^2-2^2\cdot2\) \sol{10}
% \end{esercizio}
\begin{esercizio} % \label{ese:1.17}
\(8^2-3^{\dots}\cdot5+(2^2\cdot3^2-4\cdot9):4^2+3^0\) \sol{20}
\immagine*{Testo alternativo: \(8^2-3^p\cdot5+(2^2\cdot3^2-4\cdot9):4^2+3^0\)}{}
\end{esercizio}
% \begin{esercizio} % \label{ese:1.17}
% \((2\cdot{\dots})^2-2\cdot2^4+3^3:3^2-2^3:2-2\) \sol{1}
% \end{esercizio}
% \begin{esercizio} % \label{ese:1.17}
% \((6:3+{\dots}:2-4)^3\cdot[2\cdot3:2+7:7]\) \sol{0}
% \end{esercizio}
% \begin{esercizio} % \label{ese:1.17}
% \((3^2+2^3)\cdot3-{\dots}^2:(5^2-3^2)\) \sol{50}
% \end{esercizio}
% \begin{esercizio} % \label{ese:1.17}
% \([2^{\dots}+2\cdot(2^2\cdot5+3)]:25-3^0\) \sol{1}
% \end{esercizio}
% \begin{esercizio} % \label{ese:1.17}
% \(3^2+2^2\cdot[({\dots}\cdot2^2:3+5\cdot2^2):6+1^5]\) \sol{29}
% \end{esercizio}
\begin{esercizio} % \label{ese:1.17}
\((7^2-2\cdot5+15:3):4+(3\cdot2^2+3^{\dots}-4^2)^2\) \sol{36}
\immagine*{Testo alternativo: \((7^2-2\cdot5+15:3):4+(3\cdot2^2+3^p-4^2)^2\)}{}
\end{esercizio}
\begin{esercizio} % \label{ese:1.17}
\(5^1+({\dots}^2-5\cdot3^2-2^3)-3^3:(4^2+3-10)\) \sol{13}
\immagine*{Testo alternativo: \(5^1+(p^2-5\cdot3^2-2^3)-3^3:(4^2+3-10)\)}{}
\end{esercizio}
\begin{esercizio} % \label{ese:1.17}
\(2^2+3^{\dots}+5^2-2\cdot3-8\cdot4\) \sol{0}
\immagine*{Testo alternativo: \(2^2+3^p+5^2-2\cdot3-8\cdot4\) \sol{0}}{}
\end{esercizio}
\begin{esercizio} % \label{ese:1.17}
\((5^2-3^2):2^2+9^{\dots}\cdot8^2:8^1\) \sol{12}
\immagine*{Testo alternativo: \((5^2-3^2):2^2+9^p\cdot8^2:8^1\)}{}
\end{esercizio}
\begin{esercizio} % \label{ese:1.17}
\((2^3+2^4):2+{\dots}\cdot3-2^2\cdot5\) \sol{31}
\immagine*{Testo alternativo: \((2^3+2^4):2+p\cdot3-2^2\cdot5\)}{}
\end{esercizio}
\begin{esercizio} % \label{ese:1.17}
\([(7^5\cdot7^{\dots}):(7^4)^3]:7^2\) \sol{1}
\immagine*{Testo alternativo: \([(7^5\cdot7^p):(7^4)^3]:7^2\)}{}
\end{esercizio}
\begin{esercizio} % \label{ese:1.17}
\((1^5+1^6+1^8+1^{10})\cdot4-2^{\dots}\) \sol{0}
\immagine*{Testo alternativo: \((1^5+1^6+1^8+1^{10})\cdot4-2^p\)}{}
\end{esercizio}
\begin{esercizio} % \label{ese:1.17}
\(81:3^2+32:2^2+{\dots}:5^2-(4\cdot2-2^3):3\) \sol{19}
\immagine*{Testo alternativo: \(81:3^2+32:2^2+p:5^2-(4\cdot2-2^3):3\)}{}
\end{esercizio}
\begin{esercizio} % \label{ese:1.17}
\(\{[(2^6-2^5-2^4-2^3):4+{\dots}]\cdot8-24\}+3\) \sol{3}
\immagine*{Testo alternativo: \(\{[(2^6-2^5-2^4-2^3):4+p]\cdot8-24\}+3\) 
\sol{3}}{}
\end{esercizio}
\begin{esercizio} % \label{ese:1.17}
\(3\cdot2+(2^{\dots}:2^2+3^2:3)\cdot5-(6:2+44:4):7\) \sol{29}
\immagine*{Testo alternativo: \(3\cdot2+(2^p:2^2+3^2:3)\cdot5-(6:2+44:4):7\)}{}
\end{esercizio}
\begin{esercizio} % \label{ese:1.17}
\(\{5\cdot16-(6^2-2^4)-[(3^2-{\dots}^2)\cdot10-5]\}-[(2^2\cdot5+2^3):(3^3-5^2)
]\) \sol{1}
\immagine*{Testo alternativo: \(\{5\cdot16-(6^2-2^4)-[(3^2-p^2)\cdot10-5]\}-
[(2^2\cdot5+2^3):(3^3-5^2)]\)}{}
\end{esercizio}
\begin{esercizio} % \label{ese:1.17}
\([2+15:(2^3\cdot5-3^3+2)]^4:3\cdot2-2\cdot({\dots}-5\cdot12:3)^2\) \sol{4}
\immagine*{Testo alternativo: 
\([2+15:(2^3\cdot5-3^3+2)]^4:3\cdot2-2\cdot(p-5\cdot12:3)^2\)}{}
\end{esercizio}
\begin{esercizio} % \label{ese:1.17}
\(5^2:5\cdot[(3\cdot5^2+4:2):7-2\cdot5]^2+2^{\dots}:2^2-5^2:5\) \sol{8}
\immagine*{Testo alternativo: 
\(5^2:5\cdot[(3\cdot5^2+4:2):7-2\cdot5]^2+2^p:2^2-5^2:5\)}{}
\end{esercizio}
\begin{esercizio} % \label{ese:1.17}
\(12^{10}:12^9+3^2\cdot6^2:6^2+12^2:(5\cdot2^2-19)-(5^4)^{\dots}:5^{10}\)
 \sol{140}
 \immagine*{Testo alternativo: 
\(12^{10}:12^9+3^2\cdot6^2:6^2+12^2:(5\cdot2^2-19)-(5^4)^p:5^{10}\)}{}
\end{esercizio}
\begin{esercizio} % \label{ese:1.17}
\((2^2)^3+(22-5\cdot4)^2+{\dots}^2-4^2\cdot5\) \sol{69}
\immagine*{Testo alternativo: \((2^2)^3+(22-5\cdot4)^2+p^2-4^2\cdot5\)}{}
\end{esercizio}
\begin{esercizio} % \label{ese:1.17}
\((3^5)^3:3^{13}+3^{10}:3^9+9^5\cdot9^{\dots}\cdot9^4:9^{16}\) \sol{13}
\immagine*{Testo alternativo: 
\((3^5)^3:3^{13}+3^{10}:3^9+9^5\cdot9^p\cdot9^4:9^{16}\)}{}
\end{esercizio}
\begin{esercizio} % \label{ese:1.17}
\(3^3\cdot3^7\cdot3^2:(3^6\cdot3^6)+5^2-[6^2+2^2+2\cdot50-(2^3\cdot{\dots})]:
10^2\) \sol{25}
\immagine*{Testo alternativo: 
\(3^3\cdot3^7\cdot3^2:(3^6\cdot3^6)+5^2-[6^2+2^2+2\cdot50-(2^3\cdot p)]:
10^2\)}{}
\end{esercizio}
\begin{esercizio} % \label{ese:1.17}
\((2\cdot5)^3:5^3-(2^{\dots}:2^2)\cdot\{(6-2^2)\cdot[6-5^0-(2^4:2^2)]\}\)
\sol{4}
\immagine*{Testo alternativo: 
\((2\cdot5)^3:5^3-(2^p:2^2)\cdot\{(6-2^2)\cdot[6-5^0-(2^4:2^2)]\}\)}{}
\end{esercizio}
\begin{esercizio} % \label{ese:1.17}
\(2^2\cdot2^6:2^5:2+2^6:(2^{\dots}\cdot2^2)-2^9:2^7+(6^2\cdot2^2):18+7^3:7^2\)
\sol{16}
\immagine*{Testo alternativo: 
\(2^2\cdot2^6:2^5:2+2^6:(2^p\cdot2^2)-2^9:2^7+(6^2\cdot2^2):18+7^3:7^2\)}{}
\end{esercizio}
\begin{esercizio} % \label{ese:1.17}
\(1^4+(21+{\dots}-3^3)^2-(2^2+4^2+6)\) \sol{0}
\immagine*{Testo alternativo: \(1^4+(21+p-3^3)^2-(2^2+4^2+6)\)}{}
\end{esercizio}
% \begin{esercizio} % \label{ese:1.17}
% \(10:(5^4:5^3)+18:(3^{\dots}:3^6)\) \sol{4}
% \end{esercizio}
\begin{esercizio} % \label{ese:1.17}
\((2^4)^5:2^{19}+(4^{\dots})^8:4^{47}\) \sol{6}
\immagine*{Testo alternativo: \((2^4)^5:2^{19}+(4^p)^8:4^{47}\)}{}
\end{esercizio}
\begin{esercizio} % \label{ese:1.17}
\([(7^5\cdot7^{\dots})]:[(7^3)^4]:7^2\) \sol{1}
\immagine*{Testo alternativo: \([(7^5\cdot7^p)]:[(7^3)^4]:7^2\)}{}
\end{esercizio}
\begin{esercizio} % \label{ese:1.17}
\((2\cdot2^{\dots}\cdot2^3\cdot2^4):2^9+(3^3\cdot3^5\cdot3^7):3^{14}\) 
\sol{4}
\immagine*{Testo alternativo: 
\((2\cdot2^p\cdot2^3\cdot2^4):2^9+(3^3\cdot3^5\cdot3^7):3^{14}\)}{}
\end{esercizio}
\begin{esercizio} % \label{ese:1.17}
\(\{[(3^3\cdot3^4)^2:3^6]:3^{\dots}-2\cdot3^2\}:3+\{[(5^2\cdot2-5\cdot2^2):10]
^2+1\}:5\) \sol{5}
\immagine*{Testo alternativo: 
\(\{[(3^3\cdot3^4)^2:3^6]:3^p-2\cdot3^2\}:3+\{[(5^2\cdot2-5\cdot2^2):10]
^2+1\}:5\)}{}
\end{esercizio}
\begin{esercizio} % \label{ese:1.17}
\(1+\{24^4:8^4-5^2\cdot2:[2+2^4:(2^3-2\cdot3)]\}:\{[20^{\dots}:(2\cdot10)^6-2^
2\cdot5^2]:10^2+1\}\) \sol{20}
\immagine*{Testo alternativo: 
\(1+\{24^4:8^4-5^2\cdot2:[2+2^4:(2^3-2\cdot3)]\}:
\{[20^p:(2\cdot10)^6-2^2\cdot5^2]:10^2+1\}\)}{}
\end{esercizio}
\begin{esercizio} % \label{ese:1.17}
\(\{21+[(2^9:2^6+3^2\cdot3^2\cdot5-5^3\cdot3):19]^2-
(7\cdot2^3+5^2\cdot5-{\dots}^2:2^2):29\}^2:100\) \sol{4}
\immagine*{Testo alternativo: 
\(\{21+[(2^9:2^6+3^2\cdot3^2\cdot5-5^3\cdot3):19]^2-
(7\cdot2^3+5^2\cdot5-p^2:2^2):29\}^2:100\)}{}
\end{esercizio}
% \begin{esercizio} % \label{ese:1.17}
% \([2^4+({\dots}+3^6:3^2):5-(17^6:17^6)]:17-[(17^4:17^4)+2^2\cdot(2^3-1)-2^4]:
% 13\) \sol{1}
% \immagine*{Testo alternativo: 
% \([2^4+(p+3^6:3^2):5-(17^6:17^6)]:17-[(17^4:17^4)+2^2\cdot(2^3-1)-2^4]:
% 13\)}{}
% \end{esercizio}
% \begin{esercizio} % \label{ese:1.17}
% \((3^2\cdot3^5\cdot3^3):(3^3\cdot3)^{\dots}+4-3^3:3^2+5^7:5^6\) \sol{15}
% \end{esercizio}

%\subsubsection*{1.8 - Scomposzione}
\subsubsection*{\numnameref{sec:nat_divsibilita}}

\begin{esercizio}[Crivello di Eratostene]
% \label{ese:1.17}
\begin{htmulticols}{2}
Nella tabella che segue sono rappresentati i numeri naturali fino a~100. 
Per trovare i numeri primi, cancella~1
poi seleziona 2 e cancella tutti i multipli di~2. 
Seleziona 3 e cancella i multipli di 3. 
Seleziona il primo dei numeri che non è stato cancellato, il 5, e cancella
tutti i multipli di~5. 
Procedi in questo modo fino alla fine della tabella. 
Quali sono i numeri primi minori di~100?
\columnbreak\vfil
% \eratostene
% \input{\folder lbr/tab003_eratos.pgf}
\end{htmulticols}
\immagine*{Testo alternativo: tabella 10 righe per 10 colonne. 
Si parte da 1 in alto a sinistra e si arriva a 10 in alto a destra; 
nella riga sotto si parte da 11 da sinistra e si arriva a 20 a destra; 
si continua così fino ad arrivare a 91 in basso a sinistra e si arriva 
a 100 in basso a destra.}{}
\end{esercizio}

\newcommand{\numeriboxati}{
\fbox{2} ~~ \fbox{3} ~~ \fbox{4} ~~ \fbox{5} ~~ \fbox{6} ~~ 
\fbox{7} ~~ \fbox{8} ~~ \fbox{9} ~~ \fbox{10} ~~ \fbox{11} ~~ 
\fbox{13}}

\newcommand{\itemdiv}[1]{
#1 è divisibile per \tab  \numeriboxati\\[-.5em]}

% \newcommand{\itemdiv}[1]{
% #1 è divisibile per ~~  \numeriboxati\\[-.5em]}

\begin{esercizio}
% \label{ese:1.18}
Per quali numeri sono divisibili? Segna i divisori con una crocetta.

\medskip
\TabPositions{37mm}
 \begin{enumeratees}
%  \item 1320 è divisibile per 
% \tab  \numeriboxati
 \item \itemdiv{1320}
 \item \itemdiv{2344}
 \item \itemdiv{84}
 \item \itemdiv{1255}
 \item \itemdiv{165}
 \item \itemdiv{720}
 \item \itemdiv{792}
 \item \itemdiv{462}
 \end{enumeratees}
\end{esercizio}

%\subsubsection*{1.9 - Scomposizione in fattori primi}
\subsubsection*{\numnameref{sec:nat_scomposizione}}

\begin{esercizio}
% \label{ese:1.20}
I numeri sotto elencati sono scritti come prodotto di altri numeri: 
sottolinea 
le scritture in cui ciascun
numero è scomposto in fattori primi
\begin{htmulticols}{2}
 \begin{enumeratees}
 \item \(68=17\cdot4=17\cdot2^2=2\cdot34\)
 \item \(45=5\cdot9=15\cdot3=5\cdot3^2\)
 \item \(36=6\cdot6=6^2\)
 \item \(44=2\cdot22=4\cdot11=2^2\cdot11\)
 \item \(17=17\cdot1\)
 \item \(48=6\cdot8=12\cdot4=3\cdot2^4=16\cdot3\)
 \item \(60= 2\cdot30=15\cdot4=2^2\cdot3\cdot5=10\cdot6\)
 \item \(102=6\cdot17=3\cdot34=2\cdot3\cdot17=2\cdot51\)
 \item \(200=2\cdot10^2=2^3\cdot5^2=2\cdot4\cdot25\)
 \item \(380=19\cdot10\cdot2=19\cdot5\cdot2^2\)
 \end{enumeratees}
\end{htmulticols}
\end{esercizio}

\begin{esercizio}
% \label{ese:1.21}
 Rispondi alle domande:
 \begin{enumeratees}
 \item ci può essere più di una scomposizione in fattori di un numero?
 \item ci può essere più di una scomposizione in fattori primi di un numero?
 \item quando un numero è scomposto in fattori primi?
 \end{enumeratees}
\end{esercizio}

\begin{esercizio}
% \label{ese:1.22}
Descrivi brevemente la differenza tra le seguenti frasi
\begin{enumeratees}
 \item \(a\) e~\(b\) sono due numeri primi
 \item \(a\) e~\(b\) sono due numeri primi tra di loro
\end{enumeratees}
Fai degli esempi che mettano in evidenza la differenza descritta
\end{esercizio}

% 
% \begin{esercizio}[*]
% % \label{ese:1.23}
% Scomponi i seguenti numeri in fattori primi:
%  \begin{htmulticols}{5}
%  \begin{enumeratees}
%  \item \(52\)
%  \item \(60\)
%  \item \(72\)
%  \item \(81\)
%  \item \(105\)
%  \item \(120\)
%  \item \(135\)
%  \item \(180\)
%  \item \(225\)
%  \item \(525\)
%  \end{enumeratees}
%  \end{htmulticols}
% \end{esercizio}
% 
% % \newcommand{\ebox}{\framebox(4mm,4mm)[c]{~}~~}
% 
% \newcommand{\ebox}{\fbox{\phantom{22}}~~}

\begin{esercizio}
% \label{ese:1.24}
Scomponi i seguenti numeri in fattori primi:
\begin{htmulticols}{6}
\begin{enumeratees}
\item \(52\)       % a
\item \(60\)       % b
\item \(72\)       % c
\item \(81\)       % d
\item \(105\)      % e
\item \(120\)      % f
% \item \(135\)      % g
\item \(180\)      % h
\item \(225\)      % i
\item \(525\)      % j
\item \(675\)      % k
\item \(715\)      % l
\item \(1900\)     % m
% \item \(1078\)     % n
\item \(4050\)     % o
\item \(4536\)     % p
\item \(12150\)    % q
\item \(14256\)    % r
\item \(85050\)    % s
\item \(138600\)   % t
\end{enumeratees}
\end{htmulticols}
\noindent\!\sframeop{b} \(2^2 \cdot 3 \cdot 5\); \quad 
\sframeop{a} \(2^2 \cdot 13\); \quad 
\sframeop{l} \(5 \cdot 11 \cdot 13\); \quad 
\sframeop{m} \(2^2 \cdot 5^2 \cdot 19\); \quad 
\sframeop{s} \(2 \cdot 3^5 \cdot 5^2 \cdot 7\); \quad 
% \sframeop{n} \(2 \cdot 7^2 \cdot 11\); \quad 
\sframeop{c} \(2^3 \cdot 3^2\); \quad 
\sframeop{d} \(3^4\); \quad 
\sframeop{h} \(2^2 \cdot 3^2 \cdot 5\); \quad 
\sframeop{j} \(3 \cdot 5^2 \cdot 7\); \quad 
\sframeop{r} \(2^4 \cdot 3^4 \cdot 11\); \\
\sframeop{t} \(2^3 \cdot 3^2 \cdot 5^2 \cdot 7 \cdot 11\); \quad
\sframeop{q} \(2 \cdot 3^5 \cdot 5^2\); 
% \sframeop{g} \(3^3 \cdot 5\); \quad 
\sframeop{i} \(3^2 \cdot 5^2\); \quad 
\sframeop{e} \(3 \cdot 5 \cdot 7\); \\
\sframeop{k} \(3^3 \cdot 2^5\);\quad 
\sframeop{o} \(2 \cdot 3^4 \cdot 5^2\); \quad 
\sframeop{p} \(2^3 \cdot 3^4 \cdot 7\); \quad 
\sframeop{f} \(2^3 \cdot 3 \cdot 5\). 
\end{esercizio}

%\subsubsection*{1.10 - Massimo Comune Divisisore e minimo comune multiplo}
\subsubsection*{\numnameref{sec:nat_mcdemcm}}

\begin{esercizio}
% \label{ese:1.25}
Applicando la definizione~\ref{def:mcd} trova il~\(\mcd\) tra i numeri~54 
e~132
\end{esercizio}


\begin{esercizio}
% \label{ese:1.26}
Calcola~\(\mcd\) e~\(\mcm\) dei numeri~180,~72,~90

Scomponendo in fattori si 
ha~\(180=~2^2\cdot3^2\cdot5\)~\(72 =~2^3\cdot3^2\)~\(90 =~2\cdot3^2\cdot5\)

\(\mcd =2^{\ldots}\cdot3^{\ldots}=\ldots\) \quad \quad \quad ; \quad
\immagine*{Testo alternativo: \(\mcd =2^p\cdot3^p=p\)}{}
\(\mcm =2^{\ldots}\cdot3^{\ldots}\cdot5^{\ldots}=\ldots\) \quad \quad \quad
\immagine*{Testo alternativo: \(\mcm =2^p\cdot3^p\cdot5^p=p\)}{}

\end{esercizio}

% \pagebreak %-----------------------------------------------

\begin{esercizio}
% \label{ese:1.27}
Calcola~\(\mcm\) e~\(\mcd\) tra i seguenti gruppi di numeri:
\begin{htmulticols}{4}
%  \begin{enumeratea}
 \begin{enumeratees}
 \item \(15; 5; 10\)
 \item \(2; 4; 8\)
 \item \(2; 1; 4\)
 \item \(5; 6; 8\)
 \item \(24; 12; 16\)
 \item \(6; 16; 26\)
 \item \(6; 8; 12\)
 \item \(50; 120; 180\)
 \item \(20; 40; 60\)
 \item \(16; 18; 32\)
 \item \(30; 60; 27\)
 \item \(45; 15; 35\)
 \item \(24; 12; 16\)
 \item \(6; 4; 10\)
 \item \(5; 4; 10\)
 \item \(12; 14; 15\)
 \item \(3; 4; 5\)
 \item \(6; 8; 12\)
 \item \(15; 18; 21\)
 \item \(12; 14; 15\)
 \item \(15; 18; 24\)
 \item \(100; 120; 150\)
 \item \(44; 66; 12\)
 \item \(24; 14; 40\)
 \end{enumeratees}
\end{htmulticols}
\noindent\!\sframeop{a} 5; 
\sframeop{i} 120;
\sframeop{j} 2; 
\sframeop{k} 3; 
\sframeop{p} 420;
\sframeop{q} 1; 
\sframeop{r} 24;
\sframeop{o} 20;
\sframeop{k} 540;
\sframeop{p} 1; \\
\sframeop{a} 30;
\sframeop{b} 2; 
\sframeop{q} 60;
\sframeop{r} 2; 
\sframeop{b} 8;
\sframeop{c} 1; 
\sframeop{f} 624;
\sframeop{g} 2; 
\sframeop{u} 360;
\sframeop{v} 10; \\
\sframeop{s} 3; 
\sframeop{c} 4;
\sframeop{d} 1; 
\sframeop{j} 288;
\sframeop{d} 120;
\sframeop{e} 4; 
\sframeop{g} 24;
\sframeop{l} 315;
\sframeop{m} 4; 
\sframeop{n} 60; \\
\sframeop{h} 10; 
\sframeop{h} 1800;
\sframeop{i} 20; 
\sframeop{t} 420;
\sframeop{u} 3; 
\sframeop{e} 48;
\sframeop{f} 2; 
\sframeop{s} 630;
\sframeop{t} 1; \\
\sframeop{l} 5; 
\sframeop{o} 1; 
\sframeop{v} 600;
\sframeop{w} 2; 
\sframeop{w} 132;
\sframeop{x} 2; 
\sframeop{m} 48;
\sframeop{n} 2; 
\sframeop{x} 840.

\end{esercizio}

\begin{htmulticols}{2}
\begin{esercizio}
% \label{ese:1.29}
Tre funivie partono contemporaneamente da una stessa stazione sciistica. La 
prima compie il tragitto di andata e ritorno in~15 minuti, la seconda in~18 
minuti, la terza in~20. 
Dopo quanti minuti partiranno di nuovo insieme? 
\sol{3h}
\end{esercizio}

\begin{esercizio}
% \label{ese:1.30}
Due aerei partono contemporaneamente dall'aeroporto di Milano e vi 
ritorneranno dopo aver percorso le loro rotte: il primo ogni~15 giorni e il 
secondo ogni~18 giorni. 
Dopo quanti giorni i due aerei si troveranno di nuovo insieme a Milano? 
\sol{90g}
\end{esercizio}

\begin{esercizio}
% \label{ese:1.31}
Disponendo di~56 penne,~70 matite e~63 gomme, quante confezioni uguali si 
possono fare? Come sarà composta ciascuna confezione? 
\sol{7, 8p, 10m, 9g}
\end{esercizio}

\begin{esercizio}
% \label{ese:1.32}
Una cometa passa in prossimità della Terra ogni~360 anni, una seconda 
ogni~240 anni e una terza ogni~750 anni.
Se quest'anno sono state avvistate tutte e tre, fra quanti anni sarà 
possibile vederle di nuovo tutte e tre nello stesso anno? 
\sol{18\,000}
\end{esercizio}
\end{htmulticols}


\subsection{Esercizi riepilogativi}
\begin{esercizio}
% \label{ese:1.34}
Quali delle seguenti scritture rappresentano numeri naturali?
 \begin{htmulticols}{4}
 \begin{enumeratees}
 \item \(5+3-1\)
 \item \(6+4-10\)
 \item \(5-6+1\)
 \item \(7+2-10\)
 \item \(2\cdot5:5\)
 \item \(2\cdot3:4\)
 \item \(3\cdot4-12\)
 \item \(12:4-4\)
 \item \(11:3+2\)
 \item \(27:9:3\)
 \item \(18:2-9\)
 \item \(10-1:3\)
 \end{enumeratees}
 \end{htmulticols}
\end{esercizio}


\begin{esercizio}
% \label{ese:1.35}
Calcola il risultato delle seguenti operazioni nei numeri naturali; 
alcune operazioni non sono possibili, individuale
 \begin{htmulticols}{4}
 \begin{enumeratees}
 \item \(5:5=\ldots\)
 \item \(5:0=\ldots\)
 \item \(1\cdot 5 =\ldots\)
 \item \(1-1=\ldots\)
 \item \(10:2=\ldots\)
 \item \(0:5=\ldots\)
 \item \(5\cdot1=\ldots\)
 \item \(0:0=\ldots\)
 \item \(10:5=\ldots\)
 \item \(1:5=\ldots\)
 \item \(0\cdot5=\ldots\)
 \item \(5:1=\ldots\)
 \item \(0\cdot0=\ldots\)
 \item \(1\cdot0=\ldots\)
 \item \(1:0=\ldots\)
 \item \(1:1=\ldots\)
 \end{enumeratees}
 \end{htmulticols}
\end{esercizio}

% \pagebreak %--------------------------------------

\begin{esercizio}
% \label{ese:1.36}
Aggiungi le parentesi in modo che l'espressione abbia il risultato indicato

\(2+5\cdot3+2=35\) ~~~~ \hspace{30mm} ~~~~ \(2+5\cdot3+2=27\)
\end{esercizio}

\begin{esercizio}
% \label{ese:1.37}
Traduci in espressioni aritmetiche le seguenti frasi e calcola il risultato:
\begin{enumeratees}
\item aggiungi~12 al prodotto tra~6 e~4.
\sol{36}
\item sottrai il prodotto tra~12 e~2 alla somma tra~15 e~27.
\sol{18}
\item moltiplica la differenza tra~16 e~7 con la somma tra~6 e~8.
\sol{126}
\item al doppio di~15 sottrai la somma dei prodotti di~3 con~6 e di~2 con~5.
\sol{2}
\item sottrai il prodotto di~6 per~4 al quoziente tra~100 e~2.
\sol{26}
\item moltiplica la differenza di~15 con~9 per la somma di~3 e~2.
\sol{30}
\item sottrai al triplo del prodotto di~6 e~2 il doppio del quoziente
tra~16 e~4. 
\sol{28}
\item il quadrato della somma tra il quoziente intero di~25 e~7 e il cubo 
di~2. 
\sol{121}
\item la somma tra il quadrato del quoziente intero di~25 e~7 e il quadrato 
del cubo di~2.
\sol{73}
\item la differenza tra il triplo del cubo di~5 e il doppio del quadrato 
di~5.
\sol{325}
\end{enumeratees}
% \paragraph{} a)~36;\quad b)~18;\quad c)~126;\quad d)~2;\quad e)~26;
% \quad f)~30; \quad g)~28;\quad h)~121;\quad i)~73; %\quad j)~325;
\end{esercizio}

Le espressioni che seguono sono state elaborate a partire da quelle che si 
possono trovare all'indirizzo:
\href{http://www.ubimath.org/potenze}{www.ubimath.org/potenze}
Ringrazio Ubaldo Pernigo per la competenza e disponibilità

Calcola il valore delle seguenti espressioni:

\begin{esercizio} % \label{ese:1.17}
\((11^2-10^4:10^2):3+2\cdot[(5^2-2\cdot9)^2-14\cdot3]+5\cdot2^2\) \sol{41}
\end{esercizio}
\begin{esercizio} % \label{ese:1.17}
\([10\cdot(2\cdot5-7)-3^4:3^2]:3+[(5\cdot2^2+2^2+2^1):13]^3\) \sol{15}
\end{esercizio}
\begin{esercizio} % \label{ese:1.17}
\(1+[12^4:4^4-2\cdot5^2:(2^3+2^4:2^3)]:
  \{[20^5:(10\cdot2)^3-10^2]:(3\cdot5^2)\}\) \sol{20}
\end{esercizio}
\begin{esercizio} % \label{ese:1.17}
\((2^3)^2:(5\cdot4-2^2)+[7^2:(5^2-3^2\cdot2)+13^7:13^6]:2^2+1^7\) \sol{10}
\end{esercizio}
\begin{esercizio} % \label{ese:1.17}
\((5\cdot2^2-2)^4:(2^3+1)^4+(2\cdot2^3-2\cdot5)^3:(3\cdot5-3^2)^3-2^3\)
 \sol{9}
\end{esercizio}

\begin{esercizio} % \label{ese:1.17}
\((13+3\cdot5^2:3+15+19):(3\cdot2^2)+(2^3-2^2-2)\cdot170^0\) \sol{7}
\end{esercizio}
\begin{esercizio} % \label{ese:1.17}
\(5^1+2\cdot(4^2+2\cdot7-15)-(7^2-5^2-4^2)\cdot2^2+7\) \sol{10}
\end{esercizio}
\begin{esercizio} % \label{ese:1.17}
\([2^4+(2^5:2^4+2\cdot3)\cdot2^2]:2^3+10-4^2+3^3:3^2\) \sol{3}
\end{esercizio}
\begin{esercizio} % \label{ese:1.17}
\([(9^2-7^2):(3^2-1)+(8^2-5^2):(3^2+2^2)]\cdot5\) \sol{35}
\end{esercizio}
\begin{esercizio} % \label{ese:1.17}
\([(3^2\cdot2^3-2\cdot5^2+2^{11}:2^4):(3\cdot5)-2]:(4^2-2^3)\) \sol{1}
\end{esercizio}
\begin{esercizio} % \label{ese:1.17}
\(2^{10}:2^8+3^2-2^2\cdot3^0+4^2-2^3\) \sol{17}
\end{esercizio}
\begin{esercizio} % \label{ese:1.17}
\([5+2^2\cdot3^2-5\cdot(2^4-2^2-2^2+3^2-27:3)]\cdot3^0\cdot3^2\) \sol{9}
\end{esercizio}
% \begin{esercizio} % \label{ese:1.17}
% \((3^2+2^2\cdot3-3\cdot7):5^0+7^2-6^2+2^2\cdot5^0\cdot(2^0+3^0)\) \sol{21}
% \end{esercizio}
% \begin{esercizio} % \label{ese:1.17}
% \(2^2+3^2-2^2\cdot3+5^2-4\cdot3+2^3\cdot1+2^3\cdot3-5^0\) \sol{45}
% \end{esercizio}
% \begin{esercizio} % \label{ese:1.17}
% 
% \(2^3+[(2^2+2^2):2^3+2\cdot3-(2^2+1)]\cdot\{6+[2^3+(3^2-2)]:5\}-(5^2-5)\)\
% hfill[6}
% \end{esercizio}
% \begin{esercizio} % \label{ese:1.17}
% 
% \([(2^2\cdot21+2^2\cdot3^3):2^3+2^3\cdot(15^3:15^2)]:12+(78-90:5):6\)
%  \sol{22}
% \end{esercizio}

Calcola il valore mancante nelle seguenti espressioni:

\begin{esercizio} % \label{ese:1.17}
\(2^2\cdot5-(5^2-2^3)+(24:3+{\dots}:2^3)^2:(10^2+2^2\cdot11)\) \sol{4}
\immagine*{Testo alternativo: 
\(2^2\cdot5-(5^2-2^3)+(24:3+p:2^3)^2:(10^2+2^2\cdot11)\)}{}
\end{esercizio}
\begin{esercizio} % \label{ese:1.17}
\(1+(3\cdot2^4:2^3+26^3:{\dots}^3)^2:(12^2-11^2-7\cdot3+5)^2-15:3+3\) \sol{3}
\immagine*{Testo alternativo: 
\(1+(3\cdot2^4:2^3+26^3:p^3)^2:(12^2-11^2-7\cdot3+5)^2-15:3+3\)}{}
\end{esercizio}
\begin{esercizio} % \label{ese:1.17}
\(3^2:\{5\cdot2^4+6\cdot{\dots}^2-[(21\cdot5-3^2\cdot2^3):11+2]^3\}\) \sol{1}
\immagine*{Testo alternativo: 
\(3^2:\{5\cdot2^4+6\cdot p^2-[(21\cdot5-3^2\cdot2^3):11+2]^3\}\)}{}
\end{esercizio}
\begin{esercizio} % \label{ese:1.17}
\(33^4:\{24^2:[19^3:(3^2\cdot2+4^{\dots})^2+5]^2+2^{5}\}^3\) \sol{33}
\immagine*{Testo alternativo: 
\(33^4:\{24^2:[19^3:(3^2\cdot2+4^p)^2+5]^2+2^{5}\}^3\)}{}
\end{esercizio}
\begin{esercizio} % \label{ese:1.17}
\((13+2^2+75:{\dots}+2\cdot3^2):(3\cdot2^2)+(2^3-2^2-2)\cdot17^0\) \sol{8}
\immagine*{Testo alternativo: 
\((13+2^2+75:p+2\cdot3^2):(3\cdot2^2)+(2^3-2^2-2)\cdot17^0\)}{}
\end{esercizio}
\begin{esercizio} % \label{ese:1.17}
\(35:7+13\cdot2^2-{\dots}:2^3-11\cdot3-84:7\) \sol{0}
\immagine*{Testo alternativo: 
\(35:7+13\cdot2^2-p:2^3-11\cdot3-84:7\)}{}
\end{esercizio}
\begin{esercizio} % \label{ese:1.17}
\((15:3+7^2-2\cdot5):4+[(3\cdot2^2)+{\dots}^2-4^2]^2\) \sol{36}
\immagine*{Testo alternativo: 
\((15:3+7^2-2\cdot5):4+[(3\cdot2^2)+p^2-4^2]^2\)}{}
\end{esercizio}
\begin{esercizio} % \label{ese:1.17}
\((5^2-3^2\cdot2):7+({\dots}^2-4^3):(3^0+3+3^2)\) \sol{1}
\immagine*{Testo alternativo: 
\((5^2-3^2\cdot2):7+(p^2-4^3):(3^0+3+3^2)\)}{}
\end{esercizio}
\begin{esercizio} % \label{ese:1.17}
\([(2^{\dots}\cdot7+3^3\cdot2^2):11]:(2^3\cdot15-10^2)+(52:13):2\) \sol{3}
\immagine*{Testo alternativo: 
\([(2^p\cdot7+3^3\cdot2^2):11]:(2^3\cdot15-10^2)+(52:13):2\)}{}
\end{esercizio}
\begin{esercizio} % \label{ese:1.17}
\(3^7:3^5+8^2+2^{\dots}\cdot2^7:2^{11}\) \sol{75}
\immagine*{Testo alternativo: 
\(3^7:3^5+8^2+2^p\cdot2^7:2^{11}\)}{}
\end{esercizio}
\begin{esercizio} % \label{ese:1.17}
\([({\dots}+5\cdot2-2\cdot11)\cdot2^2+(3^2-2^3)]\cdot(8^2-7\cdot9)\) \sol{1}
\immagine*{Testo alternativo: 
\([(p+5\cdot2-2\cdot11)\cdot2^2+(3^2-2^3)]\cdot(8^2-7\cdot9)\)}{}
\end{esercizio}
\begin{esercizio} % \label{ese:1.17}
\([14+(13-6)^2:(3^2-2^1)-2^{\dots}:2^4]:5+10-[(5^3:5^2+7^2-6^2):3^2]^3\)
 \sol{3}
\end{esercizio}
\begin{esercizio} % \label{ese:1.17}
\(\{18^4:6^4-2\cdot5^2:[2^4:({\dots}^3-6)+2]\}:
\{[20^5:(2\cdot10)^3-10^2]:^10^2+1\}+1\) \sol{20}
\immagine*{Testo alternativo: 
\(\{18^4:6^4-2\cdot5^2:[2^4:(p^3-6)+2]\}:
\{[20^5:(2\cdot10)^3-10^2]:^10^2+1\}+1\)}{}
\end{esercizio}

% \begin{esercizio} % \label{ese:1.17}
% \(3^3+4^2\cdot5-5^2\cdot3^{\dots}-12^0\cdot3^2+(2^2+3^0+1):6\) \sol{24}
% \end{esercizio}
% \begin{esercizio} % \label{ese:1.17}
% \(27:3^2+2^{\dots}\cdot5-20\cdot2^0+12:2^2+5^3\cdot1-8^2\) \sol{67}
% \end{esercizio}
% \begin{esercizio} % \label{ese:1.17}
% \((6^2+6)\cdot\{3^3:3^2\cdot[11\cdot2\cdot(7\cdot2^2-{\dots}\cdot2):
%   11-5 \cdot 2^2]-3^2\}:(7\cdot5)-18\) \sol{0}
% \end{esercizio}

\begin{htmulticols}{2}

\begin{esercizio}
% \label{ese:1.43}
Un'automobile percorre \(18\munit{km}\) con~1 litro di benzina. 
Quanta benzina deve aggiungere il proprietario dell'auto
sapendo che l'auto ha già~12~litri di benzina nel serbatoio, che deve 
intraprendere un viaggio di \(432\munit{km}\) e che deve
arrivare a destinazione con almeno~4~litri di benzina nel serbatoio? 
 \sol{\text{Almeno } 16}
\end{esercizio}

\begin{esercizio}
% \label{ese:1.44}
Alla cartoleria presso la scuola una penna costa~3~euro più di una matita. 
Gianni ha comprato~2~penne e~3~matite e ha speso
16~euro. Quanto spenderà Marco che ha comprato~1~penna e~2~matite? 
 \sol{9\meuro}
\end{esercizio}

\begin{esercizio}
% \label{ese:1.45}
In una città le linee della metropolitana iniziano il loro servizio alla 
stessa 
ora. La linea rossa fa una corsa ogni 25~min, la linea gialla ogni~20~min 
e la linea blu ogni~30~min. Salvo ritardi, ogni quanti minuti le tre linee
partono allo stesso momento? 
\sol{5\munit{h}}
\end{esercizio}

\begin{esercizio}
% \label{ese:1.46}
 Tre negozi si trovano sotto lo stesso porticato, ciascuno ha un'insegna 
luminosa intermittente: la prima si spegne ogni
6~secondi, la seconda ogni~7~secondi, la terza ogni~8~secondi. Se tutte le 
insegne vengono accese alle~19.00 e spente alle~21.00, quante 
volte durante la serata le tre insegne si spegneranno contemporaneamente?
\sol{42}
\end{esercizio}

\begin{esercizio}
% \label{ese:1.47}
In una gita scolastica ogni insegnante accompagna un gruppo di~12~studenti. 
Se alla gita partecipano~132~studenti, quanti insegnanti occorrono?
\sol{9}
\end{esercizio}

\begin{esercizio}
% \label{ese:1.48}
Un palazzo è costituito da~6~piani con~4~appartamenti per ogni piano. 
Se ogni appartamento ha~6~finestre con~2~vetri ciascuna, 
quanti vetri ha il palazzo?
\sol{288}
\end{esercizio}

\begin{esercizio}
% \label{ese:1.49}
Spiega brevemente il significato delle seguenti parole:
 \begin{enumeratees}
 \item numero primo
 \item numero dispari
 \item multiplo
 \item cifra
 \end{enumeratees}
\end{esercizio}

\begin{esercizio}
% \label{ese:1.50}
Rispondi brevemente alle seguenti domande:
 \begin{enumeratees}
 \item cosa vuol dire scomporre in fattori un numero?
 \item ci può essere più di una scomposizione in fattori di un numero?
 \item cosa vuol dire scomporre in fattori primi un numero?
 \end{enumeratees}
\end{esercizio}

\end{htmulticols}
