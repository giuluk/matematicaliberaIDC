% (c) 2012 - 2014 Dimitrios Vrettos - d.vrettos@gmail.com
% (c) 2014 Claudio Carboncini - claudio.carboncini@gmail.com
% (c) 2014 Daniele Zambelli - daniele.zambelli@gmail.com

\input{\folder interi_grafici.tex}

\chapter{Numeri interi relativi}

\inicapitolo{
\begin{itemize}
\item i numeri interi;
\item la funzione \emph{valore assoluto};
\item il confronto tra interi;
\item le operazioni con gli interi e le loro proprietà.
\end{itemize}
}

\section{I numeri che precedono lo zero}
\label{sec:int_negativi}

Con i numeri naturali non sempre è possibile eseguire l'operazione di 
sottrazione. 
In particolare, non è possibile sottrarre un numero più grande da un numero 
più piccolo, per esempio~\(5-12\). 
Tuttavia ci sono situazioni in cui una sottrazione di questo tipo deve essere 
eseguita.

Per esempio, è possibile acquistare un'auto di \officialeuro\ 12\,000 pur 
avendo soltanto risparmi in banca di soli \officialeuro\ 5\,000. 
In questo caso si tratta di togliere dai \officialeuro\ 
5\,000 i \officialeuro\ 12\,000 che servono per acquistare 
l'auto: materialmente non è possibile e si ricorre a un prestito.

Pensiamo a una comunicazione dei meteorologi relativa alle previsioni del 
tempo: <<domani la temperatura, a causa di una perturbazione proveniente dai 
paesi nordici, potrebbe subire un drastico calo e scendere anche di~10 
gradi>>. 
Riflettiamo: se oggi la temperatura è di~9 gradi, come possiamo esprimere 
numericamente la temperatura prevista per domani? 
Alcuni diranno: <<il liquido contenuto nel termometro si posizionerà al di 
sotto dello zero>>,
altri <<domani la temperatura sarà di un grado sotto lo zero>> e
altri ancora <<la temperatura sarà di~\(-1\) grado>>.

\affiancati{.49}{.49}{
\immagine[.8]{Rappresentazione del monte Everest e della fossa 
delle Marianne}{\everest}
}{
Leggiamo nel testo di geografia: <<Il punto più profondo della Terra si 
trova nella fossa delle Marianne; esso supera di~2\,061 metri l'altezza del 
monte Everest e si trova a~10\,916 metri sotto il livello del mare>>.
Se attribuiamo al livello del mare il valore zero, allora potremmo 
esprimere la profondità della Fossa con il numero~\(-10\,916\) e l'altezza 
del monte Everest con il numero~\(+8\,855\).
}

Per rappresentare le grandezze che hanno due sensi, come temperature, 
crediti e i debiti, latitudine nord e sud, altezze sopra il livello del 
mare e profondità marine con i numeri naturali, siamo costretti a aggiungere 
diversi giri di parole per le diverse situazioni. 
I matematici, in tutte queste situazioni usano i numeri interi relativi che 
si scrivono utilizzando gli stessi numeri naturali ma preceduti dal 
segno~``\(+\)'' se sono numeri maggiori di~0 e dal segno~``\(-\)'' se sono 
numeri minori di~0. L'insieme di questi numeri si costruisce raddoppiando 
i numeri naturali~\(\N\) e facendo precedere ciascun numero dal 
segno~``\(+\)'' o~``\(-\)'',
ad eccezione dello~0, al quale non si attribuisce segno.
\indc{\(\Z\)}{insieme \(\Z\)}
\[\Z=\lbrace\ldots,~ -5,~ -4,~ -3,~ -2,~ -1,~~~0,~ +1,~ +2,~ +3,~ +4,~ +5,~ 
\ldots \rbrace\]
%\newpage

\section{I numeri interi relativi e la retta}
\label{sec:int_retta}

Anche numeri relativi possono essere rappresentati su una retta. 
Disegniamo una \indtc{retta}{retta degli interi}, su di essa prendiamo 
un punto di riferimento al quale 
associamo il numero zero, il \indt{verso di percorrenza} da sinistra verso 
destra, un segmento~\(AB\) come un'unità di misura. 
Riportiamo questa unità di misura più volte partendo da zero e procedendo nel 
verso stabilito aggiungiamo ogni volta uno: ai punti trovati associamo gli 
interi positivi.
Ripetiamo l'operazione partendo dallo zero, ma con il verso di percorrenza 
a sinistra: ai punti trovati associamo gli interi negativi.

\immagine[.8]
{Figura: Retta su cui sono rappresentati numeri sempre più piccoli verso 
sinistra (-1, -2, -3, \dots) e sempre più grandi verso destra 
(+1, +2, +3, \dots).}
{\intrappgeo}

Possiamo interpretare questi numeri come il numero di passi da fare sulla 
retta, partendo dallo zero verso destra se il segno è positivo, verso 
sinistra se il segno è negativo.

\affiancati{.59}{.39}{
L'insieme dei \indtc{numeri relativi}{numeri interi relativi} si indica con il 
simbolo~\(\Z\). 
In particolare, l'insieme dei \indtc{numeri relativi}{numeri interi positivi} 
si indica con il 
simbolo~\indtc{\(\Z\)}{\(\Z^+\)}, l'insieme dei \indtc{numeri relativi}{numeri 
interi negativi} si indica con il simbolo~\indtc{\(\Z\)}{\(\Z^-\)}. \\
Possiamo quindi dire che:
\(\Z = \Z^- ~\scup~ \{0\} ~\scup~ \Z^+\).
}{

\immagine[.9]
{Figura: in -7 il segno è meno e il valore assoluto è 7,
in +5 il segno è più e il valore assoluto è 5.}
{\numerointero}
}

\begin{definizione}{Numeri interi}{}
Un numero intero è un numero naturale, preceduto da un segno,
%si ottiene da un numero naturale con l'aggiunta di un segno 
che può essere: \(-\) o \(+\).
\end{definizione}

\begin{osservazione}{}{}
Il sottoinsieme formato dallo zero e dai numeri interi positivi, 
\(\{0\} ~\scup~ \Z^+\) 
è isomorfo\footnote{
cioè funziona esattamente come i numeri naturali} 
all'insieme dei numeri naturali.
Questo permette di non considerare il segno \(+\) senza che sorgano 
problemi: \(+13 = 13\)
\end{osservazione}

\begin{definizione}{}{}
Due numeri relativi si dicono \emph{concordi}, se hanno lo stesso segno; 
si dicono \emph{discordi} se hanno segni opposti.
\indc{numeri relativi}{numeri concordi}
\indc{numeri relativi}{numeri discordi}
\end{definizione}

\begin{esempio}{}{}
Concordi-discordi.

\(+3\) e~\(+5\) sono concordi; \quad 
\(+3\) e~\(-5\) sono discordi; \quad 
\(-5\) e~\(-2\) sono concordi.
\end{esempio}

\begin{definizione}{Valore assoluto}{}
La funzione \emph{valore assoluto}\ind{valore assoluto}
\indc{valore assoluto}{abs(n)}
(\(\mathrm{abs}(n)\)) se riceve come argomento 
un numero intero qualsiasi dà come risultato un numero intero non negativo.
È definita come:
\[\mathrm{abs}(x) = 
\sistema{-x & \stext{ se } & x < 0 \\ +x & \stext{ se } & x \geqslant 0}\]
\end{definizione}

Il valore assoluto permette di passare dai numeri interi ai numeri naturali. 
In questo passaggio si perde, ovviamente, dell'informazione.

Nelle espressioni il valore assoluto si può indicare inserendo il numero 
relativo tra due barre verticali~(\(\valass{\,}\)).\indc{valore 
assoluto}{\(\valass{\,}\)} In linguaggio matematico:
\[\valass{a}=a,\stext{ se }a\geqslant0,\qquad 
  \valass{a}=-a,\stext{ se }a<0\]

\begin{esempio}{}{}
Valore assoluto: \\
\(\valass{+2}=+2 \quad \valass{-5}=+5 \quad 
\valass{-73}=+73 \quad \valass{+13}=+13\)
\end{esempio}

\begin{definizione}{}{}
Due numeri interi relativi sono \emph{uguali} se hanno lo stesso segno e 
lo stesso valore assoluto;
si dicono \emph{opposti}
\indc{numeri relativi}{numeri opposti} 
se hanno lo stesso valore assoluto ma segni diversi.
\end{definizione}

\begin{esempio}{}{}
Sono numeri opposti: \qquad \(+3\) e \(-3\) \quad \(+5\) e \(-5\) 
\quad \(+19\) e \(-19\).
\end{esempio}

Numeri opposti hanno lo stesso valore assoluto: \qquad 
\(\valass{-17} = \valass{+17} = +17\)

\section{Confronto di numeri relativi}
\label{sec:int_confronto}

Usando la rappresentazione dei numeri sulla retta, 
l'ordinamento\ind{ordinamento degli interi} risulta 
facile da verificare:
il verso di percorrenza della retta (la freccia) indica la direzione nella 
quale i numeri crescono.\indc{retta}{ordinamento degli interi}
Quindi dati due numeri interi relativi quello più grande è quello che, sulla 
retta orientata è rappresentato \emph{dopo} muovendosi nel verso fissato.

Le seguenti definizioni permettono di effettuare il \indtc{numeri 
relativi}{confronto tra interi} relativi usando il confronto tra numeri 
naturali.

\begin{definizione}{Confronto di interi}{}
Le seguenti tre regole permettono di confrontare due numeri interi qualunque.
\begin{enumerate} [noitemsep] %, label=\alph*)]
\item ogni numero negativo è minore di~0 e ogni numero 
positivo è maggiore di~0;
\item tra due numeri positivi il più grande è quello che ha valore 
assoluto maggiore;
\item tra due numeri negativi il più grande è quello che ha valore 
assoluto minore.
\end{enumerate}

\end{definizione}

\begin{esempio}{}{}
Confronta le seguenti coppie di numeri.
\begin{itemize} [noitemsep]
\item \(+4 > +2\): i numeri sono positivi, il maggiore è~\(+4\) perché ha 
valore assoluto maggiore;
\item \(-1 > -3\): i due numeri sono negativi, il maggiore è~\(-1\) perché 
ha valore assoluto minore;
\item \(+4 > -5\): i numeri positivi sono maggiori dei numeri negativi;
\item \(+4 > 0\): ogni numero positivo è maggiore di~0;
\item \(0 > -3\): ogni numero negativo è minore di~0.
\end{itemize}
\immagine[.8]{Figura: retta con evidenziati i numeri dell'esempio.} 
{\rettaconfronto}
\end{esempio}

% \vspazio\ovalbox{\risolvii \ref{ese:2.1}, \ref{ese:2.2}, \ref{ese:2.3}, 
% \ref{ese:2.4}, \ref{ese:2.5}}

\section{Le operazioni con i numeri relativi}
\label{sec:int_operazioni}

Con i numeri relativi anche la sottrazione (oltre all'addizione e alla 
moltiplicazione) è un'operazione interna\ind{legge di composizione interna} 
cioè la differenza di due numeri relativi è un numero relativo.

\subsection{Addizione in 
\texorpdfstring{$\Z: \quad \coppia{\Z}{+}$}{Z: (Z; +)}}
\indc{\(\Z\)}{\(\coppia{\Z}{+}\)}

\affiancati{.64}{.34}{
Il simbolo ``\(+\)'' ha ora due significati completamente diversi:
\begin{itemize} [nosep]
\item simbolo dell'addizione
\item segno del numero
\end{itemize}
Dal contesto dobbiamo capire se \(+\) è un segno del numero o è un simbolo 
di operazione. Lo stesso vale per il simbolo della sottrazione e per il 
segno meno.

All'inizio è bene usare una scrittura del tipo~\((+2)+(+5)\) per 
indicare la somma tra i numeri~\(+2\) e~\(+5\); in seguito vedremo come 
semplificare la scrittura (e complicare l'interpretazione).
}{
\immagine{Addizione tra due numeri relativi.}{\addizioneesegno}
}

La procedura per addizionare due numeri relativi dipende dal segno dei due 
addendi.

\begin{definizione}{Somma di numeri concordi}{}
La \emph{somma di due numeri concordi}\ind{numeri concordi} è il numero che 
ha per valore assoluto\ind{valore assoluto} la \emph{somma} dei valori 
assoluti e come segno lo stesso segno degli addendi.
\end{definizione}

\begin{esempio}{}{}
\( (+3)+(+5)=\ldots\): i due numeri da sommare sono concordi, 
il loro segno è~``\(+\)'', i loro valori assoluti sono~3 e~5,
la loro somma è~8. Pertanto~\((+3)+(+5)=+8\)
\end{esempio}

\begin{esempio}{}{}
~\((-2)+(-5)=\ldots\): i due numeri sono entrambi negativi, quindi sono 
concordi, 
i loro valori assoluti sono~2 e~5,
la somma ha valore assoluto~7, il segno è~``\(-\)''. Pertanto
\[(-2)+(-5)=-7.\]
\end{esempio}

\begin{definizione}{Somma di numeri discordi}{}
La \emph{somma di due numeri discordi}\ind{numeri discordi} è il numero che ha 
per valore assoluto la \emph{differenza} dei valori assoluti
e come segno il segno del numero che ha valore assoluto maggiore.
\end{definizione}

\begin{esempio}{}{}
~\((-5)+(+2)=\ldots\): i due numeri da sommare sono discordi, i loro valori 
assoluti sono~5 e~2, la differenza è~3,
il numero che ha valore assoluto maggiore è~\(-5\), pertanto il risultato 
ha lo stesso segno di~\(-5\), cioè è negativo.
In definitiva~\((-5)+(+2)=-3\)
\end{esempio}

% \begin{esempio}{}{}
% ~\((+5)+(-2)=\ldots\): i due numeri da sommare sono discordi, i loro valori 
% assoluti sono~5 e~2, la loro differenza è~3,
% il numero che ha valore assoluto maggiore è~\(+5\), pertanto il risultato 
% ha lo stesso segno di~\(+5\), cioè è positivo. 
% In definitiva~\((-5)+(-2)=+3\)
% \end{esempio}

\begin{esempio}{}{}
~\((+3)+(-7)=\ldots\): i due numeri da sommare sono discordi, i loro valori 
assoluti sono~3 e~7, la loro differenza è~4,
il numero che ha valore assoluto maggiore è~\(-7\), quindi il risultato ha 
segno negativo.
In definitiva~\((+3)+(-7)=-4\)
\end{esempio}

L'addizione si può rappresentare nella retta dei numeri come l'azione di 
partire dal punto indicato dal primo operando e 
muoversi nel verso indicato dal segno del secondo addendo: 
se è positivo ci si muove nel verso della retta, 
se è negativo ci si muove verso contrario.
\[(-3)+(+5)=2\]
% 
% \vspace{-2.5em}
\immagine[.8]{Retta dei numeri interi con l'addizione tra -3 e +5}
  {\intaddlinea}
% 
% \vspace{-2.5em}
\[ (-1)+(-3) = -4\]
% 
% \vspace{-2.5em}
\immagine[.8]{Retta dei numeri interi con l'addizione tra -1 e -3}
  {\intaddlineb}

\subsubsection{Proprietà dell'addizione in \texorpdfstring{$\Z$}{Z}}

L'addizione in \(\Z\) può essere vista come una funzione che ha per 
argomenti numeri interi e dà come risultato un numero intero.

La struttura \(\coppia{\Z}{+}\)\ind{\(\coppia{\Z}{+}\)} presenta tutte le 
proprietà della struttura \(\coppia{\N}{+}\) più un'importante proprietà:
l'esistenza dell'elemento \emph{inverso}\ind{elemento inverso} di ogni numero 
intero.
L'inverso rispetto all'addizione si chiama \emph{opposto}\ind{numeri opposti}.
Riassumendo:
\begin{itemize} [nosep]
 \item è una \emph{legge di composizione interna};
 \item \emph{Associativa}: \quad \((a + b) + c = a + (b + c)\);
\indc{proprietà}{associativa}
 \item \emph{Elemento neutro è 0}: \quad \(a + 0 = 0 + a = a\);
\indc{proprietà}{elemento neutro}
 \item \emph{Opposto}:\indc{proprietà}{elemento opposto} \quad 
 \(\text{per ogni~ } a \stext{ esiste un } a' \stext{ tale che }
 a + a' = a' + a = 0\);
 \item \emph{Commutativa}: \quad \(a + b = b + a\).
 \indc{proprietà}{commutativa}
\end{itemize}

Avendo queste proprietà, la struttura algebrica \(\coppia{\Z}{+}\) viene 
chiamata \emph{gruppo commutativo}.\ind{gruppo commutativo}

% \ovalbox{\risolvii \ref{ese:2.6}, \ref{ese:2.7}, \ref{ese:2.8}}

\subsection{Sottrazione in 
\texorpdfstring{$\Z: \quad \coppia{\Z}{-}$}{Z: (Z; -)}}
\indc{\(\Z\)}{\(\coppia{\Z}{-}\)}

La nuova proprietà dell'addizione (l'esistenza dell'opposto di ogni numero) 
trasforma la sottrazione in un'operazione interna agli interi:\ind{legge di 
composizione interna}
una funzione che per qualunque coppia di argomenti interi dà come risultato 
un numero intero.
Anzi, non solo la sottrazione si può sempre eseguire, ma addirittura \dots 
non servirà più!
Ma andiamo con ordine.

\vspace{-1em}
\[(+5) - (+7) = -2 \stext{ perché } (-2) + (+7) = +5\]

\immagine[.8]{Retta dei numeri interi con la sottrazione tra +5 e +7}
  {\intsublinea}


\begin{definizione}{Differenza in \(\Z\)}{}
La differenza di due numeri relativi si ottiene aggiungendo al primo 
numero l'opposto del secondo.

\vspace{-2em}
\[a - b = (+a) - (+b) = a + (-b) \sstext{perché:} 
  (+a) + (-b) + (+b) = a + 0 = a\]
\end{definizione}

La funzione sottrazione è quindi definita per qualunque coppia ordinata di 
numeri relativi.

\vspace{1em}
\affiancati{.54}{.44}{
\begin{esempio}{}{}
Sottrazione di numeri relativi.
\begin{enumerate} 
[leftmargin=0cm, itemindent=.5cm, noitemsep, label=(\alph*)]
\item \((+2)-(+3)=(+2)+(-3)=-1\)
\item \((+3)-(-7)=(+3)+(+7)=+10\)
% \item \((+1)-(+3)=(+1)+(-3)=-2\)
\item \((-2)-(-1)=(-2)+(+1)=-1\)
\item \((-5)-(+5)=(-5)+(-5)=-10\)
\end{enumerate}
\end{esempio}
%\end{exrig}
}{
\hspace{-7mm}
\immagine*[.8]
{Figura: Cambio il simbolo della sottrazione in addizione e
cambio il segno del secondo operando.}
{\sottrazioneint}
}
% \ovalbox{\risolvii \ref{ese:2.9}, \ref{ese:2.10}, \ref{ese:2.11}, 
% \ref{ese:2.12}, \ref{ese:2.13}}

% \ifdefined\HCode 
% \subsubsection{Proprietà della sottrazione negli interi}
% \else
% \subsubsection{Proprietà della sottrazione in \(\Z\)}
% \fi

\subsubsection{Proprietà della sottrazione in \texorpdfstring{$\Z$}{Z}}

Negli interi, la sottrazione è una legge di composizione interna e gode 
della proprietà invariantiva\indc{proprietà}{invariantiva} della 
sottrazione:
la differenza di due numeri non cambia se al minuendo e al sottraendo viene 
aggiunto o tolto lo stesso numero.

\subsection{Somma algebrica}

Poiché la sottrazione può essere trasformata in addizione, si può 
semplificare la scrittura di addizione e sottrazione di numeri relativi 
utilizzando soltanto l'operazione di addizione e omettendo di scrivere
il segno~``\(+\)'' dell'addizione. 
Questo tipo di addizione tra numeri relativi si chiama \indt{somma algebrica}.
Facciamolo passo passo:

\vspace{-1em}
\[(-5)-(-7)+(-2)+(+6)-(+3)+(-5) =\]
all'interno delle parentesi ci sono i segni dei numeri e, fuori, i simboli 
delle operazioni.\\
Eliminiamo le sottrazioni trasformandole in addizioni 
con l'opposto del sottraendo:

\vspace{-1em}
\[(-5)+(+7)+(-2)+(+6)+(-3)+(-5) =\]
eliminiamo tutti i simboli di addizione e le parentesi: 

\vspace{-1em}
\[=-5+7-2+6-3-5 =\]
ottenendo così una somma algebrica dove i simboli di addizione sono 
sottintesi: 

\vspace{-1em}
\begin{center}
\(= -5\) \quad più \quad \(+7\) \quad più \quad \(-2\) \quad più \quad 
\(+6\) \quad più \quad \(-3\) \quad più \quad \(-5 =\)\\
\end{center}

Ora possiamo risolverla andando in ordine:
\begin{enumerate} [nosep]
\item sulla retta, partiamo da \(-5\);
\item ogni volta che incontriamo un numero negativo ci spostiamo verso 
sinistra; 
\item ogni volta che incontriamo un numero positivo ci spostiamo verso 
destra.
\end{enumerate}

\immagine[.8]
{Figura: Rappresentazione sulla retta della somma algebrica precedente.}
{\sommalgebrica}

Cioè: parto da \(-5\) avanti di~7 indietro di 2 avanti di~6 indietro 
di~3 indietro di~5.

Oppure, dato che l'addizione gode della proprietà commutativa e associativa, 
e qui abbiamo tutte addizioni,
possiamo addizionare tutti i numeri positivi, tutti i numeri negativi e alla 
fine calcolare la somma dei due risultati:
\[-5+7-2+6-3-5 \quad=\quad +7+6-5-2-3-5 \quad=\quad +13 -15 \quad=\quad -2\]

\begin{esempio}{}{}
~Calcola: \((+1)+(-2)-(-10)+(+3)-(+13)-(-2)=\) 
\begin{description}
\item [trasformo le sottrazioni: \quad ] 
\(=(+1)+(-2)+(+10)+(+3)+(-13)+(+2)=\)
\item [sottintendo le addizioni: \quad ] \(=+1-2+10+3-13+2=\)
\item [eseguo le addizioni: \qquad ~~\, ] \(=+16-15=+1\)
\end{description}
\end{esempio}

Una volta capito il meccanismo, si possono riunire primi due passaggi 
e, se ci sono numeri opposti, la loro somma è nulla.

\begin{esempio}{}{}
~Calcola: \\
\((+13)+(-20)-(-60)+(+5)-(+13)-(-21)+(-60)=\) 

Trasformo in addizione algebrica e annullo i termini opposti:\\
\(=\cancel{+\;13}-20~\cancel{+\;60}+5~\cancel{-\;13}+21~\cancel{-\;60}=\)\\
addiziono i termini rimasti:\hspace{15.3mm}
\(= +26 - 20 = +6\)
\end{esempio}

% \vspazio\ovalbox{\risolvii \ref{ese:2.14}, \ref{ese:2.15}}

\subsection{Moltiplicazione in 
\texorpdfstring{$\Z: \quad \coppia{\Z}{\times}$}{Z: (Z; x)}}

Se dobbiamo moltiplicare due interi entrambi non negativi possiamo rifarci 
alla definizione data per la moltiplicazione tra naturali, ad esempio:
\[(+3) \cdot (+4) = 0 + (+3) + (+3) + (+3) + (+3) = +3+3+3+3 = +12\]
Possiamo fare riferimento alla stessa definizione anche nel caso il primo 
fattore sia negativo:
\[(-3) \cdot (+4) = 0 + (-3) + (-3) + (-3) + (-3) = -3-3-3-3 = -12\]
Se il secondo fattore è negativo, non ha molto senso aggiungere un numero 
negativo di addendi!
Possiamo però utilizzare la proprietà\indtc{proprietà}{commutativa} e 
risolvere il problema:
\[(+3) \cdot (-4) = (-4) \cdot (+3) = 
  0 + (-4) + (-4) + (-4) = -4-4-4 = -12\]
Il problema diventa più spinoso quando entrambi i fattori sono negativi.\\
Come calcolare \((-3) \cdot (-4)\)?\\
Partiamo da un'altra proprietà della moltiplicazione: zero è l'elemento 
assorbente\indc{proprietà}{elemento assorbente}, cioè se moltiplico~0 per un 
qualunque numero, il risultato 
sarà~0.
Quindi: \quad \(0 \cdot (-4) = 0\).\\
Ma zero possiamo scriverlo anche come somma di due 
\indc{numeri relativi}{numeri opposti} numeri opposti: \\ 
\(0 = (+3) + (-3)\).\\
Sostituiamo~0: \quad \(((+3) + (-3)) \cdot (-4) = 0\).\\
Per la proprietà \indtc{proprietà}{distributiva}: \quad 
\((+3) \cdot (-4) + (-3) \cdot (-4) = 0\).\\
Il primo dei due prodotti l'abbiamo già calcolato: \quad 
\((-12) + (-3) \cdot (-4) = 0\).\\
Ma per ottenere~0, il prodotto deve essere l'opposto di \(-12\) quindi: 
\quad \((-3) \cdot (-4) = +12\).

È facile generalizzare quanto visto in questo caso particolare e mostrare che 
se vogliamo che valgano le consuete proprietà delle operazioni, il prodotto 
di due numeri negativi deve essere un numero positivo.

Possiamo sintetizzare quanto visto nella seguente

\begin{definizione}{Prodotto di numeri relativi}{}
Il \emph{prodotto di due numeri interi relativi} è~0 se almeno uno dei due 
fattori è~0, altrimenti è il numero intero avente 
come valore assoluto il prodotto dei valori assoluti dei fattori e 
come segno 
il segno~``\(+\)'' se i fattori sono concordi
\indc{numeri relativi}{numeri concordi},
il segno~``\(-\)'' se i fattori sono discordi
\indc{numeri relativi}{numeri discordi}.
\end{definizione}

\begin{esempio}{}{}
Calcola i prodotti delle seguenti moltiplicazioni.
\begin{enumerate}[noitemsep, label=(\alph*)]
\item \((+3) \cdot (-2)=\) 
il prodotto dei valori assoluti è \(3 \cdot 2 = 6\) 
e i numeri sono discordi, quindi: \((+3) \cdot (-2) = -6\).
\item \((-7) \cdot (+5) =\) 
il prodotto dei valori assoluti è \(7 \cdot 5 = 35\) 
e i numeri sono discordi, quindi: \((-7) \cdot (+5) = -35\).
\item \((-20) \cdot (-4) =\) 
il prodotto dei valori assoluti è \(20 \cdot 4 = 80\) 
e i numeri sono concordi, quindi: \((-20) \cdot (-4) = +80\).
\item \((+8) \cdot (+7) =\) 
il prodotto dei valori assoluti è \(8 \cdot 7 = 56\) 
e i numeri sono concordi, quindi: \((+8) \cdot (+7) = +56\).
\item \((+8) \cdot (0) =\) 
\(+8\) e  \(0\) non sono né concordi né discordi dato che 0 non è né 
positivo né negativo, ma il prodotto di un numero per~0 è sempre~0.
\end{enumerate}
\end{esempio}

\affiancati{.79}{.19}{
Per determinare il segno di un prodotto si può ricorrere alla seguente 
\indt{regola dei segni}: nella prima riga e nella prima colonna sono 
collocati 
i segni dei fattori, all'incrocio tra la riga e la colonna c'è il segno
del risultato.
}{
\begin{center}
\scalebox{.9}{\moltsegni}
\end{center}
}

Nel caso si debbano eseguire più moltiplicazioni il segno del prodotto è 
negativo se il segno meno è presente in un numero dispari di fattori 
mentre se il segno negativo è presente un numero pari di volte il prodotto 
è positivo.

% \pagebreak %----------------------------------------
\subsubsection{Proprietà della moltiplicazione in \texorpdfstring{$\Z$}{Z}}

La moltiplicazione in \(\Z\) può essere vista come una funzione che ha per 
argomenti numeri interi e dà come risultato un numero intero.

La struttura \indc{struttura algebrica}{\(\coppia{\Z}{\times}\)} 
\(\coppia{\Z}{\times}\) presenta le stesse proprietà della struttura 
\(\coppia{\N}{\times}\).
% \indc{struttura algebrica}{\(\coppia{\N}{\times}\)}.
Riassumendo:
\begin{itemize} [noitemsep]
 \item è una \emph{legge di composizione interna};
 \item \emph{Associativa}\indc{proprietà}{associativa}: \quad 
 \((a \times b) \times c = a \times (b \times c)\);
 \item \emph{Elemento neutro è 1}\indc{proprietà}{elemento neutro}: \quad 
 \(a \times 1 = 1 \times a = a\);
 \item \emph{Commutativa}\indc{proprietà}{commutativa}: \quad 
 \(a \times b = b \times a\).
\end{itemize}
Avendo queste proprietà, la struttura algebrica \(\coppia{\Z}{\times}\) viene 
chiamata \emph{monoide commutativo}
\indc{struttura algebrica}{monoide commutativo}.

Anche nell'insieme dei numeri interi vale la proprietà 
\indtc{proprietà}{distributiva} della
moltiplicazione rispetto l'addizione.

\bigskip
Poiché negli interi valgono queste proprietà, la struttura formata dagli 
interi, dall'addizione e dalla moltiplicazione, \(\terna{\Z}{+}{\times}\),
viene chiamata \emph{anello}\indc{struttura algebrica}{anello}.

Nell'anello degli interi, possiamo sempre risolvere equazioni del tipo:
\(x + a = 0\), ma equazioni del tipo \(ax + b = 0\) non hanno sempre 
soluzione se \(a\), \(x\) e \(b\), sono numeri interi.

% \ovalbox{\risolvii \ref{ese:2.16}, \ref{ese:2.17}, \ref{ese:2.18}}

\subsection{Divisione in 
\texorpdfstring{$\Z: \quad \coppia{\Z}{:}$}{Z: (Z; :)}}
\indc{\(\Z\)}{\(\coppia{\Z}{:}\)}

La regola della divisione è del tutto analoga a quella della 
moltiplicazione.

\begin{definizione}{Quoziente di numeri interi relativi}{}
Il \emph{quoziente esatto di due numeri interi relativi}
\indc{numeri relativi}{quoziente esatto} è il numero intero, 
se esiste, avente come valore assoluto il quoziente esatto dei valori 
assoluti e come segno 
il segno~``\(+\)'' se i due numeri sono concordi,
il segno~``\(-\)'' se i due numeri sono discordi.
\end{definizione}

Osserva che mentre addizione, sottrazione e moltiplicazione sono operazioni 
interne\ind{legge di composizione interna} ai numeri interi relativi, ossia 
il risultato di queste operazioni è sempre un numero intero relativo, il 
risultato della divisione esatta\ind{divisione esatta} non 
sempre è un numero intero relativo. 
La divisione tra numeri relativi è possibile solo se è possibile la 
divisione esatta tra i loro valori assoluti, ossia se
il divisore è diverso da zero e il dividendo è un suo multiplo.

\begin{esempio}{}{}
Calcola i quozienti esatti delle seguenti divisioni.
\begin{enumerate}[noitemsep, label=(\alph*)]
\item \((+8) : (+2)=\) 
il quoziente dei valori assoluti è \(8 : 2 = 4\) 
e i numeri sono concordi, quindi: \((+8) : (+2) = +4\).
\item \((+15) : (-3) =\) 
il quoziente dei valori assoluti è \(15 : 3 = 5\) 
e i numeri sono discordi, quindi: \((+15) : (-3) = -5\).
\item \((-36) : (-4) =\) 
il quoziente dei valori assoluti è \(36 : 4 = 9\) 
e i numeri sono concordi, quindi: \((-36) : (-4) = +9\).
\item \((+8) : (+7) =\) non ha un quoziente esatto intero.
\item \((+8) : (0) =\) non è definito.
\end{enumerate}
\end{esempio}

\subsubsection{Proprietà della divisione in \texorpdfstring{$\Z$}{Z}}

Negli interi, la divisione non è una legge di composizione interna, 
comunque, se dà un risultato intero, gode della proprietà 
invariantiva\indc{proprietà}{invariantiva} 
della divisione: 
il quoziente di due interi non cambia se dividendo e divisore si 
moltiplicano o si dividono per uno stesso valore diverso da zero.


% \ovalbox{\risolvii \ref{ese:2.19}, \ref{ese:2.20}, \ref{ese:2.21}}

\subsection{Potenza in 
\texorpdfstring{$\Z: \quad \coppia{\Z}{\uparrow}$}{Z: (Z; ^)}}

Ampliando la potenza\indc{\(\Z\)}{\(\coppia{\Z}{\uparrow}\)} 
ai numeri interi distinguiamo i due casi di potenza con 
base intera e esponente naturale e di base intera e esponente intero. 

\subsubsection{Base in 
\texorpdfstring{$\Z$ e esponente in $\N$}{Z e esponente in N}}

In questo caso la definizione di potenza per un numero relativo è la stessa 
di quella data per i numeri naturali.
Se base e esponente non sono entrambi~0, la potenza si ottiene 
moltiplicando~\(+1\) per tanti fattori uguali alla base quante volte è 
indicato dall'esponente.

Ricordiamo che un qualsiasi numero, diverso da~0, elevato a~0 dà come 
risultato il numero~1 e che qualsiasi numero elevato a~1 rimane invariato.

L'unica attenzione che dobbiamo avere è quella relativa al segno:
\begin{itemize} [nosep]
\item se la base è un numero positivo il risultato della potenza sarà 
sempre positivo;
\item se la base è un numero negativo il segno dipende dall'esponente: 
\begin{itemize} [nosep]
\item se l'esponente è dispari il risultato è un numero negativo,
\item se l'esponente è pari il risultato è un numero positivo.
\end{itemize}
\end{itemize}

%\begin{exrig}
\begin{esempio}{}{}
Potenze di numeri relativi.
\begin{htmulticols}{2}
\begin{itemize} [noitemsep]
\item \((-3)^0=1\)
\item \((+5)^0=1\) 
\item \((-2)^1=-2\) 
\item \((+7)^1=+7\) 
\item \((+3)^2=(+3)\cdot(+3)=+9\)
\item \((+3)^3=(+3)\cdot(+3)\cdot(+3)=+27\)
\item \((-2)^2=(-2)\cdot(-2)=+4\)
\item \((-2)^3=(-2)\cdot(-2)\cdot(-2)=-8\)
\item \((-2)^4=+16\)
\item \((-2)^5=-32\)
\item \((-1)^6=+1\)
\item \((-1)^7=-1\)
\end{itemize}
\end{htmulticols}
\end{esempio}

% %\end{exrig}
% 
% \[a^0=1\text{ con }a\neq~0,\qquad a^1=a.\]
% %%\begin{exrig}
% \begin{esempio}{}{}
% Potenze di numeri relativi, con esponente~0 o~1.
% \[,\qquad , \qquad , \qquad  \]
% \end{esempio}

\subsubsection{Base in 
\texorpdfstring{$\Z$ e esponente in $\Z^-$}{Z e esponente in Z-}}

Abbiamo visto come trattare potenze con base negativa, ma ha un senso una 
potenza con esponente negativo\ind{potenza a esponente negativo}?

Ricordiamo la seconda proprietà delle potenze: \(a^m : a^n = a^{m-n}\). \\
Se \(m > n\) otteniamo: \(a^m : a^n = a^{m-n}\) dove \(m-n\) è 
positivo situazione già trattata.\\
Se \(m = n\) otteniamo giustamente: \(a^n : a^n = a^{n-n} = a^0 = 1\).\\
Se \(m < n\) otteniamo: \(a^m : a^n = a^{m-n}\) dove \(m-n\) è 
un numero negativo: situazione nuova.

Riprendiamo la seconda proprietà delle potenze, ma questa volta 
consideriamo \(m < n\):

\[
 a^{m-n} = a^m: a^n = \frac{a^m}{a^n}=
 \frac{1 \cdot \overbrace{\cancel{a} \cdot \cancel{a} \cdot 
       \ldots \cdot \cancel{a}}^{m \text{ volte}}}
      {1 \cdot \underbrace{\cancel{a} \cdot \cancel{a} \cdot
                           \ldots \cdot \cancel{a}}_{m \text{ volte}} 
               \underbrace{\cdot a \cdot a \ldots 
                           \cdot a}_{n-m \text{ volte}}}=
 \frac{1}{a^{n-m}}
\]

Dato che \(n-m = -(m-n)\) possiamo concludere che in generale:
\[a^n = \frac{1}{a^{-n}} = \tonda{\frac{1}{a}}^{-n}\]
Notiamo che queste uguaglianze sono valide se la base è diversa da zero.

Riassumendo, una potenza con base non nulla è uguale a una potenza che ha 
per base il reciproco della base e per esponente l'opposto dell'esponente.
Questo ci permette di risolvere i casi in cui l'esponente è negativo.

Possiamo quindi ampliare la definizione di potenza.

\begin{definizione}{Potenza in \(\Z\)}{}
Dati due numeri interi~~\(b\)~~e~~\(e\),~~\emph{non entrambi nulli}, 
\begin{itemize}[leftmargin=0cm, itemindent=.5cm]
\item 
Se \(e \geqslant 0\): l'operazione di \emph{potenza} 
associa un terzo numero~\(p\) che si ottiene moltiplicando~1 per 
\(e\) fattori uguali a~\(b\):
\begin{inaccessibleblock}[Definizione di potenza come prodotti ripetuti.]
\[\text{Se~} b = 0 \stext{e} e = 0 \quad 
  b^e   \stext{non è definita \qquad altrimenti}
  b^e = 1 \cdot \underbrace{b \cdot b \cdot \dots \cdot b}_{e~\text{ volte}} 
  = p\]
\end{inaccessibleblock}
\item 
Se \(e < 0 \stext{e} b \neq 0\): l'operazione di \emph{potenza} 
associa un terzo numero~\(p\) che si ottiene elevando il reciproco della base 
all'opposto dell'esponente:
\begin{inaccessibleblock}[Definizione di potenza come prodotti ripetuti.]
\[\text{Se~} b = 0 \stext{e} e < 0 \quad 
  b^e   \stext{non è definita \qquad altrimenti}
  b^e = \frac{1}{1 \cdot \underbrace{b \cdot b \cdot \dots 
                 \cdot b}_{-e~\text{ volte}}} 
  = p\]
\end{inaccessibleblock}
\end{itemize}
\end{definizione}

\begin{esempio}{}{}
La successione delle potenze di~2 con esponente intero:
{\footnotesize \[\dots; \quad 
  2^{-3} = \frac{1}{8}; \quad 
  2^{-2} = \frac{1}{4};\quad 
  2^{-1} = \frac{1}{2};\quad 2^{0} = 1;\quad 
  2^{+1} = 2;\quad 2^{+2} = 4; \quad 
  2^{+3} = 8; \quad \dots\]}
% \[\dots; \quad 
% %   2^{-3} = \frac{1}{8}; \quad 
%   2^{-2} = \frac{1}{4};\quad 
%   2^{-1} = \frac{1}{2};\quad 2^{0} = 1;\quad 
%   2^{+1} = 2;\quad 2^{+2} = 4; \quad 
% %   2^{+3} = 8; \quad 
%   \dots\]
\end{esempio}

\subsubsection{Proprietà della potenza in \texorpdfstring{$\Z$}{Z}}

Con i numeri interi la potenza non è una legge di composizione interna: 
come visto nell'esempio precedente, un numero intero elevato a un numero 
intero in certi casi dà, come risultato, un numero non intero.

% \ovalbox{\risolvii \ref{ese:2.22}, \ref{ese:2.23}, \ref{ese:2.24}, 
% \ref{ese:2.25}, \ref{ese:2.26}, \ref{ese:2.27}}

% \ovalbox{\risolvii \ref{ese:2.28}, \ref{ese:2.29}, \ref{ese:2.30}}

