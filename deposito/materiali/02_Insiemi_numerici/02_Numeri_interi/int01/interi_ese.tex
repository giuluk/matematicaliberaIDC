% (c) 2012 -2014 Dimitrios Vrettos - d.vrettos@gmail.com
% (c) 2014 Claudio Carboncini - claudio.carboncini@gmail.com
% (c) 2014 Daniele Zambelli - daniele.zambelli@gmail.com

\section{Esercizi}

\subsection{Esercizi dei singoli paragrafi}

%\subsubsection*{2.3 - Confronto di numeri relativi}
\subsubsection*{\numnameref{sec:int_confronto}}

\vspace{-1em}
\immagine*{Testo alternativo: si evidenzia che, se non diversamente 
specificato, all'interno dell'ambiente matematico al posto dei 
``puntini'' per indicare una parte vuota (e che dovrà successivamente 
essere riempita), è inserita la lettera ``p'' come abbreviazione di 
``puntini''.
Ciò viene fatto in quanto quando si converte da HTML a LAMBDA, i
``puntini'' non vengono rappresentati e al loro posto non
compare nulla.}{}

\begin{esercizio}
 \label{ese:2.1}
Riscrivi in ordine crescente (dal più piccolo al più grande) i seguenti 
numeri 
relativi:
\[+11\qquad-3\qquad0\qquad+2\qquad-5\qquad-7\qquad+1\]
\end{esercizio}

\begin{esercizio}
 \label{ese:2.2}
Riscrivi in ordine decrescente (dal più grande al più piccolo) i seguenti 
numeri 
relativi:
\[-5\qquad-2\qquad+3\qquad-1\qquad0\qquad+7\qquad-9\qquad+13\qquad-21\]
\end{esercizio}

\begin{esercizio}
 \label{ese:2.3}
Disponi sulla retta degli interi i seguenti numeri relativi:
\[-3;~~ +2;~~ +5;~~ -7;~~ -5;~~ -1;~~ +3\]
\begin{center}
\esec
\end{center}
\end{esercizio}

\vspace{-1em}
\immagine*{Testo alternativo: si precisa che questo esercizio non è adatto 
per studenti non vedenti, provare a disporli in ordine crescente.}{}

\begin{esercizio}
 \label{ese:2.4}
Per ciascuno dei seguenti numeri relativi scrivi il valore assoluto.
\begin{htmulticols}{3}
\begin{enumeratees}
 \item \(|+3|=\ldots\)
 \item \(|-5|=\ldots\)
 \item \(|-1|=\ldots\)
 \item \(|+10|=\ldots\)
 \item \(|-11|=\ldots\)
 \item \(|+7|=\ldots\)
\end{enumeratees}
\end{htmulticols}
\end{esercizio}

 \begin{esercizio}
\label{ese:2.5}
Scrivi tra le seguenti coppie di numeri relativi il simbolo corretto tra 
``\(>\)'' 
e ``\((<)\)''.
\begin{htmulticols}{3}
 \begin{enumeratees}
 \item \(-5\ldots-2~\)
 \immagine*{Testo alternativo: \(-5 p -2~\)}{}
 \item \(-3\ldots+5~\)
 \immagine*{Testo alternativo: \(-3 p +5~\)}{}
 \item \(-2\ldots+2~\)
 \immagine*{Testo alternativo: \(-2 p +2~\) }{}
 \item \(-5\ldots0~\)
 \immagine*{Testo alternativo: \(-5 p 0~\)}{}
 \item \(-3\ldots-5~\)
 \immagine*{Testo alternativo: \(-3 p -5~\)}{}
 \item \(-1\ldots+1~\)
 \immagine*{Testo alternativo: \(-1 p +1~\) }{}
 \item \(+3\ldots-3~\)
 \immagine*{Testo alternativo: \(+3 p -3~\)}{}
 \item \(-1\ldots-5~\)
 \immagine*{Testo alternativo: \(-1 p -5~\)}{}
 \item \(~0\ldots+1~\)
 \immagine*{Testo alternativo: \(~0 p +1~\)}{}
 \item \(+3\ldots0~\)
 \immagine*{Testo alternativo: \(+3 p 0~\)}{}
 \item \(~0\ldots -2\)
 \immagine*{Testo alternativo: \(~0 p -2\)}{}
 \item \(+7\ldots +2\)
 \immagine*{Testo alternativo: \(+7 p +2\)}{}
 \item \(-11\ldots-101~\)
 \immagine*{Testo alternativo: \(-11 p -101~\)}{}
 \item \(+100\ldots-99~\)
 \immagine*{Testo alternativo: \(+100 p -99~\)}{}
 \item \(-101\ldots+110~\)
 \immagine*{Testo alternativo: \(-101 p +110~\)}{}
%  \item \( -1010\ldots-1100~\)
%  \item \( +324\ldots -282\)
%  \item \( -714\ldots -851\)
 \end{enumeratees}
\end{htmulticols}
\end{esercizio}

%\subsubsection*{2.4 - Le operazioni con i numeri relativi}
\subsubsection*{\numnameref{sec:int_operazioni}}

% % Addizione

\begin{esercizio}
 \label{ese:2.6}
Esegui le seguenti addizioni di numeri relativi.
 \begin{htmulticols}{3}
 \begin{enumeratees}
 \item \((+3)+(+2) =~\)
 \item \((-5)+(-5) =~\)
 \item \((-3)+(+5) =~\)
 \item \((+12)+(+2) =\)
 \item \((-2)+(-3) =\)
 \item \((-3)+(+13) =\)
 \item \((+10)+(-5) =\)
 \item \((+1)+(+1) =\)
 \item \((-10)+0 =\)
%  \item \((-4)+(+4) =\)
%  \item \((+7)+(-6) =\)
%  \item \((-9)+(-3) =\)
%  \item \((-101)+(+2) =\)
%  \item \(0+(-9) =\)
%  \item \((-10)+(+10) =\)
 \end{enumeratees}
 \end{htmulticols}
\end{esercizio}


% \begin{esercizio}
%  \label{ese:2.7}
% Per ognuno dei seguenti numeri relativi scrivi il numero opposto.
%  \begin{htmulticols}{3}
%  \begin{enumeratees}
%  \item \(+3\to\ldots\)
%  \item \(-2\to\ldots\)
%  \item \(+1\to\ldots\)
%  \item \(-11\to\ldots\)
%  \item \(-3\to\ldots\)
%  \item \(+5 \to\ldots\)
%  \end{enumeratees}
%  \end{htmulticols}
% \end{esercizio}

% \newpage %------------------------------------------------------------------

% % Sottrazione

\begin{esercizio}
Esegui le seguenti sottrazioni di numeri relativi.
\label{ese:2.9}
\begin{htmulticols}{3}
\begin{enumeratees}
 \item \((-1)-(+2) =~\)
 \item \((-5)-(+3) =~\)
 \item \((-2)-(+5) =~\)
 \item \((+12)-(+2) =\)
 \item \((+1)-(-3) =\)
 \item \((-3)-(+1) =\)
 \item \((+11)-(-5) =\)
 \item \((+21)-(+11) =\)
 \item \((-1)-0 =\)
 \item \((-3)-(+4) =\)
 \item \((+7)-(-2) =\)
 \item \((-3)-(-3) =\)
 \item \(0-(-11) =\)
 \item \((-6)-(-6) =\)
 \item \((+5)-(-5) =\)
\end{enumeratees}
\end{htmulticols}
\end{esercizio}

\pagebreak %----------------------------------

\begin{esercizio}
 \label{ese:tab2}
Completa la seguente tabella.
\begin{center}
\begin{tabular}{|m{0.05\textwidth}|m{0.05\textwidth}
                   |m{0.16\textwidth}|m{0.16\textwidth}
                   |m{0.16\textwidth}|m{0.16\textwidth}|}
\hline
\(~~a\) & \(~~b\) & \(\quad a-b\) & \(\quad +a-b\) & 
\(\quad -a+b\) & \(\quad -a-b\) \\ \hline \srb{-1} & \srb{+2} & \srb{-3}  & 
\srb{-3}  & \srb{+3}  & \srb{-1} 
\\[1em] \hline
\srb{+2} & \srb{+3} & \srbop{-1}  & \srbop{-1}  & \srbop{+1}  & \srbop{-5} 
\\[1em] \hline
\srb{+1} & \srb{~~~0} & \srbop{+1}  & \srbop{+1}  & \srbop{-1}  & \srbop{-1} 
\\[1em] \hline
\srb{-2} & \srb{-3} & \srbop{+1}  & \srbop{+1}  & \srbop{-1}  & \srbop{+5} 
\\[1em] \hline
\srb{+3} & \srb{-3} & \srbop{+6}  & \srbop{+6}  & \srbop{-6}  & \srbop{~~~0} 
\\[1em] \hline
\srb{-10} & \srb{+4} & \srbop{-14}  & \srbop{-14}  & \srbop{+14}  & 
\srbop{+6} 
\\[1em] \hline
\end{tabular}
\end{center}
\end{esercizio}

\vspace{-.5em}
\immagine*{Testo alternativo: tabella 7 righe per 6 colonne 
(comprese intestazioni di colonna).
Prima riga composta dalle intestazioni di colonna: 
\(a\); \(b\); \(a-b\); \(+a-b\); \(-a+b\); \(-a-b\).
Seconda riga: -1; +2; -3; -3; +3; -1.
Terza riga: +2; +3.
Quarta riga: +1; 0.
Quinta riga: -2; -3.
Sesta riga: +3; -3.
Settima riga: -10; +4.}{}

\begin{esercizio}
 \label{ese:tab2}
Completa la seguente tabella.
\begin{center}
\begin{tabular}{|m{0.05\textwidth}|m{0.05\textwidth}
                   |m{0.16\textwidth}|m{0.16\textwidth}
                   |m{0.16\textwidth}|m{0.16\textwidth}|}
\hline
\(~~a\) & \(~~b\) & \(\quad a-b\) & \(\quad -(a-b)\) & 
\(\quad -a+b\) & \(\quad -a-(-b)\) \\ \hline \srb{-8} & \srb{+2} & 
\srbop{-10}  
& 
\srbop{+10}  & \srbop{+10}  & \srbop{+10} 
\\[1em] \hline
\srb{+6} & \srb{+3} & \srbop{+3}  & \srbop{-3}  & \srbop{-3}  & \srbop{-3} 
\\[1em] \hline
\srb{+7} & \srb{-4} & \srbop{+11}  & \srbop{-11}  & \srbop{-11}  & 
\srbop{-11} 
\\[1em] \hline
\srb{-5} & \srb{-9} & \srbop{+4}  & \srbop{-4}  & \srbop{-4}  & \srbop{-4} 
\\[1em] \hline
\srb{+2} & \srb{~~~0} & \srbop{+2}  & \srbop{-2}  & \srbop{-2}  & \srbop{-2} 
\\[1em] \hline
\srb{-8} & \srb{+6} & \srbop{-14}  & \srbop{+14}  & \srbop{+14}  & 
\srbop{+14} 
\\[1em] \hline
\end{tabular}
\end{center}
\end{esercizio}

\vspace{-.5em}
\immagine*{Testo alternativo: tabella 7 righe per 6 colonne 
(comprese intestazioni di colonna).
Prima riga composta dalle intestazioni di colonna: 
\(a\); \(b\); \(a-b\); \(-(a-b)\); \(-a+b\); \(-a-(-b)\).
Seconda riga: -8; +2.
Terza riga: +6; +3.
Quarta riga: +7; -4.
Quinta riga: -5; -9.
Sesta riga: +2; 0.
Settima riga: -8; +6.}{}

\begin{esercizio}
 \label{ese:tab3}
Completa la seguente tabella.
\begin{center}
\begin{tabular}{|m{0.045\textwidth}|m{0.045\textwidth}|m{0.045\textwidth}
                   |m{0.16\textwidth}|m{0.16\textwidth}
                   |m{0.16\textwidth}|m{0.16\textwidth}|}
\hline
\(~~a\) & \(~~b\) & \(~~c\) & \(\quad a-b+c\) & \(\quad (a-b)+c\) & 
\(\quad a+(-b+c)\) & \(\quad a-(+b+c)\) \\ \hline \srb{-1} & \srb{+2} & 
\srb{-3} 
& \srbop{-6}  & \srbop{-6}  & \srbop{-6}  & \srbop{~~~0} 
\\[1em] \hline
\srb{+2} & \srb{+3} & \srb{-5} & \srbop{-6}  & \srbop{-6}  & \srbop{-6}  & 
\srbop{+4} 
\\[1em] \hline
\srb{+1} & \srb{~~~0} & \srb{-1} & \srbop{~~~0}  & \srbop{~~~0}  & 
\srbop{~~~0} 
 
& 
\srbop{+2} 
\\[1em] \hline
\srb{-5} & \srb{-3} & \srb{+4} & \srbop{+2}  & \srbop{+2}  & \srbop{+2}  & 
\srbop{-6} 
\\[1em] \hline
\srb{+7} & \srb{-7} & \srb{+7} & \srbop{+21}  & \srbop{+21}  & \srbop{+21}  & 
\srbop{+7} 
\\[1em] \hline
\srb{-11} & \srb{~~~0} & \srb{+4} & \srbop{-7}  & \srbop{-7}  & \srbop{-7}  & 
\srbop{-15} 
\\[1em] \hline
\end{tabular}
\end{center}
\end{esercizio}

\vspace{-.5em}
\immagine*{Testo alternativo: tabella 7 righe per 7 colonne 
(comprese intestazioni di colonna).
Prima riga composta dalle intestazioni di colonna: 
\(a\); \(b\); \(c\); \(a-b+c\); \((a-b)+c\); \(a+(-b+c)\); \(a-(+b+c)\).
Seconda riga: -1; +2; -3.
Terza riga: +2; +3; -5.
Quarta riga: +1; 0; -1.
Quinta riga: -5; -3; +4.
Sesta riga: +7; -7; +7.
Settima riga: -11; 0; +4.}{}

% % Somma algebrica

\begin{esercizio}
Trasforma in somme algebriche, e poi calcola, le seguenti espressioni.
 \label{ese:as}
% \vspace{-1em}
% \begin{htmulticols}{2}
 \begin{enumeratees}
\item \((+9) - (+5) + (+6) - (+2) + (+9) =\)
 \sol{17}
\item \((+10) + (-10) + (+7) + (-8) + (+9) =\)
 \sol{8}
\item \((+12) + (-3) + (+8) - (+6) - (+7) - (+0) =\)
 \sol{4}
\item \((-1) - (-1) + (-6) - (+11) - (+2) - (-8) =\)
 \sol{-11}
\item \((+3) - (-8) + (+4) + (-12) + (-6) + (-5) =\)
 \sol{-8}
\item \((-12) - (+11) + (-6) - (+8) + (-12) + (+9) =\)
 \sol{-40}
\item \((-9) + (-6) + (+10) - (-12) + (+2) - (+9) =\)
 \sol{0}
\item \((+6) + (+6) + (-4) + (+5) - (-2) - (-11) + (-2) =\)
 \sol{24}
\item \((+6) + (+9) + (-4) + (-4) - (+2) - (+5) - (-11) =\)
 \sol{11}
\item \((+3) + (+0) + (+9) + (+0) + (-3) - (+7) - (+5) =\)
 \sol{-3}
\item \((+7) - (+4) - (-11) + (-5) + (-2) + (+6) + (+2) =\)
 \sol{15}
\item \((+8) + (-10) + (+12) + (-3) - (-9) - (+9) + (-12) =\)
 \sol{-5}
\item \((-3) + (+12) - (-7) + (-11) - (-1) - (-10) - (+3) =\)
 \sol{13}
\item \((-1) + (-9) - (-10) - (+8) + (-2) + (-2) + (-4) =\)
 \sol{-16}
\item \((-4) - (+0) + (-6) + (-4) + (+8) - (-11) - (-3) =\)
 \sol{8}
\item \((-6) - (+9) - (-11) + (-2) - (-1) - (+5) + (+5) =\)
 \sol{-5}
\item \((-4) + (+4) - (-8) - (+0) + (-4) - (-2) - (-5) =\)
 \sol{11}
\item \((+4) - (+9) - (-2) + (+3) + (-1) + (+9) + (-8) - (+10) =\)
 \sol{-10}
\item \((+0) - (+7) + (-3) + (+5) + (-2) - (-4) + (+7) - (-3) =\)
 \sol{7}
\item \((+11) - (+8) + (+10) - (-7) + (+0) - (-6) - (+10) - (+1) =\)
 \sol{15}
\item \((+0) + (-6) + (-9) + (-10) + (+2) - (-5) + (+2) + (+10) =\)
 \sol{-6}
\end{enumeratees}
%\end{htmulticols}
\end{esercizio}

\begin{esercizio}
Esegui le seguenti somme algebriche.
 \label{ese:2.14}

\vspace{-1em}
 \begin{htmulticols}{4}
 \begin{enumeratees}
 \item \(+3 -1 =\)%\ph{+2}\)
 \item \(+2 -3 =\)
 \item \(-5 +2 =\)
 \item \(-2 +2 =\)
 \item \(-5 -2 =\)
 \item \(-3 +5 =\)
 \item \(+8 -0 =\)
 \item \(-9 +0 =\)
 \item \(0 -5 =\)
 \item \(+1 -1 =\)
 \item \(-2 -2 =\)
 \item \(+9 -3 =\)
 \item \(+7 -6 =\)
 \item \(-101 +9 =\)
 \item \(-10 +5 =\)
 \item \(+7 -17 =\)
%  \item \(-5 -2 =\)
%  \item \(+3 -4 =\)
%  \item \(-1 +2 =\)
%  \item \(-3 +4 =\)
%  \item \(-6 +7 =\)
%  \item \(-1 -9 =\)
%  \item \(+8 -7 =\)
%  \item \(+2 -1 =\)
%  \item \(-6 +2 =\)
%  \item \(+5 -2 =\)
%  \item \(+4 -3 =\)
%  \item \(+4 +1 =\)
%  \item \(+4 -6 =\)
%  \item \(-10 +5 =\)
%  \item \(-16 -4 =\)
%  \item \(-3 -9 =\)
%  \item \(+14 -7 =\)
%  \item \(-10 -10 =\)
 \end{enumeratees}
\end{htmulticols}
\end{esercizio}

% % Moltiplicazione
\begin{esercizio}
 \label{ese:2.16}
Calcola i seguenti prodotti.
\begin{htmulticols}{3}
 \begin{enumeratees}
 \item \((+3)\cdot(-2)=-\ldots\)
 \immagine*{Testo alternativo: \((+3)\cdot(-2)=-p\)}{}
 \item \((-5)\cdot(-2)=+\ldots\)
 \immagine*{Testo alternativo: \((-5)\cdot(-2)=+p\)}{}
 \item \((+2)\cdot(+4) =\ldots8\)
 \immagine*{Testo alternativo: \((+2)\cdot(+4) =p 8\)}{}
 \item \((+1)\cdot(-1) =\ldots1\)
 \immagine*{Testo alternativo: \((+1)\cdot(-1) =p 1\)}{}
 \item \((+3)\cdot0 = \ldots\ldots\)
 \immagine*{Testo alternativo: \item \((+3)\cdot0 = p p\)}{}
 \item \((-2)\cdot(-2) =\ldots\ldots\)
 \immagine*{Testo alternativo: \item \((-2)\cdot(-2) =p p\)}{}
 \item \(0\cdot(-3) = \ldots\ldots\)
 \immagine*{Testo alternativo: \(0\cdot(-3) = p p\)}{}
 \item \((-2)\cdot(+2) =\ldots\ldots\)
 \immagine*{Testo alternativo: \((-2)\cdot(+2) =p p\)}{}
 \item \((+10)\cdot(-1) =\ldots\)
 \immagine*{Testo alternativo: \((+10)\cdot(-1) =p\)}{}
 \item \((+3)\cdot(+1) =\)
 \item \((+1)\cdot(-2) =\)
 \item \((+3)\cdot(-3) =\)
 \item \((-5)\cdot(-1) =\)
 \item \((+3)\cdot(-3) =\)
 \item \((-2)\cdot(+5) =\)
 \item \((-1)\cdot(-7) =\)
 \item \((+3)\cdot(+11) =\)
 \item \((+1)\cdot(-10) =\)
 \item \((-4)\cdot(+3) =\)
 \item \((+5)\cdot(-6) =\)
 \item \((-3)\cdot(-2) =\)
 \end{enumeratees}
\end{htmulticols}
\end{esercizio}

\begin{esercizio}
 \label{ese:tab2}
Completa la seguente tabella.
\begin{center}
\begin{tabular}{|m{0.05\textwidth}|m{0.05\textwidth}
                |m{0.16\textwidth}|m{0.16\textwidth}
                |m{0.16\textwidth}|m{0.16\textwidth}|}
\hline
\(~~a\) & \(~~b\) & \(\quad a \cdot b\) & \(\quad -a \cdot b\) & 
\(\quad (-a) \cdot (-b)\) & \(\quad -(a \cdot b)\) \\
\hline
\srb{-7} & \srb{+2} & \srbop{-14}  & \srbop{+14}  & \srbop{-14}  & 
\srbop{+14} 
\\[1em] \hline
\srb{+5} & \srb{+1} & \srbop{+5}  & \srbop{-5}  & \srbop{+5}  & \srbop{-5} 
\\[1em] \hline
\srb{+6} & \srb{-3} & \srbop{-18}  & \srbop{+18}  & \srbop{-18}  & 
\srbop{+18} 
\\[1em] \hline
\srb{-8} & \srb{-9} & \srbop{+72}  & \srbop{-72}  & \srbop{+72}  & 
\srbop{-72} 
\\[1em] \hline
\srb{~~~0} & \srb{-4} & \srbop{~~~0}  & \srbop{~~~0}  & \srbop{~~~0}  & 
\srbop{~~~0} 
\\[1em] \hline
\srb{-10} & \srb{+12} & \srbop{-120}  & \srbop{+120}  & \srbop{-120}  & 
\srbop{+120} 
\\[1em] \hline
\end{tabular}
\end{center}
\end{esercizio}

\vspace{-.5em}
\immagine*{Testo alternativo: tabella 7 righe per 6 colonne 
(comprese intestazioni di colonna).
Prima riga composta dalle intestazioni di colonna: 
\(a\); \(b\); \(a \cdot b\); \(-a \cdot b\); \((-a) \cdot (-b)\); 
\(-(a \cdot b)\).
Seconda riga: -7; +2.
Terza riga: +5; +1.
Quarta riga: +6; -3.
Quinta riga: -8; -9.
Sesta riga: 0; -4.
Settima riga: -10; +12.}{}

\pagebreak %-----------------------------------------------

\begin{esercizio}
 \label{ese:tab2}
Completa la seguente tabella.
\begin{center}
\begin{tabular}{|m{0.045\textwidth}|m{0.045\textwidth}
                |m{0.18\textwidth}|m{0.18\textwidth}
                |m{0.18\textwidth}|m{0.18\textwidth}|}
\hline
\(~~a\) & \(~~b\) & \((a + b) \cdot (a - b)\) & \((a + b) \cdot (a + b)\) & 
\((a - b) \cdot (a - b)\) & \((a + b) \cdot (-a + b)\) \\
\hline
\srb{-7} & \srb{+2} & \srbop{+45}  & \srbop{+25}  & \srbop{+81}  & 
\srbop{-45} 
\\[1em] \hline
\srb{+5} & \srb{+1} & \srbop{+24}  & \srbop{+36}  & \srbop{+16}  & 
\srbop{-24} 
\\[1em] \hline
\srb{+6} & \srb{-3} & \srbop{+27}  & \srbop{+9}  & \srbop{+81}  & \srbop{-27} 
\\[1em] \hline
\srb{-4} & \srb{-2} & \srbop{+12}  & \srbop{+36}  & \srbop{+4}  & \srbop{-12} 
\\[1em] \hline
\srb{~~~0} & \srb{-4} & \srbop{-16}  & \srbop{+16}  & \srbop{+16}  & 
\srbop{+16} 
\\[1em] \hline
\srb{-2} & \srb{+8} & \srbop{-60}  & \srbop{+36}  & \srbop{+100}  & 
\srbop{+60} 
\\[1em] \hline
\end{tabular}
\end{center}
\end{esercizio}

\vspace{-.5em}
\immagine*{Testo alternativo: tabella 7 righe per 6 colonne 
(comprese intestazioni di colonna).
Prima riga composta dalle intestazioni di colonna: 
\(a\); \(b\); \((a+b) \cdot (a-b)\); \((a+b) \cdot (a+b)\); 
\((a-b) \cdot (a-b)\); \((a+b) \cdot (-a+b)\).
Seconda riga: -7; +2.
Terza riga: +5; +1.
Quarta riga: +6; -3.
Quinta riga: -8; -9.
Sesta riga: 0; -4.
Settima riga: -2; +8.}{}

% % Divisione
\begin{esercizio}
\label{ese:2.19}
 Esegui le seguenti divisioni.
 
\vspace{-1em}
\begin{htmulticols}{3}
 \begin{enumeratees}
 \item \((+4):(+2) =\)
 \item \((+5):(-1) =\)
 \item \((+6):(+2) =\)
 \item \((+8):(-2) =\)
 \item \((-8):(+4) =\)
 \item \((-4):(+2) =\)
 \item \((-10):(+5) =\)
 \item \((+10):(-2) =\)
 \item \((-12):(+6) =\)
%  \item \((-12):(+4) =\)
%  \item \((+12):(-3) =\)
%  \item \((-12):(+1) =\)
 \end{enumeratees}
 \end{htmulticols}
\end{esercizio}

\begin{esercizio}
 \label{ese:tab2}
Completa la seguente tabella.
\begin{center}
\begin{tabular}{|m{0.045\textwidth}|m{0.045\textwidth}
                |m{0.18\textwidth}|m{0.18\textwidth}
                |m{0.18\textwidth}|m{0.18\textwidth}|}
\hline
\(~~a\) & \(~~b\) & \(\quad a : b\) & \(\quad -a : b\) & 
\(\quad -(a : b)\) & \(\quad a : -(b)\) \\
\hline
\srb{-24} & \srb{+2} & \srbop{-12.0}  & \srbop{+12.0}  & \srbop{+12.0}  & 
\srbop{+12.0} 
\\[1em] \hline
\srb{+18} & \srb{+1} & \srbop{+18.0}  & \srbop{-18.0}  & \srbop{-18.0}  & 
\srbop{-18.0} 
\\[1em] \hline
\srb{+48} & \srb{-3} & \srbop{-16.0}  & \srbop{+16.0}  & \srbop{+16.0}  & 
\srbop{+16.0} 
\\[1em] \hline
\srb{-18} & \srb{-9} & \srbop{+2.0}  & \srbop{-2.0}  & \srbop{-2.0}  & 
\srbop{-2.0} 
\\[1em] \hline
\srb{~~~0} & \srb{-4} & \srbop{-0.0}  & \srbop{-0.0}  & \srbop{~~~0.0}  & 
\srbop{~~~0.0} 
\\[1em] \hline
\srb{-36} & \srb{+12} & \srbop{-3.0}  & \srbop{+3.0}  & \srbop{+3.0}  & 
\srbop{+3.0} 
\\[1em] \hline
\end{tabular}
\end{center}
\end{esercizio}

\vspace{-.5em}
\immagine*{Testo alternativo: tabella 7 righe per 6 colonne 
(comprese intestazioni di colonna).
Prima riga composta dalle intestazioni di colonna: 
\(a\); \(b\); \(a : b\); \(-a : b\); \(-(a : b)\); \(a : -(b)\).
Seconda riga: -24; +2.
Terza riga: +18; +1.
Quarta riga: +48; -3.
Quinta riga: -18; -9.
Sesta riga: 0; -4.
Settima riga: -36; +12.}{}

% % Potenza di un numero relativo
\begin{esercizio}
\label{ese:2.22}
Calcola il valore delle seguenti potenze.

\vspace{-1em}
\begin{htmulticols}{3}
 \begin{enumeratees}
 \item \((+3)^2 =\)
 \item \((-1)^2 =\)
 \item \((+1)^3 =\)
 \item \((-2)^2 =\)
 \item \((-2)^3 =\)
 \item \((+2)^3 =\)
 \item \((-3)^0 =\)
 \item \((-3)^3 =\)
 \item \((-4)^1 =\)
%  \item \((+4)^1 =\)
%  \item \((-4)^2 =\)
%  \item \((-2)^4 =\)
%  \item \((-3)^0 =\)
%  \item \((-1)^5 =\)
%  \item \((-2)^4 =\)
 \end{enumeratees}
 \end{htmulticols}
\end{esercizio}

\begin{esercizio}
\label{ese:2.23}
 Applica le proprietà delle potenze.
 
\vspace{-1em}
\begin{htmulticols}{2}
 \begin{enumeratees}
 \item \((-3)^2\cdot(-3)^3 = (-3)^{\ldots}\)
 \immagine*{Testo alternativo: \((-3)^2\cdot(-3)^3 = (-3)^{p}\)}{}
 \item \((-2)^4\cdot(-2)^5 = (-2)^{\ldots}\)
 \immagine*{Testo alternativo: \((-2)^4\cdot(-2)^5 = (-2)^{p}\)}{}
 \item \((-5)\cdot(-5)^2 = (-5)^{\ldots}\)
 \immagine*{Testo alternativo: \item \((-5)\cdot(-5)^2 = (-5)^{p}\)}{}
 \item \((-10)^2\cdot(-5)^2 = (\ldots \ldots)^2\)
 \immagine*{Testo alternativo: }{\((-10)^2\cdot(-5)^2 = (p p)^2\)}
 \item \((-3)^4:(-3)^2 = (-3)^{\ldots}\)
 \immagine*{Testo alternativo: \((-3)^4:(-3)^2 = (-3)^{p}\)}{}
 \item \((-7)^3:(-7)^3=(-7)^{\ldots}\)
 \immagine*{Testo alternativo: \((-7)^3:(-7)^3=(-7)^{p}\)}{}
 \item \((-2)^4:(-2)^2=(-2)^{\ldots}\)
 \immagine*{Testo alternativo: \((-2)^4:(-2)^2=(-2)^{p}\)}{}
 \item \((-6)^4:(+2)^4=(\ldots \ldots)^4\)
 \immagine*{Testo alternativo: \((-6)^4:(+2)^4=(p p)^4\)}{}
 \item \(\big[(-3)^2\big]^3 = (-3)^{\ldots}\)
 \immagine*{Testo alternativo: \(\big[(-3)^2\big]^3 = (-3)^{p}\)}{}
 \item \(\big[(-5)^2\big]^3=(+5)^{\ldots}\)
 \item \((-3)^3\cdot(+3)^3 = \ldots\)
 \item \((-8)^2:(-4)^2= \ldots\)
 \item \(\big[(-7)^2\big]^3: (-7)^3 =\ldots\)
 \item \(\big[(-3)^3\big]^2: (-3)^4=\ldots\)
 \end{enumeratees}
 \end{htmulticols}
\end{esercizio}

\begin{esercizio}
 \label{ese:tab2}
Completa la seguente tabella.
\begin{center}
\begin{tabular}{|m{0.045\textwidth}|m{0.045\textwidth}
                |m{0.18\textwidth}|m{0.18\textwidth}
                |m{0.18\textwidth}|m{0.18\textwidth}|}
\hline
\(~~a\) & \(~~b\) & \(\quad a^b\) & \(\quad (-a)^b\) & 
\(\quad (+a)^{-b}\) & \(\quad (-a)^{-b}\) \\
\hline
\srb{-7} & \srb{+2} & \srbop{+49}  & \srbop{+49}  & \srbop{+\frac{1}{49}}  & 
\srbop{+\frac{1}{49}} 
\\[1em] \hline
\srb{-3} & \srb{+4} & \srbop{+81}  & \srbop{+81}  & \srbop{+\frac{1}{81}}  & 
\srbop{+\frac{1}{81}} 
\\[1em] \hline
\srb{-3} & \srb{+3} & \srbop{-27}  & \srbop{+27}  & \srbop{-\frac{1}{27}}  & 
\srbop{+\frac{1}{27}} 
\\[1em] \hline
\srb{-8} & \srb{-2} & \srbop{+\frac{1}{64}}  & \srbop{+\frac{1}{64}}  & 
\srbop{+64}  & 
\srbop{+64} 
\\[1em] \hline
\srb{+1} & \srb{+5} & \srbop{+1}  & \srbop{-1}  & \srbop{+1.0}  & 
\srbop{-1.0} 
\\[1em] \hline
\srb{-10} & \srb{+4} & \srbop{+10000}  & \srbop{+10000}  & \srbop{~~~0.0001}  
& 
\srbop{~~~0.0001} 
\\[1em] \hline
\end{tabular}
\end{center}
\end{esercizio}

\vspace{-.5em}
\immagine*{Testo alternativo: tabella 7 righe 6 colonne 
(comprese intestazioni di colonna).
Prima riga composta dalle intestazioni di colonna: 
\(a\); \(b\); \(a^b\); \((-a)^b\); \((+a)^{-b}\); \((-a)^{-b}\).
Seconda riga: -7; +2.
Terza riga: -3; +4.
Quarta riga: -3; +3.
Quinta riga: -8; -2.
Sesta riga: +1; +5.
Settima riga: -10; +4.}{}

\begin{esercizio}
 \label{ese:tab2}
Completa la seguente tabella.
\begin{center}
\begin{tabular}{|m{0.045\textwidth}|m{0.045\textwidth}
                |m{0.18\textwidth}|m{0.18\textwidth}
                |m{0.18\textwidth}|m{0.18\textwidth}|}
\hline
\(~~a\) & \(~~b\) & \(\quad (a + b)^2\) & \(\quad (-a + b)^2\) & 
\(\quad (+a - b)^2\) & \(\quad (-a - b)^2\) \\
\hline
\srb{-7} & \srb{+2} & \srbop{+25}  & \srbop{+81}  & \srbop{+81}  & 
\srbop{+25} 
\\[1em] \hline
\srb{-3} & \srb{+4} & \srbop{+1}  & \srbop{+49}  & \srbop{+49}  & \srbop{+1} 
\\[1em] \hline
\srb{-3} & \srb{+3} & \srbop{~~~0}  & \srbop{+36}  & \srbop{+36}  & 
\srbop{~~~0} 
\\[1em] \hline
\srb{-8} & \srb{-2} & \srbop{+100}  & \srbop{+36}  & \srbop{+36}  & 
\srbop{+100} 
\\[1em] \hline
\srb{+1} & \srb{+5} & \srbop{+36}  & \srbop{+16}  & \srbop{+16}  & 
\srbop{+36} 
\\[1em] \hline
\srb{-10} & \srb{+4} & \srbop{+36}  & \srbop{+196}  & \srbop{+196}  & 
\srbop{+36} 
\\[1em] \hline
\end{tabular}
\end{center}
\end{esercizio}

\vspace{-.5em}
\immagine*{Testo alternativo: tabella 7 righe 6 colonne 
(comprese intestazioni di colonna).
Prima riga composta dalle intestazioni di colonna: 
\(a\); \(b\); \((a+b)^2\); \((-a+b)^2\); \((+a-b)^2\); \((-a-b)^2\).
Seconda riga: -7; +2.
Terza riga: -3; +4.
Quarta riga: -3; +3.
Quinta riga: -8; -2.
Sesta riga: +1; +5.
Settima riga: -10; +4.}{}

\begin{esercizio}
 \label{ese:tab1}
Completa la seguente tabella.
\begin{center}
\begin{tabular}{|m{0.045\textwidth}
                |m{0.153\textwidth}|m{0.153\textwidth}|m{0.153\textwidth}
                |m{0.153\textwidth}|m{0.153\textwidth}|}
\hline
\(~~a\) & \(\quad a^2\) & \(\quad (-a)^2\) & 
\(\quad -a^2\) & \(\quad a^3\) & \(\quad (-a)^3\) \\
\hline
\srb{-2} & \srbop{+4}  & \srbop{+4}  & \srbop{-4}  & \srbop{-8}  & \srbop{+8} 
\\[1em] \hline
\srb{-1} & \srbop{+1}  & \srbop{+1}  & \srbop{-1}  & \srbop{-1}  & \srbop{+1} 
\\[1em] \hline
\srb{~~~0} & \srbop{~~~0}  & \srbop{~~~0}  & \srbop{~~~0}  & \srbop{~~~0}  & 
\srbop{~~~0} 
\\[1em] \hline
\srb{+1} & \srbop{+1}  & \srbop{+1}  & \srbop{-1}  & \srbop{+1}  & \srbop{-1} 
\\[1em] \hline
\srb{+2} & \srbop{+4}  & \srbop{+4}  & \srbop{-4}  & \srbop{+8}  & \srbop{-8} 
\\[1em] \hline
\srb{+3} & \srbop{+9}  & \srbop{+9}  & \srbop{-9}  & \srbop{+27}  & 
\srbop{-27} 
\\[1em] \hline
\end{tabular}
\end{center}
\end{esercizio}

\vspace{-.5em}
\immagine*{Testo alternativo: tabella 7 righe 6 colonne 
(comprese intestazioni di colonna).
Prima riga composta dalle intestazioni di colonna: 
\(a\); \((a)^2\); \((-a)^2\); \(-a^2\); \((a)^3\); \((-a)^3\).
Seconda riga: -2.
Terza riga: -1.
Quarta riga: 0.
Quinta riga: +1.
Sesta riga: +2.
Settima riga: +3.}{}

\pagebreak %------------------------------------------

\begin{esercizio}
 \label{ese:tab3}
Completa la seguente tabella.
\begin{center}
\begin{tabular}{|m{0.045\textwidth}|m{0.045\textwidth}|m{0.045\textwidth}
                |m{0.16\textwidth}|m{0.16\textwidth}
                |m{0.16\textwidth}|m{0.16\textwidth}|}
\hline
\(~~a\) & \(~~b\) & \(~~c\) & \(-2 \cdot a-b+c\) & \(-2 \cdot a-(b+c)\) & 
\(-a-2 \cdot b+c\) & \(a-b-2 \cdot c\) \\
\hline
\srb{-1} & \srb{+2} & \srb{-3} & \srbop{-3}  & \srbop{+3}  & \srbop{-6}  & 
\srbop{+3} 
\\[1em] \hline
\srb{+2} & \srb{+3} & \srb{-5} & \srbop{-12}  & \srbop{-2}  & \srbop{-13}  & 
\srbop{+9} 
\\[1em] \hline
\srb{+1} & \srb{~~~0} & \srb{-1} & \srbop{-3}  & \srbop{-1}  & \srbop{-2}  & 
\srbop{+3} 
\\[1em] \hline
\srb{-5} & \srb{-3} & \srb{+4} & \srbop{+17}  & \srbop{+9}  & \srbop{+15}  & 
\srbop{-10} 
\\[1em] \hline
\srb{+7} & \srb{-7} & \srb{+7} & \srbop{~~~0}  & \srbop{-14}  & \srbop{+14}  
& 
\srbop{~~~0} 
\\[1em] \hline
\srb{-11} & \srb{~~~0} & \srb{+4} & \srbop{+26}  & \srbop{+18}  & \srbop{+15} 
 
& 
\srbop{-19} 
\\[1em] \hline
\end{tabular}
\end{center}
\end{esercizio}

\vspace{-.5em}
\immagine*{Testo alternativo: tabella 7 righe 7 colonne 
(comprese intestazioni di colonna).
Prima riga composta dalle intestazioni di colonna: 
\(a\); \(b\); \(c\); \(-2 \cdot a-b+c\); \(-2 \cdot a-(b+c)\); 
\(-a-2 \cdot b+c\); \(a-b-2 \cdot c\).
Seconda riga: -1; +2; -3.
Terza riga: +2; +3; -5.
Quarta riga: +1; 0; -1.
Quinta riga: -5; -3; +4.
Sesta riga: +7; -7; +7.
Settima riga: -11; 0; +4.}{}


\subsection{Esercizi riepilogativi}

\begin{esercizio}
In quali delle seguenti situazioni è utile ricorrere ai numeri relativi?
 \begin{enumeratees}
 \item misurare la temperatura;
 \item contare le persone;
 \item esprimere la data di nascita di un personaggio storico;
 \item esprimere l'età di un personaggio storico;
 \item indicare il saldo attivo o passivo del conto corrente;
 \item indicare l'altezza delle montagne e le profondità dei mari.
 \end{enumeratees}
\end{esercizio}

% \clearpage
\begin{esercizio}
La somma di due numeri relativi è sicuramente positiva quando:
 \begin{htmulticols}{2}
 \noindent
 \fbox{A} \quad i due numeri sono concordi.\\
 \fbox{B} \quad i due numeri sono discordi.\\
 \fbox{C} \quad i numeri sono entrambi positivi.\\
 \fbox{D} \quad i numeri sono entrambi negativi.
 \end{htmulticols}
\end{esercizio}

\begin{esercizio}
La somma di due numeri relativi è sicuramente negativa quando:
 \begin{htmulticols}{2}
 \noindent
 \fbox{A} \quad i due numeri sono concordi.\\
 \fbox{B} \quad i due numeri sono discordi.\\
 \fbox{C} \quad i numeri sono entrambi positivi.\\
 \fbox{D} \quad i numeri sono entrambi negativi.
 \end{htmulticols}
\end{esercizio}

\begin{esercizio}
Il prodotto di due numeri relativi è positivo quando (più di una risposta 
possibile):
 \begin{htmulticols}{2}
 \noindent
 \fbox{A} \quad i due numeri sono concordi.\\
 \fbox{B} \quad i due numeri sono discordi.\\
 \fbox{C} \quad i numeri sono entrambi positivi.\\
 \fbox{D} \quad i numeri sono entrambi negativi.
 \end{htmulticols}
\end{esercizio}

\begin{esercizio}
Il prodotto di due numeri relativi è negativo quando:
 \begin{htmulticols}{2}
 \noindent
 \fbox{A} \quad i due numeri sono concordi.\\
 \fbox{B} \quad i due numeri sono discordi.\\
 \fbox{C} \quad i numeri sono entrambi positivi.\\
 \fbox{D} \quad i numeri sono entrambi negativi.
 \end{htmulticols}
\end{esercizio}

\pagebreak %------------------------------------------

\begin{esercizio}
Quali delle seguenti affermazioni sono vere?
\TabPositions{110mm}
\begin{enumeratees}
 \item ogni numero relativo è minore di zero \tab\verofalso
 \item la somma di due numeri discordi è zero \tab\verofalso
 \item il cubo di un numero intero relativo è sempre negativo 
\tab\verofalso
 \item la somma di due numeri opposti è nulla \tab\verofalso
 \item il quoziente di due numeri opposti è l'unità \tab\verofalso
 \item il quoziente di due numeri concordi è positivo \tab\verofalso
 \item il prodotto di due numeri opposti è uguale al loro quadrato 
\tab\verofalso
 \item il doppio di un numero intero negativo è positivo \tab\verofalso
 \item la somma di due interi concordi è maggiore di ciascun addendo 
\tab\verofalso
 \item il quadrato dell'opposto di un intero è l'opposto del suo quadrato 
\tab\verofalso
\end{enumeratees}
\end{esercizio}

\begin{esercizio}
Inserisci l'operazione corretta per ottenere il risultato.

\vspace{-.5em}
 \begin{htmulticols}{3}
 \begin{enumeratees}
 \item \((+2)\ldots(-1) = -2\)
 \immagine*{Testo alternativo: \((+2) p (-1) = -2\)}{}
 \item \((-10)\ldots(+5) = -2\)
 \immagine*{Testo alternativo: \((-10) p (+5) = -2\)}{}
 \item \((-18)\ldots(-19) = +1\)
 \immagine*{Testo alternativo: \((-18) p (-19) = +1\)}{}
 \item \((+15)\ldots(-20) = -5\)
 \immagine*{Testo alternativo: \item \((+15) p (-20) = -5\)}{}
 \item \((-12)\ldots(+4) = -3\)
 \immagine*{Testo alternativo: \((-12) p (+4) = -3\)}{}
 \item \((-4)\ldots0 =~0\)
 \immagine*{Testo alternativo: \((-4) p 0 =~0\)}{}
 \item \((+1)\ldots(+1) =~0\)
 \immagine*{Testo alternativo: \((+1) p (+1) =~0\)}{}
 \item \((+5)\ldots(-6) = +11\)
 \immagine*{Testo alternativo: \((+5) p (-6) = +11\)}{}
 \item \(-8\ldots(-2) = +16\)
 \immagine*{Testo alternativo: \(-8 p (-2) = +16\)}{}
 \end{enumeratees}
 \end{htmulticols}
\end{esercizio}


\begin{esercizio}
Inserisci il numero mancante.

\vspace{-.5em}
 \begin{htmulticols}{3}
 \begin{enumeratees}
 \item \(+5 + (\ldots\ldots) = -5\)
 \immagine*{Testo alternativo: \(+5 + (p p) = -5\)}{}
 \item \(-8 + (\ldots\ldots) = -6\)
 \immagine*{Testo alternativo: \item \(-8 + (p p) = -6\)}{}
 \item \(+7 - (\ldots\ldots) =~0\)
 \immagine*{Testo alternativo: \(+7 - (p p) =~0\)}{}
 \item \(0 - (\ldots\ldots) = -2\)
 \immagine*{Testo alternativo: \(0 - (p p) = -2\)}{}
 \item \(+3\cdot (\ldots\ldots) = -3\)
 \immagine*{Testo alternativo: \(+3\cdot (p p) = -3\)}{}
 \item \(-5\cdot (\ldots\ldots) =~0\)
 \immagine*{Testo alternativo: \(-5\cdot (p p) =~0\)}{}
 \item \((+16): (\ldots\ldots) = -2\)
 \immagine*{Testo alternativo: \((+16): (p p) = -2\)}{}
 \item \((-6): (\ldots\ldots) = -1\)
 \immagine*{Testo alternativo: \((-6): (p p) = -1\)}{}
 \item \((-10): (\ldots\ldots) = +5\)
 \immagine*{Testo alternativo: \((-10): (p p) = +5\)}{}
 \end{enumeratees}
 \end{htmulticols}
\end{esercizio}

\begin{esercizio}
 Scrivi tutti i numeri:
 \begin{enumeratees}
 \item interi relativi che hanno valore assoluto minore di~5;
 \item interi relativi il cui prodotto è~\(-12\)
 \item interi negativi maggiori di~\(-5\)
 \end{enumeratees}
\end{esercizio}

\begin{esercizio}
Inserisci ``\(+\)'' o ``\(-\)'' in modo da ottenere il numero più grande 
possibile:

\vspace{-.5em}
 \[-3\ldots(-3)\ldots3\ldots(-6).\]
 \immagine*{Testo alternativo: }{}
\end{esercizio}

\begin{esercizio}
Inserisci le parantesi in modo da ottenere il risultato indicato.

\vspace{-.5em}
 \begin{htmulticols}{2}
 \begin{enumeratees}
 \item \(-5 \cdot +3-1+2=-20\)
 \item \(-5+2\cdot-1+2=+5\)
 \item \(-5+7-3\cdot 2=+3\)
 \item \(-1\cdot +3-5\cdot -1-2=+12\)
%  \item \(+1-1\cdot 1 -1+3-2\cdot \tonda{-3}-2=+5\)
 \end{enumeratees}
 \end{htmulticols}
\end{esercizio}

\begin{esercizio}
Calcola il valore delle seguenti espressioni.

\vspace{-.5em}
 \begin{htmulticols}{2}
 \begin{enumeratees}
 \item \(-5+7+4-9\)
 \item \(+1-1+1-1+1-1+1\)
 \item \(+1-2+3-4+5-6\)
%  \item \(+1-2+2-3+3-4+5-6+6-7+7-8+8-9+9-10\)
 \item \((-3+10)-(2-3)\)
 \end{enumeratees}
 \end{htmulticols}
\end{esercizio}

\begin{esercizio} %38
Calcola il valore delle seguenti espressioni.
 \begin{enumeratees}
 \item \((+5-2-1)+(+2+4+6)\)  \sol{14}
 \item \((-5+7-9)+(+1-2+3)-(+4-6+8)\)  \sol{-11}
 \item \(+4-3-[+2-1-(8-3)-(-5-2)]-(2+3)\)  \sol{-7}
 \item \(-2+(-5+1)+(-7+4)-2 \cdot (-6+1)\)  \sol{1}
 \item \(15-9 \cdot (-14+12)+8 \cdot (-3+6)+ 5 \cdot(-3+1)\)  \sol{47}
 \end{enumeratees}
\end{esercizio}

\begin{esercizio} %39
Calcola il valore delle seguenti espressioni.
 \begin{enumeratees}
 \item \((50-36-25)\cdot (-15+5+20)-10\cdot (-3-7)\)  \sol{-10}
 \item \([+3-(10-5+25)]\cdot [-16+5-(-2-14)]:(9+6)\)  \sol{-9}
 \item \(20:(+15-5)-30:(-10+5)+40:(15-20)\)  \sol{0}
 \item \(18:(-3)+6\cdot [1-5\cdot (-2+4)+3]: (-6)\)  \sol{0}
%  \item \(3\cdot 4-3\cdot [18:(-2)-17+(14-26+5)\cdot 3-12]+[16-1\cdot 
% (-1-3+5)-37+16]\)
\end{enumeratees}
\end{esercizio}

\begin{esercizio} %40
Calcola il valore delle seguenti espressioni e indica dove puoi applicare le 
proprietà delle potenze.

\vspace{-.5em}
\begin{htmulticols}{2}
\begin{enumeratees}
 \item \(100:2+3^2 -2^2\cdot 6\)  \sol{35}
 \item \(2^7:2^3 -2^2\)  \sol{12}
 \item \(30-5\cdot 3 -7\cdot 2^2 -2\)  \sol{-15}
 \item \((3^2 +4^2) -(-3-4)^2\)  \sol{-24}
 \item \(5\cdot 5^3\cdot 5^4: \tonda{5^2 }^3 +5\)  \sol{30}
 \item \(32^5:16^4 +(-2)^9\)  \sol{0}
 \item \(\tonda{3^4\cdot 3^3:~3^6 }^2 +(7^2-5^2):2^2\)  \sol{15}
 \item \(\tonda{3\cdot 2^2 -10 }^4\cdot (3^3+2^3):7-10\cdot 2^3\)  \sol{0}
\end{enumeratees}
\end{htmulticols}
\end{esercizio}

\begin{esercizio} %41
Calcola il valore delle seguenti espressioni.
 \begin{enumeratees}
 \item \(-5\cdot(12-3+4)-2\cdot\big[3-16:(-2+4)\big]^2\)  \sol{-115}
 \item \(\big[-3+(-5)\cdot(-1)\big]^3+\big[-4-(1-2)\big]^2\)  \sol{17}
 \item 
\(\big[2\cdot(-3)^2+2\cdot(-3)\cdot(-2)\big]^2:\big[2^4-3\cdot(+6)\big]^2\)
  \sol{225}
 \item 
\(\big[3\cdot(-1)^2-3\cdot(-3)\cdot(-3)\big]^3:\big[2^2+5\cdot(-2)^2\big]^3\)
  \sol{-1}
 \end{enumeratees}
\end{esercizio}

\begin{esercizio} %42
Calcola il valore delle seguenti espressioni.
 \begin{enumeratees}
 \item \((-3)^2\cdot(4-1)^5:[(-4)^3:(2^5)-3^3:(-3)^3]\)  \sol{-2187}
 \item \([-(-2)\cdot2+(-10)^2:(-5)^2]\cdot[3-5+2\cdot(-3)^2-5]\)  \sol{88}
 \item \(13-3-4\cdot(-2)^2-5^3:5^2+3\cdot(2^3-3^2)-6:(-3)-(4-7+3)^4\)
   \sol{-12}
 \item \(-1-3\cdot(-3)^2-4^3:4^2+(-3-3)\cdot(2^2+3^2)-(-12):(-3)\) 
 \sol{-114}
 \end{enumeratees}
\end{esercizio}

\begin{esercizio} %43
Calcola il valore delle seguenti espressioni.
 \begin{enumeratees}
 \item \([10-6\cdot(-2)^2]:(-7)+(3^2:3)\cdot2^3-15:(-3)+[(-3)^3:(-3)^0]\)
   \sol{4}
 \item \(|-5+8|-|-11|+(-|+4|\cdot|-2\cdot(+5)|)^2\)  \sol{1592}
 \item \((-29+37)^5\cdot(-5+|23-28|)^7\)  \sol{0}
 \item \(-2\cdot(-2\cdot|-2|)^2-\tonda{|3-5|\cdot(3-5) }^2\cdot(-2)\)
   \sol{0}
 \item \((-1)^3\cdot(-1\cdot|-1|)^2-(|-3-2|\cdot(-5+3))^2\cdot(-2+1)^3\)
   \sol{99}
 \end{enumeratees}
\end{esercizio}
\begin{htmulticols}{2}
\begin{esercizio}
Traduci in una espressione matematica le seguenti frasi e motivane la 
verità o falsità:
 \begin{enumeratees}
 \item il cubo del quadrato di un numero diverso da zero è sempre positivo;
 \item il quadrato della somma di un numero con il suo opposto è sempre 
positivo;
 \item la differenza tra il triplo di~5 e l'unità è uguale all'opposto di~5;
 \item il prodotto tra il quadrato di un numero negativo e l'opposto dello 
stesso numero 
è uguale all'opposto del suo cubo.
 \end{enumeratees}
\end{esercizio}

\begin{esercizio}
 Sottrarre dal cubo di~\(-3\) la somma dei quadrati di~\(+\)2 e~\(-2\) Il 
risultato è?
\end{esercizio}

\begin{esercizio}
 Sottrarre dalla somma di~\(-\)15 e~\(+\)27 il prodotto di~\(-\)3 e~\(+\)7.
\end{esercizio}

\begin{esercizio}
 Aggiungere al prodotto di~\(-\)5 e~\(+\)3 la somma di~\(+\)5 e~\(-\)10.
\end{esercizio}

\begin{esercizio}
 Sottrarre dal prodotto di~\(+\)7 e~\(+\)4 la somma di~\(+\)1 e~\(-\)8.
\end{esercizio}

\begin{esercizio}
 Moltiplica la somma tra~\(-3\) e~\(+3\) con la differenza tra~\(+3\) e~\(-3\)
\end{esercizio}

\begin{esercizio}
 Partendo dal pian terreno scendo di~15 gradini, salgo~12 gradini, scendo di~7
gradini e risalgo di~8. A che punto mi trovo rispetto al pian terreno?
\end{esercizio}

\begin{esercizio}
 Giocando a carte contro due avversari nella prima partita ho vinto~50 
gettoni 
con il primo giocatore e perso~60
gettoni con il secondo giocatore, nella seconda partita ho perso~30 gettoni 
con 
il primo e vinto~10 gettoni
con il secondo. Quanti gettoni ho vinto complessivamente?
\end{esercizio}

% \begin{esercizio}[*]
%  Un polpo congelato è stato appena tolto dal congelatore, la sua 
% temperatura 
% è~\(-12\)\textdegree;
% viene immerso nell'acqua bollente e la sua temperatura media è aumentata 
% di~6\textdegree.
% A quale temperatura media si trova ora il polpo?
% \end{esercizio}

\begin{esercizio}
 Una lumaca sale su un muro alto~10 metri, di giorno sale di due metri ma di 
notte
scende di un metro. In quanti giorni la lumaca arriva in cima al muro?
\end{esercizio}

% \begin{esercizio}[*]
%  Un termometro segna all'inizio~\(-\)5\textdegree, poi scende 
% di~3\textdegree, 
% quindi sale di~2\textdegree,
% infine discende di~6\textdegree. Quale temperatura segna alla fine?
% \end{esercizio}
% 
% \begin{esercizio}[*]
%  Il prodotto di due numeri interi relativi è~\(+\)80, aumentando di~1 il 
% primo 
% numero il
% prodotto è~\(+\)72. Quali sono i due numeri?
% \end{esercizio}

\begin{esercizio}
Il prodotto di due numeri interi relativi è~\(+6\), la loro somma è~\(-5\). 
Quali sono i due numeri? \sol{-3;~-2}
\end{esercizio}

\begin{esercizio}
Determina due numeri relativi aventi come prodotto~\(+12\) e come 
somma~\(-7\). \sol{-3;~-4}
\end{esercizio}

\begin{esercizio}
Determina due numeri relativi aventi come prodotto~\(-12\) e come 
somma~\(-1\). \sol{+3;~-4}
\end{esercizio}

\begin{esercizio}
Determina due numeri relativi aventi come prodotto~\(-2\) e come 
somma~\(+1\). \sol{+2;~-1}
\end{esercizio}

\begin{esercizio}
Determina due numeri relativi aventi come prodotto~\(-10\) e come 
somma~\(-3\). \sol{-5;~+2}
\end{esercizio}

\begin{esercizio}
Determina due numeri relativi aventi come prodotto~\(+14\) e come 
somma~\(-9\). \sol{-7;~-2}
\end{esercizio}

\begin{esercizio}
Determina due numeri relativi aventi come prodotto~\(+15\) e come 
somma~\(-8\). \sol{-5;~-3}
\end{esercizio}

% \begin{esercizio}
%  Determina due numeri relativi aventi come prodotto~\(-7\) e come 
% somma~\(+6\). \sol{+7;~-1}
% \end{esercizio}
\end{htmulticols}
% \subsection{Risposte}

% \paragraph{2.42.}
% a)~\(-3\),\quad b)~\(+1\),\quad c)~\(-3\),\quad d)~\(-8\),\quad e)~\(+8\)
% 
% \paragraph{2.43.}
% a)~\(+14\),\quad b)~\(-11\),\quad c)~\(-7\),\quad d)~\(+1\),\quad e)~\(+47\)
% 
% \paragraph{2.44.}
% a)~\(-10\),\quad b)~\(-9\),\quad c)~\(0\),\quad d)~\(0\),\quad e)~\(+183\)
% 
% \paragraph{2.45.}
% a)~\(+35\),\quad b)~\(+12\),\quad c)~\(-15\),\quad d)~\(-24\)
% 
% \paragraph{2.46.}
% a)~\(+30\),\quad b)~\(0\),\quad c)~\(+15\),\quad d)~\(0\)
% 
% \paragraph{2.47.}
% a)~\(-115\),\quad b)~\(+17\),\quad c)~\(+225\)
% 
% \paragraph{2.48.}
% a)~\(-3^7\),\quad b)~\(+88\),\quad c)~\(-12\)
% 
% \paragraph{2.49.}
% a)~\(+4\),\quad b)~\(+1592\),\quad c)~\(0\),\quad d)~\(0\)
% 
% \paragraph{2.60.}
% \(-6\)\textdegree.
% 
% \paragraph{2.61.}
% \(-10\) \(-8\)

