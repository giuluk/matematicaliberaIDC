% (c) 2012 Claudio Carboncini - claudio.carboncini@gmail.com
% (c) 2012 Dimitrios Vrettos - d.vrettos@gmail.com
% (c) 2014 Daniele Zambelli - daniele.zambelli@gmail.com

\input{\folder calcololetterale_grafici.tex}

\chapter{Calcolo letterale}

\inicapitolo{
\begin{itemize}[left=0mm, nosep]
\item espressioni letterali;
\item modello di un problema;
\item monomi;
\item polinomi;
\item prodotti notevoli;
\item la funzione polinomiale \emph{in una variabile}.
\end{itemize}
}

\section{Espressioni letterali e valori numerici}
\label{sec:calclett_esplett}

Quando scriviamo che \(5 +7 = 7 +5\) indichiamo una frase vera relativa 
ai due numeri \emph{cinque} e \emph{sette}.
Ma la proprietà commutativa non vale solo per \(5\) e \(7\), vale anche 
per \emph{molte} altre coppie numeri.
Elencarle tutte sarebbe un problema dato che ce ne sono infinite.
Abbiamo già visto, nei moduli sugli insiemi numerici, che è possibile 
riferirsi a un generico numero usando delle lettere 
(o delle combinazioni di lettere).

In questo capitolo affronteremo il calcolo con le lettere.

\subsection{Modello di un problema}\ind{modello}
\label{subsec:calclett_modello}

\begin{problema}{}{}\label{cartoni}
Lo scatolificio ``Cartoni'' deve produrre delle scatole (aperte) partendo 
da un rettangolo di cartoncino delle dimensioni di 
\(2\munit{dm} \times 3\munit{dm}\).

Le scatole si ottengono operando quattro tagli tutti della stessa 
lunghezza paralleli ai lati lunghi del rettangolo e alla stessa distanza 
dai lati corti. 
Fatto questo, con qualche piegatura e quattro punti di colla si ottiene 
la scatola.

Lo scatolificio è interessato a sapere se le scatole così prodotte,
al variare della lunghezza del taglio, hanno tutte lo stesso volume e, 
se hanno volumi diversi, quanto deve essere lungo 
il taglio per ottenere la scatola di volume massimo. \\[.5em]
% \hspace*{\fill} \scatola \hspace*{\fill} 
\scatola 
\end{problema}

\pagebreak %-------------------------------------

Prima di affrontare il problema generale, possiamo provare a risolvere 
alcuni casi particolari.
Se il taglio è di un centimetro, cioè di \(0,1\munit{dm}\) i tre spigoli 
del parallelepipedo saranno: \\
\(s_1 = \tonda{2 - 2 \cdot 0,1}\munit{dm} = 1,8\munit{dm}; \quad 
  s_2 = \tonda{3 - 2 \cdot 0,1}\munit{dm} = 2,8\munit{dm}; \quad 
  s_3 = 0,1\munit{dm}\)\\
e di conseguenza il volume: \quad
\(V = s_1  \times s_2 \times s_3 = 
  \tonda{1,8 \times 2,8 \times 0,1} \munit{dm^3} = 0,504\munit{dm^3}\)

Se il taglio è di due centimetri, cioè di \(0,2\munit{dm}\) i tre spigoli 
del parallelepipedo saranno: \\
\(s_1 = \tonda{2 - 2 \cdot 0,2}\munit{dm} = 1,6\munit{dm}; \quad 
  s_2 = \tonda{3 - 2 \cdot 0,2}\munit{dm} = 2,6\munit{dm}; \quad 
  s_3 = 0,2\munit{dm}\)\\
e di conseguenza il volume: \quad
\(V = s_1  \times s_2 \times s_3 = 
  \tonda{1,6 \times 2,6 \times 0,2} \munit{dm^3} = 0,832\munit{dm^3}\)

Possiamo generalizzare la soluzione del problema dando un nome alla 
lunghezza del taglio. 
Un buon nome per una variabile da cui dipendono altri risultati è \(x\).

Sottintendendo le unità di misura, se il taglio è lungo \(x\) 
i tre spigoli del parallelepipedo saranno: \qquad
\(s_1 = 2 - 2 \cdot x; \qquad 
  s_2 = 3 - 2 \cdot x; \qquad 
  s_3 = x\)\\
e di conseguenza il volume: \qquad
\(V = s_1  \times s_2 \times s_3 = 
  \tonda{2 - 2 \cdot x} \times \tonda{3 - 2 \cdot x} \times x 
\)

L'ultima che abbiamo ottenuto è un'espressione che contiene una lettera: 
\(x\). 
Le espressioni matematiche che contengono lettere si dicono 
\emph{espressioni letterali}\ind{espressione letterale} e generalizzano 
soluzioni di problemi.

\affiancati{.69}{.29}{
Le espressioni letterali possono esprimere funzioni: infatti 
danno risultati che dipendono unicamente dai valori assegnati alle 
variabili identificate dalle lettere. 
In questo esempio il volume dipende dal valore dato alla variabile \(x\):
\[V(x) = \tonda{2 - 2 \cdot x} \cdot \tonda{3 - 2 \cdot x} \cdot x\]
Possiamo calcolare diversi valori del volume in corrispondenza dei 
diversi valori dati all'argomento \(x\).

Prima di completare la tabella a fianco, prova a fare qualche congettura 
sui valori della funzione e per quale valore di \(x\) il volume potrebbe 
essere massimo.
}{
\begin{center}
\begin{tabular}{cc}
taglio: x & vol.: V(x)\\
\hline
0 & 0\\
0,1 & 0,504\\
0,2 & 0,832\\
0,3 & \dots\\ % 1,008
0,4 & \dots\\ % 1,056
0,5 & \dots\\ % 1
0,6 & \dots\\ % 0,864
0,7 & \dots\\ % 0,672
0,8 & \dots\\ % 0,448
0,9 & \dots\\ % 0,216
1,0 & \dots\\ % 0
\hline
\end{tabular}
\end{center}
}

\begin{osservazione}{}{}
Attenzione, quando si usa un'espressione come \indt{modello} di una 
situazione concreta, dobbiamo stare attenti ai limiti che la situazione 
stessa pone.
\end{osservazione}

Nel nostro caso, se operiamo tagli lunghi \(2\) otteniamo come volume: 
\(V(2) = 4\) che è maggiore di tutti i volumi calcolati sopra.

Ma ha senso operare tagli di \(2\munit{dm}\) su un cartoncino di 
\(2\munit{dm} \times 3\munit{dm}\)?
Qual è il massimo valore che posso dare alla variabile \(x\) per questo 
particolare modello?

È evidente che, in questo problema, non è possibile costruire scatole 
operando tagli superiori a \(1\munit{dm}\), e non è neppure possibile 
operare tagli negativi. 
Il modello del problema precedente deve tener conto di questi due limiti 
e quindi diventa:
\[\sistema{0 \leqslant x < 1,0 \\ 
    V(x) = \tonda{2 - 2 \cdot x} \cdot \tonda{3 - 2 \cdot x} \cdot x}
\]

\subsection{Espressioni e formule}
\label{subsec:calclett_formule}

Abbiamo visto sopra il modello di un problema semplice, 
ma non semplicissimo.
Ci sono molti problemi di cui è importante conoscere la funzione che 
esprime il \indt{modello} matematico.
Ne vediamo alcuni.

\begin{center}
\begin{tabular}{ll}
problema & funzione\\
\hline\\
superficie del quadrato di lato \(l\) & \(S_{quad}(l) = l^2\)\\[.5em]
volume del cubo di spigolo \(s\) & \(V_{cubo} (s) = s^3\) \\[.5em]
superficie del triangolo con base \(b\) e altezza \(h\) & 
\(S_{tri}\coppia{b}{h} = \frac{1}{2} b \cdot h\) \\[.5em]
media tra i due numeri \(n_0\) e \(n_1\) & 
\(Media\coppia{n_0}{n_1} = \frac{1}{2}\tonda{n_0 + n_1}\) \\[.5em]
sup. del trapezio con basi \(b_0\) e \(b_1\), e altezza \(h\) & 
\(S_{trap}\terna{b_0}{b_1}{h} = \frac{1}{2} \tonda{b_0 + b_1} h\) \\[1em]
\hline
\end{tabular}
\end{center}

Vediamo che espressioni letterali possono rappresentare funzioni in una, 
due, o più variabili (con uno, due o più parametri).

Il risultato di una funzione rappresentata da un'espressione letterale è 
anche detto \indt{valore numerico dell'espressione} e si ottiene 
sostituendo le variabili con un loro valore numerico.

\begin{esempio}{}{}
Un trapezio ha basi di \(38\munit{cm}\) e \(42\munit{cm}\) e ha l'altezza 
di \(12\munit{cm}\) calcola la superficie.\\
\(S_{trap}\terna{b_0}{b_1}{h} = \dfrac{1}{2} \tonda{b_0 + b_1} h\)\\
\(S_{trap}\terna{38\munit{cm}}{42\munit{cm}}{12\munit{cm}} = 
  \dfrac{1}{2} \tonda{38\munit{cm} + 42\munit{cm}} \cdot 12\munit{cm} =
%   \dfrac{80\munit{cm} \cdot 12\munit{cm}}{2} = 
  \dots =
  480\munit{cm^2}
\)
\end{esempio}

\begin{comment}

\begin{esempio}{}{}
In tutte le villette a schiera di recente costruzione del nuovo quartiere 
Stella, vi è un terreno rettangolare di 
larghezza ~~ \(b = 5\munit{m}\) ~~ e ~~ 
lunghezza ~~ \(a = 12\munit{m}\). 
\\
Quanto misura la superficie del terreno?\\

\affiancati{.40}{.50}{
\begin{center}
\rettangolocl{12}{5}
\end{center}
}{
Il prodotto delle dimensioni rappresenta la misura richiesta: \\
\(S = a \cdot b = 12\munit{m} \cdot 5\munit{m} = 60 \munit{m^2}\).
}
\end{esempio}

Il semplice problema che abbiamo risolto è relativo a un caso particolare; 
quel terreno con quelle dimensioni. Ma se le dimensioni fossero diverse?

La procedura per determinare la misura della superficie ovviamente è 
sempre la stessa e la possiamo esprimere con la formula~\(A=b\cdot h\) 
nella quale abbiamo indicato con~\(b\) la misura di una dimensione e 
con~\(h\) la misura dell'altra dimensione, assegnate rispetto alla 
stessa unità di misura.

\begin{osservazione}{}{} 
La formula ha carattere generale; essa serve ogni 
qualvolta 
si chiede di determinare la superficie di un rettangolo, note le misure 
delle dimensioni (base e altezza) rispetto alla stessa unità di misura.
\end{osservazione}

In geometria si utilizzano tantissime formule che ci permettono di 
determinare perimetro e area delle figure piane, superficie laterale e 
totale 
e volume dei solidi. Nelle formule le lettere sostituiscono le misure di 
determinate grandezze, tipiche di quella figura o di quel solido.

% \ovalbox{\risolvi c}

\end{comment}

\subsection{Parole per descrivere schemi di calcolo}
\label{subsec:calclett_parole}

\begin{esempio}{}{}
 L'insegnante chiede agli alunni di scrivere 
 <<il doppio della somma di due numeri>>.

\begin{itemize} [nosep]
\item Antonella scrive:~\(2\cdot (3+78)\)
\item Maria chiede <<Quali sono i numeri? Se non li conosco non posso 
  soddisfare la richiesta>>;
\item Giulia scrive:~\(2\cdot (a+b)\).
\end{itemize}
Maria si è posta il problema ma non ha saputo generalizzare la richiesta. 
Antonella si è limitata a un caso particolare. 
Giulia ha espresso con una formula l'operazione richiesta dall'insegnante.
\end{esempio}

% \begin{osservazione}{}{} 
L'uso di lettere o nomi di variabili per indicare numeri ci permette 
di generalizzare uno schema di calcolo.
% \end{osservazione}


\begin{definizione}{}{}
Un'\textbf{espressione letterale}, detta anche \textbf{espressione
algebrica},\ind{espressione algebrica} è un'espressione in cui compaiono 
numeri e lettere legati dai simboli delle operazioni.
\end{definizione}

Per scrivere un'\indt{espressione letterale} ci si deve attenere a regole 
precise, quelle stesse che utilizziamo per scrivere espressioni numeriche.

Per esempio, la scrittura~``\(3\cdot 4+\)'' non è corretta, in quanto il 
simbolo~``\(+\)'' dell'addizione deve essere seguito da un altro numero 
per completare l'operazione. Analogamente non è corretta l'espressione 
letterale~``\(a \cdot + c\)''.

Come nelle espressioni numeriche, anche nelle espressioni letterali le 
parentesi indicano le operazioni da eseguire per prime.

\begin{esempio}{}{}
\begin{enumerate} [nosep]
\item 
La formula~\(a \cdot (b +c)\) rappresenta 
``il prodotto di un numero per la somma di due numeri''. 
\item 
La formula  \(a\cdot b +c\) rappresenta 
``la somma fra il prodotto di due numeri e un terzo numero''.
\end{enumerate}
\end{esempio}

% \vspazio\ovalbox{\risolvii \ref{ese:8.2}, \ref{ese:8.3}, \ref{ese:8.4}}

\subsection{Lettere per esprimere proprietà}
\label{subsec:calclett_proprieta}

Nei capitoli precedenti abbiamo già incontrato espressioni letterali, 
le abbiamo usate tutte le volte che volevamo illustrare delle proprietà 
valide che valgono usando qualunque numero. 

\begin{esempio}{}{}
\begin{enumerate} [nosep]
\item 
L'espressione ``\((a+b)+c=a+(b+c)\)'' esprime la proprietà 
associativa dell'addizione. 
\item 
L'espressione ``\(a^m \cdot a^n = a^{m + n}\)'' esprime la prima 
proprietà delle potenze. 
\end{enumerate}
Se non viene specificata qualche condizione, le lettere stanno al posto 
di numeri qualsiasi.
\end{esempio}

% \vspazio\ovalbox{\risolvii \ref{ese:8.5}, \ref{ese:8.6}, \ref{ese:8.7}, 
% \ref{ese:8.8}, \ref{ese:8.9}, \ref{ese:8.10}, \ref{ese:8.11}, 
% \ref{ese:8.12}}

\begin{comment}

\subsection{Valore numerico di un'espressione letterale}
\label{subsec:valnum}

Ogni espressione letterale rappresenta uno schema di calcolo in cui le 
lettere che vi compaiono sostituiscono numeri.
L'espressione letterale~\(2\cdot x^{2}+x\) traduce una catena di 
istruzioni che in linguaggio naturale sono così descritte: ``prendi un 
numero; fanne il quadrato; raddoppia quanto ottenuto; aggiungi al 
risultato il numero preso inizialmente''.

Questa catena di istruzioni si può anche rappresentare in modo schematico
\[x\rightarrow x^{2}\rightarrow~2\cdot x^{2}\rightarrow~2\cdot x^{2}+x\]
e può essere usata per istruire un esecutore a ``calcolare'' l'espressione 
letterale quando al posto della lettera~\(x\) si sostituisce un numero.

Calcoliamo il valore dell'espressione~\(2\cdot x^{2}+x\), sostituendo 
alla lettera il numero naturale~5.
Seguiamo la schematizzazione~\(x\rightarrow x^{2}\rightarrow~2\cdot 
x^{2}\rightarrow~2\cdot x^{2}+x\) e otteniamo:
\(5\rightarrow~25\rightarrow~50\rightarrow~55\).
Il risultato è~\(55\).
Più brevemente scriviamo~\(5\) nell'espressione letterale al posto 
di~\(x\): otteniamo l'espressione numerica~\(2\cdot 5^{2}+5\) il cui 
risultato è~\(55\).

E se al posto di~\(x\) sostituiamo~\(-5\)? Cambia il risultato?

Eseguiamo la sostituzione:  \(2\cdot (-5)^{2}+(-5)=\ldots\) Lasciamo a 
te il calcolo finale. Ti sarai accorto che il risultato è cambiato.

\begin{definizione}{}{}
In un'espressione letterale le \emph{lettere} rappresentano 
le \emph{variabili} che assumono un preciso significato quando vengono 
sostituite da numeri.
Chiamiamo \emph{valore} di un'espressione letterale il risultato numerico 
che si ottiene eseguendo le operazioni indicate dallo
schema di calcolo quando alle lettere sostituiamo un numero. 
Il valore dell'espressione letterale dipende dal \emph{valore assegnato} 
alle sue variabili.
\end{definizione}

\begin{esempio}{}{}
Calcolare il valore numerico dell'espressione:
\(3a(a-b)\) per~\(a = 1\), \(b = 1\).

\emph{Svolgimento}:~\(3\cdot 1\cdot (1-1)=3\cdot 1\cdot 0=0\).
\end{esempio}
 % \end{exrig}

% \ovalbox{\risolvii \ref{ese:8.13}, \ref{ese:8.14}, \ref{ese:8.15}, 
% \ref{ese:8.16}, \ref{ese:8.17}, \ref{ese:8.18}, \ref{ese:8.19}, 
% \ref{ese:8.20}, \ref{ese:8.21}, \ref{ese:8.22}, \ref{ese:8.23},
% \ref{ese:8.24}}

\end>{comment}

\begin{comment}

\subsection{Condizione di esistenza di un'espressione letterale}
\label{subsec:condes}

Ti proponiamo adesso alcuni casi particolari per
l'espressione~\(E=\dfrac{x-y}{3\cdot x}\).

\paragraph{Caso I}

\begin{center}
\begin{tabular*}{.2\textwidth}{@{\extracolsep{\fill}}*{3}{c}}
\toprule
\(x\) & \(y\) & \(E\)\\
1 & 1 & 0\\
\bottomrule
\end{tabular*}
\end{center}

Il numeratore della frazione è~0, mentre il denominatore vale~3; il
calcolo finale è dunque~\(\frac{0}{3}=0\).
Vi sono secondo te altre coppie che fanno
assumere ad~\(E\) quello stesso valore?

\paragraph{Caso II}
\begin{center}
\begin{tabular*}{.2\textwidth}{@{\extracolsep{\fill}}*{3}{c}}
\toprule
\(x\) &\(y\) &\(E\)\\
0 &25 &?\\
\bottomrule
\end{tabular*}
\end{center}

Invece di mettere un valore ad~\(E\), abbiamo messo punto di domanda
perché in questo caso il numeratore della frazione è~\(-25\) mentre
il denominatore vale~0; il calcolo finale è dunque~\(-{\frac{25}{0}}\), 
impossibile. Vi sono secondo te altre coppie che rendono
impossibile il calcolo del valore per~\(E\)?

Non possiamo allora concludere che per ogni coppia di numeri 
razionali~\((x,y)\) 
l'espressione~\(E\) assume un numero razionale.
Per poter calcolare il valore di~\(E\) non possiamo scegliere coppie 
aventi~\(x\) 
uguale a zero.
Scriveremo quindi come premessa alla ricerca dei valori di~\(E\) la 
\emph{Condizione di Esistenza}\,(\(\CE\))~\(x\neq~0\).

L'esempio appena svolto ci fa capire che di fronte a
un'espressione letterale dobbiamo riflettere sullo
schema di calcolo che essa rappresenta prima di assegnare valori alle
variabili che vi compaiono.

Se l'espressione letterale presenta una divisione in cui
il divisore contiene variabili, dobbiamo stabilire la~\(\CE\), 
eliminando quei valori che rendono nullo il divisore.
Per comprendere la necessità di porre le condizioni
d'esistenza ricordiamo la definizione di divisione.

Quanto fa~15 diviso~5? Perché? In forma matematica:~\(15:5=3\) perché 
\(3\cdot 
5=15\). Quindi, 
generalizzando~\(a:b=c\) se~\(c\cdot b=a\).

Vediamo ora cosa succede quando uno dei numeri è~0:

\begin{itemize} [nosep]
 \item quanto fa~0:5? Devo cercare un numero che moltiplicato per~5 mi 
dia~0: 
   trovo solo~0; infatti~\(0\cdot 5=0\).
 \item quanto fa~15:0? Devo cercare un numero che moltiplicato per~0 mi 
dia~15:
non lo trovo; infatti nessun numero moltiplicato per~0 fa~15. Quindi,
\(15:0\) è impossibile perché non esiste
\(x\) per il quale~\(x\cdot 0=15\).
 \item quanto fa~0:0? Devo cercare un numero che moltiplicato per~0 mi 
dia~0:
non ne trovo solo uno. Infatti, qualunque numero moltiplicato per~0
fa~0. Per esempio, \(0:0=33\) infatti
\(33\cdot 0=0\). Anche~\(0:0=-189,6\)
infatti~\(-189,6\cdot 0=0\). Anche \(0:0=0\)
infatti~\(0\cdot 0=0\). 
Ancora~\(0:0=10^{99}\) infatti~\(10^{99}\cdot 0=0\).
Quindi~\(0:0\) è indeterminato, perché
non è possibile determinare un~\(x\) tale che~\(x\cdot 0=0\),
per qualunque valore di~\(x\) si ha~\(x\cdot 0=0\).
\end{itemize}

Consideriamo l'espressione letterale~\(E=\frac{a-b}{a+b}\) dove~\(a\) 
e~\(b\) rappresentano numeri razionali.
Premettiamo:

\begin{enumeratea}
 \item la descrizione a parole dello schema di calcolo:
``divisione tra la differenza di due numeri e la loro
somma'';
 \item la domanda che riguarda il denominatore: ``quando
sommando due numeri razionali otteniamo~0 al
denominatore?'';
 \item la~\(\CE\): ``\(a\) e~\(b\) non devono essere numeri opposti''.
\end{enumeratea}

Siamo ora in grado di completare la tabella:
\begin{center}
\begin{tabular*}{.8\textwidth}{l@{\extracolsep{\fill}}*{5}{c}}
\toprule
\(a\) & 3 &0 & \(\frac{3}{4}\) &\(-{\frac{5}{8}}\) & \(-{\frac{19}{2}}\) \\
\(b\) & \(-3\) & \(-{\frac{1}{2}}\) & 0 &\(\frac{5}{8}\) & 
\(-{\frac{19}{2}}\) 
\\
\midrule
\(E=\frac{a-b}{a+b}\) & & & & &\\
\bottomrule
\end{tabular*}
\end{center}

Dalla~\(\CE\), ci accorgiamo subito che la
prima coppia e la quarta sono formate da numeri opposti, pertanto non
possiamo con esse calcolare il valore di~\(E\). L'ultima
coppia è formata da numeri uguali pertanto la loro differenza è~0;
il numeratore si annulla e quindi il valore di~\(E\) è~0. 
Per la coppia~\(\tonda{0,-\frac{1}{2}}\) il valore di~\(E\) è~\(-1\) 
mentre 
è~1 per la coppia~\(\tonda{\frac{3}{4},0}\).
La tabella verrà quindi così completata:

\begin{center}
\begin{tabular*}{.8\textwidth}{l@{\extracolsep{\fill}}*{5}{c}}
\toprule
\(a\) & 3 &0 & \(\frac{3}{4}\) &\(-{\frac{5}{8}}\) & \(-{\frac{19}{2}}\) \\
\(b\) & \(-3\) & \(-{\frac{1}{2}}\) & 0 &\(\frac{5}{8}\) & 
\(-{\frac{19}{2}}\) 
\\
\midrule
\(E\) &impossibile & \(-1\)& 1& impossibile&0\\
\bottomrule
\end{tabular*}
\end{center}

Cosa succede per la coppia~(0,0)?

% \vspazio\ovalbox{\risolvii \ref{ese:8.25}, \ref{ese:8.26}, 
% \ref{ese:8.27}, \ref{ese:8.28}, \ref{ese:8.29}, \ref{ese:8.30}, 
% \ref{ese:8.31}, \ref{ese:8.32}}
\end{comment}

% (c) 2012-2014 Dimitrios Vrettos - d.vrettos@gmail.com
% (c) 2014 Daniele Zambelli - daniele.zambelli@gmail.com

\section{I monomi}\ind{monomio/monomi}
\label{sec:calclett_monomi}

A questo punto, per poter fare pratica con il calcolo letterale, 
è importante:
\begin{itemize} [nosep]
\item imparare alcuni termini tecnici e definizioni,
\item abituarci ad alcune convenzioni.
\end{itemize}


\subsection{Definizioni}
\label{subsec:monomi_definizioni}

D'ora in poi quando scriveremo un'espressione letterale in cui compare
l'operazione di moltiplicazione, tralasceremo il puntino fin qui usato 
per evidenziare l'operazione.
Così l'espressione~\(5 \cdot a^2 \cdot b \cdot x^3\) 
verrà scritta in modo
più compatto~\(5a^2bx^3\).

%\begin{definizione}{}{}
% Una espressione letterale in cui numeri e lettere sono legati dalla 
% sola moltiplicazione si chiama \emph{monomio}.
% \end{definizione}
% 
% % \begin{exrig}
% \begin{esempio}{}{}
% L'espressione nelle due variabili~\(a\) e~\(b\), \(E=5\cdot 
% 2a^{2}\frac{3}{8}ab7b^{2}\)
% è un monomio perché numeri e lettere sono legate solo dalla 
% moltiplicazione.
% \end{esempio}
Quanto scritto sopra viene detto ``monomio''\ind{monomio/monomi}.

\begin{definizione}{}{}
Chiamiamo \textbf{monomio} una espressione letterale in cui non compare
l'addizione algebrica.
\end{definizione}

\begin{esempio}{}{}
\begin{enumerate} [nosep]
\item 
L'espressione: \(5 \cdot 2a^{2} \dfrac{3}{8}ab7b^{2}\)
è un monomio perché nell'espressione non appaiono addizioni o sottrazioni.
\item 
L'espressione: \(2a^{2}-ab^{2}\) non è un monomio poiché compare anche il 
segno di sottrazione.
\end{enumerate}
\end{esempio}

% \begin{osservazione}{}{}
% Gli elementi di un monomio sono \emph{fattori}, perché sono termini
% di una moltiplicazione ma possono comparire anche \emph{potenze},
% infatti la potenza è una moltiplicazione di fattori uguali. 
% % Non possono invece comparire esponenti negativi o frazionari. 
% % In un monomio gli esponenti delle variabili devono essere numeri 
% % naturali.
% \end{osservazione}

\begin{definizione}{}{}
Un monomio si dice \textbf{ridotto in forma normale}
\indc{monomio/monomi}{in forma normale} quando è scritto come 
prodotto di un solo fattore numerico e di potenze letterali con basi 
diverse, scritte in ordine alfabetico.
\end{definizione}

\begin{esempio}{}{}
Applicando le proprietà delle operazioni scrivi i seguenti monomi in 
forma normale.

\begin{enumerate*} 
\item \quad
\(5 \cdot 2a^{2} \dfrac{3}{8}ab2b^{2} = \dfrac{15}{2}a^3b^3\) \qquad~
\item \quad
\(5 x a^{4}\dfrac{3}{20}ab6b^{2} = \dfrac{9}{2}a^{5}b^{3}x\).
\end{enumerate*}
\end{esempio}

% \begin{procedura}{}{}
%  Ridurre in forma normale un monomio:
%  \begin{enumeratea}
%  \item moltiplicare tra loro i fattori numerici;
%  \item moltiplicare le potenze con la stessa base.
%  \end{enumeratea}
% \end{procedura}

% \ovalbox{\risolvi \ref{ese:9.2}}

In un monomio possiamo distinguere due parti:

\begin{definizione}{}{}
Una volta ridotto in forma normale: 
\begin{itemize} [nosep]
\item chiamiamo \textbf{coefficiente}\indc{monomio/monomi}{coefficiente}
la parte numerica;
\item chiamiamo \textbf{parte letterale} il complesso delle
lettere\indc{monomio/monomi}{parte letterale}.
\end{itemize}
\end{definizione}

Se nel monomio non appare il coefficiente, è sottinteso \(1\), 
cioè \(\textbf{a}\) vuol dire \textbf{un} \(\mathbf{a}\):~~ \(a = 1a\).

\begin{esempio}{}{}
Nella tabella seguente sono segnati alcuni monomi e i loro coefficienti 
e le loro parti letterali.

\vspace{-1em}
\begin{center}
\begin{tabular}{lcc}
monomio & coefficiente & parte letterale \\
\hline \\[-.5em]
\(-{\dfrac{1}{2}}abc\) & \(-{\dfrac{1}{2}}\) & \(abc\) \\[1em]
\(3x^{3}y^{5}\) & \(3\) & \(x^{3}y^{5}\) \\[.5em]
\(a^{5}b^{7}\) & \(1\) & \(a^{5}b^{7}\) \\[.5em]
\(-k^{2}\) & \(-1\) & \(k^{2}\) %\\[.5em]
% \hline
\end{tabular}
\end{center}
\end{esempio}

% % \begin{exrig}
% \begin{esempio}{}{}
% L'espressione letterale~\(\frac{3}{5}a^{3}bc^{2}\) è un monomio;
% il numero~\(\frac{3}{5}\) e le lettere~\(a^{3}\), \(b\), \(c^{2}\) 
% sono legate dall'operazione di moltiplicazione; il suo coefficiente 
% è il numero~\(\frac{3}{5}\) e la parte letterale è~\(a^{3}bc^{2}\).
% \end{esempio}
% 
% \begin{esempio}{}{}
%  Controesempi:
% 
%  \begin{enumeratea}
%  \item l'espressione letterale~\(\frac{3}{5}a^{3}+bc^{2}\)
%  non è un monomio dal momento che numeri e lettere sono legati oltre
% che dalla moltiplicazione anche dalla addizione;
% \item l'espressione letterale~\(\frac{3}{5}a^{-3}bc^{2}\)
% non è un monomio in quanto la potenza con esponente negativo
% rappresenta una divisione, infatti~\(a^{-3}=\frac{1}{a^{3}}\).
% \end{enumeratea}
% \end{esempio}
% % \end{exrig}

Alcuni testi non chiamano monomi quelli che hanno lettere al 
denominatore, ma noi distinguiamo tra 
\indtc{monomio/monomi}{monomi interi o fratti}.

\begin{definizione}{}{}
Diciamo che un monomio è: 
\begin{itemize} [nosep]
\item \textbf{intero} se non ha lettere al denominatore
(o se le lettere non hanno esponenti negativi);
\item \textbf{fratto} se ha lettere al denominatore
(o se ci sono lettere con esponenti negativi).
\end{itemize}
\end{definizione}

% \begin{esempio}{}{}
% \begin{itemize} [nosep]
% \item Il monomio \quad \(\dfrac{3}{5}a^{3}bc^{2}\) \quad è intero;
% \item il monomio \quad \(5a^{-3}b^{-1}c^{2}=
%              5 \dfrac{c^{2}}{a^{3}b} =
%              \dfrac{5c^{2}}{a^{3}b}\) \quad è fratto.
% \end{itemize}
%\end{esempio}

\begin{esempio}{}{}
Classifica i seguenti monomi:

\hfill \labelitemi~~ \(\dfrac{3}{5}a^{3}bc^{2}\):\quad intero; 
\hfill
\labelitemi~~ \(5a^{-3}b^{-1}c^{2}=
             5 \dfrac{c^{2}}{a^{3}b} =
             \dfrac{5c^{2}}{a^{3}b}\):\quad fratto. \hfill~
\end{esempio}

Vediamo ora il significato di alcuni termini che vengono utilizzati con i 
monomi.

\begin{definizione}{}{}
Due o più monomi sono \textbf{simili}\indc{monomio/monomi}{monomi simili} se
hanno \indtc{monomio/monomi}{parte letterale} identica.
\end{definizione}

\begin{esempio}{}{}
Il monomio~\(\dfrac{3}{5}a^{3}bc^{2}\) è simile
a~\(68a^{3}bc^{2}\) e anche a~\(-0,5a^{3}bc^{2}\), ma non è simile
a~\(\dfrac{3}{5}a^{2}bc^{3}\). 
L'ultimo monomio ha le stesse lettere degli altri ma sono elevate a 
esponenti diversi.
\end{esempio}

\begin{definizione}{}{}
Si dicono \textbf{opposti}\indc{monomio/monomi}{monomi opposti} due monomi
simili che hanno \indtc{monomio/monomi}{coefficiente} opposto.
\end{definizione}

\begin{esempio}{}{}
\begin{itemize} [nosep]
\item
I monomi~\(\dfrac{3}{5}a^{3}bc^{2}\) e~\(-{\dfrac{3}{5}}a^{3}bc^{2}\) 
sono opposti, infatti sono simili e hanno coefficienti opposti.
\item
I monomi~\(\dfrac{3}{5}a^{3}bc^{2}\) e \(-\dfrac{5}{3}a^{3}bc^{2}\) 
non sono opposti ma semplicemente simili. 
\item
I monomi~\(\dfrac{3}{5}a^{3}bc^{2}\) e \(-\dfrac{3}{5}a^{2}bc^{3}\) 
hanno i coefficienti opposti ma non sono opposti perché non sono 
simili. 
\end{itemize}
\end{esempio}

\begin{definizione}{}{}
Se il coefficiente del monomio è zero il \textbf{monomio} si dice
\textbf{nullo}\indc{monomio/monomi}{monomio nullo}, qualunque sia la parte
letterale; tutti i monomi nulli sono equivalenti.
\end{definizione}

% \ovalbox{\risolvi \ref{ese:9.3}}

\begin{definizione}{}{}
Il \textbf{grado complessivo} di un monomio\indc{monomio/monomi}{grado di
un monomio} è la somma degli esponenti della parte letterale.

Quando il monomio è ridotto a forma normale, l'esponente di una sua 
variabile ci indica il
\textbf{grado} del monomio \textbf{rispetto a quella variabile}.
\end{definizione}

Dobbiamo ricordarci che quando non c'è l'esponente è sottinteso 1: \quad 
\(a = a^1\).

\begin{esempio}{}{}
Il monomio~\(\dfrac{3}{5}a^{3}bc^{2}\) ha grado complessivo~6,
ottenuto sommando gli esponenti della sua parte letterale~\((3+1+2=6)\).\\
Rispetto alla variabile~\(a\) è di terzo grado, rispetto alla
variabile~\(b\) è di primo grado, rispetto alla variabile
\(c\) è di secondo grado.
\end{esempio}

% Abbiamo detto che gli esponenti della parte letterale del monomio sono
% numeri naturali, dunque possiamo anche avere una o più variabili
% elevate ad esponente~0. Cosa succede allora nel monomio?
% 
% Consideriamo il monomio~\(56a^{3}b^{0}c^{2}\), sappiamo che qualunque
% numero diverso da zero elevato a zero è uguale a~1, quindi possiamo
% sostituire la variabile~\(b\) che ha esponente~0 con~1 e
% otteniamo~\(56a^{3}\cdot 1\cdot c^{2}=56a^{3}c^{2}\). Se in un monomio 
% ogni variabile ha esponente~0, il monomio rimane solamente con il suo
% coefficiente numerico: per esempio~\(-3a^{0}x^{0}=-3\cdot 1\cdot 1=-3\).


\begin{osservazione}{}{} 
Esistono \emph{monomi di grado~0}; essi presentano solo il
coefficiente e pertanto \emph{sono} equiparabili ai \emph{numeri}.

Possiamo quindi vedere l'insieme dei monomi come un'estensione dei numeri.

Esiste un sottoinsieme dei monomi che si comporta esattamente come i 
numeri, cioè è isomorfo all'insieme numerico a cui appartengono i 
coefficienti.
\end{osservazione}

\begin{esempio}{}{}
il numero \(7,2\) può essere visto come:\\ 
\labelitemi~ monomio di grado zero di coefficiente \(7,2\) 
\quad o \quad 
\labelitemi~ numero \(7,2\).
\end{esempio}


\begin{comment}


\subsection{Valore di un monomio}
\label{subsec:monomi_valore}

Poiché il monomio è un'espressione letterale,
possiamo calcolarne il valore quando alle sue variabili sostituiamo 
numeri.

\begin{esempio}{}{}
 Calcola il valore del monomio~\(3x^{4}y^{5}z\) per i valori~\(x=-3\), 
\(y=5\) 
e~\(z=0\).

Sostituendo i valori assegnati 
otteniamo~\(3\cdot (-3)^{4}\cdot 5^{5}\cdot 0=0\) 
essendo uno dei fattori nullo.
\end{esempio}


\begin{osservazione}{}{} 
Il valore di un monomio è nullo quando almeno una 
delle sue 
variabili assume il valore~0.
\end{osservazione}


Molte formule di geometria sono scritte sotto forma di monomi: area del
triangolo~\(\frac{1}{2}bh\) area del quadrato~\(l^{2}\)
perimetro del quadrato~\(4l\) area del rettangolo~\(bh\) volume del 
cubo~\(l^{3}\) ecc.
Esse acquistano significato quando alle lettere sostituiamo
numeri che rappresentano le misure della figura considerata.

% \ovalbox{\risolvii \ref{ese:9.4}, \ref{ese:9.5}, \ref{ese:9.6}, 
% \ref{ese:9.7}, \ref{ese:9.8}, \ref{ese:9.9}, \ref{ese:9.10}, 
% \ref{ese:9.11}, \ref{ese:9.12}}

\end{comment}


\subsection{Operazioni con i monomi}
\label{subsec:monomi_operazioni}

Ci proponiamo ora di introdurre nell'insieme dei monomi
le operazioni di addizione, sottrazione, moltiplicazione, potenza,
divisione.

Ricordiamo che definire in un insieme un'operazione
significa stabilire una legge che associa a due elementi
dell'insieme un altro elemento dell'insieme stesso.

\subsubsection{Moltiplicazione di monomi}
\label{subsubsec:monomi_moltiplicazione}

La moltiplicazione di due monomi si indica con lo stesso simbolo della
moltiplicazione tra numeri; i suoi termini si chiamano fattori e il
risultato si chiama prodotto, proprio come negli insiemi numerici.
Dato che nella moltiplicazione vale la proprietà commutativa:

\begin{definizione}{}{}
Il \textbf{prodotto di monomi}\indc{monomio/monomi}{prodotto di monomi}
è il monomio che ha per coefficiente il prodotto dei coefficienti e per 
parte letterale il prodotto delle parti letterali.
\end{definizione}

\begin{esempio}{}{}
Dati i monomi~\(m_{1}=-4x^{2}yz^{3}\) e~\(m_{2}=\dfrac{5}{6}x^{3}z^{6}\),
il monomio prodotto è:

\vspace{-1.5em}
\[m_{1} \cdot m_{2}=\tonda{-4x^{2}yz^{3}} \tonda{\frac{5}{6}x^{3}z^{6}}=
\tonda{-4\cdot {\frac{5}{6}}}\tonda{x^{2}\cdot x^{3}}\cdot 
y\cdot \tonda{z^{3}\cdot z^{6}}=-\frac{10}{3}x^{5}yz^{9}.\]
\end{esempio}


% \begin{procedura}{}{}
% [per moltiplicare due monomi]
% La moltiplicazione tra monomi si effettua moltiplicando prima i
% coefficienti numerici e dopo le parti letterali:
% 
% \begin{enumeratea}
%  \item nella moltiplicazione tra i coefficienti usiamo le regole note 
% della moltiplicazione tra numeri razionali;
%  \item nella moltiplicazione tra le parti letterali applichiamo la 
% regola del prodotto di potenze con la stessa base.
% \end{enumeratea}
% \end{procedura}

% \subsubsection{Proprietà della moltiplicazione}
% 
% \begin{enumeratea}
% \item commutativa:~\(m_{1}\cdot m_{2}=m_{2}\cdot m_{1}\)
% \item associativa:~\(m_{1}\cdot m_{2}\cdot m_{3}=(m_{1}\cdot m_{2})\cdot 
% m_{3}=m_{1}\cdot (m_{2}\cdot m_{3})\)
% \item 1 è l'elemento neutro:~\(1\cdot m=m\cdot 1=m\)
% \item se uno dei fattori è uguale a~0 il prodotto è~0, cioè~\(0\cdot 
% m=m\cdot 0=0\).
% \end{enumeratea}

% \ovalbox{\risolvii \ref{ese:9.13}, \ref{ese:9.14}, \ref{ese:9.15}, 
% \ref{ese:9.16}}

\subsubsection{Potenza di un monomio}
\label{subsubsec:monomi_potenza}

% Ricordiamo che tra i numeri l'operazione di elevamento a
% potenza ha un solo termine, la base, sulla quale si agisce a seconda
% dell'esponente.
% 
% \[\text{Potenza }=\text{ base }^\text{ esponente}= 
% \underbrace{(\text{ base 
% }\cdot \text{ base }\cdot\text{ base }\cdot\ldots\cdot \text{ base 
% })}_{\text{tanti fattori quanti ne indica l'esponente}}.\]
% 
% Analogamente viene indicata la potenza di un monomio: la base è un
% monomio e l'esponente è un numero naturale.

\begin{definizione}{}{}
La \textbf{potenza di un monomio}\indc{monomio/monomi}{potenza di monomio}
è un monomio che ha per coefficiente la potenza del coefficiente e per 
parte letterale la potenza della parte letterale.
\end{definizione}

\begin{esempio}{}{}
Calcoliamo il quadrato e il cubo del 
monomio~\(m_{1}=-{\dfrac{1}{2}}a^{2}b\).

\vspace{-.5em}
\[\tonda{-{\frac{1}{2}}a^{2}b}^{2}
=\tonda{-{\frac{1}{2}}}^{2}\cdot\tonda{a^{2}}^{2}\cdot (b)^{2}
=\frac{1}{4}a^{4}b^{2}\]

\vspace{-.5em}
\[\tonda{-{\frac{1}{2}}a^{2}b}^{3}
=\tonda{-{\frac{1}{2}}}^{3}\cdot\tonda{a^{2}}^{3}\cdot (b)^{3}
=-{\frac{1}{8}}a^{6}b^{3}\]
% \begin{align*}
% \tonda{-{\frac{1}{2}}a^{2}b}^{2}
% &=\tonda{-{\frac{1}{2}}}^{2}\cdot\tonda{a^{2}}^{2}\cdot (b)^{2}
% =\frac{1}{4}a^{4}b^{2}
% \end{align*}
% \begin{align*}
% \tonda{-{\frac{1}{2}}a^{2}b}^{3}
% &=\tonda{-{\frac{1}{2}}}^{3}\cdot\tonda{a^{2}}^{3}\cdot (b)^{3}
% =-{\frac{1}{8}}a^{6}b^{3}
% \end{align*}
\end{esempio}

\begin{esempio}{}{}
Calcoliamo il quadrato e il cubo del monomio~\(m_{2}=5a^{3}b^{2}c^{2}\).

\vspace{-.5em}
\[\tonda{5a^{3}b^{2}c^{2}}^{2}
=\tonda{5}^{2}\cdot \tonda{a^{3}}^{2}\cdot\tonda{b^{2}}^{2}\cdot 
\tonda{c^{2}}^{2}
=25a^{6}b^{4}c^{4}\]

\vspace{-.5em}
\[\tonda{5a^{3}b^{2}c^{2}}^{3}
=\tonda{5}^{3}\cdot \tonda{a^{3}}^{3}\cdot\tonda{b^{2}}^{3}\cdot 
\tonda{c^{2}}^{3}
=125a^{9}b^{6}c^{6}\]
\end{esempio}

% \begin{procedura}{}{}
% Eseguire la potenza di un monomio:
% 
% \begin{enumeratea}
%  \item applichiamo la proprietà relativa alla potenza di un prodotto,
% eseguiamo cioè la potenza di ogni singolo fattore del monomio;
%  \item applichiamo la proprietà relativa alla potenza di potenza,
% moltiplicando l'esponente della variabile per l'esponente delle potenza.
% \end{enumeratea}
% \end{procedura}

% \ovalbox{\risolvii \ref{ese:9.17}, \ref{ese:9.18}, \ref{ese:9.19}, 
% \ref{ese:9.20}}

\subsubsection{Divisione tra due monomi}
\label{subsubsec:monomi_divisione}

Premessa: ricordiamo che il quoziente della divisione tra un dividendo 
\(n\) e un divisore \(d \neq 0\) è il numero \(q\) che moltiplicato per 
il divisore dà come risultato il dividendo: \\
\hspace*{39mm}\(n : d = q \sLRarrow q \cdot d = n\)\\
Con i numeri razionali la divisione si trasforma nella moltiplicazione 
tra il dividendo e il reciproco del divisore: \quad
\(n : d = n \cdot \dfrac{1}{d}\) \quad
se questo reciproco esiste.

Possiamo usare la stessa regola anche per i monomi se prima definiamo 
cosa è il reciproco di un monomio:

\begin{definizione}{}{}
Il \textbf{reciproco} di un monomio è il monomio che ha per coefficiente 
il reciproco del coefficiente e per parte letterale il reciproco della 
parte letterale\indc{monomio/monomi}{monomio reciproco}.
\end{definizione}

\begin{esempio}{}{}
 il reciproco di \(\dfrac{3}{4}a^3bc^2\) \quad è\\
\(\dfrac{4}{3}\dfrac{1}{a^3bc^2}\) \quad o anche \quad
\(\dfrac{4}{3a^3bc^2}\) \quad o anche \quad
\(\dfrac{4}{3}a^{-3}b^{-1}c^{-2}\)
\end{esempio}

Dobbiamo tenere presente che il reciproco di \(0\) non è definito, 
quindi possiamo calcolare il reciproco di un monomio solo se il suo 
coefficiente e ogni sua lettera sono \textbf{diversi da zero}.

Usando il reciproco, definire la divisione è facile.

\begin{definizione}{}{}
Il \textbf{quoziente}\indc{monomio/monomi}{monomio quoziente} tra due
monomi si ottiene moltiplicando il primo per il reciproco del secondo.
\end{definizione}

\begin{esempio}{}{}
Calcola: \(\tonda{+36x^{5}y^{2}z^2}:\tonda{-18x^{3}y}\)

\vspace{-.5em}
\[\tonda{+36x^{5}y^{2}z^2}:\tonda{-18x^{3}y} =
\tonda{+36x^{5}y^{2}z^2} \cdot \tonda{-\frac{1}{18x^{3}y}} =-2x^{2}yz^2\]

\vspace{-.5em}
Infatti~\((-2x^{2}yz^2)\cdot(-18x^{3}y)=(+36x^{5}y^{2}z^2)\).\\
Dato che la divisione è definita solo se il divisore è diverso da 0, il 
risultato ottenuto è valido solo se:
\(x \neq 0 \text{ e } y\neq 0\)
\end{esempio}
% % \end{exrig}
% 
% \begin{procedura}{}{}
% [Calcolare il quoziente di due monomi]
% Il quoziente di due monomi è così composto:
% 
% \begin{enumeratea}
%  \item il coefficiente è il quoziente dei coefficienti dei monomi dati;
%  \item la parte letterale ha gli esponenti ottenuti sottraendo gli 
% esponenti
% delle stesse variabili;
%  \item se la potenza di alcune lettere risulta negativa il risultato 
% della divisione non è un monomio.
% \end{enumeratea}
% \end{procedura}
% 
% % \begin{exrig}
\begin{esempio}{}{}
Calcola: 
\(\tonda{\dfrac{7}{2}a^{3}x^{4}y^{2}}:\tonda{-\dfrac{21}{8}ax^{2}y}\).
\[\tonda{\frac{7}{2}a^{3}x^{4}y^{2}}:\tonda{-\frac{21}{8}ax^{2}y} =
  \tonda{\frac{7}{2}a^{3}x^{4}y^{2}}\cdot \tonda{-\frac{8}{21ax^{2}y}} =
  -\frac{4}{3}a^{2}x^{2}y\]
con: \(a \neq 0 \text{ e } x \neq 0 \text{ e } y\neq 0\)
\end{esempio}

\begin{esempio}{}{}
Calcola: 
\(\tonda{\dfrac{9}{20}a^{2}b^{4}c}:\tonda{-{\dfrac{1}{8}}a^{5}b^{2}}\)

Si può eseguire la divisione solo se~\(a\neq~0\text{ e }b\neq~0\), 
in questo caso:
\[\tonda{\frac{9}{20}a^{2}b^{4}c}:\tonda{-{\frac{1}{8}}a^{5}b^{2}} =
\tonda{\frac{9}{20}a^{2}b^{4}c} \cdot \tonda{-\frac{8}{a^{5}b^{2}}} =
-\frac{18b^2c}{5a^3} = -\frac{18}{5} \frac{b^2c}{a^3}\]
con: 
\(a \neq 0 \text{ e } b \neq 0\)
\end{esempio}

% In conclusione, l'operazione di divisione tra due monomi
% ha come risultato un monomio se ogni variabile del dividendo ha
% esponente maggiore o uguale all'esponente con cui
% compare nel divisore.


% \vspazio\ovalbox{\risolvii \ref{ese:9.21}, \ref{ese:9.22}, 
% \ref{ese:9.23}}

\subsubsection{Addizione di due monomi}
\label{subsubsec:monomi_addizione}

Di solito quando si affrontano le operazioni, si inizia con l'addizione, 
invece con i monomi l'abbiamo tenuta per ultima perché c'è un problema: 
non sempre la somma di due monomi è un monomio. 
L'addizione di monomi non è una legge di composizione interna.

\paragraph{Addizione di monomi simili}
~

\begin{definizione}{}{}
La \textbf{somma algebrica di due monomi simili} è un monomio simile agli
addendi che ha come coefficiente la somma algebrica dei 
coefficienti\indc{monomio/monomi}{addizione fra monomi}.
\end{definizione}

\begin{esempio}{}{}
Calcoliamo~\(3x^{3}+(-7x^{3})\).

I due addendi sono monomi simili dunque la somma è ancora un monomio
ed è simile ai singoli addendi. Precisamente
\(3x^{3}+(-7x^{3})=(3+(-7))x^{3}=-4x^{3}\).

La somma di monomi simili si riduce alla somma algebrica di numeri.
\end{esempio}

% \ovalbox{\risolvi \ref{ese:9.24}}

% \subsubsection{Proprietà della addizione}
% 
% \begin{enumeratea}
%  \item commutativa:~\(m_{1}+m_{2}=m_{2}+m_{1}\)
%  \item 
% associativa:~\(m_{1}+m_{2}+m_{3}=(m_{1}+m_{2})+m_{3}=m_{1}+(m_{2}+m_{3})\)
%  \item 0 è l'elemento neutro:~\(0+m=m+0=m\)
%  \item per ogni monomio m esiste il monomio opposto, cioè un
%  monomio~\(m\Ast\) tale che
%  \[m + m\Ast = m\Ast +m=0.\]
% \end{enumeratea}
% 
% L'ultima proprietà enunciata ci permette di definire
% nell'insieme dei monomi simili anche la sottrazione di
% monomi. Essa si indica con lo stesso segno della sottrazione tra numeri
% e il suo risultato si chiama differenza.

% \begin{osservazione}{}{} 
% Per sottrarre due monomi simili si aggiunge al primo
% \end{osservazione}

% l'opposto del secondo.
% 
% % \begin{exrig}
% \begin{esempio}{}{}
% Assegnati~\(m_{1}=\frac{1}{2}a^{2}b\), \(m_{2}=-\text{5a}^{2}b\) 
% determina~\(m_{1} - m_{2}\).
% 
% L'operazione richiesta
% \(\frac{1}{2}a^{2}b-(-5a^{2}b)\) diventa
% \(\frac{1}{2}a^{2}b+5a^{2}b=\frac{11}{2}a^{2}b\).
% \end{esempio}
% % \end{exrig}
% 
% Sulla base di quanto detto, possiamo unificare le due operazioni di
% addizione e sottrazione di monomi simili in un'unica
% operazione che chiamiamo \textbf{somma algebrica di monomi}.
% 
% 
% \begin{osservazione}{}{} 
% La somma algebrica di due monomi simili è un monomio 
% simile 
% % agli addendi avente per coefficiente la somma algebrica dei 
% % coefficienti.
% \end{osservazione}

\begin{esempio}{}{}
Determiniamo la somma 
\quad 
\(\dfrac{3}{5}x^{4}-\dfrac{1}{3}x^{4}+x^{4}+
  \dfrac{4}{5}x^{4}-2x^{4}-\dfrac{1}{2}x^{4}\)

\vspace{.5em}
Osserviamo che tutti gli addendi sono tra loro simili dunque:

\vspace{-1.5em}
\[\frac{3}{5}x^{4}-\frac{1}{3}x^{4}+x^{4}+
  \frac{4}{5}x^{4}-2x^{4}-\frac{1}{2}x^{4}=
  \tonda{\frac{3}{5}-\frac{1}{3}+1+\frac{4}{5}-2-\frac{1}{2}}x^{4}=
  -{\frac{13}{30}}x^{4}\]
\end{esempio}

\paragraph{Addizione di monomi non simili}
~

Analizziamo il caso dell'addizione:\\
\(7a^{3}b^{2}-5a^{2}b^{3}+a^{3}b^{2}\), si vuole determinarne la somma. 
I monomi addendi non sono tutti tra loro simili; 
lo sono però il primo e il terzo. 
Le proprietà commutativa e associativa ci consentono di riscrivere
l'addizione precedente ``avvicinando'' i monomi simili e
sostituendo ad essi la loro somma:
\[7a^{3}b^{2}-5a^{2}b^{3}+a^{3}b^{2}=(7a^{3}b^{2}+a^{3}b^{2})-5a^{2}b^{3}
=8a^{3}b^{2}-5a^{2}b^{3}.\]
L'espressione così ottenuta è la somma richiesta, e non è un monomio.
% \vspazio\ovalbox{\risolvi \ref{ese:9.25}}\vspazio
Il procedimento che abbiamo seguito per determinare il risultato
dell'addizione assegnata viene chiamato 
\emph{riduzione dei termini simili}
\indc{monomio/monomi}{riduzione termini simili}.

L'operazione di addizione tra monomi ha come risultato 
un \emph{monomio} se tutti gli addendi sono monomi simili altrimenti 
il risultato è un \emph{polinomio}\indc{polinomio/polinomi}{polinomio}.

\begin{esempio}{}{}
Calcola la seguente somma:\quad \(3a-7a+2a+a\).

Il risultato è un monomio poiché gli addendi sono tutti monomi
simili: \(-a\).
\end{esempio}

\begin{esempio}{}{}
Calcola la seguente somma:\quad 
\(\dfrac{1}{2}a^{3}+b-\dfrac{3}{4}a^{3}-\dfrac{6}{5}b\).

Gli addendi non sono tutti simili, ottengo un polinomio: \quad
\(-{\dfrac{1}{4}}a^{3}-\dfrac{1}{5}b\).
\end{esempio}

% \ovalbox{\risolvii \ref{ese:9.26}, \ref{ese:9.27}, \ref{ese:9.28}, 
% \ref{ese:9.29}, \ref{ese:9.30}, \ref{ese:9.31}, \ref{ese:9.32}}

\subsection{Espressioni con i monomi}
\label{subsec:monomi_espressioni}

\noindent Consideriamo l'espressione letterale: \\ 
\(\qquad E(a,~b)=\tonda{-{\dfrac{1}{2}}a^{2}b}^{3}:\tonda{a^{5}b}+
  \tonda{-2ab}\cdot\tonda{\dfrac{1}{2}b+b}+5ab^{2}\)

Vediamo che è in due variabili, le variabili sono infatti~\(a\) e~\(b\). 
Inoltre, i termini delle operazioni che vi compaiono sono monomi.

Se volessimo calcolare il valore di~\(E\) per~\(a = 10,~~b = -2\), 
dovremmo sostituire nell'espressione tali valori e risolvere
l'espressione numerica che ne risulta. 
Inoltre se dovessimo calcolare il valore di~\(E\) per altre coppie di 
valori dovremmo ogni volta applicare questo procedimento.

Dal momento che abbiamo studiato come eseguire le operazioni razionali
con i monomi, prima di sostituire i numeri alle lettere, applichiamo le
regole del calcolo letterale in modo da ridurre~\(E\), se possibile, 
in una espressione più semplice.

Prima di procedere, essendovi una divisione poniamo innanzi tutto
la~\(\CE a \neq~0\) e~\(b \neq~0\) e eseguiamo rispettando la precedenza
delle operazioni come facciamo nelle espressioni numeriche.
% \newpage
\begin{esempio}{}{}
 \begin{align*}
 &\tonda{-{\frac{1}{2}}a^{2}b}^{3}:(a^{5}b)+
  (-2ab)\cdot\tonda{\frac{1}{2}b+b}+5ab^{2} =
&& \text{eseguiamo il cubo}\\
=&\tonda{-{\frac{1}{8}}a^{6}b^{3}:a^{5}b}+
  (-2ab)\cdot{\frac{3}{2}}b+5ab^{2} =
&& \text{eseguiamo div. e molt.}\\
=&-{\frac{1}{8}}ab^{2}-3ab^{2}+5ab^{2} 
= \frac{15}{8}ab^{2} \quad \stext{con} a, b \ne 0 
&& \text{sommiamo i mon. simili}
\end{align*}
Ora è più semplice calcolarne il valore: per~\(a=10\) e~\(b=-2\) si ha:\\
\(E(10, -2) =
  \dfrac{15}{8}\cdot 10\cdot(-2)^{2}=\dfrac{15}{8}\cdot 10\cdot 4=75\).
\end{esempio}

\begin{esempio}{}{}
 \begin{align*}
 &\tonda{\frac{2}{3}ab^{2}c}^{2}:\tonda{-3ab^{3}}-\frac{2}{9}abc^{2} =
&& \text{Sviluppiamo le potenze}\\
 =&\frac{4}{9}a^{2}b^{4}c^{2}:\tonda{-3ab^{3}}-\frac{2}{9}abc^{2} =
&& \text{eseguiamo div. e molt.}\\
 =&-{\frac{4}{27}}abc^{2}-\frac{2}{9}abc^{2} =
&& \text{sommiamo i monomi simili}\\
=&\frac{-4-6}{27}abc^{2} 
 =-{\frac{10}{27}}abc^{2} \quad \stext{con} a, b \ne 0 && 
  \text{e otteniamo il risultato}
 \end{align*}
\end{esempio}

\begin{esempio}{}{}
\(\quadra{\tonda{-{\dfrac{14}{16}}x^{2}y^{2}}:\tonda{-{\dfrac{14}{4}}
xy}}^{3}+\dfrac{1}{2}xy\cdot{\dfrac{1}{4}}x^{2}y^{2}\).

Trasformiamo la divisione in moltiplicazione, poi
\begin{align*}
 =&\quadra{+{\frac{14}{16}}\cdot{\frac{4}{14}}xy}^{3}+\frac{1}{2}xy\cdot 
{\frac{1}{4}}x^{2}y^{2}= && \text{calcoliamo i prodotti}\\
=&\quadra{\frac{1}{4}xy}^{3}+\frac{1}{8}x^{3}y^{3} =
&& \text{sviluppiamo il cubo}\\
=&\frac{1}{64}x^{3}y^{3}+\frac{1}{8}x^{3}y^{3}= && 
\text{sommiamo i monomi simili}\\
=&\frac{1+8}{64}x^{3}y^{3} 
=\frac{9}{64}x^{3}y^{3} \quad \stext{con} x, y \ne 0 && 
  \text{e otteniamo il risultato}
\end{align*}
\end{esempio}

% \ovalbox{\risolvii \ref{ese:9.33}, \ref{ese:9.34}, \ref{ese:9.35}, 
% \ref{ese:9.36}, \ref{ese:9.37}, \ref{ese:9.38}, \ref{ese:9.39}, 
% \ref{ese:9.40}, \ref{ese:9.41}, \ref{ese:9.42}}

\subsection{Massimo Comune Divisore e minimo comune multiplo tra monomi}
\label{subsec:monomi_mcdemcm}

Attenzione: in questa sezione considereremo solo monomi interi.

\subsubsection{Massimo Comune Divisore}
\indc{monomio/monomi}{MCD di monomi}

\begin{definizione}{}{}
Si dice che il monomio \(B\) è \textbf{divisore} del monomio \(A\) se 
\(B\) non ha lettere diverse da quelle presenti in \(A\) e ogni 
lettera di \(B\) è elevata a un esponente minore o uguale a quello della 
lettera corrispondente in \(A\). 
\end{definizione}

\begin{esempio}{}{}
Quale dei seguenti è un divisore di: \({8a^{2}b^{3}c}\): 

\begin{enumerate*}
\item \({7b^{3}c}\); \qquad~
\item \({8a^{2}b^{4}c}\); \qquad~
\item \({a^{2}b^{3}c^{2}}\); \qquad~
\item \({4a^{2}b^{3}x}\).
\end{enumerate*}

Il primo monomio ha tutte le caratteristiche richieste.
Il secondo ha l'esponente della lettera \(b\) troppo grande.
Il terzo ha l'esponente della lettera \(c\) troppo grande.
Il quarto ha una lettera, \(x\) non presente nel monomio dell'esercizio. 
\end{esempio}

\begin{definizione}{}{}
 Il \textbf{massimo comune divisore} tra due o più monomi è il
monomio che, tra tutti i divisori comuni dei monomi dati, ha grado
massimo.

Si ottiene moltiplicando tra di loro solo i \emph{fattori comuni} presi con 
il \emph{grado minimo}.
\end{definizione}

Il \indtc{monomio/monomi}{coefficiente} numerico può essere un qualunque 
numero reale: se i coefficienti sono tutti interi è opportuno scegliere 
il loro~\(\mcd\), se non sono interi è opportuno scegliere~1.

% \begin{esempio}{}{}
% Calcola il~\(\mcd\) dei monomi \quad 
% \(12a^{3}b^{2}c\) \quad e \quad \(16a^{2}bx\) 
% 
% I fattori comuni sono: \quad 
% \(4;\quad a;\quad b\)
% 
% Prendendo questi fattori con il grado minimo presente nei monomi 
% si ottiene:\quad \(4a^{2}b\).
% \end{esempio}

% \begin{procedura}{}{}
% Calcolare il~\(\mcd\) tra monomi
% 
% Il~\(\mcd\) di un gruppo di monomi è il monomio che ha:
% 
% \begin{enumeratea}
%  \item per coefficiente numerico il~\(\mcd\) dei valori assoluti dei
% coefficienti dei monomi qualora
% questi siano numeri interi, se non sono interi si prende~1;
%  \item la parte letterale formata da tutte le lettere comuni ai monomi
% dati, ciascuna presa una sola volta e con l'esponente minore con cui 
% compare.
% \end{enumeratea}
% \end{procedura}

\begin{esempio}{}{}
Calcolare~\(\mcd \terna{14a^{3}b^{4}c^{2}}{4ab^{2}}{8a^{2}b^{3}c}\) 

I fattori comuni sono: \quad 
\(2; \quad a; \quad b\)

Prendendo questi fattori con il grado minimo presente nei monomi 
si ottiene:
\[\mcd \terna{14a^{3}b^{4}c^{2}}{4ab^{2}}{8a^{2}b^{3}c}=2ab^{2}\]
\end{esempio}

\begin{esempio}{}{}
Calcolare il massimo comune divisore tra: 

\(\quad 5x^{3}y^{2}z^{3}; \quad -\dfrac{1}{8}xy^{2}z^{2}; 
  \quad 7x^{3}z^{2}\)

Si osservi che i coefficienti numerici dei monomi non sono numeri interi.
Scegliamo~1 come coefficiente del~\(\mcd\).
Le lettere in comune sono~\(xz\), prese ciascuna con
l'esponente minore con cui compaiono si ha:
\[\mcd \terna{5x^{3}y^{2}z^{3}}{-\frac{1}{8}xy^{2}z^{2}}{7x^{3}z^{2}} = 
  xz^{2}\]
\end{esempio}


\begin{osservazione}{}{} 
La scelta di porre uguale a~1 il coefficiente 
numerico del~\(\mcd\), nel
caso in cui i monomi abbiano coefficienti razionali, è dovuta al
fatto che una qualsiasi frazione divide tutte le altre e quindi una
qualsiasi frazione potrebbe essere il coefficiente del~\(\mcd\).
Ad essere più precisi, dovremmo dichiarare a quale degli insiemi 
numerici appartengono i coefficienti. 
Qui consideriamo coefficienti che appartengono all'insieme numerico 
più ampio che conosciamo.
\end{osservazione}


\begin{definizione}{}{}
Due monomi si dicono \textbf{primi tra loro}
\indc{monomio/monomi}{monomi primi tra loro} se il loro~\(\mcd\) è~1.
\end{definizione}

\subsubsection{Minimo comune multiplo}
\indc{monomio/monomi}{mcm di monomi}

\begin{definizione}{}{}
Si dice che il monomio \(B\) è \textbf{multiplo} del monomio \(A\) se in 
\(B\) sono presenti tutte le lettere presenti in \(A\) e ogni 
lettera di \(B\) è elevata a un esponente maggiore o uguale a quello della 
lettera corrispondente in \(A\). 
\end{definizione}

\begin{esempio}{}{}
Quale dei seguenti non è un multiplo di: \({8a^{2}b^{3}c}\): 

\begin{enumerate*}
\item \({7a^{2}b^{3}c}\); \qquad~
\item \({3a^{2}b^{4}cz^{3}}\); \qquad~
\item \({a^{2}b^{2}c}\); \qquad~
\item \({4a^{2}b^{3}c^{2}x}\).
\end{enumerate*}

Il terzo non è un multiplo perché l'esponente della lettera \(b\) è 
troppo piccolo. 
\end{esempio}

% \begin{definizione}{}{}
% Un monomio~\(A\) si dice \textbf{multiplo} di un monomio~\(B\) se esiste
% un monomio~\(C\) per il quale~\(A=B\cdot C\). 
% In questo caso diremo anche che~\(B\) è \textbf{divisore} del monomio~\(A\).
% \end{definizione}

\begin{definizione}{}{}
Il \textbf{minimo comune multiplo di due o più monomi}
è il monomio che, tra tutti i monomi multipli comuni,
ha il grado minore.

Si ottiene moltiplicando i fattori comuni e non comuni presi con il massimo 
esponente.
\end{definizione}

Il \indtc{monomio/monomi}{coefficiente} numerico può essere un qualunque 
numero reale: se i coefficienti sono tutti interi è opportuno scegliere 
il loro~\(\mcm\), se non lo sono è opportuno scegliere~1.

% \begin{esempio}{}{}
% Calcola il minimo comune multiplo tra~\(5a^{3}bc\) e~\(10a^{2}b^{2}\) 
% 
% I fattori primi comuni e non comuni sono: \quad 
% \(2;~5;~a;~b;~c\)
% 
% Moltiplicando tutti questi fattori presi con il massimo esponente 
% otteniamo:
% \[\mcm\tonda{5a^{3}bc;~10a^{2}b^{2}} = 10a^{3}b^{2}c\]
% \end{esempio}
% 
% In realtà applicando la definizione è poco pratico calcolare il~\(\mcm\), 
% è utile invece la seguente procedura.
% 
% \begin{procedura}{}{}
% [Calcolo del~\(\mcm\) tra due o più monomi]
% Il~\(\mcm\) di un gruppo di monomi è il monomio che ha:
% 
% \begin{enumeratea}
% \item per coefficiente numerico il~\(\mcm\) dei valori assoluti dei
% coefficienti dei monomi qualora
% questi siano numeri interi, se non sono interi si prende~1;
% \item la parte letterale formata da tutte le lettere comuni e non comuni
% ai monomi dati, ciascuna
% presa una sola volta e con l'esponente maggiore con
% cui compare.
% \end{enumeratea}
% \end{procedura}
% 
% % \begin{exrig}
\begin{esempio}{}{}
Calcola \(\mcm \tonda{5a^{3}bc;~12ab^{2}c;~10a^{3}bc^{2}}\)

I fattori primi comuni e non comuni sono: \quad 
\(2;~3;~5;~a;~b;~c\)

Moltiplicandoli con il massimo esponente si ottiene:
\[\mcm \tonda{5a^{3}bc;~12ab^{2}c;~10a^{3}bc^{2}} = 60a^{3}b^{2}c^{2}\]
\end{esempio}

\begin{esempio}{}{}
Calcola 
\(\mcm \tonda{6x^{2}y;~-\dfrac{1}{2}xy^{2}z;~\dfrac{2}{3}x^{3}yz}\)

Almeno uno dei coefficienti è una frazione quindi non consideriamo i 
coefficienti.
Moltiplicando le singole lettere con il massimo esponente si ottiene:
\[\mcm \tonda{6x^{2}y;~-\frac{1}{2}xy^{2}z;~\frac{2}{3}x^{3}yz} = 
x^{3}y^{2}z\]
\end{esempio}

Assegnati due monomi, per esempio~\(x^{2}y\) e~\(xy^{2}z\),
calcoliamo~\(\mcd\) e \(\mcm\).\\
\hspace*{\fill}\(\mcd (x^{2}y;xy^{2}z)=xy\)\quad e \quad 
\(\mcm (x^{2}y;xy^{2}z)=x^{2}y^{2}z\).\hspace*{\fill}\\
Moltiplichiamo ora~\(\mcd\) e~\(\mcm\), abbiamo: \quad 
\(xy\cdot x^{2}y^{2}z = x^{3}y^{3}z.\)\\
Moltiplichiamo ora i monomi assegnati, abbiamo: \quad 
\((x^{2}y) \cdot (xy^{2}z)=x^{3}y^{3}z\).

Il prodotto\indc{monomio/monomi}{prodotto di monomi} dei due monomi è 
uguale al prodotto tra il~\(\mcd\) e il~\(\mcm\). 
Si può dimostrare che questa proprietà vale in generale.

\begin{teorema}{}{}
Dati due monomi, il prodotto tra il loro massimo comune
divisore e il loro minimo comune multiplo è uguale al prodotto tra i
monomi stessi.
\end{teorema}

% \ovalbox{\risolvii \ref{ese:9.41}, \ref{ese:9.42}, \ref{ese:9.43}, 
% \ref{ese:9.44}, \ref{ese:9.45}, \ref{ese:9.46}, \ref{ese:9.47}}

\section{Polinomi $\Pol$}
\label{subsec:poli_polinomi}

\subsection{Definizioni fondamentali}
\label{subsec:poli_definizioni}

\begin{definizione}{}{}
Un polinomio\indc{polinomio/polinomi}{polinomio} è un'\indt{espressione 
letterale} che consiste in una somma algebrica di monomi interi.
\end{definizione}

\begin{osservazione}{}{}
Se in un polinomio non tutti i monomi sono interi, otteniamo un 
\emph{polinomio fratto}\indc{polinomio/polinomi}{polinomio fratto}, 
cioè formato da \emph{frazioni algebriche}.
Per studiare questi oggetti, abbiamo bisogno di alcuni strumenti 
matematici che ancora non sappiamo usare. 
Dovete portare pazienza: li studieremo dopo aver imparato a scomporre 
in fattori i polinomi.
\end{osservazione}


\begin{esempio}{}{}
Sono polinomi:
\(6a+2b\),\quad \(5a^2b+3b^2\),\quad \(6x^2-5y^2x-1\),\quad 
\(7ab-2a^2b^3+4\).
\end{esempio}

Se tra i termini di un polinomio non sono presenti monomi simili, il 
polinomio si dice in \emph{forma normale} o
\emph{ridotto}; se al contrario si presentano dei termini simili, 
possiamo eseguire la riduzione del polinomio sommando i termini simili. 
Tutti i polinomi sono quindi riducibili 
\indtc{polinomio/polinomi}{in forma normale}.

Un polinomio in forma normale può presentare tra i suoi termini un 
monomio di grado~0 che viene comunemente chiamato 
\emph{termine noto}\indc{polinomio/polinomi}{termine noto}.

\begin{esempio}{}{}
Il polinomio~\(3ab+b^2-2ba+4-6ab^2+5b^2\) ridotto in forma normale 
diventa~\(ab+6b^2-6ab^2+4\). Il termine noto è~\(4\).
\end{esempio}

% \ovalbox{\risolvi\ref{ese:10.1}}\vspazio

Un polinomio può anche essere costituito da un unico termine, pertanto 
un monomio è anche un polinomio.
Un polinomio che, ridotto in forma normale, è somma algebrica di due, 
tre, quattro monomi non nulli si dice rispettivamente binomio, trinomio, 
quadrinomio.

\begin{esempio}{}{}
Binomi, trinomi, quadrinomi.
\begin{enumeratea}
\item \(xy-5x^3y^2\) \quad è un binomio;
\item \(3ab^2 +a-4a^3\) \quad è un trinomio;
\item \(a-6ab^2+3ab-5b\) \quad è un quadrinomio.
\end{enumeratea}
\end{esempio}

\begin{definizione}{}{}
Due polinomi, ridotti in forma normale, formati da termini uguali si dicono 
\textbf{uguali}\indc{polinomio/polinomi}{polinomi uguali, opposti, nulli}.

% Due polinomi, ridotti in forma normale, formati da termini uguali si 
% dicono 
% \emph{uguali}, più precisamente vale il \emph{principio di identità dei 
% polinomi}:
% due polinomi~\(p(x)\) e~\(q(x)\) sono uguali se, e solo se, sono uguali
% i coefficienti dei termini simili.

Se due polinomi sono formati da termini opposti, allora si dicono 
\textbf{polinomi opposti}.

Definiamo, inoltre, un \textbf{polinomio nullo} se i suoi termini sono
a coefficienti nulli. 
Il polinomio nullo coincide con il monomio nullo e quindi con il numero~0.
\end{definizione}

\begin{esempio}{}{}
Polinomi uguali, opposti, nulli.
\begin{enumeratea}
\item I polinomi
\(\quad \frac{1}{3}xy+2y^3-x\) \(\quad~2y^3-x+\frac{1}{3}xy \quad\) 
sono uguali;
\item i polinomi~\(\quad~6ab-3a+2b\) \(\quad~3a-2b-6ab \quad\) 
sono opposti;
\item il polinomio~\(\quad~7ab+4a^2-ab+b^3-4a^2-2b^3-6ab+b^3 \quad\) è un 
polinomio nullo, infatti riducendolo in forma normale otteniamo il 
monomio nullo~\(0\).
\end{enumeratea}
\end{esempio}

\begin{definizione}{}{}
Il \textbf{grado complessivo} (o semplicemente
\textbf{grado}\indc{polinomio/polinomi}{grado del polinomio}) di un
polinomio è il \textbf{massimo} dei gradi complessivi dei suoi termini.
Si chiama, invece, \textbf{grado di un polinomio rispetto a una
data lettera} l'esponente maggiore con cui quella lettera compare nel 
polinomio, dopo che è stato ridotto a forma normale.
\end{definizione}

\begin{esempio}{}{} Grado di un polinomio.
\begin{itemize} [nosep]
\item Il polinomio~\(2ab+3-4a^2b^2\) ha grado complessivo~\(4\) perché il 
monomio con grado massimo è~\(-4a^2b^2 \), che è un monomio di quarto 
grado;
\item il grado del polinomio~\(a^3+3b^2a-4b^5a^2\) rispetto alla 
lettera~\(a\) è~\(3\) perché l'esponente più grande con cui tale lettera 
compare è~\(3\).
\end{itemize}
\end{esempio}

% \ovalbox{\risolvi\ref{ese:10.2}}

\begin{definizione}{}{}
Un polinomio si dice \textbf{omogeneo}\indc{polinomio/polinomi}{polinomio
omogeneo} se tutti i termini che lo compongono sono dello stesso grado.
\end{definizione}

\begin{esempio}{}{}
Il polinomio~\(a^3-b^3+ab^2\) è un polinomio omogeneo di grado~\(3\).
\end{esempio}

% \ovalbox{\risolvi\ref{ese:10.3}}

\begin{definizione}{}{}
Un polinomio si dice \textbf{ordinato secondo le potenze decrescenti
(crescenti) di una lettera}\indc{polinomio/polinomi}{polinomio ordinato}, 
se i suoi termini sono ordinati in maniera tale che gli esponenti di tale 
lettera decrescono (crescono), leggendo il polinomio da sinistra verso 
destra.
\end{definizione}

L'essere ordinato non è una caratteristica del polinomio, ma di come 
viene rappresentato.

\begin{esempio}{}{}
Il polinomio~\(\dfrac{1}{2}x^3+\dfrac{3}{4}x^2y-2xy^2+\dfrac{3}{8}y^3\) 
è ordinato secondo le potenze decrescenti della lettera~\(x\), e secondo 
le potenze crescenti della lettera~\(y\).\\
\(\dfrac{3}{8}y^3+\dfrac{1}{2}x^3+\dfrac{3}{4}x^2y-2xy^2\) è lo stesso
polinomio ma non ordinato
\end{esempio}

\begin{definizione}{}{}
Un polinomio di grado~\(n\) rispetto a una data lettera si dice 
\textbf{completo}\indc{polinomio/polinomi}{polinomio completo} se contiene
anche tutte le potenze di tale lettera di grado 
inferiore a~\(n\), compreso il termine di grado zero.
\end{definizione}

\begin{esempio}{}{}
Il polinomio~\(x^4-3x^3+5x^2+6x-7\) è completo di 
grado~\(4\) e inoltre risulta ordinato rispetto alla lettera~\(x\). Il 
termine di grado zero è~\(-7\).
% Il polinomio~\(x^4-3x^3+5x^2+\dfrac{1}{2}x-\dfrac{3}{5}\) è completo di 
% grado~\(4\) e inoltre risulta ordinato rispetto alla lettera~\(x\). Il 
% termine 
% noto è~\(-\dfrac{3}{5}\).
\end{esempio}

\begin{osservazione}{}{}
Ogni polinomio può essere scritto sotto forma ordinata e completa: 
l'ordinamento si può effettuare per la proprietà commutativa della 
somma, 
la completezza si può ottenere aggiungendo termini 
con coefficiente uguale a~\(0\).

Per esempio, il polinomio~\(x^4-x+1+4x^2\) può essere scritto sotto forma 
ordinata e completa come~\(x^4+0x^3+4x^2-x+1\).
\end{osservazione}

% \vspazio\ovalbox{\risolvii 
% \ref{ese:10.4},\ref{ese:10.5},\ref{ese:10.6},\ref{ese:10.7},
% \ref{ese:10.8},
% \ref{ese:10.9},\ref{ese:10.10}}

\subsection{Somma algebrica di polinomi}
\label{subsec:poli_somma}

I polinomi sono somme algebriche di monomi e quindi le espressioni 
letterali che si ottengono dalla somma
o differenza di polinomi sono ancora somme algebriche di monomi.

L'addizione è una legge di composizione interna all'insieme dei polinomi.

\begin{definizione}{}{}
La \textbf{somma di due o più polinomi}\indc{polinomio/polinomi}{somma di
polinomi} è un polinomio avente per termini tutti i termini dei polinomi 
addendi.
\end{definizione}

La differenza di polinomi si può trasformare nella somma del primo 
con l'opposto del secondo.

\begin{esempio}{}{}
Differenza di polinomi.
\begin{equation*}
\begin{split}
&\tonda{3a^2+2b - \frac{1}{2}ab} - \tonda{2a^2+ab - \frac{1}{2}b} =
3a^2+2b - \frac{1}{2}ab - 2a^2 - ab + \frac{1}{2}b = \\
&=a^2 + \frac{-1-2}{2}ab + \frac{4+1}{2}b = 
a^2 - \frac{3}{2}ab + \frac{5}{2}b
\end{split}
\end{equation*}
\end{esempio}

% \ovalbox{\risolvii \ref{ese:10.11}, \ref{ese:10.12}, \ref{ese:10.13}}

\subsection{Prodotto di un polinomio per un monomio}
\label{subsec:poli_prodottopermonomio}

Per eseguire il prodotto tra un monomio e un polinomio si applica la 
proprietà distributiva: \quad 
\( a \cdot \tonda{b + c} = a \cdot b + a \cdot c\)

\begin{definizione}{}{} 
Il \textbf{prodotto di un monomio per un polinomio} si ottiene moltiplicando
il monomio per ciascun termine del polinomio\indc{polinomio/polinomi}
{prodotto per un monomio}.
% Il prodotto di un monomio per un polinomio è
% un polinomio avente come termini i prodotti del monomio per ciascun
% termine del polinomio.
\end{definizione}


% \newpage
\begin{esempio}{}{}
 Prodotto di un monomio per un polinomio.\\ [.5em]
% \(\tonda{3x^{2}y} \cdot \tonda{2{xy}-5x^{3}y^{2}}=
% \tonda{3x^{2}y} \cdot \tonda{2{xy}} +
% \tonda{3x^{2}y} \cdot \tonda{-5x^{3}y^{2}} =
% 6x^{3}y^{2}-15x^{5}y^{3}\)\\[.5em]
\(
\tonda{3x^{3}y}\cdot\tonda{\dfrac{1}{2}x^{2}y^{2}+\dfrac{4}{3}{xy}^{3}}
=\tonda{3x^{3}y}\cdot\tonda{\dfrac{1}{2}x^{2}y^{2}}
+\tonda{3x^{3}y}\cdot\tonda{\dfrac{4}{3}{xy}^{3}} = \newline
=\dfrac{3}{2}x^{5}y^{3}+4x^{4}y^{4}
\)
\end{esempio}

% \ovalbox{\risolvii \ref{ese:10.14}, \ref{ese:10.15}}

\subsection{Quoziente tra un polinomio e un monomio}
\label{subsec:poli_quozientepermonomio}

Il quoziente\indc{polinomio/polinomi}{divisione per un monomio} tra un 
polinomio e un monomio si calcola applicando la
proprietà distributiva della divisione rispetto
all'addizione.

\begin{definizione}{}{}
Si dice che un \textbf{polinomio è divisibile per un monomio non
nullo} se esiste un polinomio che, moltiplicato per il monomio, dà
come risultato il polinomio dividendo; il monomio si dice
\textbf{divisore} del polinomio.
\end{definizione}

\begin{definizione}{}{}
Il \textbf{quoziente tra un polinomio e un monomio suo divisore} è un
polinomio ottenuto dividendo ogni termine del polinomio per il monomio
divisore.
\end{definizione}

\begin{esempio}{}{}
Quoziente tra un polinomio e un monomio.\\ [.5em]
\(\tonda{6x^{5}y+9x^{3}y^{2}}:\tonda{3x^{2}y} = 
2x^{(5-2)}y^{(1-1)}+3x^{(3-2)}y^{(2-1)} = 2x^{3}+3{xy}\)
\end{esempio}

\begin{osservazione}{}{}
\begin{enumerate}[nosep, label=\alph*)]
\item Poiché ogni monomio è divisibile per qualsiasi numero diverso
da zero, allora anche ogni polinomio è divisibile per un qualsiasi
numero diverso da zero;
\item un polinomio è divisibile per un monomio, non nullo, se ogni
fattore letterale del monomio compare in ogni monomio del polinomio
con grado uguale o maggiore;
\item la divisione tra un polinomio e un qualsiasi monomio non nullo è
sempre possibile, tuttavia il risultato è un polinomio solo nel caso
in cui il monomio sia divisore di tutti i termini del polinomio.
\end{enumerate}
\end{osservazione}

\begin{esempio}{}{}
Quoziente tra un polinomio e un monomio.
\begin{align*}
\tonda{8a^{4}b-14a^{3}b^{2}}:\tonda{2a^{4}b^3} &= 
4a^{(4-4)}b^{(1-3)}-7a^{(3-4)}b^{(2-3)} = \\
& = 4b^{-2}-7a^{-1}b^{-1} = \frac{4}{b^2}-\frac{7}{ab}
\end{align*}
% \[\tonda{8a^{4}b-14a^{3}b^{2}}:\tonda{2a^{4}b^3} = 
% 4a^{(4-4)}b^{(1-3)}-7a^{(3-4)}b^{(2-3)} = 4b^{-2}-7a^{-1}b^{-1} =
% \frac{4}{b^2}-\frac{7}{ab}\]
Il risultato non è un polinomio, ma un 
\indtc{polinomio/polinomi}{polinomio fratto} o frazione algebrica.
\end{esempio}

% \ovalbox{\risolvii \ref{ese:10.16}, \ref{ese:10.17}, \ref{ese:10.18}}

\subsection{Prodotto di polinomi}
\label{subsec:poli_prodotto}

\begin{definizione}{}{}
Il \textbf{prodotto di due polinomi} è il
polinomio\indc{polinomio/polinomi}{prodotto tra due polinomi} che si 
ottiene moltiplicando ogni termine del primo polinomio per ciascun 
termine del secondo polinomio 
(e poi riducendo i termini simili).
\end{definizione}

% \newpage
\begin{esempio}{}{}
 Esegui i seguenti prodotti riducendo i termini simili.

\begin{enumerate} [left=0mm]
\item 
\(\tonda{a^{2}b+3a-4{ab}}\tonda{\dfrac{1}{2}a^{2}b^{2}-a+3{ab}^{2}}=\)

\(=\dfrac{1}{2}a^{4}b^{3}-a^{3}b+\underline{3a^{3}b^{3}}+
   \dfrac{3}{2}a^{3}b^{2}
-3a^{2}+9a^{2}b^{2} -\underline{2a^{3}b^{3}}+4a^{2}b-12a^{2}b^{3}=\newline
=\dfrac{1}{2}a^{4}b^{3}-a^{3}b+a^{3}b^{3}+
 \dfrac{3}{2}a^{3}b^{2}-3a^{2}+9a^{2}b^{2}+4a^{2}b-12a^{2}b^{3}\)
\item 
\(\tonda{x-y^{2}-3{xy}}\tonda{-2x^{2}y-3y}=-2x^{3}y+3{xy}+2x^{2}y^
{3}-3y^{3}+6x^{3}y^{2}+9{xy}^{2}\)
\item 
\(\tonda{\dfrac{1}{2}x^{3}-2x^{2}}\tonda{\dfrac{3}{4}x+1}=
\dfrac{3}{8}x^{4}+\underline{\dfrac{1}{2}x^{3}}-
 \underline{\dfrac{3}{2}x^{3}}-2x^{2}
=\dfrac{3}{8}x^{4}-x^{3}-2x^{2}\)
\end{enumerate}
\end{esempio}
% \ovalbox{\risolvi \ref{ese:10.19}}

\subsection{Quoziente di polinomi}
\label{subsec:poli_quoziente}

\begin{definizione}{}{}
Il \textbf{quoziente di due polinomi}
\indc{polinomio/polinomi}{quoziente tra due polinomi}, se esiste, è il 
polinomio che moltiplicato per il \textbf{divisore} dà il \textbf{dividendo}.
\end{definizione}

\begin{osservazione}{}{}
Il \emph{quoziente di due polinomi} non sempre è un polinomio.

Esiste una \emph{divisione con resto} di polinomi mentre, 
per poter avere sempre il quoziente esatto di due polinomi, 
dobbiamo estendere l'insieme \(\Pol\) ai 
polinomi fratti o frazioni algebriche.
\end{osservazione}

Affronteremo il problema della divisione tra polinomi dopo aver 
affrontato la scomposizione in fattori di polinomi.


\subsection{Proprietà delle operazioni con i polinomi}
\label{subsec:poli_proprieta}

\begin{osservazione}{}{}
Rispetto alle operazioni aritmetiche, i polinomi si comportano in modo 
analogo ai numeri interi. 
\end{osservazione}

Poiché nelle operazioni con i polinomi (\(\Pol\)) valgono le stesse 
proprietà che valgono nei numeri interi, anche la struttura formata dai 
polinomi, dall'addizione e dalla moltiplicazione, 
\(\terna{\Pol}{+}{\times}\), viene chiamata anello.

\section{Prodotti notevoli}
\label{sec:poli_prodnot}

Con l'espressione \indtc{polinomio/polinomi}{prodotti notevoli} si 
indicano alcune identità che si ottengono in seguito alla moltiplicazione 
di polinomi aventi caratteristiche particolari facili da ricordare.
Per cui è più utile imparare a memoria il prodotto che calcolarlo 
eseguendo la moltiplicazione.

\subsection{Quadrato di un binomio}
\indc{polinomio/polinomi}{quadrato di un binomio}
\label{subsec:prodnot_quadratobinomio}

Consideriamo il binomio~\(A+B\) in cui~\(A\) e~\(B\) rappresentano due 
monomi ed analizziamo che cosa succede moltiplicando il binomio per se
stesso, eseguendo cioè la
moltiplicazione~\(\tonda{A+B}\tonda{A+B}\), che sotto forma di 
potenza si scrive~\(\tonda{A+B}^{2}\).
\[\tonda{A+B}^{2}=\tonda{A+B}\tonda{A+B}=A^{2}+{AB}+{BA}+B^{2
}=A^{2}+2{AB}+B^{2}.\]
Pertanto, senza effettuare i passaggi intermedi si ha:

\begin{teorema}{}{} 
Il quadrato di un binomio è
uguale alla somma tra il quadrato del primo termine, il doppio prodotto 
del primo termine per il secondo e il quadrato del secondo termine: \quad 
\(\tonda{A+B}^{2}=A^{2}+2{AB}+B^{2}\)
\end{teorema}

\affiancati{.74}{.24}{
È possibile dare anche
un'interpretazione geometrica di questo prodotto notevole 
sostituendo~\(A\) e~\(B\) rispettivamente con le misure~\(a\) e~\(b\)
di due segmenti.

Costruiamo il segmento di lunghezza~\(a+b\) e il quadrato 
di lato~\(a+b\), il quale avrà area~\((a+b)^{2}\).
Possiamo dividere il quadrato in quattro parti come nella figura a fianco.
Il quadrato di lato~\(a+b\) è composto da due quadrati
di area rispettivamente~\(a^{2}\) e~\(b^{2}\) e
da due rettangoli di area~\(ab\). 
}{
% \scalebox{.6}{\input{\folder lbr/fig001_quad.pgf}}
 \centering \quadratobin{1.8}{3}
}

Di conseguenza l'area del quadrato è uguale 
a:~\((a+b)^{2}=a^{2}+b^{2}+{ab}+{ab}=a^{2}+2{ab}+b^{2}\).

\begin{osservazione}{}{} 
Nel quadrato di un binomio:
\begin{itemize} [nosep]
\item I termini quadrati~\(A^{2}\) e~\(B^{2}\) sono sempre positivi.
\item Il termine rettangolare~\(2AB\) è positivo se~\(A\) e~\(B\) sono 
concordi, negativo se sono discordi.
\end{itemize}
\end{osservazione}

% Nella identità precedente, \(A\) e~\(B\) rappresentano due monomi 
% qualsiasi,
% quindi la scrittura~\(A+B\) deve intendersi come somma algebrica di due
% monomi che, rispetto al segno, possono essere concordi o discordi.
% 
% Ne consegue che:
% 
% \begin{wrapfloat}{figure}{r}{0pt}
% \scalebox{.6}{\input{\folder lbr/fig001_quad.pgf}}
% \end{wrapfloat}

\begin{esempio}{}{} Calcola i seguenti quadrati:
\begin{enumerate} [nosep]
\item \(\tonda{3a^2-5ab}^2 = 9a^4 - 30 a^3 b + 25a^2 b^2\)
\item \(\tonda{-2a^2-7ab}^2 = 4a^4 + 28 a^3 b + 49a^2 b^2\)
\item \(\tonda{xy^2+2xz}^2 = x^2 y^4 + 4 x^2 y^2 z + 4x^2 y^2\)
\end{enumerate}
\end{esempio}

\begin{esempio}{}{} Trova quale quadrato dà come risultato il seguente 
trinomio:\\
\(9x^4 - 12 x^2 y + 4y^2= 
\tonda{3x^2}^2 +2\tonda{3x^2}\tonda{-2y}+\tonda{-2y}^2=
\tonda{3x^2-2y}^2\)
\end{esempio}

% \vspazio\ovalbox{\risolvii \ref{ese:11.1}, \ref{ese:11.2}, 
% \ref{ese:11.3}, \ref{ese:11.4}, \ref{ese:11.5}, \ref{ese:11.6}, 
% \ref{ese:11.7}, \ref{ese:11.8}, \ref{ese:11.9}, \ref{ese:11.10}}

\subsection{Quadrato di un polinomio}
\label{subsec:prodnot_quadratopolinomio}

\affiancati{.69}{.29}{
Possiamo estendere la definizione al quadrato di un polinomio qualunque.

Si consideri il trinomio~\(A+B+C\), il suo quadrato sarà:

\vspace{-1.5em}
\begin{align*}
&\tonda{A+B+C}^{2}=\tonda{A+B+C}\cdot\tonda{A+B+C}=\\
&=A^{2}+{AB}+{AC}+{BA}+B^{2}+{BC}+{CA}+{CB}+C^{2}=\\
&=A^{2}+B^{2}+C^{2}+2{AB}+2{AC}+2{BC}
\end{align*}

\vspace{-.5em}
Perciò: \quad
\(\tonda{A+B+C}^{2}=A^{2}+B^{2}+C^{2}+2{AB}+2{AC}+2{BC}\)
}{
\immagine*{Quadrato che ha per lato a+b+c.}
{\hspace*{-3mm}\quadratotri{1.2}{2}{2.8}}
}


\begin{teorema}{}{} 
Il quadrato di un polinomio è uguale alla somma
dei quadrati dei monomi che lo compongono e dei doppi prodotti di ogni
termine per ciascuno dei successivi.

Nel caso di un polinomio composto da quattro monomi si ha:
\[\tonda{A+B+C+D}^{2}=A^{2}+B^{2}+C^{2}+D^{2}+2{AB}+2{AC}+2{AD}+2{BC}+2
{BD}+2{CD}\]
\end{teorema}

\begin{esempio}{}{} 
Calcola il seguente quadrato:\\
\(\tonda{2a^3-4b-c}^2 = 4a^6 + 16b^2 +c^2 -16a^3 b -4 a^3 c +8 bc\)
\end{esempio}

\begin{esempio}{}{} 
Trova quale quadrato dà come risultato il seguente 
polinomio:\\
\(x^2 + 4y^2 + z^2 -4xy +2xz -4yz = \tonda{x-2y+z}^2\)
\end{esempio}

% \ovalbox{\risolvii \ref{ese:11.11}, \ref{ese:11.12}, \ref{ese:11.13}, 
% \ref{ese:11.14}, \ref{ese:11.15}}

\subsection{Prodotto della somma fra due monomi per la loro differenza}
\label{subsec:prodnot_sommaperdifferenza}

Osserviamo i seguenti prodotti:
\begin{align*}
\tonda{A-B}\tonda{A+B}   &= A^{2}+{AB}-{AB}-B^{2}=A^{2}-B^{2}\\
\tonda{-A+B}\tonda{A+B}  &= -A^{2}-{AB}+{AB}+B^{2}=-A^{2}+B^{2}\\
\tonda{-A+B}\tonda{-A-B} &= A^{2}+{AB}-{AB}-B^{2}=A^{2}-B^{2}\\
\tonda{A-B}\tonda{-A-B}  &= A^{2}-{AB}+{AB}-B^{2}=-A^{2}+B^{2}
\end{align*}

\affiancati{.59}{.29}{
Quando eseguiamo il prodotto\indc{polinomio/polinomi}
{somma per differenza} tra due binomi che hanno due termini uguali e due 
termini opposti i prodotti incrociati si annullano e rimangono i due 
prodotti del termine uguale per se stesso e dei due termini opposti: 
il primo, prodotto tra i due termini che non cambiano segno, risulterà 
positivo, il secondo, prodotto tra i termini che cambiano segno, 
risulterà negativo. 
}{
\immagine*{Differenza di due quadrati e rettangolo equivalente}
{\hspace{-15mm}\sommadifferenza{3}{1}}
}

\begin{definizione}{}{} 
Il \textbf{prodotto tra due binomi} che hanno due termini uguali e
due termini opposti è uguale alla somma dei prodotti dei due termini 
uguali e dei due termini opposti.
\[\tonda{A-B}\tonda{A+B} = A^{2} - B^{2} \qquad
  \tonda{A-B}\tonda{-A-B} = -A^{2} + B^{2}\]
\end{definizione}

\begin{esempio}{}{}
Alcuni esempi:
\begin{enumerate}
\item \(\tonda{-3a^{2}-5{ab}} \tonda{3a^{2}-5{ab}}=-9a^{2}+25a^{2}b^{2}\)
\item \(9a^2b^4-4x^6 = \tonda{3ab^{2}+2x^3} \tonda{3ab^{2}-2x^3}\)
\item \(\tonda{-{\dfrac{1}{4}}x^{2}+b}\cdot \tonda{+{\dfrac{1}{4}}x^{2}+b}=
        -\dfrac{1}{16}x^{4}+b^{2}\)
\item calcola il prodotto~\(28 \cdot 32\).~~
Soluzione:~\(28\cdot 32=(30-2)(30+2)=900-4=896\)
\item \((2x+1-y)(2x+1+y)=\)
\(\tonda{(\underbrace{2x+1}_{A})-\underbrace{y}_{B}}
  \tonda{(\underbrace{2x+1}_{A})+\underbrace{y}_{B}}= \newline
\hspace*{38mm}
=\underbrace{(2x+1)^{2}}_{A^{2}}-\underbrace{y^{2}}_{B^{2}}=
4x^{2}+4x+1-y^{2}\)
\end{enumerate}
\end{esempio}

% \ovalbox{\risolvii \ref{ese:11.16}, \ref{ese:11.17}, \ref{ese:11.18}, 
% \ref{ese:11.19}, \ref{ese:11.20}, \ref{ese:11.21}, \ref{ese:11.22}, 
% \ref{ese:11.23}}


\subsection{Prodotto particolare}
\label{subsec:prodnot_particolare}

Consideriamo la moltiplicazione tra due binomi di primo grado che hanno 
i coefficienti dei termini di primo grado uguali a 
uno\indc{polinomio/polinomi}{prodotto particolare}:

\affiancati{.69}{.29}{
\vspace*{-.5em}
\[\tonda{x +3} \tonda{x+4} = x^2 +3x +4x +12 = x^2 +7x +12\]
Ora generalizziamo il calcolo mettendo al posto dei numeri dei parametri:

\vspace*{-1.5em}
\[\tonda{x +a} \tonda{x+b} = x^2 +ax +bx +ab = x^2 +\tonda{a+b}x +ab\]
}{
% \vspace*{-1em}
\immagine{}{\trinot{3}{1}{2}}
}

% \vspace*{-.5em}
Chiamando:\\
\(a+b=s\) \quad somma dei termini di grado zero e\\
\(ab=p\) \quad prodotto degli stessi due termini,\\
possiamo dire che il prodotto tra i due binomi è un trinomio di secondo 
grado che ha per coefficienti rispettivamente 
\(~~ 1, ~~ s,~~ p\) ~cioè:~~
\(\tonda{x +a} \tonda{x+b} = x^2 +sx +p\).

\begin{definizione}{}{} 
Il \textbf{prodotto tra due binomi} che hanno il primo termine
uguale è un trinomio costruito 
secondo lo schema seguente:
\[\tonda{A+B}\tonda{A+C} = A^{2} + (B+C)A + BC\]
\end{definizione}

\begin{esempio}{}{}
Alcuni esempi:
\begin{enumerate}
\item \(\tonda{x+7} \tonda{x+5}\) \quad poiché: \quad 
\(s=7+5=12\) ~~ e ~~ \(p=7 \cdot 5=35\), \\ 
\(\tonda{x+7} \tonda{x+5}=x^2+12x+35\)
\item \(\tonda{x-3} \tonda{x+5}\) \quad poiché: \quad 
\(s=-3+5=+2\) ~~ e ~~ \(p=-3 \cdot 5=-15\):\\ 
\(\tonda{x-3} \tonda{x+5}=x^2+2x-15\)
% \item \(\tonda{x-4} \tonda{x-6}\) \quad poiché: 
% \quad \(s=-4-6=-10\) ~~ e ~~ \(p=-4 \cdot \tonda{-6}=+24\): \\ 
% \(\tonda{x-4} \tonda{x-6}=x^2-10x+24\)
\end{enumerate}
\end{esempio}

\begin{osservazione}{}{}
I prodotti notevoli ``somma per differenza'' e ``quadrato del binomio'' 
possono essere visti come un caso particolare di trinomio particolare.
\end{osservazione}

\begin{esempio}{}{}
Alcuni esempi:
\begin{enumerate}
\item \(\tonda{x-3} \tonda{x+3}\) \quad poiché: \quad 
\(s=-3+3=0\) ~~ e ~~ \(p=-3 \cdot 3 = -9\): \\ 
\(\tonda{x-3} \tonda{x+3}=x^2-9\)
\item \(\tonda{x-3} \tonda{x-3}\) \quad poiché: \quad 
\(s=-3-3=-6\) ~~ e ~~ \(p=-3 \cdot -3 = +9\): \\ 
\(\tonda{x-3} \tonda{x-3}=x^2-6x+9\)
\end{enumerate}
\end{esempio}

\begin{esempio}{}{}
Trova la moltiplicazione di binomi che dia come prodotto il seguente 
trinomio: \(x^2+3x-40\)\\
Devo trovare i due numeri \(a\) e \(b\) tali che: 
\(a+b=+3\) e \(a \cdot b=-40\). \\
Si inizia sempre dal valore assoluto del prodotto:\quad
\(40 = 1 \cdot 40; \quad 2 \cdot 20; \quad 4 \cdot 10; 
  \quad 5 \cdot 8;~\dots\)\\
I due fattori che hanno per somma algebrica \(3\) sono : 
\(5\) e \(8\).\\
Inseriamoli nello schema della moltiplicazione: \quad
\(\tonda{x \quad 5} \tonda{x \quad 8}\)\\
Ora aggiustiamo i segni in modo che 
la somma sia \(+3\) e il prodotto \(-40\):\\[-1em]
\begin{center}\(x^2+3x-40 = \tonda{x-5} \tonda{x+8}\)\end{center}
\end{esempio}

\pagebreak %---------------------------------------
\subsection{Cubo di un binomio}
\label{subsec:prodnot_cubo}

Si consideri il binomio~\(A+B\), possiamo ottenere il suo cubo 
in diversi modi\indc{polinomio/polinomi}{cubo di un binomio}:

\affiancati{.62}{.36}{
\begin{align*} \small
&\tonda{A+B}^{3}=\tonda{A+B}^{2}\tonda{A+B}= \\
&=\tonda{A^{2}+2{AB}+B^{2}}\tonda{A+B}=\\
%\tonda{A+B}^{3}&=\tonda{A+B}^{2}\tonda{A+B}=\\
%&=\tonda{A^{2}+2{AB}+B^{2}}\tonda{A+B}=\\
&=A^{3}+A^{2}B+2A^{2}B+2{AB}^{2}+{AB}^{2}+B^{3}=\\
&=A^{3}+3A^{2}B+3{AB}^{2}+B^{3}
\end{align*}
Al posto dei due monomi possiamo considerare le lunghezze di due 
segmenti \(a\) e \(b\). 
Osservando il cubo che ha per spigolo \(a + b\), possiamo vedere che è 
composto da due cubi e altri 6 parallelepipedi, a 3 a 3 uguali.

\vspace{1em}
Oppure possiamo sviluppare il cubo come la moltiplicazione tra tre 
binomi uguali, il prodotto si ottiene moltiplicando tutti i primi 
elementi, poi il primo per il primo per il secondo, poi il primo per il 
secondo per il primo e così via esaurendo tutte le possibilità:
\begin{align*} \small
&\tonda{a+a}^{3} = \tonda{a+b}\tonda{a+b}\tonda{a+b} = \\
&= aaa + aab + aba + abb + baa + bab + bba + bbb = \\
&= a^3 + 3a^2b + 3ab^2 +b^3
\end{align*}
}{
\immagine*{Cubo scomposto in due cubi e sei parallelepipedi.}
{\hspace{-10mm}\cubobin{3}{1}{0}}

\vspace{+4mm}
\immagine*{Cubo scomposto in due cubi e sei parallelepipedi, esploso.}
{\hspace{-10mm}\cubobin{3}{1}{1.4}}
}

% Quindi:

\begin{teorema}{}{} 
Il cubo di un binomio è uguale alla somma tra il
cubo del primo monomio, il triplo prodotto del quadrato del primo
% monomio 
per il secondo, il triplo prodotto del primo 
% monomio 
per il quadrato del secondo e il cubo del secondo monomio.

\vspace{-2.em}
\[\tonda{A+B}^3=A^3+3A^2B+3AB^2+B^3 \qquad
  \tonda{A-B}^3=A^3-3A^2B+3AB^2-B^3\]

\vspace{-3.em}
\[\tonda{-A+B}^3=-A^3+3A^2B-3AB^2+B^3 \hspace{5.8mm}
  \tonda{-A-B}^3=-A^3-3A^2B-3AB^2-B^3\]
\end{teorema}

% Essendo~\(\tonda{A-B}^{3}=\left[A+\tonda{-B}\right]^{3}\), il
% cubo della differenza di due monomi si ottiene facilmente dal cubo
% della somma, quindi
% \(\tonda{A-B}^{3}=A^{3}-3A^{2}B+3{AB}^{2}-B^{3}\).

\begin{esempio}{}{} Calcola i seguenti cubi di binomi:
% \begin{htmulticols}{2}
\begin{enumeratea}
\item \(\tonda{x-2}^3= x^3-6x^2+12x-8\)
\item \(\tonda{-x+1}^3= -x^3+3x^2-3x+1\)
\item \(\tonda{-x-4}^3= -x^3-12x^2-48x-64\)
\item \(\tonda{2x+3y}^3= 8x^3+12x^2y+18xy^2+27\)
\end{enumeratea}
% \end{htmulticols}
\end{esempio}

\begin{esempio}{}{}
Trova il cubo del binomio equivalente al seguente quadrinomio:\\
\(x^3-9x^2a+27xa^2-27a^3\)\\
Osserviamo che: \(x^3\) è il cubo di \(x\) \quad 
e \quad \(-27a^3\) è il cubo di \(-3a\)\\
possiamo sospettare che: \(x^3-9x^2a+27xa^2-27a^3 = \tonda{x-3a}^3\)\\
eseguendo il prodotto notevole, ne abbiamo la conferma.
\end{esempio}

% \vspazio\ovalbox{\risolvii \ref{ese:11.24}, \ref{ese:11.25}, 
% \ref{ese:11.26}, \ref{ese:11.27}}

\subsection{Potenza n-esima di un binomio}
\indc{polinomio/polinomi}{potenze di un binomio}
\label{subsec:prodnot_potenzabinomio}

\affiancati{.54}{.44}{
Finora abbiamo calcolato le potenze del binomio~\(a+b\) fino
all'ordine tre, in questo paragrafo ci si propone di
fornire un criterio che permetta di calcolare la potenza~\((a+b)^{n}\),
con~\(n\in \N\). Osserviamo le potenze ottenute:
}{
\begin{align*}
&(a+b)^0 = 1\\
&(a+b)^1 = a+b\\
&(a+b)^2 = a^2+2ab+b^2\\
&(a+b)^3 = a^3+3a^2b+3ab^2+b^3
\end{align*}
}
Si può notare che:

\begin{itemize} [nosep]
\item lo sviluppo di ciascuna potenza dà origine a un polinomio
omogeneo dello stesso grado dell'esponente della
potenza, completo e ordinato secondo le potenze decrescenti di~\(a\) e 
crescenti di~\(b\)
\item il primo e l'ultimo coefficiente sono sempre uguali a~1;
\item i coefficienti di ciascuna riga si ottengono utilizzando una
disposizione dei numeri a triangolo, detto (in Italia) 
\emph{triangolo di Tartaglia}.
\ind{triangolo di Tartaglia}
\end{itemize}

\affiancati{.44}{.44}{
\immagine*{Triangolo di Tartaglia.}{\hspace*{-5mm}\tartaglia}
}{
\immagine*{Come ricavare il triangolo di Tartaglia.}{\hspace*{-2mm}\tartaglib}
}

In questo triangolo i numeri di ciascuna riga 
% (tranne il primo e l'ultimo che sono uguali a~1) 
sono la somma dei due che stanno sopra.
Con questa semplice regola si hanno gli sviluppi:

\begin{itemize} [nosep]
\item \((a+b)^{0}=1\)
\item \((a+b)^{1}=a+b\)
\item \((a+b)^{2}=a^{2}+2{ab}+b^{2}\)
\item \((a+b)^{3}=a^{3}+3a^{2}b+3{ab}^{2}+b^{3}\)
\item \((a+b)^{4}=a^{4}+4a^{3}b+6a^{2}b^{2}+4{ab}^{3}+b^{4}\)
\item \((a+b)^{5}=a^{5}+5a^{4}b+10a^{3}b^{2}+10a^{2}b^{3}+5{ab}^{4}+b^{5}\)
\item \dots
\end{itemize}

\begin{esempio}{}{} Calcola le seguenti potenze di binomi:
\begin{enumeratea}
\item \(\tonda{-3x-4}^0  = 1\)
\item \(\tonda{-5x+4}^1  = -5x+4\)
\item \(\tonda{-2x-3y}^2 = 4x^2+12xy+9y^2\)
\item \(\tonda{-2x-3y}^3 = -8x^3-36x^2y-54xy^2-27y^3\)
\item \(\tonda{x-2}^4    = x^4 -8x^3+24x^2-32x+16\)
\item \(\tonda{-x+1}^5   = -x^5 +5x^4-10x^3+10x^2-5x+1\)
\end{enumeratea}
\end{esempio}

% \ovalbox{\risolvii \ref{ese:11.28}, \ref{ese:11.29}, \ref{ese:11.30}}

\section{Valori un polinomio}
\label{sec:poli_funzione}

\subsection{Riprendiamo il problema di partenza}

Nel problema di inizio capitolo (pag.~\pageref{cartoni}) abbiamo trovato 
che il volume dipende 
dalla lunghezza del taglio secondo la seguente funzione:
\[V(x) = \tonda{2 - 2 \cdot x} \cdot \tonda{3 - 2 \cdot x} \cdot x\]

Ora che abbiamo imparato a usare monomi e polinomi possiamo riscrivere 
questa espressione in modo più compatto:
\[V(x) = \tonda{-2x +2}\tonda{-2x +3}x\]
Possiamo eseguire il prodotto tra i due binomi:
\[V(x) = \tonda{-2x +2}\tonda{-2x +3}x = 
  \tonda{+4x^2-6x-4x+6}x = \tonda{+4x^2-10x+6}x\]
e, moltiplicando il polinomio per il monomio, otteniamo un
trinomio di terzo grado:
\[V(x) = +4x^3 -10x^2 +6x\]

\affiancati{.59}{.39}{
Fornendo a \(x\) diversi valori compresi tra \(0\) e \(1\) abbiamo 
trovato un massimo per \(x=0,4\): \\
\(V(0,4) = 1,056\).\\
Ma forse il massimo si trova un po' prima o un po' dopo a \(0,4\).
Facciamo qualche prova:\\ 
\(V(0,35) = +4 \cdot 0,35^3 -10 \cdot 0,35^2 +6 \cdot 0,35 = 1,0465\)\\
\(V(0,45) = +4 \cdot 0,45^3 -10 \cdot 0,45^2 +6 \cdot 0,45 = 1,0395\)\\
Chiaramente il massimo è compreso tra \(0,35\) e \(0,40\).

Costruiamo la tabella per questi valori della variabile \(x\).

% Punti stazionari: \(\dfrac{5\mp\sqrt{7}}{6}\)\\
% \(x_1 \approx 0,3923747814892349;\quad x_2 \approx 1,2742918851774319\)\\
% \(f(x_1) \approx 1,0563058954611901\quad 
%   f(x_2) \approx -0,3155651547204489\)
}{
\begin{center}
\begin{tabular}{cc}
taglio: x & volume: V(x)\\
\hline
0,35 & 1,046500\\ 
0,36 & 1,050624\\ % 
0,37 & 1,053612\\ % 
0,38 & 1,055488\\ % 
0,39 & 1,056276\\ % 
0,40 & 1,056000\\ %
\hline
\end{tabular}
\end{center}
}

Il massimo volume trovato è di \(1,056276\munit{dm^2}\) per un taglio di 
\(3,9\munit{cm}\), ma il vero massimo potrebbe essere un po' prima o un 
po' dopo al valore di \(3,9\munit{cm}\).

La fustellatrice dello scatolificio ``Cartoni'' è in grado di effettuare 
tagli con la precisione di mezzo millimetro.
Calcoliamo allora il volume per i valori \(0,385\) e \(0,395\):\\ 
\(V(0,385) = +4 \cdot 0,385^3 -10 \cdot 0,385^2 +6 \cdot 0,385=1,0560165\)\\
\(V(0,395) = +4 \cdot 0,395^3 -10 \cdot 0,395^2 +6 \cdot 0,395=1,0562695\)

Il confronto dei tre volumi dà:\quad 
\(V(0,385) < V(0,390) > V(0,395)\) \quad 
quindi con quella fustellatrice, il massimo volume lo si può ottenere con 
il taglio di lunghezza \(x = 0,390\).

Come consulenti abbiamo svolto il nostro compito, ma come ricercatori 
possiamo essere soddisfatti?

Continuando con il metodo di approssimazioni successive possiamo trovare 
un ``massimo'' per \(x=0,3923747814892349\), ma siamo sicuri che il 
massimo non si trovi un po' prima o un po' dopo a questo valore?

Per rispondere in modo preciso a questa domanda avremo bisogno di 
esplorare altre regioni della matematica.

\subsection{Funzione polinomiale}\ind{funzione polinomiale}

Nel nostro problema, abbiamo visto che non avevano senso tagli 
\(x < 0\) o \(x \geqslant 1\).
Ma possiamo astrarre da quel problema e considerare la funzione 
\[f(x):~ x \mapsto 4x^3 -10x^2 +6x\]
senza particolari limitazioni al valore della variabile \(x\).

Ora possiamo interpretare i diversi valori della variabile \(x\) come le 
ascisse di punti nel piano cartesiano e i corrispondenti valori della 
funzione come le ordinate \(y\) degli stessi punti.
Riportando alcuni di questi punti in un piano cartesiano, otteniamo una 
figura più o meno definita a seconda di quanti valori di \(x\) 
consideriamo.

\affiancati{.39}{.59}{
{\footnotesize
\begin{center}
\begin{tabular}{cl}
\(x\) & \(y=f(x)\)\\
\hline
\(-0.2\) & \( -1.632\) \\
\(-0.1\) & \( -0.704\) \\
\(+0.0\) & \(   +0.0\) \\
\(+0.1\) & \( +0.504\) \\
\(+0.2\) & \( +0.832\) \\
\(+0.3\) & \( +1.008\) \\
\(+0.4\) & \( +1.056\) \\
\(+0.5\) & \(   +1.0\) \\
\(+0.6\) & \( +0.864\) \\
\(+0.7\) & \( +0.672\) \\
\(+0.8\) & \( +0.448\) \\
\(+0.9\) & \( +0.216\) \\
\(+1.0\) & \(   +0.0\) \\
\(+1.1\) & \( -0.176\) \\
\(+1.2\) & \( -0.288\) \\
\(+1.3\) & \( -0.312\) \\
\(+1.4\) & \( -0.224\) \\
\(+1.5\) & \(   +0.0\) \\
\(+1.6\) & \( +0.384\) \\
\(+1.7\) & \( +0.952\) \\
\(+1.8\) & \( +1.728\) \\
\(+1.9\) & \( +2.736\) \\
\(+2.0\) & \(   +4.0\) \\
\hline
\end{tabular}
\end{center}
}
}{
\begin{center}
\funzpol
\end{center}
}


