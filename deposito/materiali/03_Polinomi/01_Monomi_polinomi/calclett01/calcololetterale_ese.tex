% (c) 2012 -2014 Claudio Carboncini - claudio.carboncini@gmail.com
% (c) 2012 -2014 Dimitrios Vrettos - d.vrettos@gmail.com
% (c) 2014 Daniele Zambelli - daniele.zambelli@gmail.com

\section{Esercizi}

\subsection{Esercizi dei singoli paragrafi}

% ------------------- Espressioni letterali ------------------

\subsubsection*{\numnameref{sec:calclett_esplett}}

% \begin{htmulticols}{2}
\begin{esercizio}
\label{ese:8.1}
~

\affiancati{.69}{.29}{
Esprimi con una formula l'area della superficie della zona colorata,
indicando con~\(l\) la misura del lato~\(AB\) e
con~\(b\) la misura di~\(AC\)

\noindent\emph{Svolgimento}: l'area del quadrato è \ldots\ldots,\\
l'area di ciascuno dei quadratini bianchi è \ldots\ldots.\\ 
Pertanto l'area della superficie in grigio è \ldots\ldots
}{
\noindent\hspace*{\fill}\quatagliato{4}{.8}\hspace{\fill}
}
\end{esercizio}

\begin{esercizio}
\label{ese:8.2}
Scrivi l'espressione algebrica letterale relativa alla frase ``eleva al 
quadrato la differenza tra il cubo di un numero e il doppio del suo 
quadrato''.

\emph{Svolgimento}: detto~\(a\) il numero generico, il cubo di~\(a\) si 
indica con \ldots, il doppio quadrato di~\(a\) si indica con \ldots e 
infine il quadrato della differenza sarà: \ldots
\end{esercizio}

\begin{esercizio}
\label{ese:8.3}
Traduci in parole della lingua italiana il seguente schema di 
calcolo:~\((a-b)^{3}\)

\emph{Svolgimento}: 
``Eleva al \ldots\ldots la differenza tra \ldots\ldots''
\end{esercizio}

% \begin{esercizio}
%  \label{ese:8.4}
% Individua tra le espressioni letterali sottostanti, quelle scritte 
% correttamente:
% \begin{enumeratea}
% \spazielenx
%  \item \(b\cdot {\dfrac{4}{5}}+\tonda{3-\dfrac{7}{2}}\cdot a-a\)
%  \item \(a\cdot +2-b^{4}\)
%  \item \(x\cdot (a-b)^{2}+(x-3)\)
%  \item \(x^{y}-a:2\)
%  \item \(-a+4b+c\)
%  \item \(\dfrac{a\cdot 1}{2}-\dfrac{a}{2}\)
% \end{enumeratea}
% \end{esercizio}

% \end{htmulticols}

\begin{esercizio}
\label{ese:8.5}
Collega con una freccia la proprietà dell'operazione con la sua 
scrittura  attraverso lettere:
\begin{htmulticols}{2}
\noindent
Commutativa dell'addizione\\
Associativa della moltiplicazione\\
Distributiva prodotto rispetto alla somma\\
\(a\cdot (x+y)=a\cdot x+a\cdot y\)\\
\(\tonda{a\cdot b}\cdot c=a\cdot \tonda{b\cdot c}\)\\
\({a+b=b+a}\)
\end{htmulticols}
\end{esercizio}

\begin{esercizio}
\label{ese:8.6}
Esprimi con le lettere la proprietà commutativa della moltiplicazione

\emph{Svolgimento}: ``considerati~\(a\) e~\(b\) due numeri qualsiasi, 
la proprietà commutativa si esprime per mezzo dell'espressione 
\ldots\ldots; cioè \ldots\ldots\ldots''
\end{esercizio}

% \begin{htmulticols}{2}
\begin{esercizio}
\label{ese:8.7}
Scrivi la formula dell'area di un trapezio, poi calcola l'area di un 
trapezio con base maggiore~\(B=5\munit{cm}\), base minore~\(b=2\munit{cm}\) 
e altezza~\(h=4\munit{cm}\)
\end{esercizio}

\begin{esercizio}
\label{ese:8.8}
Scrivi la formula che permette di calcolare il l'area di un quadrato di 
perimetro~\(2p\), poi calcola l'area di un quadrato di perimetro
\(l=48\munit{cm}\).
\end{esercizio}

\begin{esercizio}
\label{ese:8.9}
Scrivi la formula per calcolare l'altezza~\(h_i\) relativa 
all'ipotenusa~\(BC\) del triangolo rettangolo~\(ABC\),
poi calcola l'altezza \(h_i\) del triangolo con i 
cateti:~\(\overline{AB}=8\munit{m}\) e \(\overline{AC}=15\munit{m}\).
\end{esercizio}

\begin{esercizio}
\label{ese:8.10}
Scrivi la formula per calcolare il volume della scatola con le dimensioni 
\(a\), \(b\), \(c\),\\
Poi calcola il volume della scatola con 
\(a=10\munit{cm}\), \(b=4\munit{cm}\), \(c=2\munit{cm}\).

\vspace{.5em}
\affiancati{.49}{.49}{
Se raddoppiamo ciascuna dimensione allora il volume diventa:
\begin{enumeratea}
\item \(2\cdot a\cdot b\cdot c\)
\item \(a^{2}\cdot b^{2}\cdot c^{2}\)
\item \(6\cdot a\cdot b\cdot c\)
\item \(8\cdot \cdot b\cdot c\)
\end{enumeratea}
}{
\noindent\hspace*{\fill}\parall{10}{4}{2}\hspace{\fill}
}
\end{esercizio}

\pagebreak %--------------------------------------------------

\begin{esercizio}
\label{ese:8.12}
Scrivi sotto forma di espressioni letterali le seguenti frasi:
\begin{enumeratea}
\item moltiplica~\(a\) per l'inverso di~\(a\)
\item sottrai ad~\(a\) l'inverso di~\(b\)
\item sottrai il doppio di~\(a\) al cubo di~\(a\)
\item moltiplica~\(a\) per l'opposto del cubo di~\(a\);
\item somma al triplo di~\(a\) il doppio quadrato di~\(b\);
\item moltiplica l'inverso di~\(b\) per il quadrato dell'inverso di~\(a\);
\item somma al cubo di~\(a\) il quadrato della somma di~\(a\) e~\(b\);
\item dividi il quadrato di~\(a\) per il triplo cubo di~\(b\);
\item moltiplica il quadrato di~\(b\) per l'inverso del cubo di~\(a\);
%  \item il cubo di un numero, aumentato di~2, è uguale al quadrato della 
%    differenza tra quel numero e uno;
\item il reciproco della somma dei quadrati di~\(a\) e di~\(b\);
\item il cubo della differenza tra~1 e il cubo di~\(a\);
\item la somma dei quadrati di~\(a\) e di~\(b\) per il quadrato della 
  differenza tra~\(a\) e~\(b\);
\end{enumeratea}
\end{esercizio}
% \end{htmulticols}

% \begin{esercizio}[*]
% \label{ese:8.14}
% Calcola il valore dell'espressione letterale 
% \(E=\dfrac{3}{7}\cdot a\cdot b-\dfrac{1}{2}\cdot (a-b)+a-b\) le cui 
% variabili~\(a\), \(b\) rappresentano numeri razionali, per i valori 
% assegnati 
% nella tabella sottostante.
% 
% \emph{Svolgimento}: se vogliamo calcolare il valore dell'espressione 
% letterale dobbiamo scegliere due numeri razionali, uno da
% assegnare alla variabile~\(a\), l'altro alla variabile~\(b\)
% 
% \begin{tabular*}{.9\textwidth}{@{\extracolsep{\fill}}*{5}{c}}
%  \toprule
%  \(a\) & \(3\) & \(0\) & \(2\) & \(-{2}\)\vspace{1.05ex}\\
%  \(b\) & \(-3\) & \(-\dfrac{1}{2}\) & \(0\) & \(-\dfrac{3}{2}\) \\
%  \midrule
%  \(E=\dfrac{3}{7} ab-\dfrac{1}{2}(a-b)+a-b\)& & & &\\
%  \bottomrule
%  \end{tabular*}
% \end{esercizio}

\begin{comment}

\begin{esercizio}
\label{ese:8.15}
Calcola il valore numerico 
dell'espressione:~\(\dfrac{a}{a-3}+\dfrac{b}{3-b}\) per~\(a = -1\), 
\(b = 0\)

\emph{Svolgimento}:~\(\dfrac{-1}{-1-3}+\dfrac{0}{3-0}= \ldots\ldots\)
\end{esercizio}

% \newpage

\begin{esercizio}
\label{ese:8.16}
Calcola il valore dell'espressione~\(E=\dfrac{x-y}{3x}\) costruita con le 
variabili~\(x\) e~\(y\) che rappresentano numeri razionali. 
L'espressione letterale assegnata traduce il seguente schema di calcolo:
``la divisione tra la differenza di due numeri e il triplo del primo 
numero''. 
Completa la seguente tabella:

\begin{tabular*}{.9\textwidth}{@{\extracolsep{\fill}}*{7}{c}}
\toprule
\(x\) & \(\dfrac{3}{4}\) & \(\dfrac{19}{3}\) & \(\dfrac{3}{4}\) & 
\(-4\) & \(\ldots\) & \(\ldots\) \vspace{1.05ex}\\
\(y\) & \(-2\) & \(0\) & \(0\) & \(-2\) & \(\ldots\) & \(\ldots\) \\
\midrule
\(E=\dfrac{x-y}{3x}\)& & & & & &\\
\bottomrule
\end{tabular*}

Ti sarai accorto che in alcune caselle compare lo stesso valore 
per~\(E\): perché secondo te succede questo fatto?

Vi sono, secondo te, altre coppie che fanno assumere ad~\(E\) quello 
stesso valore?

\end{esercizio}

% \begin{esercizio}[*]
% \label{ese:8.17}
% Scrivi con una frase le seguenti espressioni
% \begin{htmulticols}{2}
%  \begin{enumeratea}
% \spazielenx
%  \item \(3a\)
%  \item \(\dfrac{2a}{3b^{2}}\)
%  \end{enumeratea}
% \end{htmulticols}
% \end{esercizio}

\begin{esercizio}
\label{ese:8.18}
Scrivi con una frase le seguenti espressioni
\begin{htmulticols}{4}
\begin{enumeratea}
\spazielenx
\item \(2b-5a\)
\item \(a {\dfrac{1}{a}}\)
\item \((a+b)^{2}\)
\item \(\dfrac{3x+y}{2x^{2}}\)
\end{enumeratea}
\end{htmulticols}
\end{esercizio}

% \newpage

\begin{esercizio}
\label{ese:8.19}
Completa la tabella sostituendo nella espressione della prima colonna i 
valori indicati.

\begin{tabular*}{.93\textwidth}{l@{\extracolsep{\fill}}*{8}{c}}
\toprule
Espressione & \(x=1\) & \(x=-1\) & \(x=0\) & \(x=2\) & 
\(x=\dfrac{1}{2}\) & \(x=-\dfrac{1}{2}\) & \(x=0,1\) & 
\(x=\dfrac{1}{10}\)\\
\midrule
\(2x+1\) & & & & & & & &\\
\(-(3x-2)\) & & & & & & & &\\
\(x^{2}+2x+2\) & & & & & & & &\\
\(x^{2}-x\) & & & & & & & &\\
\(-x^{2}+x-1\) & & & & & & & &\\
\(x^{3}-1\) & & & & & & & &\\
\(x^{3}+3x^{2}\) & & & & & & & &\\
\(-x^{3}+x^{2}-x\) & & & & & & & &\\
\(-(x+1)^{2}\) & & & & & & & &\\
\bottomrule
\end{tabular*}
\end{esercizio}

\begin{esercizio}
\label{ese:8.20}
Calcola il valore numerico delle seguenti espressioni algebriche:
\begin{enumeratea}
\spazielenx
\item \(3x^{2}-\dfrac{1}{4}x^{2}\quad\) per~\(x=\dfrac{1}{2}\quad\) 
\emph{Svolgimento}:~\(3\cdot \tonda{\dfrac{1}{2}}^{2}-\dfrac{1}{4}\cdot 
\tonda{\dfrac{1}{2}}^{2}=\ldots \ldots \ldots =\dfrac{11}{16}\)
\item \(5a^{2}b\quad\) per~\(a=-{\dfrac{1}{2}}\), \(b=\dfrac{3}{5}\quad\) 
\emph{Svolgimento}:~\(5\cdot \tonda{-{\dfrac{1}{2}}}^{2}\cdot 
\tonda{\dfrac{3}{5}}=\ldots \ldots \ldots \ldots \)
\item \(\dfrac{3}{2}\cdot a^{2}+\dfrac{1}{2}a-1\quad\) per~\(a=0\), 
per~\(a=-1\) 
e~\(a=2\)
\item \(2\cdot x^{5}-8\cdot x^{4}+3\cdot x^{3}+2\cdot x^{2}-7\cdot 
x+8\quad\) 
per~\(x=+1\) e~\(x=-1\)
\end{enumeratea}
\end{esercizio}

\begin{esercizio}
\label{ese:8.21}
Calcola il valore numerico delle seguenti espressioni algebriche:
\begin{enumeratea}
\spazielenx
\item \((x-1)\cdot (x-2)\cdot (x+3)\quad\) per~\(x=0\), \(x= -1\) 
e~\(x= 2\)
\item \(x^{2}+2x+1\quad\) per~\(x=0\), \(x= -1\) e~\(x= 1\)
\item \(-a^{2}\cdot b\cdot c^{3}\quad\) per~\(a=1\), \(b=-1\), \(c=-2\) 
e~\(a=-1\), 
\(b=\dfrac{9}{16}\), \(c=\dfrac{4}{3}\)
\item \(-{\dfrac{3}{2}}a+2b^{2}+11\quad\) per~\(a=-20\), 
\(b=-{\dfrac{1}{2}}\) 
e~\(a=\dfrac{2}{3}\), \(b=0\)
\item \(-a^{2}+\dfrac{1}{a}-3\cdot a^{3}\quad\) per~\(a=\dfrac{1}{3}\), 
\(a=-1\) e~\(a=+1\)
\end{enumeratea}
\end{esercizio}

% \begin{esercizio}[*]
% \label{ese:8.22}
% Calcola il valore numerico delle seguenti espressioni algebriche:
%  \begin{enumeratea}
% \spazielenx
%  \item \(4a+a^{3}\quad\) per~\(a=2\) e~\(a=1\) \hfill[2; 5]
%  \item \(2a+5a^{2}\quad\) per~\(a=-1\) e~\(a=0\) \hfill[3; 0]
%  \item \(3x+2y^{2}(xy)\quad\) 
%   per~\(x=1\), \(y=-{\dfrac{1}{2}}\) e~\(x=\dfrac{1}{3}\), \(y=-1\) 
% \sol{\dfrac{11}{4}}
%  \item \(a^{2}-b^{-1}+ab\quad\) 
%   per~\(a=1\), \(b=\dfrac{1}{2}\) e~\(a=0\), \(b=-1\) 
% \sol{-\dfrac{1}{2}}
%  \item \(3a^{2}b-7ab+a\quad\) per~\(a=1\), \(b=3\) e~\(a=-1\), 
% \(b=-3\) \hfill[-11]
%  \end{enumeratea}
% \end{esercizio}

% \newpage

\begin{esercizio}[*]
\label{ese:8.23}
Calcola il valore numerico delle seguenti espressioni algebriche:
\begin{enumeratea}
\spazielenx
\item \(3xy-2x^{2}+3y^{2}\quad\) 
per~\(x=\dfrac{1}{2}\), \(y=2\) e~\(x=2\), \(y=\dfrac{1}{2}\) 
\sol{\dfrac{29}{2}}
\item \(\dfrac{2}{3}a\tonda{a^2-b^2 } \quad\) 
per~\(a=-3\), \(b=-1\) e~\(a=\dfrac{1}{3}\), \(b=0\) \sol{-16}
\item \(\dfrac{xy}{x}+3xy^{3}\quad\) 
per~\(x=2\), \(y=-1\) e~\(x=-2\), \(y=+1\) \sol{-7}
\item \(\dfrac{1}{2}\dfrac{(a+b)^{2}}{a^{2}b^{2}}+2a+3b\quad\) 
per~\(a=\dfrac{1}{4}\), \(b=-2\) e \(a=\dfrac{1}{2}\),
\(b=-{\dfrac{1}{2}}\) \sol{\dfrac{5}{8}}
\item \(3x^{3}+2xy\tonda{\dfrac{x^{2}}{y}}+2y^{2}\quad\) 
per~\(x=-2\), \(y=\dfrac{3}{4}\) e~\(x=-1\), \(y=-1\) 
\sol{\dfrac{311}{8}}
\end{enumeratea}
\end{esercizio}

% \begin{esercizio}[*]
% \label{ese:8.24}
% Calcola il valore numerico delle seguenti espressioni algebriche:
%  \begin{enumeratea}
% \spazielenx
%  \item \(\dfrac{4a-7b}{(2a+3b)^{3}}\cdot ab^{3}\quad\) 
%  per~\(a=-{\dfrac{1}{2}}\), \(b=1\) e~\(a=-{\dfrac{1}{4}}\), 
% \(b=\dfrac{2}{3}\) 
% \sol{\dfrac{9}{16}}
%  \item \(\dfrac{4x^{2}-5xy+3y}{6x+y^{2}}\quad\) 
%  per~\(x=-1\), \(y=2\) e~\(x=0\), \(y=-2\) \hfill[-10]
%  \item \(\dfrac{x}{x+3}+y^{2}-\dfrac{xy-3x+y}{(\mathit{xy})^{2}}\quad\) 
%  per~\(x=3\), \(y=\dfrac{1}{3}\) e~\(x=1\), \(y=-1\) 
% \sol{\dfrac{149}{18}}
%  \item \(\dfrac{(4a-2b)\cdot {2a^{2}}}{3b^{3}}\cdot 
% {\dfrac{3}{4}}ab+a^{3}\quad\) 
%  per~\(a=1\), \(b=-1\) e~\(a=0\), \(b=-3\) \hfill[4]
%  \end{enumeratea}
% \end{esercizio}

\subsubsection*{\numnameref{subsec:condes}}

\begin{esercizio}
\label{ese:8.25}
Se~\(E(x)=-{\dfrac{x-2}{2}x^{2}}\) trova quanto vale \(E\) quando 
\(x\) vale: 
\( -2; \quad -{\dfrac{5}{8}}; \quad 0; \quad +\dfrac{3}{4}; \quad +2\)\\
\end{esercizio}

\begin{esercizio}
\label{ese:8.26}
Calcola il valore numerico dell'espressione:~\(\dfrac{3x-1}{x}\) 
per~\(x = 0\)

\emph{Svolgimento}: Sostituendo alla~\(x\) il valore assegnato si ha una
divisione per \ldots e quindi \dotfill
\end{esercizio}

\begin{esercizio}[*]
\label{ese:8.27}
Sostituendo alle lettere i numeri, a fianco indicati, stabilisci se le
seguenti espressioni hanno significato:
\TabPositions{8cm}
\begin{enumeratea}
\item \(\dfrac{x+3}{x}\)\quad per~\(x=0.\) \hfill\sino
\item \(\dfrac{x^{2}+y}{x}\)\quad per~\(x=3, y=0.\) \hfill\sino
\item \(\dfrac{(a+b)^{2}}{(a-b)^{2}}\)\quad per~\(a=1, b=1\) \hfill\sino
\item \(\dfrac{5x^{2}+3y-xy}{(x^{2}+y)^{3}}\)\quad per~\(x=2, y=-2\) 
\hfill\sino
\item \(\dfrac{a^{3}+b+6a^{2}}{a^{2}+b^{2}+3ab-3a^{2}}\)\quad per~\(a=1, 
b=\dfrac{4}{3}\) \hfill\sino
\end{enumeratea}
\end{esercizio}

\begin{esercizio}
\label{ese:8.28}
Sostituendo alle lettere numeri razionali
arbitrari, determina se le seguenti uguaglianze tra formule sono
vere o false
\TabPositions{8cm}
\begin{enumeratea}
\item \(a^{2}+b^{2}=(a+b)^{2}\) \hfill\verofalso
\item \((a-b)\cdot (a^{2}+a\cdot b+b^{2})=a^{3}-b^{3}\) \hfill\verofalso
\item \((5a-3b)\cdot (a+b)=5a^{2}+ab-3b^{2}\) \hfill\verofalso
\end{enumeratea}
\end{esercizio}

\newpage

\begin{esercizio}
\label{ese:8.29}
Se~\(n\) è un qualunque numero naturale,
l'espressione~\(2\cdot n+1\) dà origine:
\begin{htmulticols}{2}
\fbox{A}\quad ad un numero primo

\fbox{B}\quad ad un numero dispari

\fbox{C}\quad ad un quadrato perfetto

\fbox{D}\quad ad un numero divisibile per~3
\end{htmulticols}
\end{esercizio}

\begin{esercizio}
\label{ese:8.30}
Quale formula rappresenta un multiplo di~5, qualunque sia il numero 
naturale~\(n\)?
\begin{center}
\fbox{A}\quad~\(5+n\) \quad\fbox{B}\quad~\(n^{5}\) \quad\fbox{C}\quad~\(5\cdot 
n\) 
\quad\fbox{D}\quad~\(\dfrac{n}{5}\)
\end{center}
\end{esercizio}

\begin{esercizio}
\label{ese:8.31}
La tabella mostra i valori assunti da~\(y\) al variare di~\(x\) 
Quale delle seguenti è la relazione tra~\(x\) e~\(y\)?

\begin{center}
\begin{tabular*}{.4\textwidth}{l@{\extracolsep{\fill}}*{4}{c}}
\toprule
\(x\) & 1 & 2 & 3 & 4\\
\(y\) & 0 & 3 & 8 & 15\\
\bottomrule
\end{tabular*}

\vspace{1.10ex}\fbox{A}\quad~\(y=x+1\) \quad\fbox{B}\quad~\(y=x^{2}-1\) 
\quad\fbox{C}\quad~\(y=2x-1\) \quad\fbox{D}\quad~\(y=2x^{2}-1\)
\end{center}
\end{esercizio}

\begin{esercizio}
\label{ese:8.32}
Verifica che sommando tre numeri dispari consecutivi si ottiene un
multiplo di~3. Utilizza terne di numeri dispari che cominciano per~3;
7; 11; 15; 21. Per esempio~\(3+5+7= \ldots\) multiplo di? Vero. 
Continua tu.
\end{esercizio}

% \subsection{Risposte}
% 
% \begin{htmulticols}{2}
% \paragraph{\ref{ese:8.11}}
% a)~\(a \cdot\dfrac{1}{a}\),\quad b)~\(a-\dfrac{1}{b}\),\quad c)~
% \(a^3-2a\)
% \paragraph{\ref{ese:8.14}}
% \(a=3\) \(b=-3\rightarrow -\dfrac{6}{7}\),\quad~\(a=0\) 
% \(b=-\dfrac{1}{2}\rightarrow 
% \dfrac{1}{4}\),\quad~\(a=-\dfrac{3}{2}\) \(b=-\dfrac{3}{2}\rightarrow 
% -\dfrac{27}{28}\)
% \paragraph{\ref{ese:8.17}}
% a)~ Il triplo di~\(a\),\quad ~b)~Dividi il doppio di~\(a\) 
% per il triplo del quadrato di~\(b\)
% \paragraph{\ref{ese:8.22}}
% a)~\(a=2 \rightarrow~16\) \(a=1 \rightarrow~5\), \quad b)~\(a=-1 
% \rightarrow~3\) 
% \(a=0 \rightarrow~0\),
% c)~\(x=1\) \(y=-\dfrac{1}{2} \rightarrow \dfrac{11}{4}\),\quad d)~\
% (a=1\) 
% \(b=\dfrac{1}{2}\rightarrow -\dfrac{1}{2}\), \quad e)~\(a=1\) \(b=3 
% \rightarrow -11\)
% \paragraph{\ref{ese:8.23}}
% a)~\(x=\dfrac{1}{2}\) \(y=2 \rightarrow \dfrac{29}{2}\), \quad b)~\
% (a=-3\) \(b=-1 
% \rightarrow -16\), \quad c)~\(x=2\),\(y=-1 \rightarrow -7\),
% d)~\(a=\dfrac{1}{4}\) \(b=-2 \rightarrow \dfrac{5}{8}\),\quad e)
% \(x=-2\) \(y=\dfrac{3}{4} \rightarrow -\dfrac{311}8{}\),\quad
% \paragraph{\ref{ese:8.24}}
% a)~\(a=-\dfrac{1}{2}\) \(b=1 \rightarrow \dfrac{9}{16}\),
% b)~\(x=-1\) \(y=2 \rightarrow -10\),\quad c)~\(x=3\) \(y=\dfrac{1}{3} 
% \rightarrow \dfrac{149}{18}\),\quad d)~\(a=1\) \(b=-1 \rightarrow~4\)
% \paragraph{\ref{ese:8.27}}
% a)~Non ha significato perché \(\dfrac{4}{0}\) non è un numero.
% \end{htmulticols}
\end{comment}
% ------------------- Monomi ------------------

%\subsubsection*{9.1 - L'insieme dei monomi}
\subsubsection*{\numnameref{sec:calclett_monomi}}

% \begin{esercizio}
% \label{ese:9.1}
% Individua tra le espressioni letterali di seguito elencate, 
% quelle che sono monomi.
% \[E_{1}=35x^{2}+y^{2}; E_{2}=-4^{-1}ab^{4}c^{6}; 
% E_{3}=\dfrac{4}{x}y^{2}; 
% E_{4}=-{\dfrac{87}{2}}x^{2}z.\]
% 
% Per rispondere in modo corretto devo individuare quelle espressioni in
% cui compare solamente la \dotfill; pertanto sono monomi \dotfill
% \end{esercizio}

\begin{esercizio}
\label{ese:9.2}
Scrivi in forma normale i seguenti monomi:
\begin{htmulticols}{2}
\(\dfrac{4}{9}ab18c^{3}2^{-2}a^{3}b=
\dfrac{\ldots }{\ldots }a^{\ldots}b^{\ldots }c^{\ldots };\)

\(-x^{5}\dfrac{1}{9}y^{4}\tonda{-1+5}^{2}y^{7}=\dotfill~\)
\end{htmulticols}
\end{esercizio}

\begin{esercizio}
\label{ese:9.3}
Nell'insieme:\\
\(M=\graffa{-{\dfrac{34}{5}}a^{3}b;\quad 3^{2}a^{2}b^{4};\quad 
\dfrac{1}{3}ab^{3};\quad a^{3}b;\quad -a;\quad 7a^{2}b^{4};\quad 
-\dfrac{1}{3}ab^{3};\quad -89a^{3}b}\),\\
determina i sottoinsiemi dei monomi simili; 
rappresenta con un diagramma di Venn.
\end{esercizio}

% %\subsubsection*{9.2 - Valore di un monomio}
% \subsubsection*{\numnameref{subsec:monomi_valore}}

\begin{esercizio}
\label{ese:9.4}
Calcola l'area di un triangolo che ha altezza \quad \(h=2,5\) \quad e 
base \quad \(b=\dfrac{3}{4}\).
\end{esercizio}

\begin{esercizio}
\label{ese:9.5}
Calcola il valore dei seguenti monomi in corrispondenza dei valori 
indicati per ciascuna lettera.

\begin{htmulticols}{2}
\begin{enumeratea}
\spazielenx
\item \(-\dfrac{2}{9}xz\),~~ \( x=\dfrac{1}{2}\), \(z=-1\)
\item \(-\dfrac{8}{5}x^{2}y\),~~ \( x=-1 \), \(y=+10\)
%  \item \(-\dfrac{1}{2}a^{2}bc^3\),~~ \( a=-\dfrac{1}{2} \), 
% \(b=\dfrac{3}{2},c=-1\)
\item \(\dfrac{7}{2}a^{3}x^{4}y^2\),~~ \( a=\dfrac{1}{2}\), \(x=2\), 
\(y=-\dfrac{1}{2}\)
\item \(\dfrac{8}{3}abc^2\),~~ \( a=-3 \), 
\(b=-\dfrac{1}{3}\), \(c=\dfrac{1}{2}\)
\end{enumeratea}
\end{htmulticols}
\end{esercizio}

% \newpage

\begin{esercizio}
\label{ese:9.6}
Il grado complessivo di un monomio è:

\begin{enumeratea}
\item l'esponente della prima variabile che compare nel monomio;
\item la somma di tutti gli esponenti che compaiono sia ai fattori
numerici sia a quelli letterali;
\item il prodotto degli esponenti delle variabili che compaiono nel 
monomio;
\item la somma degli esponenti di tutte le variabili che vi compaiono.
\end{enumeratea}
\end{esercizio}

\begin{esercizio}
\label{ese:9.7}
Due monomi sono simili se:

\begin{enumeratea}
\item hanno lo stesso grado;
\item hanno le stesse variabili;
\item hanno lo stesso coefficiente;
\item hanno le stesse variabili con rispettivamente gli stessi esponenti.
\end{enumeratea}
\end{esercizio}

\begin{esercizio}
\label{ese:9.8}
Individua e sottolinea i monomi tra le seguenti espressioni letterali:
\[3+ab;\quad -2a;\quad -\dfrac{7}{3}ab^2;\quad 
-\tonda{\dfrac{4}{3}}^{3};\quad a^{2}bc\cdot{\dfrac{-2}{a^{3}}};\quad 
4a^{-3}b^{2}c^{5};\quad -x;\quad 8x^{4}-4x^{2};\quad 
% -y\cdot\tonda{2x^{4}+6z};\quad 
\dfrac{abc^{9}}{3+7^{-2}}\]
\end{esercizio}

\begin{esercizio}
\label{ese:9.9}
Nel monomio\quad \(m=-{\dfrac{5}{2}}a^{3}x^{2}y^{4}z^{8}\) distinguiamo:\\
coefficiente~\(=\ldots\),
parte letterale~\(=\ldots\ldots\ldots\),
grado complessivo~\(=\ldots\),
il grado della lettera~\(x=\ldots\)
\end{esercizio}

\begin{esercizio}
\label{ese:9.10}
Motiva brevemente la verità o falsità delle seguenti proposizioni:
\TabPositions{8.5cm}
\begin{enumeratea}
\item ``Se due monomi hanno ugual grado allora sono simili''
\hfill\verofalso\quad perché\dotfill
\item ``Se due monomi sono simili allora hanno lo stesso grado''
\hfill\verofalso\quad perché\dotfill
\end{enumeratea}
\end{esercizio}

\begin{esercizio}
\label{ese:9.11}
Quale diagramma di Venn rappresenta in modo corretto la seguente 
proposizione: 
<<alcune espressioni letterali non sono monomi>>.
\(L\): insieme delle espressioni letterali, \(M\): insieme dei monomi.\\

\noindent\hspace*{\fill}\esevenn\hspace*{\fill}
\end{esercizio}

\begin{esercizio}
\label{ese:9.12}
Attribuisci il valore di verità alle seguenti proposizioni:
\begin{enumeratea}
\spazielenx
\item Il valore del monomio~\(-a\) è negativo per qualunque a diverso da 
zero.\hfill\verofalso
\item Il valore del monomio~\(-a^{2}\) è negativo per qualunque a diverso 
da zero.\hfill\verofalso
\item Il valore del monomio~\((-a)^{2}\) è negativo per qualunque a diverso 
da zero.\hfill\verofalso
\item Il monomio~\(b^{6}\) è il cubo di~\(b^{2}\). \hfill\verofalso
\item L'espressione~\(ab^{-1}\) è un monomio. \hfill\verofalso
\item Il valore del monomio \(ab\) è nullo per
\(a = 1\) e \(b =-1\). \hfill\verofalso
\item Il valore del monomio è nullo se una delle sue lettere 
vale \(0\). \hfill\verofalso
\end{enumeratea}
\end{esercizio}

\begin{esercizio}
\label{ese:9.12}
Dato il monomio \(M(x, y) = 3x^2y\) calcola il valore di:

\(M(2, 1) = \ldots\ldots\); \quad \(M(0, 6) = \ldots\ldots\); \quad 
\(M(1, 1) = \ldots\ldots\); \quad \(M(3, 2) = \ldots\ldots\) 
\end{esercizio}

% \newpage

%\subsubsection*{9.3 - Operazioni con i monomi}
\subsubsection*{\numnameref{subsec:monomi_operazioni}}

% %\subsubsection*{9.4 - Potenza di un monomio}
% \subsubsection*{\numnameref{subsubsec:monomi_moltiplicazione}}

\begin{esercizio}
\label{ese:9.13}
Determina il prodotto dei seguenti monomi.
\begin{htmulticols}{2}
\begin{enumeratea}
\spazielenx
\item \(\tonda{-x^{2}y^{4}}\cdot \tonda{-{\dfrac{8}{5}}x^{2}y}\)
\item 
\(\tonda{-{\dfrac{15}{28}}xy^{3}}\cdot\tonda{-\dfrac{7}{200}x^{2}y^{2}}\)
\item 
\(\tonda{a^{5}b^{5}y^{2}}\cdot\tonda{-\dfrac{8}{5}a^{2}y^{2}b^{3}}\)
\item \(2ab^{2}\cdot \tonda{-{\dfrac{1}{6}}a^{2}b}\cdot 3a\)
\item 
\(\tonda{-{\dfrac{2}{9}}xz}\tonda{-{\dfrac{1}{4}}z^{3}}(27x)\)
\item \(-8\tonda{\dfrac{1}{4}x}\tonda{\dfrac{4}{5}x^{3}a^{4}}\)
\item \(5x^{3}y^{2}\cdot \tonda{-{\dfrac{1}{3}}x^{3}y^{2}}\cdot 
\tonda{-{\dfrac{1}{3}}}\)
\item \(6ab\cdot \tonda{-{\dfrac{1}{3}}a^{2}}\cdot {\dfrac{1}{2}ab\cdot 
4a^{2}}\)
\item \(\tonda{\dfrac{7}{2}a^{3}x^{4}y^{2}}\cdot 
        \tonda{-{\dfrac{8}{21}}ax^{2}y}\)
\item \(\tonda{\dfrac{4}{3}a^{3}bc^{2}}\cdot 
        \tonda{-{\dfrac{21}{4}}ab^{2}c^3}\)
\end{enumeratea}
\end{htmulticols}
\end{esercizio}


\begin{esercizio}
\label{ese:9.14}
Determina il prodotto dei seguenti monomi.
\begin{htmulticols}{3}
\begin{enumeratea}
\spazielenx
\item \((-2xy)\cdot (+3ax)\)
\item \(6a(-2ab)\tonda{-3a^{2}b^{2}}\)
\item \((-1)(-ab)\)
\item \(1,5a^{2}b\cdot \tonda{-{\dfrac{2}{3}}a^{2}b}\)
\item \(-{\dfrac{7}{5}}xy^{3}\tonda{-{\dfrac{10}{3}}xy^{2}z}\)
\item \(-x\tonda{14x^{2}}\)
\end{enumeratea}
\end{htmulticols}
\end{esercizio}

\begin{esercizio}
\label{ese:9.15}
Determina il prodotto delle seguenti coppie di monomi.
\begin{htmulticols}{3}
\begin{enumeratea}
\item \(1,\overline{6}xa\tonda{1,2xy^{2}}\phantom{\tonda{\dfrac{1}{1}}}\)
\item \(\tonda{\dfrac{12}{7}m^{2}n^{3}}\tonda{-{\dfrac{7}{4}}mn}\)
\item \(\tonda{-{\dfrac{5}{4}}ax^{2}}\tonda{\dfrac{3}{10}x^{3}y}\)
\item \(12ab\tonda{-{\dfrac{1}{2}}a^{3}b^{3}}\)
\item \(\tonda{-{\dfrac{15}{8}}at^{2}}\tonda{\dfrac{6}{5}t^{3}x}\)
\item \(\tonda{\dfrac{12}{4}a^{2}n^{2}}\tonda{-{\dfrac{7}{4}}ax}\)
\end{enumeratea}
\end{htmulticols}
\end{esercizio}


\begin{esercizio}
\label{ese:9.16}
Osservando gli esercizi precedenti puoi concludere che il grado del 
prodotto è:

\begin{enumeratea}
\item il prodotto dei gradi dei suoi fattori;
\item la somma dei gradi dei suoi fattori;
\item minore del grado di ciascuno dei suoi fattori;
\item uguale al grado dei suoi fattori.
\end{enumeratea}
\end{esercizio}

% \newpage %------------------------------------------------------

% %\subsubsection*{9.4 - Potenza di un monomio}
% \subsubsection*{\numnameref{subsubsec:monomi_potenza}}

\begin{esercizio}
\label{ese:9.17}
Esegui le potenze indicate.
\begin{htmulticols}{2}
\begin{enumeratea}
\spazielenx
\item \(\tonda{-{\dfrac{3}{5}}abx^{3}y^{5}}^{3}=\dfrac{\ldots 
}{\ldots}a^{3}b^{3}x^{\ldots }y^{\ldots}\)
\item \(\tonda{-a^{4}b^{2}}^{7}=\ldots\)
\item \(\tonda{-3x^{3}y^{4}z}^{2}=9x^{6}y^{\ldots }z^{\ldots }\)
\item 
\(\tonda{\dfrac{1}{2}a^{2}bc^{5}}^{4}=\dfrac{1}{\ldots}a^{\ldots}b^{\
ldots }c^
{\ldots}\)
\item \(\tonda{a^{3}b^{2}}^{8}=\ldots\)
\item \(\tonda{-5ab^{2}c}^{3}=\ldots\)
\end{enumeratea}
\end{htmulticols}
\end{esercizio}

% \newpage %----------------------------------------------

\begin{esercizio}
\label{ese:9.18}
Esegui le potenze indicate.
\begin{htmulticols}{3}
\begin{enumeratea}
\spazielenx
\item \(\tonda{+2ax^{3}y^{2}}^{2}\)
\item \(\tonda{-{\dfrac{1}{2}}axy^{2}}^{3}\)
\item \(\tonda{\dfrac{3}{4}x^{4}y}^{3}\)
\item \(\tonda{\dfrac{2}{3}xy^{2}}^{3}\)
\item \(\tonda{-{\dfrac{1}{2}}ab}^{4}\)
\item \(\tonda{-{\dfrac{3}{2}}a^{5}}^{2}\)
\end{enumeratea}
\end{htmulticols}
\end{esercizio}

\begin{esercizio}
\label{ese:9.19}
Esegui le operazioni indicate.
\begin{htmulticols}{2}
\begin{enumeratea}
\spazielenx
\item \(\quadra{\tonda{-rs^{2}t}^{2}}^{3}\)
\item \(\quadra{\tonda{-{\dfrac{1}{2}}x^{2}y^{3}}^{2}}^{3}\)
\item \(\quadra{\tonda{-{\dfrac{3}{2}}a^{2}b^{3}}^{2}}^{2}\)
\item \(\tonda{-xy}^{2}\tonda{-{\dfrac{1}{2}}xy^{2}}^{3}\)
\item 
\(-\tonda{\dfrac{3}{2}xy^{2}}^{0}\cdot\tonda{-{\dfrac{1}{6}}xy}^{2}\)
\item 
\(-\tonda{-{\dfrac{1}{3}}x^{3}y^{2}}^{2}\cdot \tonda{-\dfrac{1}{3}}^{-3}\)
\item \(\tonda{\dfrac{2}{3}ab^{2}c}^{2}\cdot\tonda{-3ab^{3}}^{2}\)
\item 
\(\quadra{\tonda{-{\dfrac{1}{2}}a^{2}b}^{2}\cdot{\dfrac{2}{3}a^{2}b}}^
{ 2}\)
%  \item \(\tonda{\dfrac{2}{3}x\cdot{\dfrac{1}{6}}x\cdot 
% {\dfrac{1}{2}}x}^{2}\cdot\tonda{-{\dfrac{1}{6}}ab^{2}}^{2}\)
\end{enumeratea}
\end{htmulticols}
\end{esercizio}

% %\subsubsection*{9.5 - Divisione di due monomi}
% \subsubsection*{\numnameref{subsubsec:monomi_divisione}}

\begin{esercizio}
\label{ese:9.21}
Indica le Condizioni di esistenza, \(\CE\), del quoziente e 
esegui le divisioni:
\begin{htmulticols}{2}
\begin{enumeratea}
\spazielenx
\item \((-34x^{5}y^{2}):(-2yz^{3})\)
\item \(21a^{3}x^{4}b^{2}:(7ax^{2}b)\)
\item \(a^{6}:(20a^{2})\)
\item \(15b^{8}:\tonda{-{\dfrac{40}{3}}b^{3}}\)
\item 
\(\tonda{-{\dfrac{13}{72}}x^{2}y^{5}z^{3}}:\tonda{-{\dfrac{26}{27}}xyz}\)
\item \((-\dfrac{5}{6}a^{7}):(8a^{7})\)
\item \(20ax^{4}y:(2xy)\)
\item \(-72a^{4}b^{2}y^{2}:(-3ab^{2})\)
\item \(48a^{5}bx:a^{2}b\)
\item \(\tonda{\dfrac{1}{2}a^{3}}:(-4a^{5})\)
\item \(\tonda{-{\dfrac{12}{2}}a^{7}b^{5}c^{2}}:(-18ab^{4}c)\)
%  \item 
% \(\quadra{-\tonda{-{\dfrac{1}{3}}x^{3}y^{2}}^{2}:\tonda{-{\dfrac{1}{3}}
% }}^{2}:(x^{3}y^{2})^{2}\)
\item 
\(\quadra{\dfrac{3}{5}x^{4}:\tonda{\dfrac{1}{3}x^{4}}}\cdot\quadra{x^{4}
:\tonda{\dfrac{4}{5}x^{4}}}\)
%  \item \(\tonda{\dfrac{2}{3}ab^{2}c}^{2}:(-3ab^{3})\)
\end{enumeratea}
\end{htmulticols}
\end{esercizio}

% %\subsubsection*{9.6 - Addizione di due monomi simili}
% \subsubsection*{\numnameref{subsubsec:monomi_addizione}}

\begin{esercizio}
\label{ese:9.24}
Determina la somma dei monomi 
simili~\(8a^{2}b+(-{\dfrac{2}{3}})a^{2}b+\dfrac{1}{6}a^{2}b\)\\
La somma è un monomio \dotfill agli
addendi; il suo coefficiente è dato da\\
\(8-\dfrac{2}{3}+\dfrac{1}{6}=\ldots \), 
la parte letterale è~\(\dotfill~\) 
quindi la somma è~\(\dotfill\)
\end{esercizio}

\begin{esercizio}
\label{ese:9.25}
Determina la somma~\(S=2a-3ab-a+17ab+41a\)

I monomi addendi non sono tra loro simili, modifico la scrittura
dell'operazione applicando le proprietà associativa e commutativa
in modo da affiancare i monomi simili:
\[
S=2a-3ab-a+17ab+41a=(\ldots\ldots\ldots)+
(\ldots\ldots\ldots)=\ldots\ldots\ldots\]
La somma ottenuta non è un \ldots\ldots\ldots\ldots\ldots
\end{esercizio}

\begin{esercizio}
\label{ese:9.26}
Esegui la somma algebrica dei seguenti monomi.
\begin{htmulticols}{3}
\begin{enumeratea}
\spazielenx
\item \(6x+2x-3x\)
\item \(-3a+2a-5a\)
\item \(5a^{2}b-3a^{2}b\)
\item \(a^{2}b^{2}-3a^{2}b^{2}\)
\item \(2xy-3xy+xy\)
\item \(2y^{2}-3y^{2}+7y^{2}-4y^{2}\)
% \end{enumeratea}
% \end{htmulticols}
% \end{esercizio}
% 
% \begin{esercizio}
%  \label{ese:9.27}
% Esegui la somma algebrica dei seguenti monomi.
% \begin{htmulticols}{3}
% \begin{enumeratea}
\spazielenx
\item \(-2xy^{2}+xy^{2}\)
\item \(-3ax-5ax\)
\item \(5ab-2ab\)
\item \(-3xy^{2}+3xy^{2}\)
\item \(7xy^{3}-2xy^{3}\)
\item \(+2xy^{2}-4xy^{2}\)
% \end{enumeratea}
% \end{htmulticols}
% \end{esercizio}
% 
% \begin{esercizio}
%  \label{ese:9.28}
% Esegui la somma algebrica dei seguenti monomi.
% \begin{htmulticols}{2}
% \begin{enumeratea}
\spazielenx
\item \(\dfrac{1}{2}a^{2}-a^{2}\)
\item \(+2xy^{2}-4xy^{2}+xy^{2}\)
\item \(-5x^{2}+3x^{2}\)
\item \(\dfrac{1}{2}a+2a\)
\item \(5a^{2}b+2a^{2}b+a^{2}b-3a^{2}b-a^{2}b\)
\item \(0,1x-5x-1,2x+3x\)
% \end{enumeratea}
% \end{htmulticols}
% \end{esercizio}
% 
% \begin{esercizio}
%  \label{ese:9.29}
% Esegui la somma algebrica dei seguenti monomi.
% \begin{htmulticols}{2}
% \begin{enumeratea}
\spazielenx
\item \(\dfrac{1}{4}a^{3}b^{2}-\dfrac{1}{2}a^{3}b^{2}\)
\item \(\dfrac{2}{3}x-\dfrac{2}{5}x-2x+\dfrac{3}{10}x\)
\item 
\(\dfrac{2}{5}ab-\dfrac{1}{2}ab+\dfrac{27}{2}ab-\dfrac{1}{10}ab-
\dfrac{5}{2}ab\)
\item \(-\tonda{-{\dfrac{1}{2}}ax^{2}}-3ax^{2}\)
\item \(-{\dfrac{9}{2}}xy-(-xy)\)
\item \(2xy^{2}-\dfrac{3}{2}xy^{2}-xy^{2}\)
\end{enumeratea}
\end{htmulticols}
\end{esercizio}

\pagebreak %--------------------------------------------------

\begin{esercizio}
\label{ese:9.30}
Esegui la somma algebrica dei seguenti monomi.
\begin{htmulticols}{2}
\begin{enumeratea}
\spazielenx
\item \(\dfrac{1}{2}a+2a+(2a-a)-\tonda{3a-\dfrac{1}{2}a}\)
\item \(6xy^{2}+\dfrac{1}{3}xy^{2}-\dfrac{1}{4}xy^{2}-6xy^{2}\)
\item \(\dfrac{1}{2}xy^{2}+\dfrac{3}{2}xy^{2}\)
\item \(\tonda{\dfrac{2}{3}a+a}-\tonda{\dfrac{2}{3}a-a}\)
\item \(5ab-2ab+(-ab)-(+2ab)+ab\)
\item \(-1,2x^{2}+0,1x^{2}+(-5x)^{2}-(-25x)^{2}\)
\end{enumeratea}
\end{htmulticols}

\end{esercizio}

% \begin{esercizio}
%  \label{ese:9.31}
% Esegui le operazioni indicate.
% \begin{htmulticols}{2}
% \begin{enumeratea}
% \spazielenx
%  \item \(6ab-\dfrac{1}{3}a^{2}+\dfrac{1}{2}ab+4a^{2}\)
%  \item 
% 
% \(\tonda{\dfrac{1}{4}x^{2}-\dfrac{3}{4}x^{2}+x^{2}}-
% \tonda{-{\dfrac{1}{3}}x^{2}+\dfrac{1}{2}x^{2}}\)
%  \item 
% \(-{\dfrac{4}{3}}a^{2}b^{3}-2a^{2}b^{3}+\dfrac{1}{3}a^{2}b^{3}-a^{2}
% b^{3}\)
%  \item
% \(\tonda{-xy}^{2}\tonda{-{\dfrac{1}{2}}xy^{2}}+
% \dfrac{3}{2}xy^{2}\tonda{-\dfrac{1}{6}xy}^{2}\)
% \end{enumeratea}
% \end{htmulticols}
% \end{esercizio}

\begin{esercizio}
\label{ese:9.32}
Esegui la somma algebrica dei seguenti monomi.

\begin{enumeratea}
\spazielenx
\item 
\(\dfrac{1}{2}x^{2}-2x^{2}-\tonda{-{\dfrac{1}{2}}x^{2}+
\dfrac{3}{4}x^{2}-2x^{2}-\dfrac{3}{5}x^{2}}\)
\item 
\(5x^{3}y^{2}+\tonda{-{\dfrac{1}{3}}x^{3}y^{2}}+
\tonda{-{\dfrac{1}{3}}}-\tonda{x^{3}y^{2}}+
\tonda{-{\dfrac{1}{4}}x^{3}y^{2}}-\tonda{-{\dfrac{1}{3}}}\)
\item 
\(\tonda{2xy^{2}-\dfrac{3}{2}xy^{2}}-\tonda{xy^{2}+2xy^{2}-4xy^{2}
}+\tonda{xy^{2}+\dfrac{1}{2}xy^{2}}\)
\end{enumeratea}
\end{esercizio}

%*{9.7 - Espressioni con i monomi}
\subsubsection*{\numnameref{subsec:monomi_espressioni}}

\begin{esercizio}[*]
\label{ese:9.33}
Esegui le operazioni tra monomi.

\begin{enumeratea}
\spazielenx
\item 
\(\tonda{\dfrac{1}{2}a^{2}-a^{2}}\tonda{\dfrac{1}{2}a+2a}+
(2a-a)\tonda{3a-\dfrac{1}{2}a}a\)
\sol{\dfrac{5}{4}a^3}
\item 
\(\tonda{\dfrac{2}{3}a-\dfrac{5}{2}a}a+\tonda{7a-\dfrac{1}{3}a}^{2}:2\)
\sol{\dfrac{389}{6}a^2}
\item 
\(\dfrac{1}{2}x^{2}\tonda{x^{2}+\dfrac{1}{2}x^{2}}-\dfrac{1}{6}x^{3}
\tonda{12x-\dfrac{18}{5}x}\)
\sol{-\dfrac{13}{20}x^4}
\item 
\(\tonda{-{\dfrac{3}{4}}x^{4}a^{2}b}:\tonda{\dfrac{1}{2}x^{2}ab}+
\dfrac{2}{3}x^{2}a\)
\sol{-\dfrac{5}{6}x^2a \stext{se} a,~b,~x \ne 0}
\item 
\(\tonda{\dfrac{1}{2}a-\dfrac{1}{4}a}^{2}:\tonda{\dfrac{3}{2}a-2a}\)
\sol{-\dfrac{1}{8}a \stext{se} a \ne 0}
\item 
\((3a-2a)(2x+2x):2a\)
\sol{2x \stext{se} a \ne 0}
\item 
\(\tonda{\dfrac{1}{4}x^{2}-\dfrac{2}{3}x^{2}+x^{2}}
\tonda{-{\dfrac{1}{3}}x+\dfrac{1}{2}x}\)
\sol{\dfrac{7}{72}x^{3}}
\item 
\(\tonda{\dfrac{1}{5}x-\dfrac{5}{2}x+x}-
\tonda{2x-\dfrac{8}{3}x+\dfrac{1}{4}x+x}-\dfrac{7}{60}x\)
\sol{-2x}
\item 
\(5a+\graffa{-{\dfrac{3}{4}}a-
\quadra{2a-\dfrac{1}{2}a+\tonda{3a-a}+0,5a}-a}\)
\sol{-\dfrac{3}{4}a}
\item 
\(-12x^{2}\tonda{\dfrac{1}{3}x}^{2}+
\quadra{0,1x^{2}\tonda{-5x}^{2}-\tonda{- x^{2}}^{2}}\)
\sol{\dfrac{1}{6}x^{4}}
%  \item 
% 
% \(-{\dfrac{3}{5}}x^{2}y^{2}\tonda{-{\dfrac{10}{9}}xz^{2}}(-15xy)-0,\
% overline{
% 6}x^{4}yz\tonda{-0,\overline{7}xy^{2}z}\)
%   \hfill[]
%  \item 
% 
% 
% \(\dfrac{1}{2}ab^{2}c+\quadra{\dfrac{3}{4}a^{3}b^{6}c^{3}-
% \tonda{-{\dfrac{1}{4} 
% ab ^{2
% 
% }c}}^{3}-\tonda{-{\dfrac{1}{2}ab^{2}}}^{2}\tonda{-{\dfrac{1}{16}ab^{
% 2} c^
% {3}}}}:%
%  \tonda{-{\dfrac{5}{4}a^{2}b^{4}c^{2}}}\)
%   \hfill\(\quadra{-\dfrac{1}{8}ab^{2}c}\)
% \end{enumeratea}
% \end{esercizio}
% 
% \begin{esercizio}[*]
%  \label{ese:9.35}
% Esegui le operazioni tra monomi.
% 
% \begin{enumeratea}
\item 
\(\tonda{2xy^{2}-\dfrac{3}{2}xy^{2}}-\tonda{xy^{2}+2xy^{2}-4xy^{2}
}+\tonda{xy^{2}+\dfrac{1}{2}xy^{2}}\)
\sol{3xy^{2}}
\item 
\(\dfrac{1}{4}x^{4}y^{2}-
\quadra{\dfrac{3}{2}x^{5}y^{4}:\tonda{\dfrac{1}{2}xy}^{2}-3x^{3}y^{2}}
\tonda{-{\dfrac{1}{3}}x} +
\tonda{-{\dfrac{1}{2}}x^{2}y}^{2}\)
\sol{\dfrac{3}{2}x^{4}y^{2}}
\item 
\(a^{2}-\graffa{a-\quadra{2\tonda{\dfrac{a}{2}-\dfrac{a}{3}}}}^{2} +
\tonda{\dfrac{2}{3}a+a}\tonda{\dfrac{2}{3}a-a}\)
\sol{0}
%  \item 
% 
% \(\quadra{\tonda{-{\dfrac{1}{2}}a^{2}b}^{2}\cdot\tonda{-{\dfrac{2}{3}}
% b^{2}}^{2}-%
% 
% \tonda{+{\dfrac{1}{3}}b^{3}a^{2}}^{2}}:\tonda{\dfrac{2}{3}a-\dfrac{1
% }{ 6}
% a+\dfrac{1}{2}a}+\tonda{-{\dfrac{1}{6}}ab^{2}}^2%
%  \tonda{-{\dfrac{2}{5}}ab^2}\)
%   \hfill\(\quadra{-\dfrac{1}{90}a^{3}b^{6}}\)
%  \item \begin{multline*}
%  
% \quadra{\tonda{2x+\dfrac{7}{4}x}^{2}:\tonda{\dfrac{1}{3}x+x+\dfrac{3}{4}
% x}}^{2}%
% :\tonda{18x-\dfrac{9}{2}x+\dfrac{27}{2}x}\\
%  
% 
% +\quadra{\tonda{-{\dfrac{2}{3}}abx}^{2}-\tonda{\dfrac{1}{3}abx}^{2}}
% :(a^{2}b^{2}x)-x;
%  \end{multline*}
%  \item \begin{multline*}
%  
% 
% \tonda{\dfrac{1}{4}xy^{2}}\tonda{-{\dfrac{16}{5}}x^{2}y}-8x^{2}y^{2}
% \tonda{-2xy}%
%  -\dfrac{2}{5}x\tonda{-{\dfrac{5}{3}}x^{2}}\tonda{+3y^{3}}\\
%  
% 
% +\tonda{\dfrac{12}{7}xy^{2}}\tonda{-{\dfrac{7}{4}}x^{2}y}+\dfrac{9}{
% 5} x^
% {3}y^{3}.
%  \end{multline*}
% \end{enumeratea}
% \end{esercizio}
% 
% 
% \begin{esercizio}[*]
%  \label{ese:9.36}
% Esegui le operazioni tra monomi.
% 
% \begin{enumeratea}
%  \item 
% 
% \(\dfrac{2}{3}a^{2}b-\quadra{3a-\dfrac{1}{3}a^{2}b-\tonda{\dfrac{2}{5}a+
% \dfrac{1} {2}
% a-3a}+\tonda{\dfrac{2}{5}a^{2}b+\dfrac{1}{2}a^{2}b-2a^{2}b}}%
%  -\dfrac{1}{10}a^{2}b+\dfrac{51}{10}a\)
%   \sol{2a^{2}b}
\item 
\(\tonda{\dfrac{1}{3}x+\dfrac{1}{2}x-2x}\tonda{-{\dfrac{1}{2}x^{2}}}+
\tonda{\dfrac{3}{4}x^{2}-2x^{2}}\tonda{-{\dfrac{3}{5}x}}%
-\dfrac{4}{3}\tonda{x^{3}+\dfrac{1}{2}x^{3}}\)
\sol{-\dfrac{2}{3}x^{3}}
%  \item \begin{multline*}
%  
% 
% 
% \quadra{\dfrac{3}{5}ab^{2}+\dfrac{1}{2}b-
% ab^{2}\cdot\tonda{-{\dfrac{3}{10}}+\dfrac{4}{5}-\dfrac{1}{2}}-
% 2b+\dfrac{3}{2}b+\dfrac{1}{15}ab^{2}}^{2}\\%
% % 
% 
% :\quadra{\tonda{b+\dfrac{3}{2}b}^{2}-\dfrac{5}{10}b^{2}+
%  \dfrac{1}{2}b^{2}}\cdot\tonda{-{\dfrac{5}{2}ab}}^{2};
%  \end{multline*}
\item 
\(\quadra{\tonda{\dfrac{3}{2}xy}^{2}\cdot\dfrac{4}{15}y}^{2}-
\tonda{\dfrac{3}{2}xy^{2}}^{2}\cdot\tonda{\dfrac{2}{3}}^{3}%
+\dfrac{8}{75}x^{2}y^{4}:\tonda{\dfrac{10}{3}x^{2}y}\)
\sol{-\dfrac{3}{25}y^{3}}
\item 
\(\tonda{\dfrac{1}{2}x+2x}\tonda{\dfrac{1}{2}x-2x}
\tonda{\dfrac{1}{4}x^2-4x^{2}} - 16\tonda{x \cdot x^{3}}\)
\sol{-\dfrac{31}{16}x^{4}}
\end{enumeratea}
\end{esercizio}


\begin{esercizio}
\label{ese:9.37}
Dati i monomi:\quad
\(m_{1}=\dfrac{3}{8}a^{2}b^{2}\),\quad 
\(m_{2}=-{\dfrac{8}{3}}ab^{3}\),\quad 
\(m_{3}=-3a\),\quad 
\(m_{4}=-{\dfrac{1}{2}}b\), \quad \(m_{5}=2b^{3}\),\\
calcola il risultato delle seguenti operazioni, 
ponendo le opportune~\(\CE\):

\vspace{-0.8em}
\begin{htmulticols}{3}
\begin{enumeratea}
\item \(m_{1}\cdot m_{2}\cdot (m_{4})^{2}\)
\item \(-m_{2}\cdot m_{1}\cdot (m_{3})^{2}\cdot m_{5}\)
\item \((m_{3}\cdot m_{4})^{2}-m_{1}\)
\item \(m3\cdot m_{5}-m_{2}\)
\item \(m_{2}:m_{3}+m_{5}\)
\item \(m_{1}:m_{2}\)
\end{enumeratea}
\end{htmulticols}
\end{esercizio}


\begin{esercizio}
\label{ese:9.38}
Quando sottraiamo due monomi opposti otteniamo:

\vspace{-0.8em}
\begin{htmulticols}{2}
\begin{enumeratea}
\item il doppio del primo termine;
\item il doppio del secondo termine;
\item il monomio nullo;
\item 0.
\end{enumeratea}
\end{htmulticols}
\end{esercizio}

\begin{esercizio}
\label{ese:9.39}
Quando dividiamo fra loro due monomi opposti otteniamo:
\begin{center}
\fbox{A}\quad\(-1\)
\qquad\fbox{B}\quad~0
\qquad\fbox{C}\quad~1
\qquad\fbox{D}\quad il quadrato del primo monomio
\end{center}
\end{esercizio}

\begin{esercizio}
\label{ese:9.40}
Attribuisci il valore di verità alle seguenti proposizioni:
\TabPositions{11cm}
\begin{enumeratea}
\item la somma di due monomi opposti è il monomio nullo \hfill\verofalso
\item il quoziente di due monomi simili è il quoziente dei loro 
coefficienti 
\hfill\verofalso
\item la somma di due monomi è un monomio \hfill\verofalso
\item il prodotto di due monomi è un monomio \hfill\verofalso
\item l'opposto di un monomio ha sempre il coefficiente negativo 
\hfill\verofalso
\end{enumeratea}
\end{esercizio}

\begin{esercizio}
\label{ese:9.41}
Un quadrato è formato da \(16\) piastrelle quadrate, tutte di lato \(2x\). 
Determina perimetro e area del quadrato. \sol{32x;\quad 64x^2}
\end{esercizio}

\begin{esercizio}
\label{ese:9.42}
Di un triangolo equilatero di lato \(l\) si triplicano due lati e 
si dimezza il terzo lato, che triangolo si ottiene? 
Qual'è la differenza tra i perimetri dei due triangoli? 
\sol{\dfrac{7}{2}l}
\end{esercizio}

%\subsubsection*{9.8 - Massimo Comune Divisore e minimo comune multiplo 
% tra monomi}
\subsubsection*{\numnameref{subsec:monomi_mcdemcm}}

\begin{esercizio}
\label{ese:9.43}
Vero o falso?

\TabPositions{10cm}
\begin{enumeratea}
\spazielenx
\item \(12a^{3}b^{2}c\) è un multiplo di~\(abc\) \hfill\verofalso
\item \(2xy\) è un divisore di~\(x^{2}\) \hfill\verofalso
\item \(2a\) \ è divisore di~\(4ab\) \hfill\verofalso
\item \(-5b^{2}\) è divisore di~\(15ab\) \hfill\verofalso
\item \(8ab\) è multiplo di~\(a^{2}b^{2}\) \hfill\verofalso
\item \(12a^{5}b^{4}\) è multiplo di~\(60a^{5}b^{7}\) \hfill\verofalso
\item \(5\) è divisore di~\(15a\) \hfill\verofalso
\end{enumeratea}
\end{esercizio}

\begin{esercizio}
\label{ese:9.44}
Vero o falso?

\TabPositions{10cm}
\begin{enumeratea}
\spazielenx
\item il~\(\mcm\) fra monomi è divisibile per tutti i monomi dati 
\hfill\verofalso
\item il~\(\mcd\) fra monomi è multiplo di almeno un monomio dato 
\hfill\verofalso
\item il~\(\mcm\) è il prodotto dei monomi tra di loro \hfill\verofalso
\end{enumeratea}
\end{esercizio}

% \newpage %------------------------------------------------------

\begin{esercizio}
\label{ese:9.45}
Calcola il~\(\mcm\) e il~\(\mcd\) dei seguenti gruppi di monomi.
\TabPositions{15mm, 30mm}
% \begin{htmulticols}{2}
\begin{enumeratea}
\spazielenx
\item \(18x^{3}y^{2}\),\tab \(xy\),\tab \(4x^{3}y^{4}\) 
\sol{36x^{3}y^{4};\quad 2xy}
\item \(5xyz^{5}\),\tab \(x^{3}y^{2}z^{2}\) 
\sol{5x^{3}y^{2}z^{5};\quad xyz^{2}}
\item \(4ab^{2}\),\tab \(2a^{3}b^{2}\),\tab \(10ab^{5}\) 
\sol{20a^{3}b^{5};\quad 2ab^{2}}
\item \(2a^{2}bc^{3}\),\tab \(ab^{4}c^{2}\),\tab \(24a^{3}bc\)
\sol{24a^3b^4c^3;\quad a^2bc}
\item \(6a^{2}x\),\tab \(2ax^{3}\),\tab \(4x^{2}c^{3}\)
\sol{24a^2c^3x^3;\quad 2ax}
\item \(30ab^{2}c^{4}\),\tab \(5a^{2}c^{3}\),\tab \(15abc\)
\sol{30a^2b^2c^4;\quad 5ac}
\item \(x^{2}y^{4}z^{2}\),\tab \(xz^{3}\),\tab \(24y^{2}z\)
\sol{24x^2y^4z^3;\quad z}
\item \(4a^{2}y\),\tab \(y^{3}c\),\tab \(15ac^{5}\)
\sol{60a^2c^5y^3;\quad }
\item \(15xyc^{2}\),\tab \(x^{2}y^{3}c^{2}\),\tab \(6c^{4}\)
\sol{30xc^4x^2y^3;\quad xc^2}
% \end{enumeratea}
% % \end{htmulticols}
% \end{esercizio}
% 
% \begin{esercizio}
%  \label{ese:9.48}
% Calcola il~\(\mcm\) e il~\(\mcd\) dei seguenti gruppi di monomi.
% 
% \begin{enumeratea}
% \spazielenx
\item \(-3xy^{3}z\),\tab \(-6x^{3}yz\),\tab \(8x^{3}z\) 
\sol{24x^{3}y^{3}z;\quad 2xz}
\item \(\dfrac{1}{6}ab^{2}c\),\tab \(-2a^{2}b^{2}c\),\tab 
\(-{\dfrac{1}{3}}ab^{2}c^{2}\) 
\sol{a^{2}b^{2}c^{2};\quad ab^{2}c}
\item \(\dfrac{2}{4}x^{2}y^{2}\),\tab \(\dfrac{1}{8}xy^{2}\),\tab 
\(\dfrac{5}{6}xyz^{2}\) 
\sol{x^{2}y^{2}z^{2};\quad xy}
\item \(a^{4n}b^{m+4}\),\tab \(a^{3n}b^{m+3}\),\tab 
\(a^{n}b^{m}z^{2m+1}\)
\sol{a^{4n}b^{m+4}z^{2m+1};\quad a^{n}b^{m}}
\end{enumeratea}
\end{esercizio}

\begin{esercizio}
\label{ese:9.49}
Dati i monomi~\(3xy^{2}\) e~\(xz^{3}\)

\begin{enumeratea}
\spazielenx
\item calcola il loro~\(\mcd\)
\item calcola il loro~\(\mcm\)
\item verifica che il loro prodotto è uguale al prodotto tra il 
loro~\(\mcm\) e il loro~\(\mcd\)
\item verifica che il loro~\(\mcd\) è uguale al quoziente tra il loro 
prodotto e il loro~\(\mcm\)
\end{enumeratea}
\end{esercizio}

% \subsection{Risposte}

% \paragraph{\ref{ese:9.33}} 
% d)~\(-\dfrac{5}{6}ax^{2}\)
% \paragraph{\ref{ese:9.34}} 
% a)~\(\dfrac{7}{72}x^{3}\),\quad b)~\(-2x\), \quad c)~\(-\dfrac{3}{4}a\), 
% \quad 
% d)~\(\dfrac{1}{6}x^{4}\), \quad f)~\(-\dfrac{1}{8}ab^{2}c\)
% \paragraph{\ref{ese:9.35}} 
% a)~\(3xy^{2}\),\quad b)~\(\dfrac{3}{2}x^{4}y^{2}\), \quad c)~0, \quad 
% d)~\(-\dfrac{1}{90}a^{3}b^{6}\), %\quad e)~\(\dfrac{49}{48}x\), \quad 
% f)~\(16x^{3}y^{3}\)
% \paragraph{\ref{ese:9.36}} 
% a)~\(2a^{2}b\),\quad b)~\(-\dfrac{2}{3}x^{3}\), \quad 
% %c)~\(\dfrac{4}{9}a^{4}b^{4}\), \quad 
% d)~\(-\dfrac{3}{25}y^{3}\)
% \paragraph{\ref{ese:9.41}} 
% \(24x\) \( 36x^2 \)
% \paragraph{\ref{ese:9.42}} 
% \(\dfrac{3}{2}a\)
% \paragraph{\ref{ese:9.45}} 
% a)~\(28x^{3}y^{4}; xy\),\quad b)~\(x^{3}y^{2}z^{5}; xyz^{2}\), \quad 
% c)~\(20a^{3}b^{5}; ab^{2}\)
% \paragraph{\ref{ese:9.48}} 
% a)~\(a^{4n}b^{m+4}z^{2m+1}; a^{n}b^{m}\),\quad b)~\(24x^{3}y^{3}z; 
% 2xz\),\quad 
% c)~\(a^{2}b^{2}c^{2};ab^{2}c\),\quad d)~\(x^{2}y^{2}z^{2};xy\)

% ------------------- Polinomi ------------------

%\subsubsection*{10.1 - Definizioni fondamentali}
\subsubsection*{\numnameref{subsec:poli_polinomi}}

%\subsubsection*{10.1 - Definizioni fondamentali}
\subsubsection*{\numnameref{subsec:poli_definizioni}}

\begin{esercizio}
\label{ese:10.1}
Riduci in forma normale il seguente polinomio:
\[5a^3-4ab-1+2a^3+2ab-a-3a^3.\]
\emph{Svolgimento}: Evidenziamo i termini simili e sommiamoli tra di loro:
\[\uline{5a^3}-\uuline{4ab}+1+\uline{2a^3}+\uuline{2ab}-a-\uline{3a^3}\]
in modo da ottenere \dotfill Il termine noto è \dotfill
\end{esercizio}

\begin{esercizio}
\label{ese:10.2}
Il grado di:
\begin{enumeratea}
\item \(x^2y^2-3y^3+5yx-6y^2x^3\) rispetto alla lettera~\(y\) è \dotfill, 
il grado complessivo è \dotfill
\item \(5a^2-b+4ab\) rispetto alla lettera~\(b\) è \dotfill, il grado 
complessivo è \dotfill
\end{enumeratea}
\end{esercizio}

\begin{htmulticols}{2}
\begin{esercizio}
\label{ese:10.3}
Stabilire quali dei seguenti polinomi sono omogenei:

\begin{enumeratea}
\item \(x^3y+2y^2x^2-4x^4\)
\item \(2x+3-xy\)
\item \(2x^3y^3-y^4x^2+5x^6\)
\end{enumeratea}
\end{esercizio}

\begin{esercizio}
\label{ese:10.4}
Individuare i polinomi ordinati con potenze crescenti:

\begin{enumeratea}
\item \(2-6x^2+x\)
\item \(8-x+3x^2+5x^3\)
\item \(3x^4-x^3+2x^2-x+7\)
\end{enumeratea}
\end{esercizio}

\begin{esercizio}
\label{ese:10.6}
Scrivere un polinomio di terzo grado nelle variabili~\(a\) e~\(b\) che sia 
omogeneo, completo e ordinato.
\end{esercizio}

\begin{esercizio}
\label{ese:10.5}
Il polinomio~\(b^2+a^4+a^3+a^2\):
\begin{itemize} [nosep]
\item ha grado massimo è \ldots~ il grado rispetto alla lettera~\(a\) è 
\ldots ~
rispetto alla lettera~\(b\) è \ldots
\item è ordinato rispetto alla a? \hfill\verofalso
\item è completo? \hfill\verofalso
\item è omogeneo? \hfill\verofalso
\end{itemize}
\end{esercizio}

\begin{esercizio}
\label{ese:10.7}
Scrivere un polinomio di terzo grado nelle variabili~\(a\) e~\(b\) che sia 
omogeneo ma non completo.
\end{esercizio}

\begin{esercizio}
\label{ese:10.7}
Scrivere un polinomio di quarto grado nelle variabili~\(x\) e~\(y\) che sia 
omogeneo e ordinato secondo le potenze decrescenti della seconda variabile.
\end{esercizio}

\begin{esercizio}
\label{ese:10.8}
Scrivere un polinomio di quinto grado nelle variabili~\(r\) e~\(s\) che sia 
omogeneo e ordinato secondo le potenze crescenti della prima variabile.
\end{esercizio}

\begin{esercizio}
\label{ese:10.9}
Scrivere un polinomio di quarto grado nelle variabili~\(z\) e~\(w\) che sia 
omogeneo e ordinato secondo le potenze crescenti della prima variabile.
\end{esercizio}

\begin{esercizio}
\label{ese:10.10}
Scrivere un polinomio di sesto grado nelle variabili~\(x\), \(y\) e~\(z\) 
che sia completo e ordinato secondo le potenze decrescenti della seconda 
variabile.
\end{esercizio}

\end{htmulticols}

%\subsubsection*{10.2 - Somma algebrica di polinomi}
\subsubsection*{\numnameref{subsec:poli_somma}}

\begin{esercizio}
\label{ese:10.12}
Calcolare la somma dei due polinomi:\quad 
\(2x^2+5-3y^2x\),\quad \(x^2-xy+2-y^2x+y^3\)\\
indichiamo la somma: \quad 
\((2x^2+5-3y^2x)+(x^2-xy+2-y^2x+y^3)\), \\
eliminiamo le parentesi: \quad
\(\uline{2x^2}+\dotuline{5}-\uuline{3y^2x}+\uuline{x^2}-xy+\dotuline{2}-
\uuline{y^2x}+y^3\), \\
sommando i monomi simili otteniamo: \quad 
\(3x^2-4x^{\ldots}y^{\ldots}-\ldots xy+y^3+\ldots\)
\end{esercizio}

% \pagebreak %-------------------------------------------------

\begin{esercizio}
\label{ese:10.13}
Somma le seguenti coppie di polinomi.
\begin{htmulticols}{2}
\begin{enumeratea}
\spazielenx
\item \(a+b-b\),\quad \(a+b-2b\)
\item \(a+b-(-2b)\),\quad \(a-(b-2b)\)
\item \(2a+(3a+b)\),\quad \(2a+2b+(2a+b)+2a\)
\item \(2a+b-(-3a-b)\),\quad \(-3b-(-3b-2a)\)
\end{enumeratea}
\end{htmulticols}
\end{esercizio}

\begin{esercizio}[*]
\label{ese:10.14}
Esegui le seguenti somme di polinomi.

\begin{enumeratea}
\spazielenx
\item 
\(\tonda{2a^{2}-3b}+\tonda{4b+3a^{2}}+\tonda{a^{2}-2b}\)
\sol{6a^2-b}
\item 
\(\tonda{3a^{3}-3b^{2}}+\tonda{6a^{3}+b^{2}}+\tonda{a^{3}-b^{2}}\)
\sol{10a^3-3b^2}
\item 
\(\tonda{\dfrac{1}{5}x^{3}-5x^{2}+\dfrac{1}{5}x-1}-
\tonda{3x^{3}-\dfrac{7} {3}x^{2}+\dfrac{1}{4}x-1}\)
\sol{-\dfrac{14}{5}x^3-\dfrac{8}{3}x^2-\dfrac{1}{20}x}
\item 
\(\tonda{\dfrac{1}{2}+2a^{2}}-\tonda{\dfrac{2}{5}a^{2}+\dfrac{1}{2}{ax}}+
\quadra{-\tonda{-{\dfrac{3}{2}}-2{ax}+x^{2}}+\dfrac{1}{3}a^{2}}
-\dfrac{3}{2}{ax}\)
\sol{-x^{2}+\dfrac{29}{15}a^{2}-2}
\item 
\(\tonda{\dfrac{3}{4}a+\dfrac{1}{2}b-\dfrac{1}{6}{ab}}-
\tonda{\dfrac{9}{8}{ab}+\dfrac{1}{2}a^{2}-2b}+{ab}-\dfrac{3}{4}a\)
\sol{-{\dfrac{a^{2}}{2}}-\dfrac{7}{24}ab+\dfrac{5}{2}b}
\end{enumeratea}
\end{esercizio}

\pagebreak %-------------------------------------------------

%\subsubsection*{10.3 - Prodotto di un polinomio per un monomio}
\subsubsection*{\numnameref{subsec:poli_prodottopermonomio}}

\begin{esercizio}
\label{ese:10.15}
Esegui i seguenti prodotti di un monomio per un polinomio.
\begin{htmulticols}{3}
\begin{enumeratea}
\spazielenx
\item \((a + b)b\)
\item \((a - b)b\)
\item \((a +b)(-b)\)
\item \((a - b + 51)b\)
\item \((-a - b -51)(-b)\)
\item \((a^{2} - a)a\)
\item \((a^{2} - a)(-a)\)
\item \((a^{2}- a - 1)a^{2}\)
\item \((a^{2}b-ab - 1)(ab)\)
\item \((ab- ab - 1)(ab)\)
\item \((a^{2}b- ab -1)(a^{2}b^{2})\)
\item \((a^{2}b-ab - 1)(ab)^{2}\)
\item \(ab(a^{2}b- ab -1)ab\)
\item \(-2a(a^{2} - a - 1)(-a^{2})\)
\item \((x^{2}a- ax+2)(2x^{2}a^{3})\)
\end{enumeratea}
\end{htmulticols}
\end{esercizio}

\begin{esercizio}
\label{ese:10.16}
Esegui i seguenti prodotti di un monomio per un polinomio.
\begin{htmulticols}{2}
\begin{enumeratea}
\spazielenx
\item \(\dfrac{3}{4}x^{2}y\cdot\tonda{2{xy}+\dfrac{1}{3}x^{3}y^{2}}\)
\item 
\(\tonda{\dfrac{a^{4}}{4}+\dfrac{a^{3}}{8}+
\dfrac{a^{2}}{2}}\tonda{2a^{2}}\)
\item \(\tonda{\dfrac{1}{2}a-3+a^{2}}\tonda{-{\dfrac{1}{2}}a}\)
\item \(\tonda{5x+3{xy}+\dfrac{1}{2}y^{2}}\tonda{3x^{2}y}\)
\item 
\(\tonda{\dfrac{2}{3}xy^{2}+\dfrac{1}{2}x^{3}-\dfrac{3}{4}{xy}}(6{xy})\)
\item \(-\dfrac{1}{3}y\tonda{6x^{2}y-3{xy}}\)
\item \(-3xy^2\tonda{\dfrac{1}{3}x+1}\)
\item \(\tonda{\dfrac{7}{3}b-b}\tonda{a-\dfrac{1}{2}b+1}(3a-2a)\)
\end{enumeratea}
\end{htmulticols}
\end{esercizio}
%\newpage

% \pagebreak %-------------------------------------------------

%\subsubsection*{10.4 - Quoziente tra un polinomio e un monomio}
\subsubsection*{\numnameref{subsec:poli_quozientepermonomio}}

\begin{esercizio}
\label{ese:10.17}
Svolgi le seguenti divisioni tra polinomi e monomi.
\begin{htmulticols}{2}
\begin{enumeratea}
\spazielenx
\item \(\tonda{2x^{2}y+8{xy}^{2}}:\tonda{2{xy}}\)
\item \(\tonda{a^{2}+a}:a\)
\item \(\tonda{a^{2}-a}:(-a)\)
\item \(\tonda{\dfrac{1}{2}a-\dfrac{1}{4}}:\dfrac{1}{2}\)
\item \(\tonda{\dfrac{1}{2}a-\dfrac{1}{4}}:2\)
\item \((2a-2):\dfrac{1}{2}\)
\item \(\tonda{\dfrac{1}{2}a-\dfrac{a^{2}}{4}}:\dfrac{a}{2}\)
% \end{enumeratea}
% \end{htmulticols}
% \end{esercizio}
% 
% \begin{esercizio}
% \label{ese:10.18}
%  Svolgi le seguenti divisioni tra polinomi e monomi.
%  \begin{htmulticols}{2}
% \begin{enumeratea}
% \spazielenx
\item \(\tonda{a^{2}-a}:a\)
\item \(\tonda{a^{3}+a^{2}-a}:a\)
\item \(\tonda{8a^{3}+4a^{2}-2a}:2a\)
\item \(\tonda{a^{3}b^{2}+a^{2}b-ab}:b\)
\item \(\tonda{a^{3}b^{2}-a^{2}b^{3}-ab^{4}}:(-{ab}^{2})\)
\item \(\tonda{a^{3}b^{2}+a^{2}b-ab}:ab\)
\item \(\tonda{16x^{4}-12x^{3}+24x^{2}}:\tonda{4x^{2}}\)
\item \(\tonda{-x^{3}+3x^{2}-10x+5}:(-5)\)
% \end{enumeratea}
% \end{htmulticols}
% \end{esercizio}
% 
% % \newpage %-----------------------------------------
% 
% \begin{esercizio}
% \label{ese:10.19}
%  Svolgi le seguenti divisioni tra polinomi e monomi.
% 
% \begin{htmulticols}{2}
% \begin{enumeratea}
\item \(\tonda{a^{3}b^{2}-a^{4}b+a^{2}b^{3}}:\tonda{a^{2}b}\)
\item \(\tonda{a^{2}-a^{4}+a^{3}}:\tonda{a^{2}}\)
%  \item \(\tonda{-3a^{2}b^{3}-2a^{2}b^{2}+6a^{3}b^{2}}:(-3{ab})\)
\item \(\tonda{\dfrac{4}{3}a^{2}b^{3}-\dfrac{3}{4}a^{3}b^{2}}:
      \tonda{-{\dfrac{3}{2} a^{2}b^{2}}}\)
\item \(\tonda{2a+\dfrac{a^{2}}{2}-\dfrac{a^{3}}{4}}:
      \tonda{\dfrac{a}{2}}\)
\item \(\tonda{\dfrac{1}{2}a-\dfrac{a^{2}}{4}-\dfrac{a^{3}}{8}}:
        \tonda{\dfrac{1}{2} a}\)
\item 
\(\tonda{-4x+\dfrac{1}{2}x^{3}}\tonda{2x^{2}-3x+\dfrac{1}{2}}\)
\end{enumeratea}
\end{htmulticols}
\end{esercizio}

\pagebreak %--------------------------------------------

%\subsubsection*{10.5 - Prodotto di polinomi}
\subsubsection*{\numnameref{subsec:poli_prodotto}}

\begin{esercizio}
\label{ese:10.20}
Esegui le seguenti moltiplicazioni tra polinomi.
\begin{enumeratea}
\spazielenx
\item \(\tonda{- 3 x - 10}\tonda{7 x}\)
\sol{- 21 x^{2} - 70 x}
\item \(\tonda{2 x - 4}\tonda{x - 7}\)
\sol{2 x^{2} - 18 x + 28}
\item \(\tonda{8 x - 5}\tonda{5 x - 1}\)
\sol{40 x^{2} - 33 x + 5}
\item \(\tonda{9 x + 9}\tonda{- 7 x + 9}\)
\sol{- 63 x^{2} + 18 x + 81}
\item \(\tonda{12 x + 2}\tonda{- 7 x + 4}\)
\sol{- 84 x^{2} + 34 x + 8}
% \item \(\tonda{- 12 x - 1}\tonda{5 x - 9}\)
%   \sol{- 60 x^{2} + 103 x + 9}
% \item \(\tonda{- 7 x - 2}\tonda{4 x + 11}\)
%   \sol{- 28 x^{2} - 85 x - 22}
% \item \(\tonda{12 x - 8}\tonda{- 10 x + 5}\)
%   \sol{- 120 x^{2} + 140 x - 40}
% \item \(\tonda{11 x - 1}\tonda{6 x^{2} + 8}\)
%   \sol{66 x^{3} - 6 x^{2} + 88 x - 8}
\item \(\tonda{5 x^{2} + 6 x - 6}\tonda{x - 11}\)
\sol{5 x^{3} - 49 x^{2} - 72 x + 66}
\item \(\tonda{12 x^{2} + 5 x - 4}\tonda{5 x + 6}\)
\sol{60 x^{3} + 97 x^{2} + 10 x - 24}
\item \(\tonda{12 x^{2} + 3 x - 5}\tonda{2 x - 9}\)
\sol{24 x^{3} - 102 x^{2} - 37 x + 45}
\item \(\tonda{- 5 x^{2} + x - 7}\tonda{- 4 x - 4}\)
\sol{20 x^{3} + 16 x^{2} + 24 x + 28}
\item \(\tonda{- 2 x^{2} - 12 x + 7}\tonda{- x + 8}\)
\sol{2 x^{3} - 4 x^{2} - 103 x + 56}
% \item \(\tonda{x^{2} - 2 x - 11}\tonda{- 7 x + 12}\)
%   \sol{- 7 x^{3} + 26 x^{2} + 53 x - 132}
% \item \(\tonda{- 9 x^{2} - 12 x - 7}\tonda{- 10 x + 6}\) 
% \hfill 
% [\(90 x^{3} + 66 x^{2} - 2 x - 42}
% \item \(\tonda{6 x + 2}\tonda{- 10 x^{2} - 7 x - 12}\) \hfill 
% [\(- 60 x^{3} - 62 x^{2} - 86 x - 24}
% \item \(\tonda{12 x + 7}\tonda{- 4 x^{2} + 9 x + 6}\)
%   \sol{- 48 x^{3} + 80 x^{2} + 135 x + 42}
% \item \(\tonda{- 6 x + 12}\tonda{- 5 x^{2} + 7 x - 12}\) 
% \hfill 
% [\(30 x^{3} - 102 x^{2} + 156 x - 144}
\item \(\tonda{- 8 x + 11}\tonda{7 x^{2} - 11 x + 10}\) 
\sol{- 56 x^{3} + 165 x^{2} - 201 x + 110}
\item \(\tonda{4 x^{2} + 4 x - 8}\tonda{- 8 x^{2} - x - 2}\) 
\sol{- 32 x^{4} - 36 x^{3} + 52 x^{2} + 16}
\item \(\tonda{- 6 x^{2} - 12 x - 10}\tonda{2 x^{2} - 4 x - 9}\) 
\sol{- 12 x^{4} + 82 x^{2} + 148 x + 90}
\item \(\tonda{- 6 x^{2} + 7 x + 7}\tonda{12 x^{2} - 6 x + 6}\) 
\sol{- 72 x^{4} + 120 x^{3} + 6 x^{2} + 42}
% \item \(\tonda{4 x^{2} + 8 x - 2}\tonda{4 x^{2} + 9 x + 9}\) 
% \sol{16 x^{4} + 68 x^{3} + 100 x^{2} + 54 x - 18}
% \item \(\tonda{- 6 x^{2} - 7 x + 10}\tonda{x^{2} + 5 x - 9}\) 
% \sol{- 6 x^{4} - 37 x^{3} + 29 x^{2} + 113 x - 90}
% \item \(\tonda{- 8 x^{2} + 12 x + 8}\tonda{- 11 x^{2} - 6 x - 
% 10}\)
%   \sol{88 x^{4} - 84 x^{3} - 80 x^{2} - 168 x - 80}
\item \(\tonda{- \dfrac{x}{4} + 3}\tonda{- \dfrac{5 x}{7} - 3}\)
\sol{\dfrac{5 x^{2}}{28} - \dfrac{39 x}{28} - 9}
\item \(\tonda{\dfrac{4 x}{3} + \dfrac{1}{3}}\tonda{- x + 
\dfrac{2}{5}}\)
\sol{- \dfrac{4 x^{2}}{3} + \dfrac{x}{5} + \dfrac{2}{15}}
\item \(\tonda{x^{2} - \dfrac{12 x}{5} - 9}\tonda{- x + 
\dfrac{2}{3}}\)
\sol{- x^{3} + \dfrac{46 x^{2}}{15} + \dfrac{37 x}{5} - 6}
\item \(\tonda{- x - \dfrac{9}{11}}\tonda{\dfrac{x}{2} - 
\dfrac{3}{4}}\)
\sol{- \dfrac{x^{2}}{2} + \dfrac{15 x}{44} + \dfrac{27}{44}}
\item \(\tonda{- x - \dfrac{5}{4}}\tonda{\dfrac{7 x}{10} - 
\dfrac{1}{4}}\)
\sol{- \dfrac{7 x^{2}}{10} - \dfrac{5 x}{8} + \dfrac{5}{16}}
\item \(\tonda{\dfrac{9 x}{4} + \dfrac{1}{2}}\tonda{\dfrac{x}{6} - 
\dfrac{1}{3}}\)
\sol{\dfrac{3 x^{2}}{8} - \dfrac{2 x}{3} - \dfrac{1}{6}}
\item \(\tonda{\dfrac{3 x}{2} - \dfrac{1}{10}}\tonda{\dfrac{7 x}{9} - 
\dfrac{3}{10}}\)
\sol{\dfrac{7 x^{2}}{6} - \dfrac{19 x}{36} + \dfrac{3}{100}}
\item \(\tonda{- \dfrac{3 x}{2} - \dfrac{6}{11}}\tonda{\dfrac{11 x}{3} + 
\dfrac{5}{3}}\)
\sol{- \dfrac{11 x^{2}}{2} - \dfrac{9 x}{2} - \dfrac{10}{11}}
\item \(\tonda{- \dfrac{7 x}{6}}\tonda{\dfrac{x^{2}}{2} - \dfrac{2 x}{11} 
- \dfrac{3}{2}}\)
\sol{- \dfrac{7 x^{3}}{12} + \dfrac{7 x^{2}}{33} + \dfrac{7 x}{4}}
\item \(\tonda{\dfrac{4 x^{2}}{5} - \dfrac{3 x}{5} + 
\dfrac{3}{5}}\tonda{- 3 x - \dfrac{4}{3}}\)
\sol{- \dfrac{12 x^{3}}{5} + \dfrac{11 x^{2}}{15} - x - 
\dfrac{4}{5}}
\item \(\tonda{\dfrac{6 x^{2}}{5} + \dfrac{x}{3} - 
\dfrac{4}{3}}\tonda{x + \dfrac{7}{2}}\)
\sol{\dfrac{6 x^{3}}{5} + \dfrac{68 x^{2}}{15} - \dfrac{x}{6} - 
\dfrac{14}{3}}
\item \(\tonda{x^{2} + 4 x - \dfrac{9}{8}}\tonda{- \dfrac{4 x}{5} - 
\dfrac{3}{4}}\)
\sol{- \dfrac{4 x^{3}}{5} - \dfrac{79 x^{2}}{20} - 
\dfrac{21 x}{10} + \dfrac{27}{32}}
% \item \(\tonda{- \dfrac{x}{5} + \dfrac{1}{4}}
% \tonda{\dfrac{2 x^{2}}{7} - 3 x - \dfrac{3}{2}}\)
%   \sol{- \dfrac{2 x^{3}}{35} + \dfrac{47 x^{2}}{70} - 
%   \dfrac{9 x}{20} - \dfrac{3}{8}}
% \item \(\tonda{3 x + \dfrac{4}{3}}\tonda{- \dfrac{7 x^{2}}{2} - \dfrac{4 
% x}{3} + \dfrac{4}{3}}\)
%   \sol{- \dfrac{21 x^{3}}{2} - \dfrac{26 x^{2}}{3} + 
%   \dfrac{20 x}{9} + \dfrac{16}{9}}
% \item \(\tonda{4 x^{2} - \dfrac{5 x}{7} + \dfrac{1}{11}}\tonda{\dfrac{11 
% x}{10} - \dfrac{5}{7}}\)
%   \sol{\dfrac{22 x^{3}}{5} - \dfrac{51 x^{2}}{14} + 
%   \dfrac{299 x}{490} - \dfrac{5}{77}}
% \item \(\tonda{2 x^{2} - \dfrac{8 x}{11} - 
% \dfrac{4}{7}}\tonda{- \dfrac{9 x}{11} - \dfrac{1}{6}}\)
%   \sol{- \dfrac{18 x^{3}}{11} + \dfrac{95 x^{2}}{363} + 
%   \dfrac{136 x}{231} + \dfrac{2}{21}}
% \item \(\tonda{- \dfrac{2 x}{5} + 1}\tonda{- \dfrac{9 x^{2}}{11} + 
% \dfrac{11 x}{7} - \dfrac{2}{11}}\)
%   \sol{\dfrac{18 x^{3}}{55} - \dfrac{557 x^{2}}{385} + 
%   \dfrac{633 x}{385} - \dfrac{2}{11}}
% \item \(\tonda{- \dfrac{7 x^{2}}{2} - \dfrac{x}{5} + 
% \dfrac{4}{3}}\tonda{- \dfrac{7 x}{10} + \dfrac{11}{9}}\)
%   \sol{\dfrac{49 x^{3}}{20} - \dfrac{931 x^{2}}{225} - 
%   \dfrac{53 x}{45} + \dfrac{44}{27}}
% \item \(\tonda{- \dfrac{11 x^{2}}{8} + \dfrac{11 x}{12} - 
% \dfrac{12}{11}}\tonda{\dfrac{x}{2} - \dfrac{1}{11}}\)
%   \sol{- \dfrac{11 x^{3}}{16} + \dfrac{7 x^{2}}{12} - 
%   \dfrac{83 x}{132} + \dfrac{12}{121}}
% \item \(\tonda{\dfrac{3 x^{2}}{7} - 10 x + 
% \dfrac{9}{8}}\tonda{x^{2} + 8 x - \dfrac{4}{9}}\)
%   \sol{\dfrac{3 x^{4}}{7} - \dfrac{46 x^{3}}{7} - 
%   \dfrac{13283 x^{2}}{168} + \dfrac{121 x}{9} - \dfrac{1}{2}}
% \item \(\tonda{- \dfrac{8 x}{11} + \dfrac{3}{10}}
% \tonda{- \dfrac{3 x^{2}}{2} - \dfrac{2 x}{7} - \dfrac{8}{11}}\)
%   \sol{\dfrac{12 x^{3}}{11} - \dfrac{373 x^{2}}{1540} + 
%   \dfrac{1877 x}{4235} - \dfrac{12}{55}}
% \item \(\tonda{\dfrac{9 x^{2}}{2} + \dfrac{4 x}{7} - 1}
% \tonda{\dfrac{2 x^{2}}{3} - \dfrac{x}{3} - \dfrac{5}{11}}\)
%   \sol{3 x^{4} - \dfrac{47 x^{3}}{42} - \dfrac{447 x^{2}}{154} + 
% \dfrac{17 x}{231} + \dfrac{5}{11}}
% \item \(\tonda{\dfrac{11 x^{2}}{12} - \dfrac{7 x}{6} + 
% \dfrac{12}{11}}\tonda{- 3 x^{2} + 4 x - \dfrac{5}{7}}\)
%   \sol{- \dfrac{11 x^{4}}{4} + \dfrac{43 x^{3}}{6} - 
%   \dfrac{2647 x^{2}}{308} + \dfrac{343 x}{66} - \dfrac{60}{77}}
% \item \(\tonda{x^{2} + \dfrac{x}{2} - \dfrac{11}{7}}
% \tonda{- \dfrac{12 x^{2}}{11} + \dfrac{3 x}{5} + \dfrac{2}{5}}\)
%   \sol{- \dfrac{12 x^{4}}{11} + \dfrac{3 x^{3}}{55} + 
%   \dfrac{169 x^{2}}{70} - \dfrac{26 x}{35} - \dfrac{22}{35}}
% \item \(\tonda{\dfrac{12 x^{2}}{7} + \dfrac{10 x}{3} - 2}
% \tonda{\dfrac{7 x^{2}}{9} - \dfrac{3 x}{11} + \dfrac{12}{11}}\)
%   \sol{\dfrac{4 x^{4}}{3} + \dfrac{4418 x^{3}}{2079} - 
%   \dfrac{412 x^{2}}{693} + \dfrac{46 x}{11} - \dfrac{24}{11}}
% \item \(\tonda{- \dfrac{x^{2}}{6} + \dfrac{x}{10} + \dfrac{1}{8}}
% \tonda{\dfrac{12 x^{2}}{11} + \dfrac{4 x}{3} - \dfrac{4}{7}}\)
%   \sol{- \dfrac{2 x^{4}}{11} - \dfrac{56 x^{3}}{495} + 
%   \dfrac{281 x^{2}}{770} + \dfrac{23 x}{210} - \dfrac{1}{14}}
\end{enumeratea}
\end{esercizio}


% ------------------- Prodotti notevoli ------------------
\subsubsection*{\numnameref{sec:poli_prodnot}}

%\subsubsection*{11.1 - Quadrato di un binomio}
\subsubsection*{\numnameref{subsec:prodnot_quadratobinomio}}

\begin{esercizio}
\label{ese:11.1}
Completa:

\begin{enumeratea}
\spazielenx
\item \(\tonda{3x + y}^{2} = \tonda{3x}^{2} + 2(3x)(y) + (y)^{2} = 
\dotfill~\)
\item \((-2x + 3y)^{2} = (-2x)^{2} + 2(-2x)(3y) +(3y)^{2} = \dotfill~\)
\item \((-3x -5y)^{2} = (-3x)^{2} + 2(-3x)(-5x)+(-5x)^{2}= \dotfill~\)
\item \((3x - y)^{2} = (3x)^{2} +2(3x)(-y) + (-y)^{2} = \dotfill~\)
\item 
\(\tonda{2x+3y}^{2}=\tonda{2x}^{2} +
2\cdot\tonda{2x}\tonda{3y}+\tonda{3y}^{2}=\dotfill\)
% \item 
% \(\tonda{x^{2}-\dfrac{1}{2}y}^{2}=\tonda{x^{2}}^{\ldots}+2\cdot%
% \tonda{\ldots \ldots }\tonda{-\ldots \ldots%
% }+\tonda{-{\dfrac{1}{2}}y}^{\ldots }=\dotfill~\)
\end{enumeratea}
\end{esercizio}

\begin{esercizio}
\label{ese:11.2}
Quali dei seguenti polinomi sono quadrati di binomi?

\vspace{-.5em}
\begin{htmulticols}{2}
% \TabPositions{4cm}
\begin{enumeratea}
\spazielenx
\item \(a^{2}+4{ab}+4b^{2}\) \hfill\sino
\item \(a^{2}-2{ab}-b^{2}\) \hfill\sino
\item \(25a^{2}-15{ab}+3b\) \hfill\sino
\item \(\dfrac{49}{4}a^{4}-21a^{2}b^{2}+9b^{2}\) \hfill\sino
\item \(a^{6}+b^{4}+2a^{3}b^{2}\) \hfill\sino
\item \(25a^{2}+4b^{2}-20{ab}^{2}\) \hfill\sino
\item \(-25a^{4}-\dfrac{1}{16}b^{4}+\dfrac{5}{2}a^{2}b^{2}\) \hfill\sino
\item \(\dfrac{1}{4}a^{6}+\dfrac{1}{9}b^{4}+\dfrac{1}{6}a^{3}b^{2}\) 
\hfill\sino
\end{enumeratea}
\end{htmulticols}
\end{esercizio}

\begin{esercizio}
\label{ese:11.3}
Completa in modo da formare un quadrato di binomio.
\begin{htmulticols}{3}
\begin{enumeratea}
\spazielenx
\item \(\dfrac{9}{16}x^{2}+\ldots +y^{2}\)
\item \(x^{2} + 2x + \ldots \)
\item \(4x^{2}y^{2} - 2xyz \ldots \)
\item \(\dfrac{a^{4}}{4}-\ldots+4b^{4}\)
\item \(9+6x+ \ldots \)
\item \(1-x+ \ldots \)
\item \(x^{2}+4y^{2}-\ldots \)
\item \(4x^{2}-4{xy}+ \ldots \)
\item \(4x^{2}-20x+\ldots \)
\end{enumeratea}
\end{htmulticols}
\end{esercizio}

% \pagebreak %----------------------------------------

% \begin{esercizio}
%  \label{ese:11.4}
% Sviluppa i seguenti quadrati di binomi.
% \begin{htmulticols}{4}
% \begin{enumeratea}
%  \item \(\tonda{x+1}^{2}\)
%  \item \(\tonda{x+2}^{2}\)
%  \item \(\tonda{x-3}^{2}\)
%  \item \(\tonda{2x-1}^{2}\)
%  \item \(\tonda{x+y}^{2}\)
%  \item \(\tonda{x-y}^{2}\)
%  \item \(\tonda{2x+y}^{2}\)
%  \item \(\tonda{x+2y}^{2}\)
%  \item \(\tonda{-a+b}^{2}\)
%  \item \(\tonda{-a-1}^{2}\)
%  \item \(\tonda{-a+3}^{2}\)
%  \item \(\tonda{-a+2b}^{2}\)
%  \item \(\tonda{2a+3b}^{2}\)
%  \item \(\tonda{2a-3b}^{2}\)
%  \item \(\tonda{3a+2b}^{2}\)
%  \item \(\tonda{-2+3b}^{2}\)
% \end{enumeratea}
% \end{htmulticols}
% \end{esercizio}

\begin{esercizio}
\label{ese:11.4}
Sviluppa i seguenti quadrati di binomi e associali al corrispondente 
risultato.
\begin{htmulticols}{4}
\begin{enumeratea}
\spazielenx
\item \(\tonda{x - 1}^{2}\) % A
% \bresult{x^{2} - 8 x + 16} % B
\item \(\tonda{x - 4}^{2}\) % B
% \bresult{x^{2} - 2 x + 1} % A
\item \(\tonda{x - 6}^{2}\) % C
% \bresult{x^{2} - 2 x + 1} % D
\item \(\tonda{- x + 1}^{2}\) % D
% \bresult{9 x^{2} - 36 x + 36} % J
\item \(\tonda{4 x + 1}^{2}\) % E
% \bresult{36 x^{2} - 48 x + 16} % K
\item \(\tonda{-4 x + 1}^{2}\) % F
% \bresult{16 x^{2} - 8 x + 1} % F
\item \(\tonda{2 x + 4}^{2}\) % G
% \bresult{4 x^{2} + 4 x + 1} % M
\item \(\tonda{- 3 x + 1}^{2}\) % H
% \bresult{16 x^{2} + 8 x + 1} % E
\item \(\tonda{- 3 x - 1}^{2}\) % I
% \bresult{9 x^{2} - 30 x + 25} % L
\item \(\tonda{3 x - 6}^{2}\) % J
% \bresult{9 x^{2} - 6 x + 1} % H
\item \(\tonda{6 x - 4}^{2}\) % K
% \bresult{x^{2} - 12 x + 36} % C
\item \(\tonda{3 x - 5}^{2}\) % L
% \bresult{9 x^{2} + 6 x + 1} % I
\item \(\tonda{- 2 x - 1}^{2}\) % M
% \bresult{4 x^{2} + 16 x + 16} % G
% \item \(\tonda{4 x + 2}^{2}\)  \bresult{16 x^{2} + 16 x + 4}
% \item \(\tonda{5 x - 4}^{2}\)  \bresult{25 x^{2} - 40 x + 16}
% \item \(\tonda{- 2 x + 3}^{2}\)  \bresult{4 x^{2} - 12 x + 9}
% \item \(\tonda{6 x + 6}^{2}\)  \bresult{36 x^{2} + 72 x + 36}
% \item \(\tonda{- 3 x + 3}^{2}\)  \bresult{9 x^{2} - 18 x + 9}
% \item \(\tonda{4 x - 4}^{2}\)  \bresult{16 x^{2} - 32 x + 16}
% \item \(\tonda{5 x + 4}^{2}\)  \bresult{25 x^{2} + 40 x + 16}
% \item \(\tonda{- 6 x + 3}^{2}\)  \bresult{36 x^{2} - 36 x + 9}
% \item \(\tonda{- 2 x + 6}^{2}\)  \bresult{4 x^{2} - 24 x + 36}
% \item \(\tonda{- 6 x - 2}^{2}\)  \bresult{36 x^{2} + 24 x + 4}
% \item \(\tonda{- 5 x + 6}^{2}\)  \bresult{25 x^{2} - 60 x + 36}
% \item \(\tonda{- 6 x + 6}^{2}\)  \bresult{36 x^{2} - 72 x + 36}
% \item \(\tonda{- x + 1}^{2}\)  \bresult{x^{2} - 2 x + 1}
% \item \(\tonda{3 x + 1}^{2}\)  \bresult{9 x^{2} + 6 x + 1}
% \item \(\tonda{3 x + 4}^{2}\)  \bresult{9 x^{2} + 24 x + 16}
% \item \(\tonda{- 3 x - 1}^{2}\)  \bresult{9 x^{2} + 6 x + 1}
% \item \(\tonda{- 4 x + 1}^{2}\)  \bresult{16 x^{2} - 8 x + 1}
\item \(\tonda{- x - \dfrac{1}{2}}^{2}\) % N
\item \(\tonda{+ x + \dfrac{1}{2}}^{2}\) % N
%   \bresult{4 x^{2} + 10 x + \dfrac{25}{4}} % R???
\item \(\tonda{\dfrac{x}{2} + 2}^{2}\) % O
%   \bresult{x^{2} + x + \dfrac{1}{4}} % N
\item \(\tonda{6 x - \dfrac{2}{3}}^{2}\) % P
%   \bresult{\dfrac{16 x^{2}}{25} - 4 x + \dfrac{25}{4}} % X
\item \(\tonda{- 3 x + \dfrac{5}{3}}^{2}\) % Q
%   \bresult{\dfrac{x^{2}}{16} + \dfrac{x}{6} + \dfrac{1}{9}} % Y
\item \(\tonda{- 2 x - \dfrac{5}{2}}^{2}\) % R
%   \bresult{9 x^{2} - 10 x + \dfrac{25}{9}} % Q
\item \(\tonda{x + \dfrac{2}{3}}^{2}\) % S
%   \bresult{\dfrac{25 x^{2}}{9} - 4 x + \dfrac{36}{25}} % Z
\item \(\tonda{- \dfrac{x}{6} + 1}^{2}\) % T
%   \bresult{\dfrac{x^{2}}{36} - \dfrac{x}{3} + 1} % T
\item \(\tonda{- 3 x - \dfrac{6}{5}}^{2}\) % U
%   \bresult{9 x^{2} + \dfrac{36 x}{5} + \dfrac{36}{25}} % U
\item \(\tonda{- \dfrac{2 x}{3} - \dfrac{3}{4}}^{2}\) % V
%   \bresult{\dfrac{4 x^{2}}{9} + x + \dfrac{9}{16}} % V
\item \(\tonda{\dfrac{x}{2} + \dfrac{1}{3}}^{2}\) % W
%   \bresult{\dfrac{x^{2}}{4} + \dfrac{x}{3} + \dfrac{1}{9}} % W
\item \(\tonda{- \dfrac{4 x}{5} + \dfrac{5}{2}}^{2}\) % X
%   \bresult{36 x^{2} - 8 x + \dfrac{4}{9}} % P
% \item \(\tonda{\dfrac{x}{4} + \dfrac{1}{3}}^{2}\) % Y
%   \bresult{\dfrac{x^{2}}{4} + 2 x + 4} % O
% \item \(\tonda{- \dfrac{5 x}{3} + \dfrac{6}{5}}^{2}\) % Z
%   \bresult{x^{2} + \dfrac{4 x}{3} + \dfrac{4}{9}} % S
% \item \(\tonda{- \dfrac{x}{2} - \dfrac{1}{5}}^{2}\)
%   \bresult{\dfrac{x^{2}}{4} + \dfrac{x}{5} + \dfrac{1}{25}}
% \item \(\tonda{- \dfrac{x}{6} + \dfrac{1}{2}}^{2}\)
%   \bresult{\dfrac{x^{2}}{36} - \dfrac{x}{6} + \dfrac{1}{4}}
% \item \(\tonda{- \dfrac{x}{3} - \dfrac{5}{4}}^{2}\)
%   \bresult{\dfrac{x^{2}}{9} + \dfrac{5 x}{6} + \dfrac{25}{16}}
% \item \(\tonda{- \dfrac{x}{5} + \dfrac{3}{4}}^{2}\)
%   \bresult{\dfrac{x^{2}}{25} - \dfrac{3 x}{10} + \dfrac{9}{16}}
% \item \(\tonda{\dfrac{3 x}{2} + \dfrac{5}{2}}^{2}\)
%   \bresult{\dfrac{9 x^{2}}{4} + \dfrac{15 x}{2} + \dfrac{25}{4}}
% \item \(\tonda{- \dfrac{2 x}{3} - \dfrac{2}{3}}^{2}\)
%   \bresult{\dfrac{4 x^{2}}{9} + \dfrac{8 x}{9} + \dfrac{4}{9}}
% \item \(\tonda{- \dfrac{5 x}{6} - \dfrac{4}{5}}^{2}\)
%   \bresult{\dfrac{25 x^{2}}{36} + \dfrac{4 x}{3} + \dfrac{16}{25}}
% \item \(\tonda{\dfrac{5 x}{6} + \dfrac{5}{6}}^{2}\)
%   \bresult{\dfrac{25 x^{2}}{36} + \dfrac{25 x}{18} + \dfrac{25}{36}}
\item \(\tonda{+ x + \dfrac{1}{2}}^{2}\) % N
\end{enumeratea}
\end{htmulticols}
\noindent\!\sframeop{b} \(x^{2} - 8 x + 16\); \quad 
\sframeop{a} \(x^{2} - 2 x + 1\); \quad 
\sframeop{d} \(x^{2} - 2 x + 1\); \quad 
\sframeop{j} \(9 x^{2} - 36 x + 36\); \quad 
\sframeop{k} \(36 x^{2} - 48 x + 16\); \quad 
\sframeop{f} \(16 x^{2} - 8 x + 1\); \quad 
\sframeop{m} \(4 x^{2} + 4 x + 1\); \quad 
\sframeop{e} \(16 x^{2} + 8 x + 1\); \quad 
\sframeop{l} \(9 x^{2} - 30 x + 25\); \quad 
\sframeop{h} \(9 x^{2} - 6 x + 1\); \quad 
\sframeop{c} \(x^{2} - 12 x + 36\); \quad 
\sframeop{i} \(9 x^{2} + 6 x + 1\); \quad 
\sframeop{g} \(4 x^{2} + 16 x + 16\); \quad 
\sframeop{r} \(4 x^{2} + 10 x + \dfrac{25}{4}\); \quad 
\sframeop{n} \(x^{2} + x + \dfrac{1}{4}\); \quad % Y
\sframeop{x} \(\dfrac{16 x^{2}}{25} - 4 x + \dfrac{25}{4}\); \quad 
% \sframeop{y} \(\dfrac{x^{2}}{16} + \dfrac{x}{6} + \dfrac{1}{9}\); \quad 
\sframeop{q} \(9 x^{2} - 10 x + \dfrac{25}{9}\); \quad 
% \sframeop{z} \(\dfrac{25 x^{2}}{9} - 4 x + \dfrac{36}{25}\); \quad 
\sframeop{t} \(\dfrac{x^{2}}{36} - \dfrac{x}{3} + 1\); \quad 
\sframeop{u} \(9 x^{2} + \dfrac{36 x}{5} + \dfrac{36}{25}\); \quad 
\sframeop{v} \(\dfrac{4 x^{2}}{9} + x + \dfrac{9}{16}\); \quad 
\sframeop{w} \(\dfrac{x^{2}}{4} + \dfrac{x}{3} + \dfrac{1}{9}\); \quad 
\sframeop{p} \(36 x^{2} - 8 x + \dfrac{4}{9}\); \quad 
\sframeop{o} \(\dfrac{x^{2}}{4} + 2 x + 4\); \quad 
\sframeop{s} \(x^{2} + \dfrac{4 x}{3} + \dfrac{4}{9}\); \quad 
\end{esercizio}

% \begin{esercizio}
%  \label{ese:11.6}
% Sviluppa i seguenti quadrati di binomi.
% \begin{htmulticols}{3}
% \begin{enumeratea}
%  \item \(\tonda{x+1}^{2}\)
%  \item \(\tonda{\dfrac{1}{2}a+\dfrac{3}{4}b}^{2}\)
%  \item \(\tonda{-2x^{2}-\dfrac{7}{4}y}^{2}\)
%  \item \(\tonda{5x^{3}-\dfrac{4}{3}y^{2}}^{2}\)
%  \item \(\tonda{-1+\dfrac{3}{2}a^{2}x}^{2}\)
%  \item \(\tonda{a^{2}+a}^{2}\)
%  \item \(\tonda{3a-\dfrac{1}{3}a^{2}}^{2}\)
%  \item \(\tonda{-2-\dfrac{1}{2}x}^{2}\)
%  \item \(\tonda{\dfrac{3}{2}x^{2}-2x}^{2}\)
%  \item \(\tonda{x^{2}-\dfrac{1}{2}x}^{2}\)
% %  \item \(\tonda{\dfrac{1}{2}a^{2}-b^{2}}^{2}\)
%  \item \(\tonda{x^{n+1}+x^{n}}^{2}\)
%  \item \(\tonda{-{\dfrac{2}{3}x-\dfrac{3}{5}x^{2}}}^{2}\)
%  \item \(\tonda{x^{2n}-\dfrac{1}{2}x^{n}}^{2}\)
%  \item \(\tonda{-2^{2}-\dfrac{1}{2}x^{n}}^{2}\)
%  \item \(\tonda{-2x^{2n}-\dfrac{1}{4}y^{m}}^{2}\)
% \end{enumeratea}
% \end{htmulticols}
% \end{esercizio}

\begin{esercizio}[*]
\label{ese:11.8}
Semplifica le seguenti espressioni contenenti quadrati di binomi.

\begin{enumeratea}
\spazielenx
\item \(\tonda{x-2y}^{2}-\tonda{2x-y}^{2}\) 
\sol{3y^{2}-3x^{2}}
\item \(3(x-y)^{2}-2(x+2y)^{2}\)
\sol{x^{2}-14xy-5y^{2}}
\item \(3(2x+5)^{2}-4(2x+5)(2x-5)+10(2x-5)^{2}\)
\item \(\tonda{x^{2}+1}^{2}-6\tonda{x^{2}+1}+8\)
\item \(\dfrac{1}{2}\tonda{x-\dfrac{1}{2}}^{2}-
      2\tonda{x-\dfrac{1}{2}}\)
\sol{ \dots }
\item 
\(\dfrac{1}{2}x(y-1)^{2}-\dfrac{3}{2}y(x+1)^{2}+\dfrac{1}{2}{xy}(3x-y+8)\)
\sol{\dfrac{1}{2}x-\dfrac{3}{2}y}
\item \(\tonda{3x-\dfrac{1}{2}y}^{2}-
    \tonda{\dfrac{1}{2}x+y}^{2}+3x(2-y)^{2}
    -3y^{2}\tonda{x-\dfrac{1}{4}}+4x(4y-3)\)
\sol{\dfrac{35}{4}x^{2}}
\item \(\tonda{x-1}^{2}-\tonda{2x+3}^{2}\)
\sol{-3x^{2}-14x-8}
\item \(\dfrac{1}{2}\tonda{2x+\dfrac{1}{2}}^{2}-2\tonda{2x-
    \dfrac{1}{2}}^{2}\)
\sol{-6x^{2}+5x-\dfrac{3}{8}}
\item \((2a+b)^{2}(a-b)^{2}-2(3-b)^{2}(3+b)^{2}-
    (6b+2a^{2})^{2}+a^{2}b[4a+3(b+8)]\)
\sol{2{ab}^{3}-b^{4}-162}
%  \item \(\tonda{\dfrac{3}{2}x^{2}-2x}^{2}+\tonda{x^{2}-
%         \dfrac{1}{2}x}^{2}-\tonda{\dfrac{3}{2}x^{2}-2x}
%         \tonda{x^{2}-\dfrac{1}{2}x}\)
%   \sol{\dfrac{7}{4}x^4-\dfrac{26}{4}x^3+\dfrac{21}{4}x^2}
%  \item \((x+1)^{2}+(x-2)^{2}+\tonda{x-\dfrac{1}{3}}^{2}-
%         2x\tonda{x-\dfrac{1}{2}}^{2}\)
%   \sol{ \dots }
\end{enumeratea}
\end{esercizio}

% \pagebreak %--------------------------------------------------

\begin{esercizio}
\label{ese:11.11}
Risali alle moltiplicazioni partendo dai prodotti e associali al 
corrispondente risultato.

\begin{htmulticols}{3}
\begin{enumeratea}
\spazielenx
\item \(x^{2} + 10 x + 25\) % A
% \bresult{\tonda{x + 12}^{2}} % B
\item \(x^{2} + 24 x + 144\) % B
% \bresult{\tonda{3 x - 7}^{2}} % C
\item \(9 x^{2} - 42 x + 49\) % C
% \bresult{\tonda{8 x + 5}^{2}} % F
\item \(36 x^{2} + 12 x + 1\) % D
% \bresult{\tonda{6 x + 5}^{2}} % E
\item \(36 x^{2} + 60 x + 25\) % E
% \bresult{4 \tonda{x - 6}^{2}} % H
\item \(64 x^{2} + 80 x + 25\) % F
% \bresult{\tonda{12 x + 1}^{2}} % J
\item \(25 x^{2} + 60 x + 36\) % G
% \bresult{4 \tonda{3 x + 1}^{2}} % K
\item \(4 x^{2} - 48 x + 144\) % H
% \bresult{4 \tonda{2 x - 1}^{2}} % L
\item \(81 x^{2} - 144 x + 64\) % I
% \bresult{\tonda{2 x - 11}^{2}} % P
\item \(144 x^{2} + 24 x + 1\) % J
% \bresult{\tonda{5 x + 6}^{2}} % G
\item \(36 x^{2} + 24 x + 4\) % K
% \bresult{4 \tonda{3 x - 1}^{2}} % M
\item \(16 x^{2} - 16 x + 4\) % L
% \bresult{\tonda{11 x - 2}^{2}} % N
\item \(36 x^{2} - 24 x + 4\) % M
% \bresult{49 \tonda{x + 1}^{2}} % O
\item \(121 x^{2} - 44 x + 4\) % N
% \bresult{\tonda{x + 5}^{2}} % A
\item \(49 x^{2} + 98 x + 49\) % O
% \bresult{\tonda{9 x - 8}^{2}} % I
\item \(4 x^{2} - 44 x + 121\) % P
% \bresult{\tonda{6 x + 1}^{2}} % D
\item \(121 x^{2} + 88 x + 16\) % Z
% \bresult{\tonda{11 x + 4}^{2}} % Z
% \item \(16 x^{2} - 88 x + 121\) % 
% \bresult{\tonda{4 x - 11}^{2}} %  % 
% \item \(100 x^{2} - 140 x + 49\) % 
% \bresult{\tonda{10 x - 7}^{2}} % 
% \item \(100 x^{2} + 80 x + 16\) % 
% \bresult{4 \tonda{5 x + 2}^{2}} % 
% \item \(121 x^{2} - 154 x + 49\) % 
% \bresult{\tonda{11 x - 7}^{2}} % 
% \item \(144 x^{2} - 120 x + 25\) % 
% \bresult{\tonda{12 x - 5}^{2}} % 
% \item \(121 x^{2} - 242 x + 121\) % 
% \bresult{121 \tonda{x - 1}^{2}} % 
% \item \(100 x^{2} - 220 x + 121\) % 
% \bresult{\tonda{10 x - 11}^{2}} % 
% \item \(100 x^{2} + 100 x + 25\) % 
% \bresult{25 \tonda{2 x + 1}^{2}} % 
% \item \(x^{2} - 12 x + 36\) \hfill 
% \bresult{\tonda{x - 6}^{2}} % 
\item \(\dfrac{25}{4} x^{2} + 40 x + 64\) % Q
% \bresult{\tonda{\dfrac{7}{6} x + 6}^{2}} % R
\item \(\dfrac{49}{36} x^{2} + 14 x + 36\) % R
% \bresult{\tonda{\dfrac{11}{10}x - 8}^{2}} % X
\item \(\dfrac{4}{9} x^{2} + \dfrac{8}{3} x + 4\) % S
% \bresult{\tonda{\dfrac{1}{2} x - \dfrac{2}{3}}^{2}} % V
\item \(\dfrac{9}{16} x^{2} + \dfrac{3}{2} x + 1\) % T
% \bresult{\tonda{\dfrac{5}{2} x + 8}^{2}} % Q
\item \(9 x^{2} + \dfrac{9}{5} x + \dfrac{9}{100}\) % U
% \bresult{\tonda{\dfrac{2}{3}x + 2}^{2}} % S
\item \(\dfrac{1}{4}x^{2} - \dfrac{2}{3} x + \dfrac{4}{9}\) %V 
% \bresult{\tonda{\dfrac{3}{4} x + 1}^{2}} % T
\item \(\dfrac{1}{36}x^{2} - \dfrac{1}{8}x + \dfrac{9}{64}\) % W
% \bresult{\tonda{\dfrac{2}{3}x - \dfrac{5}{3}}^{2}} % Y
\item \(\dfrac{121}{100} x^{2} - \dfrac{88}{5} x + 64\) % X
% \bresult{\tonda{\dfrac{1}{6}x - \dfrac{3}{8}}^{2}} % 
\item \(\dfrac{4}{9} x^{2} - \dfrac{20}{9} x + \dfrac{25}{9}\) % Y
% \bresult{\tonda{3 x + \dfrac{3}{10}}^{2}} % U
% \item \(\dfrac{4 x^{2}}{9} - \dfrac{32 x}{9} + \dfrac{64}{9}\) % 
% \bresult{\dfrac{4 \tonda{x + 3}^{2}}{9}} % 
% \item \(\dfrac{9 x^{2}}{4} - \dfrac{21 x}{2} + \dfrac{49}{4}\) % 
% \bresult{\dfrac{\tonda{3 x - 7}^{2}}{4}} % 
% \item \(\dfrac{81 x^{2}}{121} - \dfrac{216 x}{11} + 144\) % 
% \bresult{\dfrac{9 \tonda{3 x - 44}^{2}}{121}} % 
% \item \(\dfrac{25 x^{2}}{9} + \dfrac{70 x}{9} + \dfrac{49}{9}\) % 
% \bresult{\dfrac{\tonda{5 x + 7}^{2}}{9}} % 
% \item \(\dfrac{25 x^{2}}{36} - \dfrac{25 x}{6} + \dfrac{25}{4}\) % 
% \bresult{\dfrac{25 \tonda{x - 3}^{2}}{36}} % 
% \item \(\dfrac{x^{2}}{64} + \dfrac{9 x}{20} + \dfrac{81}{25}\) % 
% \bresult{\dfrac{\tonda{5 x + 72}^{2}}{1600}} % 
% \item \(\dfrac{81 x^{2}}{4} - \dfrac{21 x}{2} + \dfrac{49}{36}\) % 
% \bresult{\dfrac{\tonda{27 x - 7}^{2}}{36}} % 
% \item \(\dfrac{64 x^{2}}{9} - \dfrac{16 x}{15} + \dfrac{1}{25}\) % 
% \bresult{\dfrac{\tonda{40 x - 3}^{2}}{225}} % 
% \item \(\dfrac{x^{2}}{144} + \dfrac{2 x}{5} + \dfrac{144}{25}\) % 
% \bresult{\dfrac{\tonda{5 x + 144}^{2}}{3600}} % 
% \item \(\dfrac{81 x^{2}}{121} - \dfrac{6 x}{11} + \dfrac{1}{9}\) % 
% \bresult{\dfrac{\tonda{27 x - 11}^{2}}{1089}} % 
% \item \(\dfrac{121 x^{2}}{25} - \dfrac{44 x}{45} + \dfrac{4}{81}\) % 
% \bresult{\dfrac{\tonda{99 x - 10}^{2}}{2025}} % 
% \item \(\dfrac{64 x^{2}}{49} - \dfrac{32 x}{35} + \dfrac{4}{25}\) % 
% \bresult{\dfrac{4 \tonda{20 x - 7}^{2}}{1225}} % 
% \item \(\dfrac{144 x^{2}}{49} + \dfrac{72 x}{35} + \dfrac{9}{25}\) % 
% \bresult{\dfrac{9 \tonda{20 x + 7}^{2}}{1225}} % 
% \item \(\dfrac{36 x^{2}}{25} + \dfrac{96 x}{55} + \dfrac{64}{121}\) % 
% \bresult{\dfrac{4 \tonda{33 x + 20}^{2}}{3025}} % 
% \item \(\dfrac{81 x^{2}}{25} + \dfrac{216 x}{35} + \dfrac{144}{49}\) % 
% \bresult{\dfrac{9 \tonda{21 x + 20}^{2}}{1225}} % 
% \item \(\dfrac{121 x^{2}}{100} - \dfrac{121 x}{35} + \dfrac{121}{49}\) % 
% \bresult{\dfrac{121 \tonda{7 x - 10}^{2}}{4900}}
\end{enumeratea}
\end{htmulticols}
\noindent\!\sframeop{b} \(\tonda{x + 12}^{2}\); \quad 
\sframeop{c} \(\tonda{3 x - 7}^{2}\); \quad 
\sframeop{f} \(\tonda{8 x + 5}^{2}\); \quad 
\sframeop{e} \(\tonda{6 x + 5}^{2}\); \quad \\
\sframeop{h} \(4 \tonda{x - 6}^{2}\); \quad 
\sframeop{j} \(\tonda{12 x + 1}^{2}\); \quad 
\sframeop{k} \(4 \tonda{3 x + 1}^{2}\); \quad 
\sframeop{l} \(4 \tonda{2 x - 1}^{2}\); \quad \\
\sframeop{p} \(\tonda{2 x - 11}^{2}\); \quad 
\sframeop{g} \(\tonda{5 x + 6}^{2}\); \quad 
\sframeop{m} \(4 \tonda{3 x - 1}^{2}\); \quad 
\sframeop{n} \(\tonda{11 x - 2}^{2}\); \quad \\
\sframeop{z} \(\tonda{11 x + 4}^{2}\); \quad 
\sframeop{o} \(49 \tonda{x + 1}^{2}\); \quad 
\sframeop{a} \(\tonda{x + 5}^{2}\); \quad 
\sframeop{i} \(\tonda{9 x - 8}^{2}\); \quad \\
\sframeop{d} \(\tonda{6 x + 1}^{2}\); \quad 
\sframeop{r} \(\tonda{\dfrac{7}{6} x + 6}^{2}\); \quad 
\sframeop{x} \(\tonda{\dfrac{11}{10}x - 8}^{2}\); \quad 
\sframeop{v} \(\tonda{\dfrac{1}{2} x - \dfrac{2}{3}}^{2}\); \quad \\
\sframeop{q} \(\tonda{\dfrac{5}{2} x + 8}^{2}\); \quad 
\sframeop{s} \(\tonda{\dfrac{2}{3}x + 2}^{2}\); \quad 
\sframeop{t} \(\tonda{\dfrac{3}{4} x + 1}^{2}\); \quad 
\sframeop{y} \(\tonda{\dfrac{2}{3}x - \dfrac{5}{3}}^{2}\); \quad \\
\sframeop{w} \(\tonda{\dfrac{1}{6}x - \dfrac{3}{8}}^{2}\); \quad 
\sframeop{u} \(\tonda{3 x + \dfrac{3}{10}}^{2}\).
\end{esercizio}
% 
%\subsubsection*{11.2 - Quadrato di un polinomio}
\subsubsection*{\numnameref{subsec:prodnot_quadratopolinomio}}

\begin{esercizio}
\label{ese:11.11}
Completa i seguenti quadrati.

\begin{enumeratea}
\spazielenx
\item \(\tonda{x+3y-1}^{2}=x^{2}+\ldots \ldots +1+6xy-2x-6y\)
\item 
\(\tonda{x^{2}-\dfrac{1}{2}y+1}^{2}=x^{4}+\dfrac{1}{4}y^{2}+\ldots\
ldots -x^{2}y+\ldots\ldots -y\)
\item \(\tonda{2x^{2}-\dfrac{x}{2}+\dfrac{1}{2}}^{2}=\ldots\ldots 
+\dfrac{x^{2}}{4}%
+\dfrac{1}{4}-2x^{\ldots }+2x^{\ldots}-\dfrac{\ldots}{\ldots}\ldots \)
\end{enumeratea}
\end{esercizio}
% 
% \begin{esercizio}
%  \label{ese:11.12}
% Sviluppa i seguenti quadrati di polinomi.
% 
% \begin{htmulticols}{3}
% \begin{enumeratea}
% \item \(\tonda{a+b-c}^{2}\)
% \item \(\tonda{a-b+c}^{2}\)
% \item \(\tonda{x^{2}+x+1}^{2}\)
% \item \(\tonda{x-x^{2}+1}^{2}\)
% \item \(\tonda{2x^{2}-x+3}^{2}\)
% \item \(\tonda{-x^{2}-2x+1}^{2}\)
% \item \(\tonda{3x^{2}+2z-y^{2}}^{2}\)
% \item \(\tonda{-a+b-c}^{2}\)
% \item \(\tonda{6a-3y^{3}-2z^{2}}^{2}\)
% \item \(\tonda{1-x-x^{2}}^{2}\)
% \item \(\tonda{-2{ba}+4-6{ab}^{2}+5b^{2}}^{2}\)
% \item \(\tonda{2{ab}+3-4a^{2}b^{2}-2b^{3}}^{2}\)
% \end{enumeratea}
% \end{htmulticols}
% \end{esercizio}
% 
% 
% \begin{esercizio}
%  \label{ese:11.13}
% Sviluppa i seguenti quadrati di polinomi.
% 
% \begin{htmulticols}{3}
% \begin{enumeratea}
% \spazielenx
% \item 
% \(\tonda{\dfrac{1}{3}x^{3}-\dfrac{4}{5}x^{2}-\dfrac{1}{4}x}^{2}\)
% \item \(\tonda{3x^{3}+\dfrac{1}{2}y^{2}-\dfrac{3}{4}}^{2}\)
% \item \(\tonda{5a^{3}-\dfrac{1}{2}{ab}-1-a}^{2}\)
% \item \(\tonda{\dfrac{1}{2}x+2y^{2}-3}^{2}\)
% \item \(\tonda{\dfrac{2}{3}y^{2}-3x^{4}+\dfrac{7}{4}z}^{2}\)
% \item \(\tonda{2a+\dfrac{1}{2}{ab}^{2}-3b}^{2}\)
% % \item \(\tonda{2x^{3}y^{2}-y^{2}x+5x^{2}}^{2}\)
% \item \(\tonda{\dfrac{1}{2}x^{2}+\dfrac{3}{4}x^{2}x-2{xy}}^{2}\)
% \item \(\tonda{\dfrac{2}{3}y^{2}-3x^{2}+\dfrac{3}{4}{xy}}^{2}\)
% \item \(\tonda{a-b+\dfrac{1}{2}}^{2}\)
% \end{enumeratea}
% \end{htmulticols}
% \end{esercizio}

% \newpage %------------------------------------------------------

\begin{esercizio}
\label{ese:11.13}
Sviluppa i seguenti quadrati di polinomi.

% \begin{htmulticols}{2}
\begin{enumeratea}
\spazielenx
\item \(\tonda{- 2 x^{2} + 2 x + 1}^{2}\)
\sol{4 x^{4} - 8 x^{3} + 4 x + 1}
\item \(\tonda{- x^{2} - 3 x + 4}^{2}\)
\sol{x^{4} + 6 x^{3} + x^{2} - 24 x + 16}
\item \(\tonda{x^{2} + 4 x + 3}^{2}\)
\sol{x^{4} + 8 x^{3} + 22 x^{2} + 24 x + 9}
\item \(\tonda{x^{2} + 5 x - 1}^{2}\)
\sol{x^{4} + 10 x^{3} + 23 x^{2} - 10 x + 1}
\item \(\tonda{x^{2} - 2 x + 5}^{2}\)
\sol{x^{4} - 4 x^{3} + 14 x^{2} - 20 x + 25}
\item \(\tonda{- 2 x^{2} - x - 1}^{2}\)
\sol{4 x^{4} + 4 x^{3} + 5 x^{2} + 2 x + 1}
\item \(\tonda{- x^{2} + 3 x - 5}^{2}\)
\sol{x^{4} - 6 x^{3} + 19 x^{2} - 30 x + 25}
\item \(\tonda{2 x^{2} + 3 x + 1}^{2}\)
\sol{4 x^{4} + 12 x^{3} + 13 x^{2} + 6 x + 1}
\item \(\tonda{4 x^{2} - 2 x - 1}^{2}\)
\sol{16 x^{4} - 16 x^{3} - 4 x^{2} + 4 x + 1}
\item \(\tonda{2 x^{2} + 4 x - 1}^{2}\)
\sol{4 x^{4} + 16 x^{3} + 12 x^{2} - 8 x + 1}
\item \(\tonda{3 x^{2} - 5 x - 2}^{2}\)
\sol{9 x^{4} - 30 x^{3} + 13 x^{2} + 20 x + 4}
\item \(\tonda{4 x^{2} - 6 x - 4}^{2}\)
\sol{16 x^{4} - 48 x^{3} + 4 x^{2} + 48 x + 16}
% \item \(\tonda{6 x^{2} - 5 x + 3}^{2}\)
%   \sol{36 x^{4} - 60 x^{3} + 61 x^{2} - 30 x + 9}
% \item \(\tonda{6 x^{2} + 3 x + 3}^{2}\)
%   \sol{36 x^{4} + 36 x^{3} + 45 x^{2} + 18 x + 9}
% \item \(\tonda{3 x^{2} + 2 x - 4}^{2}\)
%   \sol{9 x^{4} + 12 x^{3} - 20 x^{2} - 16 x + 16}
% \item \(\tonda{4 x^{2} - 4 x + 3}^{2}\)
%   \sol{16 x^{4} - 32 x^{3} + 40 x^{2} - 24 x + 9}
% \item \(\tonda{4 x^{2} + 4 x + 4}^{2}\)
%   \sol{16 x^{4} + 32 x^{3} + 48 x^{2} + 32 x + 16}
% \item \(\tonda{- 5 x^{2} + x - 6}^{2}\)
%   \sol{25 x^{4} - 10 x^{3} + 61 x^{2} - 12 x + 36}
% \item \(\tonda{- 3 x^{2} + 6 x - 1}^{2}\)
%   \sol{9 x^{4} - 36 x^{3} + 42 x^{2} - 12 x + 1}
% \item \(\tonda{6 x^{2} - 2 x - 6}^{2}\)
%   \sol{36 x^{4} - 24 x^{3} - 68 x^{2} + 24 x + 36}
% \item \(\tonda{- 2 x^{2} + 4 x + 5}^{2}\)
%   \sol{4 x^{4} - 16 x^{3} - 4 x^{2} + 40 x + 25}
% \item \(\tonda{- 6 x^{2} - 3 x + 3}^{2}\)
%   \sol{36 x^{4} + 36 x^{3} - 27 x^{2} - 18 x + 9}
% \item \(\tonda{- 3 x^{2} - 4 x - 5}^{2}\)
%   \sol{9 x^{4} + 24 x^{3} + 46 x^{2} + 40 x + 25}
% \item \(\tonda{- 5 x^{2} - 3 x - 5}^{2}\)
%   \sol{25 x^{4} + 30 x^{3} + 59 x^{2} + 30 x + 25}
% \item \(\tonda{- 4 x^{2} + 3 x - 4}^{2}\)
%   \sol{16 x^{4} - 24 x^{3} + 41 x^{2} - 24 x + 16}
% \item \(\tonda{- 5 x^{2} + 2 x - 6}^{2}\)
%   \sol{25 x^{4} - 20 x^{3} + 64 x^{2} - 24 x + 36}
% \item \(\tonda{x^{2} + x + 1}^{2}\)
%   \sol{x^{4} + 2 x^{3} + 3 x^{2} + 2 x + 1}
% \item \(\tonda{x^{2} + 2 x - 1}^{2}\)
%   \sol{x^{4} + 4 x^{3} + 2 x^{2} - 4 x + 1}
% \item \(\tonda{- 3 x^{2} - 6 x - 6}^{2}\)
%   \sol{9 x^{4} + 36 x^{3} + 72 x^{2} + 72 x + 36}
\item \(\tonda{x^{2} - \dfrac{x}{2} - 3}^{2}\)
\sol{x^{4} - x^{3} - \dfrac{23 x^{2}}{4} + 3 x + 9}
\item \(\tonda{2 x^{2} + \dfrac{x}{2} - 2}^{2}\)
\sol{4 x^{4} + 2 x^{3} - \dfrac{31 x^{2}}{4} - 2 x + 4}
\item \(\tonda{2 x^{2} - \dfrac{x}{2} - 3}^{2}\)
\sol{4 x^{4} - 2 x^{3} - \dfrac{47 x^{2}}{4} + 3 x + 9}
\item \(\tonda{- 5 x^{2} + x - \dfrac{5}{2}}^{2}\)
\sol{25 x^{4} - 10 x^{3} + 26 x^{2} - 5 x + \dfrac{25}{4}}
\item \(\tonda{t+x+y+z}^{2}\)
\sol{t^2+x^2+y^2+z^2+2tx+2ty+2tz+2xy+2xz+2yz}
\item \(\tonda{t-x+y-z}^{2}\)
\sol{t^2+x^2+y^2+z^2-2tx+2ty-2tz-2xy+2xz-2yz}
% \item \(\tonda{2t-x+y-3z}^{2}\)
%   \sol{4t^2+x^2+y^2+9z^2-4tx+4ty-12tz-2xy+6xz-6yz}
\item \(\tonda{x^3+x^2+x+2}^{2}\)
\sol{x^6 + 2x^5 + 3x^4 + 6x^3 + 5x^2 + 4x + 4}
\item \(\tonda{x^3+2x^2-2x+1}^{2}\)
\sol{x^{6} + 4 x^{5} - 6 x^{3} + 8 x^{2} - 4 x + 1}
% \item \(\tonda{- \dfrac{3 x^{2}}{4} - 4 x - 2}^{2}\)
%   \sol{\dfrac{9 x^{4}}{16} + 6 x^{3} + 19 x^{2} + 16 x + 4}
% \item \(\tonda{- 2 x^{2} - x + \dfrac{5}{4}}^{2}\)
%   \sol{4 x^{4} + 4 x^{3} - 4 x^{2} - \dfrac{5 x}{2} + 
%   \dfrac{25}{16}}
% \item \(\tonda{\dfrac{x^{2}}{6} + x - 4}^{2}\)
%   \sol{\dfrac{x^{4}}{36} + \dfrac{x^{3}}{3} - \dfrac{x^{2}}{3} - 
%   8 x + 16}
% \item \(\tonda{- \dfrac{2 x^{2}}{3} + x + 1}^{2}\)
%   \sol{\dfrac{4 x^{4}}{9} - \dfrac{4 x^{3}}{3} - \dfrac{x^{2}}{3} + 
%   2 x + 1}
% \item \(\tonda{\dfrac{3 x^{2}}{2} - 2 x - \dfrac{2}{3}}^{2}\)
%   \sol{\dfrac{9 x^{4}}{4} - 6 x^{3} + 2 x^{2} + \dfrac{8 x}{3} + 
% \dfrac{4}{9}}
% \item \(\tonda{- \dfrac{5 x^{2}}{4} - \dfrac{x}{2} + 2}^{2}\)
%   \sol{\dfrac{25 x^{4}}{16} + \dfrac{5 x^{3}}{4} - 
%   \dfrac{19 x^{2}}{4} - 2 x + 4}
% \item \(\tonda{- \dfrac{x^{2}}{2} - x + \dfrac{1}{3}}^{2}\)
%   \sol{\dfrac{x^{4}}{4} + x^{3} + \dfrac{2 x^{2}}{3} - 
%   \dfrac{2 x}{3} + \dfrac{1}{9}}
% \item \(\tonda{\dfrac{4 x^{2}}{5} - x - \dfrac{1}{2}}^{2}\)
%   \sol{\dfrac{16 x^{4}}{25} - \dfrac{8 x^{3}}{5} + 
%   \dfrac{x^{2}}{5} + x + \dfrac{1}{4}}
% \item \(\tonda{- \dfrac{2 x^{2}}{5} + \dfrac{3 x}{2} - 5}^{2}\)
%   \sol{\dfrac{4 x^{4}}{25} - \dfrac{6 x^{3}}{5} + 
%   \dfrac{25 x^{2}}{4} - 15 x + 25}
% \item \(\tonda{- \dfrac{5 x^{2}}{6} + \dfrac{6 x}{5} + 1}^{2}\)
%   \sol{\dfrac{25 x^{4}}{36} - 2 x^{3} - \dfrac{17 x^{2}}{75} + 
%   \dfrac{12 x}{5} + 1}
% \item \(\tonda{\dfrac{x^{2}}{3} + \dfrac{5 x}{6} + 5}^{2}\)
%   \sol{\dfrac{x^{4}}{9} + \dfrac{5 x^{3}}{9} + 
%   \dfrac{145 x^{2}}{36} + \dfrac{25 x}{3} + 25}
% \item \(\tonda{- \dfrac{x^{2}}{3} - 3 x + \dfrac{1}{5}}^{2}\)
%   \sol{\dfrac{x^{4}}{9} + 2 x^{3} + \dfrac{133 x^{2}}{15} - 
%   \dfrac{6 x}{5} + \dfrac{1}{25}}
% \item \(\tonda{\dfrac{x^{2}}{4} - x - \dfrac{4}{3}}^{2}\)
%   \sol{\dfrac{x^{4}}{16} - \dfrac{x^{3}}{2} + \dfrac{x^{2}}{3} + 
%   \dfrac{8 x}{3} + \dfrac{16}{9}}
% \item \(\tonda{\dfrac{5 x^{2}}{3} + \dfrac{6 x}{5} + \dfrac{5}{2}}^{2}\)
%   \sol{\dfrac{25 x^{4}}{9} + 4 x^{3} + \dfrac{733 x^{2}}{75} + 
%   6 x + \dfrac{25}{4}}
% \item \(\tonda{\dfrac{x^{2}}{3} - \dfrac{x}{4} + \dfrac{4}{5}}^{2}\)
%   \sol{\dfrac{x^{4}}{9} - \dfrac{x^{3}}{6} + 
%   \dfrac{143 x^{2}}{240} - \dfrac{2 x}{5} + \dfrac{16}{25}}
% \item \(\tonda{\dfrac{x^{2}}{2} + \dfrac{6 x}{5} + \dfrac{2}{3}}^{2}\)
%   \sol{\dfrac{x^{4}}{4} + \dfrac{6 x^{3}}{5} + 
%   \dfrac{158 x^{2}}{75} + \dfrac{8 x}{5} + \dfrac{4}{9}}
% \item \(\tonda{- \dfrac{4 x^{2}}{5} - \dfrac{x}{2} + \dfrac{1}{2}}^{2}\)
%   \sol{\dfrac{16 x^{4}}{25} + \dfrac{4 x^{3}}{5} - 
%   \dfrac{11 x^{2}}{20} - \dfrac{x}{2} + \dfrac{1}{4}}
% \item \(\tonda{- \dfrac{x^{2}}{5} - \dfrac{5 x}{3} - \dfrac{5}{4}}^{2}\)
%   \sol{\dfrac{x^{4}}{25} + \dfrac{2 x^{3}}{3} + 
%   \dfrac{59 x^{2}}{18} + \dfrac{25 x}{6} + \dfrac{25}{16}}
% \item \(\tonda{- \dfrac{3 x^{2}}{5} - \dfrac{2 x}{5} + 
% \dfrac{1}{2}}^{2}\)
%   \sol{\dfrac{9 x^{4}}{25} + \dfrac{12 x^{3}}{25} - 
%   \dfrac{11 x^{2}}{25} - \dfrac{2 x}{5} + \dfrac{1}{4}}
\end{enumeratea}
% \end{htmulticols}
\end{esercizio}

\begin{esercizio}[*]
\label{ese:11.14}
Semplifica le seguenti espressioni che contengono quadrati di polinomi.

\begin{enumeratea}
\item \((x+y-1)^{2}-(x-y+1)^{2}\)
\sol{4{xy}-4x}
\item \((2a+b-x)^{2}+(2x-b-a)^{2}-5(x+a+b)^{2}+b(4a+3b)\)
\sol{-18ax-16bx}
\item \(\tonda{x^{2}+x+1}^{2}-(x+1)^{2}\)
\sol{x^{4}+2x^{3}+2x^{2}}
\item \((a+b+1)^{2}-(a-b-1)^{2}\)
\sol{4ab+4a}
% \end{enumeratea}
% \end{esercizio}
% 
% \begin{esercizio}
%  \label{ese:11.15}
% Semplifica le seguenti espressioni che contengono quadrati di polinomi.
% 
% \begin{enumeratea}
\item \((a-3b+1)^{2}-(a-3b)^{2}-(3b-1)^{2}+(a-3b)(a+3b-1)\)
\item 
\(\tonda{\dfrac{1}{2}a^{2}-b^{2}}^{2}+\tonda{a-b+\dfrac{1}{2}}^{2}
-\tonda{a+b-\dfrac{1}{2}}^{2}\)
\item \((a+b-1)^{2}-(a+b)^{2}-(a-1)^{2}-(b-1)^{2}\)
\end{enumeratea}
\end{esercizio}

%\subsubsection*{11.3 - Prodotto della somma fra due monomi per la loro 
% differenza}
\subsubsection*{\numnameref{subsec:prodnot_sommaperdifferenza}}

% \begin{esercizio}
%  \label{ese:11.17}
% Esegui le seguenti moltiplicazioni del tipo somma per differenza
% \(\tonda{A+B}\tonda{A-B}=A^{2}-B^{2}\)
%  \begin{htmulticols}{3}
% \begin{enumeratea}
%  \item \(\tonda{x-1}\tonda{x+1}\)
%  \item \(\tonda{a+1}\tonda{a-1}\)
%  \item \(\tonda{l+\dfrac{1}{2}m}\tonda{l-\dfrac{1}{2}m}\)
%  \item \(\tonda{b-2}\tonda{b+2}\)
%  \item \(\tonda{2a+b}\tonda{2a-b}\)
%  \item \(\tonda{\dfrac{1}{2}u+v}\tonda{\dfrac{1}{2}u-v}\)
%  \item \(\tonda{a+2b}\tonda{a-2b}\)
%  \item \(\tonda{2a+3b}\tonda{2a-3b}\)
%  \item \(\tonda{x-\dfrac{1}{2}}\tonda{x+\dfrac{1}{2}}\)
% %  \item \(\tonda{3a-5y}\tonda{-3a-5y}\)
% \end{enumeratea}
% \end{htmulticols}
% \end{esercizio}

% \newpage % ---------------------------------------------
\begin{esercizio}
\label{ese:11.17}
Esegui le seguenti moltiplicazioni del tipo somma per differenza
e associale al corrispondente risultato.

\vspace{-.5em}
\begin{htmulticols}{3}
\begin{enumeratea}
\spazielenx
\item \(\tonda{- x - 6}\tonda{- x + 6}\) % A
%   \bresult{25 x^{2} - 36} % M
\item \(\tonda{3 x - 3}\tonda{3 x + 3}\) % B
%   \bresult{9 x^{2} - 9} % B
\item \(\tonda{3 x - 1}\tonda{3 x + 1}\) % C
%   \bresult{4 x^{2} - 25} % I
\item \(\tonda{5 x - 1}\tonda{5 x + 1}\) % D
%   \bresult{9 x^{2} - 1} % C
\item \(\tonda{5 x - 4}\tonda{5 x + 4}\) % E
%   \bresult{36 x^{2} - 25} % L
\item \(\tonda{3 x + 6}\tonda{3 x - 6}\) % F
%   \bresult{25 x^{2} - 1} % D
\item \(\tonda{5 x + 3}\tonda{5 x - 3}\) % G
%   \bresult{25 x^{2} - 9} % G
\item \(\tonda{- 3 x - 1}\tonda{- 3 x + 1}\) % H
%   \bresult{9 x^{2} - 1} % H
\item \(\tonda{2 x - 5}\tonda{2 x + 5}\) % I
%   \bresult{9 x^{2} - 36} % F
\item \(\tonda{4 x - 5}\tonda{4 x + 5}\) % J
%   \bresult{16 x^{2} - 25} % J
\item \(\tonda{- 3 x + 2}\tonda{- 3 x - 2}\) % K
%   \bresult{9 x^{2} - 4} % K
\item \(\tonda{6 x - 5}\tonda{6 x + 5}\) % L
%   \bresult{25 x^{2} - 16} % E
\item \(\tonda{5 x + 6}\tonda{5 x - 6}\) % M
%   \bresult{x^{2} - 36} % A
% \item \(\tonda{- 3 x - 6}\tonda{- 3 x + 6}\) % 
%   \bresult{9 x^{2} - 36} % 
% \item \(\tonda{- 5 x - 1}\tonda{- 5 x + 1}\) % 
%   \bresult{25 x^{2} - 1} % 
% \item \(\tonda{- 5 x + 1}\tonda{- 5 x - 1}\) % 
%   \bresult{25 x^{2} - 1} % 
% \item \(\tonda{- 6 x - 1}\tonda{- 6 x + 1}\) % 
%   \bresult{36 x^{2} - 1} % 
% \item \(\tonda{- 3 x - 6}\tonda{- 3 x + 6}\) % 
%   \bresult{9 x^{2} - 36} % 
% \item \(\tonda{- 4 x + 6}\tonda{- 4 x - 6}\) % 
%   \bresult{16 x^{2} - 36} % 
% \item \(\tonda{- 5 x - 4}\tonda{- 5 x + 4}\) % 
%   \bresult{25 x^{2} - 16} % 
% \item \(\tonda{- 4 x + 4}\tonda{- 4 x - 4}\) % 
%   \bresult{16 x^{2} - 16} % 
% \item \(\tonda{- 6 x + 5}\tonda{- 6 x - 5}\) % 
%   \bresult{36 x^{2} - 25} % 
% \item \(\tonda{- 5 x - 4}\tonda{- 5 x + 4}\) % 
%   \bresult{25 x^{2} - 16} % 
% \item \(\tonda{x - 1}\tonda{x + 1}\) % 
%   \bresult{x^{2} - 1} % 
% \item \(\tonda{x + 2}\tonda{x - 2}\) % 
%   \bresult{x^{2} - 4} % 
% \item \(\tonda{x + 1}\tonda{x - 1}\) % 
%   \bresult{x^{2} - 1} % 
% \item \(\tonda{x + 6}\tonda{x - 6}\) % 
%   \bresult{x^{2} - 36} % 
\item \(\tonda{\dfrac{x}{2} + 2}\tonda{\dfrac{x}{2} - 2}\) % N
%   \bresult{25 x^{2} - \dfrac{1}{9}} % Y
\item \(\tonda{x + \dfrac{4}{5}}\tonda{x - \dfrac{4}{5}}\) % O
%   \bresult{x^{2} - \dfrac{9}{16}} % R
\item \(\tonda{- x + \dfrac{1}{2}}\tonda{- x - \dfrac{1}{2}}\) % P
%   \bresult{\dfrac{x^{2}}{4} - 4} % N
\item \(\tonda{- \dfrac{x}{2} + 3}\tonda{- \dfrac{x}{2} - 3}\) % Q
%   \bresult{9 x^{2} - \dfrac{16}{25}} % W
\item \(\tonda{- x - \dfrac{3}{4}}\tonda{- x + \dfrac{3}{4}}\) % R
%   \bresult{x^{2} - \dfrac{16}{25}} % O
\item \(\tonda{3 x - \dfrac{1}{2}}\tonda{3 x + \dfrac{1}{2}}\) % S
%   \bresult{x^{2} - \dfrac{1}{4}} % P
\item \(\tonda{\dfrac{3 x}{2} + 4}\tonda{\dfrac{3 x}{2} - 4}\) % T
%   \bresult{16 x^{2} - \dfrac{25}{9}} % V
\item \(\tonda{5 x - \dfrac{1}{2}}\tonda{5 x + \dfrac{1}{2}}\) % U
%   \bresult{9 x^{2} - \dfrac{1}{4}} % S
\item \(\tonda{4 x - \dfrac{5}{3}}\tonda{4 x + \dfrac{5}{3}}\) % V
%   \bresult{9 x^{2} - \dfrac{9}{4}} % X
\item \(\tonda{3 x - \dfrac{4}{5}}\tonda{3 x + \dfrac{4}{5}}\) % W
%   \bresult{25 x^{2} - \dfrac{1}{4}} % U
\item \(\tonda{- 3 x + \dfrac{3}{2}}\tonda{- 3 x - \dfrac{3}{2}}\) % X
%   \bresult{\dfrac{9 x^{2}}{4} - 16} % T
\item \(\tonda{- 5 x + \dfrac{1}{3}}\tonda{- 5 x - \dfrac{1}{3}}\) % Y
%   \bresult{\dfrac{x^{2}}{4} - 9} % Q
% \item \(\tonda{- \dfrac{3 x}{5} + 1}\tonda{- \dfrac{3 x}{5} - 1}\) % 
%   \bresult{\dfrac{9 x^{2}}{25} - 1} % 
% \item \(\tonda{- 2 x - \dfrac{5}{6}}\tonda{- 2 x + 
% \dfrac{5}{6}}\) % 
%   \bresult{4 x^{2} - \dfrac{25}{36}} % 
% \item \(\tonda{- \dfrac{x}{2} + \dfrac{1}{2}}\tonda{- \dfrac{x}{2} - 
% \dfrac{1}{2}}\) % 
%   \bresult{\dfrac{x^{2}}{4} - \dfrac{1}{4}} % 
% \item \(\tonda{- \dfrac{x}{2} - \dfrac{1}{6}}\tonda{- \dfrac{x}{2} + 
% \dfrac{1}{6}}\) % 
%   \bresult{\dfrac{x^{2}}{4} - \dfrac{1}{36}} % 
% \item \(\tonda{\dfrac{5 x}{2} - \dfrac{3}{2}}\tonda{\dfrac{5 x}{2} + 
% \dfrac{3}{2}}\) % 
%   \bresult{\dfrac{25 x^{2}}{4} - \dfrac{9}{4}} % 
% \item \(\tonda{- \dfrac{x}{5} + \dfrac{4}{5}}\tonda{- \dfrac{x}{5} - 
% \dfrac{4}{5}}\) % 
%   \bresult{\dfrac{x^{2}}{25} - \dfrac{16}{25}} % 
% \item \(\tonda{\dfrac{3 x}{2} - \dfrac{3}{4}}\tonda{\dfrac{3 x}{2} + 
% \dfrac{3}{4}}\) % 
%   \bresult{\dfrac{9 x^{2}}{4} - \dfrac{9}{16}} % 
% \item \(\tonda{\dfrac{5 x}{4} - \dfrac{1}{2}}\tonda{\dfrac{5 x}{4} + 
% \dfrac{1}{2}}\) % 
%   \bresult{\dfrac{25 x^{2}}{16} - \dfrac{1}{4}} % 
% \item \(\tonda{\dfrac{6 x}{5} - \dfrac{1}{5}}\tonda{\dfrac{6 x}{5} + 
% \dfrac{1}{5}}\) % 
%   \bresult{\dfrac{36 x^{2}}{25} - \dfrac{1}{25}} % 
% \item \(\tonda{- \dfrac{5 x}{4} - \dfrac{3}{4}}\tonda{- \dfrac{5 x}{4} + 
% \dfrac{3}{4}}\) % 
%   \bresult{\dfrac{25 x^{2}}{16} - \dfrac{9}{16}} % 
\end{enumeratea}
\end{htmulticols}
\noindent\!\sframeop{m} \(25 x^{2} - 36\); \quad 
\sframeop{b} \(9 x^{2} - 9\); \quad 
\sframeop{i} \(4 x^{2} - 25\); \quad 
\sframeop{c} \(9 x^{2} - 1\); \quad 
\sframeop{l} \(36 x^{2} - 25\); \quad 
\sframeop{d} \(25 x^{2} - 1\); \quad 
\sframeop{g} \(25 x^{2} - 9\); \quad 
\sframeop{h} \(9 x^{2} - 1\); \quad 
\sframeop{f} \(9 x^{2} - 36\); \quad 
\sframeop{j} \(16 x^{2} - 25\); \quad 
\sframeop{k} \(9 x^{2} - 4\); \quad 
\sframeop{e} \(25 x^{2} - 16\); \quad 
\sframeop{a} \(x^{2} - 36\); \quad 
\sframeop{y} \(25 x^{2} - \dfrac{1}{9}\); \quad 
\sframeop{r} \(x^{2} - \dfrac{9}{16}\); \quad 
\sframeop{n} \(\dfrac{x^{2}}{4} - 4\); \quad 
\sframeop{w} \(9 x^{2} - \dfrac{16}{25}\); \quad 
\sframeop{o} \(x^{2} - \dfrac{16}{25}\); \quad 
\sframeop{p} \(x^{2} - \dfrac{1}{4}\); \quad 
\sframeop{v} \(16 x^{2} - \dfrac{25}{9}\); \quad 
\sframeop{s} \(9 x^{2} - \dfrac{1}{4}\); \quad 
\sframeop{x} \(9 x^{2} - \dfrac{9}{4}\); \quad 
\sframeop{u} \(25 x^{2} - \dfrac{1}{4}\); \quad 
\sframeop{t} \(\dfrac{9 x^{2}}{4} - 16\); \quad 
\sframeop{q} \(\dfrac{x^{2}}{4} - 9\).
\end{esercizio}

% \newpage %------------------------------------------

\begin{esercizio}
\label{ese:11.16}
Calcola a mente i seguenti prodotti applicando la regola 
\((A+B)(A-B)=A^2-B^2\)
%Senza utilizzare la calcolatrice, calcolare mentalmente i seguenti
%prodotti:

\vspace{-.5em}
\begin{htmulticols}{4}
\begin{enumeratea}
\item \(18\cdot 22\)
\item \(15\cdot 25\)
\item \(43\cdot 37\)
\item \(195\cdot 205\)
\end{enumeratea}
\end{htmulticols}
\end{esercizio}

% \begin{esercizio}
%  \label{ese:11.17}
% Esegui i seguenti prodotti applicando la regola
% \(\tonda{A+B}\tonda{A-B}=A^{2}-B^{2}\)
%  \begin{htmulticols}{2}
% \begin{enumeratea}
%  \item \(\tonda{\dfrac{2}{3}x+\dfrac{3}{2}y}
%         \tonda{\dfrac{2}{3}x-\dfrac{3}{2}y}\)
%  \item \(\tonda{-{\dfrac{2}{5}}x-\dfrac{3}{7}y}
%         \tonda{-{\dfrac{2}{5}}x+\dfrac{3}{7}y}\)
%  \item \(\tonda{x^{2}+\dfrac{1}{2}z}
%         \tonda{x^{2}-\dfrac{1}{2}z}\)
%  \item \(\tonda{\dfrac{2}{3}x^{2}+3y^{2}}
%         \tonda{-{\dfrac{2}{3}}x^{2}+3y^{2}}\)
%  \item \(\tonda{\dfrac{2}{3}a^{3}+\dfrac{1}{2}y^{3}}
%         \tonda{-{\dfrac{2}{3}}a^{3}+\dfrac{1}{2}y^{3}}\)
%  \item 
% \(\tonda{-2a^{3}-\dfrac{7}{3}y}\tonda{-2a^{3}+\dfrac{7}{3}y}\)
%  \item \(\tonda{5x^{2}-\dfrac{6}{5}y^{3}}
%         \tonda{5x^{2}+\dfrac{6}{5}y^{3}}\)
%  \item \(\tonda{a^{5}+\dfrac{1}{2}y^{4}}
%         \tonda{a^{5}-\dfrac{1}{2}y^{4}}\)
%  \item \(\tonda{-{\dfrac{8}{3}}x^{4}-\dfrac{1}{2}x^{3}}
%         \tonda{\dfrac{8}{3}x^{4}-\dfrac{1}{2}x^{3}}\)
%  \item \(\tonda{2x^{5}+\dfrac{3}{2}y^{5}}
%         \tonda{2x^{5}-\dfrac{3}{2}y^{5}}\)
%  \item \(\tonda{-x-\dfrac{1}{2}}\tonda{-x+\dfrac{1}{2}}\)
%  \item \(\tonda{-x-\dfrac{1}{2}}\tonda{-{\dfrac{1}{2}}+x}\)
%  \item \(\tonda{-{\dfrac{2}{3}x-\dfrac{3}{5}x^{2}}}
%         \tonda{\dfrac{2}{3}x-\dfrac{3}{5}x^{2}}\)
%  \item \(\tonda{-{\dfrac{2}{3}x-\dfrac{3}{5}x^{2}}}
%         \tonda{\dfrac{3}{5}x^{2}-\dfrac{2}{3}x}\)
%  \item \(\tonda{\dfrac{2}{3}x-\dfrac{3}{5}x^{2}}
%         \tonda{-{\dfrac{2}{3}x-\dfrac{3}{5}x^{2}}}\)
%  \item \(\tonda{\dfrac{2}{3}x+\dfrac{3}{5}x^{2}}
%         \tonda{\dfrac{2}{3}x-\dfrac{3}{5}x^{2}}\)
% \end{enumeratea}
% \end{htmulticols}
% \end{esercizio}

\begin{esercizio}[*]
\label{ese:11.21}
Applica la regola della somma per differenza ai seguenti casi.

% \vspace{-.5em}
\begin{enumeratea}
\spazielenx
\item \((2a+b+1)(2a+b-1)\) 
\sol{2 a^{3} + a^{2} b + a^{2} + 2 a b - 2 a + b^{2} - 1 }
\item \((3x-b+c)(3x+b-c)\) 
\sol{- b^{2} + 2 b c - c^{2} + 9 x^{2} }
\item \((3x-y-1)(3x+y-1)\)
\sol{9x^{2}-6x-y^{2}+1}
\item \(\quadra{(2x+y)+(3y-1)}\quadra{(2x+y)-(3y-1)}\) 
\sol{4 x^{2} + 4 x y - 8 y^{2} + 6 y - 1 }
\item \((ab-2b-a)(-{ab}+2b-a)\)
\sol{a^{2}-a^{2}b^{2}+4{ab}^{2}-4b^{2}}
\item \(\tonda{\dfrac{1}{2}a+1+b+ab}\tonda{\dfrac{1}{2}a+1-b-ab}\)
\sol{-a^{2}b^{2}+\dfrac{1}{4}a^{2}-2{ab}^{2}+a-b^{2}+1}
% \item \(\tonda{a-2ab+3b}\tonda{a-2ab-3b}\)
% \sol{4 a^{2} b^{2} - 4 a^{2} b + a^{2} - 9 b^{2} }
\end{enumeratea}
\end{esercizio}

% \pagebreak %-------------------------------------

\begin{esercizio}[*]
\label{ese:11.22}
Semplifica le seguenti espressioni con prodotti notevoli.

\begin{enumeratea}
\spazielenx
\item \((a+b)(a-b)-(a+b)^{2}\)
\sol{-2{ab}-2b^{2}}
\item \([(x-1)(1+x)]^{2}\)
\sol{x^{4}-2x^{2}+1}
\item 
\(\tonda{\dfrac{2}{3}a-b}\tonda{\dfrac{2}{3}a+b}-\dfrac{2}{3}(a-b)^{ 2} +
2\tonda{\dfrac{1}{3}a}^{2}\)
\sol{\dfrac{4}{3}{ab}-\dfrac{5}{3}b^{2}}
%  \item \begin{multline*}
%  
% 
% 
% \tonda{2x-\dfrac{1}{2}y}\tonda{\dfrac{1}{2}y+2x}+\tonda{5x-\dfrac{
% 1}{ 5}
% }\tonda{5x+\dfrac{1}{5}}\\
%  
% 
% 
% +\tonda{\dfrac{1}{5}-5x}\tonda{5x+\dfrac{1}{5}}-\tonda{2x+\dfrac{1
% }{2 }
% y}\tonda{\dfrac{1}{2}y-2x}.
% 	 \end{multline*}
%   \sol{\quadra{8x^2-\dfrac{1}{2}y^2}\)
%  \end{enumeratea}
% \end{esercizio}
% 
% \begin{esercizio}[*]
%  \label{ese:11.23}
% Semplifica le seguenti espressioni con prodotti notevoli.
% 
%  \begin{enumeratea}


\item \(\tonda{\dfrac{2}{3}a-b}\tonda{\dfrac{2}{3}a+b}
      \tonda{b^{2}+\dfrac{4}{9}a^{2}}\)
\sol{\dfrac{16}{81}a^{4}-b^{4}}
\item \(\tonda{-{\dfrac{1}{2}x-\dfrac{1}{2}}}
      \tonda{-\dfrac{1}{2}x+\dfrac{1}{2}}+
      \tonda{x-\dfrac{1}{2}}\tonda{-x-\dfrac{1}{2}}-
      2x\tonda{x-\dfrac{1}{4}}^{2}\)
\sol{-2x^3-\dfrac{1}{4}x^2-\dfrac{1}{8}x}
\item \((a+b-1)^{2}+(a-b)^{2}+\tonda{a-\dfrac{1}{2}b}
      \tonda{a+\dfrac{1}{2}b}+2a\tonda{a-\dfrac{1}{2}}-
      a(5a+3)-(2b-1)\) \\
\sol{\dfrac{7}{4}b^{2}-4b-6a+2}
\item \(\tonda{x^{2}+2x}
      \tonda{\dfrac{1}{2}x+1}+\tonda{\dfrac{1}{2}x-1}^{2}-
      \tonda{\dfrac{1}{2}x+1}
      \tonda{-{\dfrac{1}{2}}x+1}-\dfrac{1}{2}x^{2}(x+5)\)
\sol{x}
\end{enumeratea}
\end{esercizio}

\begin{esercizio}
\label{ese:11.11}
Risali alle moltiplicazione partendo dai prodotti e associali al 
corrispondente risultato.

\begin{htmulticols}{4}
\begin{enumeratea}
\spazielenx
\item \(x^{2} - 1\) % A
% \bresult{\tonda{x - 8} \tonda{x + 8}} % B
\item \(x^{2} - 64\) % B
% \bresult{4 \tonda{x - 6} \tonda{x + 6}} % F
\item \(x^{2} - 16\) % C
% \bresult{\tonda{x - 10} \tonda{x + 10}} % N
\item \(x^{2} - 100\) % D
% \bresult{\tonda{4x - 1} \tonda{4x + 1}} % H
\item \(9 x^{2} - 36\) % E
% \bresult{\tonda{x - 4} \tonda{x + 4}} % C
\item \(4 x^{2} - 144\) % F
% \bresult{81 \tonda{x - 1} \tonda{x + 1}} % G
\item \(81 x^{2} - 81\) % G
% \bresult{\tonda{6x - 5} \tonda{6x + 5}} % K
\item \(16 x^{2} - 1\) % H
% \bresult{\tonda{5 x - 7} \tonda{5 x + 7}} % J
\item \(25 x^{2} - 4\) % I
% \bresult{\tonda{7 x - 11} \tonda{7 x + 11}} % P
\item \(25 x^{2} - 49\) % J
% \bresult{\tonda{9 x - 8} \tonda{9 x + 8}} % L
\item \(36 x^{2} - 25\) % K
% \bresult{\tonda{7 x - 4} \tonda{7 x + 4}} % M
\item \(81 x^{2} - 64\) % L
% \bresult{9 \tonda{x - 2} \tonda{x + 2}} % E
\item \(49 x^{2} - 16\) % M
% \bresult{\tonda{11 x - 7} \tonda{11 x + 7}} % O
\item \(x^{2} - 100\) % N
% \bresult{\tonda{x - 1} \tonda{x + 1}} % A
\item \(121 x^{2} - 49\) % O
% \bresult{\tonda{5 x - 2} \tonda{5 x + 2}} % I
\item \(49 x^{2} - 121\) % P
% \bresult{\tonda{x - 10} \tonda{x + 10}} % D
% \item \(16 x^{2} - 121\) % 
% \bresult{\tonda{4 x - 11} \tonda{4 x + 11}} % 
% \item \(100 x^{2} - 81\) % 
% \bresult{\tonda{10 x - 9} \tonda{10 x + 9}} % 
% \item \(100 x^{2} - 81\) % 
% \bresult{\tonda{10 x - 9} \tonda{10 x + 9}} % 
% \item \(25 x^{2} - 121\) % 
% \bresult{\tonda{5 x - 11} \tonda{5 x + 11}} % 
% \item \(100 x^{2} - 49\) % 
% \bresult{\tonda{10 x - 7} \tonda{10 x + 7}} % 
% \item \(144 x^{2} - 100\) % 
% \bresult{4 \tonda{6 x - 5} \tonda{6 x + 5}} % 
% \item \(121 x^{2} - 100\) % 
% \bresult{\tonda{11 x - 10} \tonda{11 x + 10}} % 
% \item \(100 x^{2} - 121\) % 
% \bresult{\tonda{10 x - 11} \tonda{10 x + 11}} % 
\item \(100 x^{2} - 169\) % Z
% \bresult{\tonda{3x - \dfrac{2}{9}} \tonda{3x + \dfrac{2}{9}}} % W
\item \(9x^{2} - 16\) % Q
% \bresult{\tonda{x - \dfrac{5}{2}} \tonda{x + \dfrac{5}{2}}} % T
\item \(144 x^{2} - 9\) % R
% \bresult{\tonda{6x - \dfrac{11}{3}} \tonda{6x + \dfrac{11}{3}}} % Y
\item \(144 x^{2} - 4\) % S
% \bresult{\tonda{10 x - 13} \tonda{10 x + 13}} % Z
\item \(x^{2} - \dfrac{25}{4}\) % T
% \bresult{\tonda{x - \dfrac{1}{2}} \tonda{x + \dfrac{1}{2}}} % V
\item \(\dfrac{x^{2}}{36} - 25\) % U
% \bresult{9 \tonda{4 x - 1} \tonda{4 x + 1}} % R
\item \(x^{2} - \dfrac{1}{4}\) % V
% \bresult{\tonda{\dfrac{1}{6}x - 5} \tonda{\dfrac{1}{6}x + 5}} % U
\item \(9 x^{2} - \dfrac{4}{81}\) % W
% \bresult{4 \tonda{6 x - 1} \tonda{6 x + 1}} % S
\item \(\dfrac{9 x^{2}}{16} - 25\) % X
% \bresult{\tonda{3x - 4} \tonda{3x + 4}} % Q
\item \(36 x^{2} - \dfrac{121}{9}\) % Y
% \bresult{\tonda{\dfrac{3}{4}x - 5} \tonda{\dfrac{3}{4}x - 5}} % X
% \item \(9 x^{2} - \dfrac{121}{49}\) % 
% \bresult{\dfrac{\tonda{21 x - 11} \tonda{21 x + 11}}{49}} % 
% \item \(\dfrac{x^{2}}{4} - \dfrac{16}{81}\) % 
% \bresult{\dfrac{\tonda{9 x - 8} \tonda{9 x + 8}}{324}} % 
% \item \(\dfrac{x^{2}}{16} - \dfrac{9}{100}\) % 
% \bresult{\dfrac{\tonda{5 x - 6} \tonda{5 x + 6}}{400}} % 
% \item \(\dfrac{4 x^{2}}{9} - \dfrac{4}{49}\) % 
% \bresult{\dfrac{4 \tonda{7 x - 3} \tonda{7 x + 3}}{441}} % 
% \item \(\dfrac{x^{2}}{4} - \dfrac{121}{49}\) % 
% \bresult{\dfrac{\tonda{7 x - 22} \tonda{7 x + 22}}{196}} % 
% \item \(\dfrac{9 x^{2}}{16} - \dfrac{1}{144}\) % 
% \bresult{\dfrac{\tonda{9 x - 1} \tonda{9 x + 1}}{144}} % 
% \item \(\dfrac{16 x^{2}}{81} - \dfrac{9}{4}\) % 
% \bresult{\dfrac{\tonda{8 x - 27} \tonda{8 x + 27}}{324}} % 
% \item \(\dfrac{9 x^{2}}{121} - \dfrac{1}{4}\) % 
% \bresult{\dfrac{\tonda{6 x - 11} \tonda{6 x + 11}}{484}} % 
% \item \(\dfrac{64 x^{2}}{121} - \dfrac{81}{4}\) % 
% \bresult{\dfrac{\tonda{16 x - 99} \tonda{16 x + 99}}{484}} % 
% \item \(\dfrac{4 x^{2}}{25} - \dfrac{36}{49}\) % 
% \bresult{\dfrac{4 \tonda{7 x - 15} \tonda{7 x + 15}}{1225}} % 
% \item \(\dfrac{25 x^{2}}{16} - \dfrac{144}{49}\) % 
% \bresult{\dfrac{\tonda{35 x - 48} \tonda{35 x + 48}}{784}} % 
% \item \(\dfrac{16 x^{2}}{49} - \dfrac{121}{16}\) % 
% \bresult{\dfrac{\tonda{16 x - 77} \tonda{16 x + 77}}{784}} % 
% \item \(\dfrac{16 x^{2}}{25} - \dfrac{49}{81}\) % 
% \bresult{\dfrac{\tonda{36 x - 35} \tonda{36 x + 35}}{2025}} % 
% \item \(\dfrac{144 x^{2}}{25} - \dfrac{25}{64}\) % 
% \bresult{\dfrac{\tonda{96 x - 25} \tonda{96 x + 25}}{1600}} % 
% \item \(\dfrac{100 x^{2}}{81} - \dfrac{121}{64}\) % 
% \bresult{\dfrac{\tonda{80 x - 99} \tonda{80 x + 99}}{5184}} % 
% \item \(\dfrac{64 x^{2}}{121} - \dfrac{121}{16}\) % 
% \bresult{\dfrac{\tonda{32 x - 121} \tonda{32 x + 121}}{1936}} % 
% \item \(\dfrac{16 x^{2}}{121} - \dfrac{121}{64}\) % 
% \bresult{\dfrac{\tonda{32 x - 121} \tonda{32 x + 121}}{7744}} % 
\end{enumeratea}
\end{htmulticols}
\noindent\!\sframeop{b} \(\tonda{x - 8} \tonda{x + 8}\); \quad 
\sframeop{f} \(4 \tonda{x - 6} \tonda{x + 6}\); \quad 
\sframeop{n} \(\tonda{x - 10} \tonda{x + 10}\); \quad \\
\sframeop{h} \(\tonda{4x - 1} \tonda{4x + 1}\); \quad 
\sframeop{c} \(\tonda{x - 4} \tonda{x + 4}\); \quad 
\sframeop{g} \(81 \tonda{x - 1} \tonda{x + 1}\); \quad \\
\sframeop{k} \(\tonda{6x - 5} \tonda{6x + 5}\); \quad 
\sframeop{j} \(\tonda{5 x - 7} \tonda{5 x + 7}\); \quad 
\sframeop{p} \(\tonda{7 x - 11} \tonda{7 x + 11}\); \quad \\
\sframeop{l} \(\tonda{9 x - 8} \tonda{9 x + 8}\); \quad 
\sframeop{m} \(\tonda{7 x - 4} \tonda{7 x + 4}\); \quad 
\sframeop{e} \(9 \tonda{x - 2} \tonda{x + 2}\); \quad \\
\sframeop{o} \(\tonda{11 x - 7} \tonda{11 x + 7}\); \quad 
\sframeop{a} \(\tonda{x - 1} \tonda{x + 1}\); \quad 
\sframeop{i} \(\tonda{5 x - 2} \tonda{5 x + 2}\); \quad \\
\sframeop{d} \(\tonda{x - 10} \tonda{x + 10}\); \quad 
\sframeop{t} \(\tonda{x - \dfrac{5}{2}} \tonda{x + \dfrac{5}{2}}\); \quad 
\sframeop{y} \(\tonda{6x -\dfrac{11}{3}} \tonda{6x +\dfrac{11}{3}}\); \quad 
\sframeop{z} \(\tonda{10 x - 13} \tonda{10 x + 13}\); \quad 
\sframeop{v} \(\tonda{x -\dfrac{1}{2}} \tonda{x +\dfrac{1}{2}}\); \quad 
\sframeop{r} \(9 \tonda{4 x - 1} \tonda{4 x + 1}\); \quad \\
\sframeop{u} \(\tonda{\dfrac{1}{6}x - 5} \tonda{\dfrac{1}{6}x + 5}\); \quad 
\sframeop{s} \(4 \tonda{6 x - 1} \tonda{6 x + 1}\); \quad 
\sframeop{q} \(\tonda{3x - 4} \tonda{3x + 4}\); \quad \\
\sframeop{x} \(\tonda{\dfrac{3}{4}x - 5} \tonda{\dfrac{3}{4}x - 5}\); \quad 
\sframeop{w} \(\tonda{3x - \dfrac{2}{9}} \tonda{3x + \dfrac{2}{9}}\).
\end{esercizio}

\pagebreak %------------------------------------------

\subsubsection*{\numnameref{subsec:prodnot_particolare}}

\begin{esercizio}
\label{ese:11.}
Esegui le seguenti moltiplicazioni tra binomi particolari e associali al 
corrispondente risultato.
\begin{htmulticols}{3}
\begin{enumeratea}
\spazielenx
\item \(\tonda{x - 3}\tonda{x + 3}\) % A
\item \(\tonda{x + 2}\tonda{x - 2}\) % B
\item \(\tonda{x - 4}\tonda{x + 4}\) % C
\item \(\tonda{x - 2}\tonda{x + 1}\) % D
\item \(\tonda{x + 3}\tonda{x - 2}\) % E
\item \(\tonda{x + 5}\tonda{x - 4}\) % F
\item \(\tonda{x + 6}\tonda{x - 5}\) % G
\item \(\tonda{x + 4}\tonda{x - 3}\) % H
\item \(\tonda{x - 5}\tonda{x + 1}\) % I
\item \(\tonda{x - 1}\tonda{x - 6}\) % J
\item \(\tonda{x - 1}\tonda{x - 5}\) % K
\item \(\tonda{x + 2}\tonda{x + 1}\) % L
\item \(\tonda{x + 1}\tonda{x - 5}\) % M
% \item \(\tonda{x - 5}\tonda{x - 1}\) % 
%   \bresult{x^{2} - 6 x + 5} % 
% \item \(\tonda{x - 3}\tonda{x - 2}\) % 
%   \sol{x^{2} - 5 x + 6} % 
% \item \(\tonda{x + 6}\tonda{x - 1}\) % 
%   \sol{x^{2} + 5 x - 6} % 
% \item \(\tonda{x + 1}\tonda{x + 4}\) % 
%   \sol{x^{2} + 5 x + 4} % 
% \item \(\tonda{x + 1}\tonda{x + 4}\) % 
%   \sol{x^{2} + 5 x + 4} % 
% \item \(\tonda{x - 2}\tonda{x + 5}\) % 
%   \sol{x^{2} + 3 x - 10} % 
% \item \(\tonda{x + 3}\tonda{x - 5}\) % 
%   \sol{x^{2} - 2 x - 15} % 
% \item \(\tonda{x - 2}\tonda{x + 5}\) % 
%   \sol{x^{2} + 3 x - 10} % 
% \item \(\tonda{x + 2}\tonda{x + 6}\) % 
%   \sol{x^{2} + 8 x + 12} % 
% \item \(\tonda{x - 5}\tonda{x - 4}\) % 
%   \sol{x^{2} - 9 x + 20} % 
% \item \(\tonda{x + 6}\tonda{x + 2}\) % 
%   \sol{x^{2} + 8 x + 12} % 
% \item \(\tonda{x - 2}\tonda{x + 5}\) % 
%   \sol{x^{2} + 3 x - 10} % 
% \item \(\tonda{x + 5}\tonda{x + 6}\) % 
%   \sol{x^{2} + 11 x + 30} % 
% \item \(\tonda{x + 1}\tonda{x - 3}\) % 
%   \sol{x^{2} - 2 x - 3} % 
% \item \(\tonda{x + 2}\tonda{x + 3}\) % 
%   \sol{x^{2} + 5 x + 6} % 
% \item \(\tonda{x - 1}\tonda{x - 2}\) % 
%   \sol{x^{2} - 3 x + 2} % 
% \item \(\tonda{x + 6}\tonda{x + 3}\) % 
%   \sol{x^{2} + 9 x + 18} % 
\item \(\tonda{x - 6}\tonda{x - \dfrac{1}{6}}\) % N
\item \(\tonda{x - \dfrac{2}{3}}\tonda{x - \dfrac{1}{3}}\) % O
\item \(\tonda{x + \dfrac{3}{2}}\tonda{x - 1}\) % P
\item \(\tonda{x - 1}\tonda{x + \dfrac{1}{2}}\) % Q
\item \(\tonda{x - \dfrac{2}{3}}\tonda{x + 1}\) % R
\item \(\tonda{x + 1}\tonda{x - \dfrac{3}{5}}\) % S
\item \(\tonda{x + \dfrac{2}{3}}\tonda{x + 1}\) % T
\item \(\tonda{x + \dfrac{1}{4}}\tonda{x - 1}\) % U
\item \(\tonda{x - \dfrac{5}{3}}\tonda{x + \dfrac{6}{5}}\) % V
\item \(\tonda{x + \dfrac{3}{4}}\tonda{x - 6}\) % W
% \item \(\tonda{x - \dfrac{2}{5}}\tonda{x + 6}\) % X
% \item \(\tonda{x - \dfrac{5}{4}}\tonda{x + 5}\) % Y
% \item \(\tonda{x + \dfrac{5}{3}}\tonda{x - \dfrac{3}{2}}\) % 
%   \sol{x^{2} + \dfrac{x}{6} - \dfrac{5}{2}} % 
% \item \(\tonda{x + \dfrac{1}{3}}\tonda{x - \dfrac{1}{6}}\) % 
%   \sol{x^{2} + \dfrac{x}{6} - \dfrac{1}{18}} % 
% \item \(\tonda{x + \dfrac{2}{3}}\tonda{x + \dfrac{2}{3}}\) % 
%   \sol{x^{2} + \dfrac{4 x}{3} + \dfrac{4}{9}} % 
% \item \(\tonda{x + \dfrac{1}{4}}\tonda{x + \dfrac{3}{2}}\) % 
%   \sol{x^{2} + \dfrac{7 x}{4} + \dfrac{3}{8}} % 
% \item \(\tonda{x + \dfrac{4}{3}}\tonda{x - \dfrac{1}{4}}\) % 
%   \sol{x^{2} + \dfrac{13 x}{12} - \dfrac{1}{3}} % 
% \item \(\tonda{x + \dfrac{1}{4}}\tonda{x + \dfrac{4}{5}}\) % 
%   \sol{x^{2} + \dfrac{21 x}{20} + \dfrac{1}{5}} % 
% \item \(\tonda{x + \dfrac{5}{4}}\tonda{x + \dfrac{3}{2}}\) % 
%   \sol{x^{2} + \dfrac{11 x}{4} + \dfrac{15}{8}} % 
% \item \(\tonda{x - \dfrac{3}{2}}\tonda{x - \dfrac{1}{5}}\) % 
%   \sol{x^{2} - \dfrac{17 x}{10} + \dfrac{3}{10}} % 
% \item \(\tonda{x + \dfrac{1}{6}}\tonda{x - \dfrac{5}{4}}\) % 
%   \sol{x^{2} - \dfrac{13 x}{12} - \dfrac{5}{24}} % 
% \item \(\tonda{x + \dfrac{3}{5}}\tonda{x + \dfrac{1}{6}}\) % 
%   \sol{x^{2} + \dfrac{23 x}{30} + \dfrac{1}{10}} % 
\end{enumeratea}
\end{htmulticols}
\noindent\!\sframeop{h} \(x^{2} + x - 12\) % H
\sframeop{a} \(x^{2} - 9\) % A
\sframeop{j} \(x^{2} - 7 x + 6\) % J
\sframeop{d} \(x^{2} - x - 2\) % D
\sframeop{e} \(x^{2} + x - 6\) \\ % E
\sframeop{f} \(x^{2} + x - 20\) % F
\sframeop{m} \(x^{2} - 4 x - 5\) % M
\sframeop{b} \(x^{2} - 4\) % B
\sframeop{i} \(x^{2} - 4 x - 5\) % I
\sframeop{c} \(x^{2} - 16\) \\ % C
\sframeop{g} \(x^{2} + x - 30\) % G
\sframeop{l} \(x^{2} + 3 x + 2\) % L
\sframeop{k} \(x^{2} - 6 x + 5\) % K
\sframeop{o} \(x^{2} - x + \dfrac{2}{9}\) % O
\sframeop{n} \(x^{2} - \dfrac{37 x}{6} + 1\) % N
\sframeop{q} \(x^{2} - \dfrac{x}{2} - \dfrac{1}{2}\) % Q
\sframeop{r} \(x^{2} + \dfrac{x}{3} - \dfrac{2}{3}\) % R
\sframeop{s} \(x^{2} + \dfrac{2 x}{5} - \dfrac{3}{5}\) % S
\sframeop{t} \(x^{2} + \dfrac{5 x}{3} + \dfrac{2}{3}\) % T
\sframeop{v} \(x^{2} - \dfrac{7 x}{15} - 2\) % V
\sframeop{p} \(x^{2} + \dfrac{x}{2} - \dfrac{3}{2}\) % P
% \sframeop{y} \(x^{2} + \dfrac{15 x}{4} - \dfrac{25}{4}\) % Y
\sframeop{w} \(x^{2} - \dfrac{21 x}{4} - \dfrac{9}{2}\) % W
\sframeop{u} \(x^{2} - \dfrac{3 x}{4} - \dfrac{1}{4}\) % U
% \sframeop{x} \(x^{2} + \dfrac{28 x}{5} - \dfrac{12}{5}\) % X
\end{esercizio}


\begin{esercizio}
\label{ese:11.11}
Risali alle moltiplicazioni partendo dai prodotti e associali al 
corrispondente risultato.

\begin{htmulticols}{4}
\begin{enumeratea}
\spazielenx
\item \(x^{2} - 14 x + 49\) % A
\item \(x^{2} - 49\) % B
\item \(x^{2} + x - 6\) % C
\item \(x^{2} - x - 12\) % D
\item \(x^{2} - x - 56\) % E
\item \(x^{2} - x - 30\) % F
\item \(x^{2} + x - 90\) % G
\item \(x^{2} + 8 x - 9\) % H
\item \(x^{2} + 5 x + 4\) % I
\item \(x^{2} + 3 x - 4\) % J
\item \(x^{2} - 5 x + 4\) % K
\item \(x^{2} + 9 x + 18\) % L
\item \(x^{2} - 3 x - 28\) % M
\item \(x^{2} + 6 x - 27\) % N
\item \(x^{2} + 7 x + 12\) % O
\item \(x^{2} + 4 x - 21\) % P
\item \(x^{2} + 8 x - 48\) % Q
\item \(x^{2} - 6 x - 72\) % R
\item \(x^{2} - 6 x - 72\) % S
\item \(x^{2} - 5 x - 50\) % T
\item \(x^{2} - 6 x - 40\) % U
\item \(x^{2} + 12 x + 11\) % V
\item \(x^{2} + 10 x - 11\) % W
\item \(x^{2} - 12 x + 11\) % X
\item \(x^{2} - 21 x + 108\) % Y
\item \(x^{2} - 23 x + 132\) % Z

% \item \(x^{2} - x - 2\) % O
% \sframeop{} \(\dfrac{\tonda{x + 1} \tonda{2 x + 1}}{2}\) % 
% \item \(x^{2} - \dfrac{16 x}{3} + 7\) % P
% \sframeop{} \(\dfrac{\tonda{x - 3} \tonda{3 x - 7}}{3}\) % 
% \item \(x^{2} - x - \dfrac{55}{36}\) % Q
% \sframeop{} \(\tonda{x - 2} \tonda{x + 1}\) % O
% \item \(x^{2} - \dfrac{x}{2} - \dfrac{3}{2}\) % R
% \sframeop{} \(\dfrac{\tonda{6 x - 11} \tonda{6 x + 5}}{36}\) % 
% \item \(x^{2} + \dfrac{3 x}{2} + \dfrac{1}{2}\) % S
% \sframeop{} \(\dfrac{\tonda{x - 3} \tonda{9 x + 2}}{9}\) % 
% \item \(x^{2} - \dfrac{2 x}{9} - \dfrac{7}{9}\) % T
% \sframeop{} \(\dfrac{\tonda{x + 8} \tonda{3 x - 2}}{3}\) % 
% \item \(x^{2} - \dfrac{5 x}{3} - \dfrac{2}{3}\) % U
% \sframeop{} \(\dfrac{\tonda{x - 3} \tonda{9 x + 7}}{9}\) % 
% \item \(x^{2} - \dfrac{25 x}{9} - \dfrac{2}{3}\) % V
% \sframeop{} \(\dfrac{\tonda{x + 1} \tonda{9 x + 5}}{9}\) % 
% \item \(x^{2} - \dfrac{20 x}{9} - \dfrac{7}{3}\) % W
% \sframeop{} \(\dfrac{\tonda{x + 1} \tonda{2 x - 3}}{2}\) % 
% \item \(x^{2} + \dfrac{14 x}{9} + \dfrac{5}{9}\) % X
% \sframeop{} \(\dfrac{\tonda{x - 1} \tonda{9 x + 7}}{9}\) % 
% \item \(x^{2} + \dfrac{22 x}{3} - \dfrac{16}{3}\) % Y
% \sframeop{} \(\dfrac{\tonda{x - 2} \tonda{3 x + 1}}{3}\) % 
% \item \(x^{2} + \dfrac{x}{3} - \dfrac{7}{48}\) % Z
% \sframeop{} \(\dfrac{\tonda{4 x - 1} \tonda{12 x + 7}}{48}\) % 
% \item \(x^{2} - \dfrac{94 x}{9} + \dfrac{40}{9}\) % 
% \sframeop{} \(\dfrac{\tonda{x - 10} \tonda{9 x - 4}}{9}\) % 
% \item \(x^{2} - \dfrac{5 x}{8} + \dfrac{1}{16}\) % 
% \sframeop{} \(\dfrac{\tonda{2 x - 1} \tonda{8 x - 1}}{16}\) % 
% \item \(x^{2} + \dfrac{14 x}{9} + \dfrac{8}{27}\) % 
% \sframeop{} \(\dfrac{\tonda{3 x + 4} \tonda{9 x + 2}}{27}\) % 
% \item \(x^{2} - \dfrac{67 x}{24} + \dfrac{1}{3}\) % 
% \sframeop{} \(\dfrac{\tonda{3 x - 8} \tonda{8 x - 1}}{24}\) % 
% \item \(x^{2} + \dfrac{16 x}{15} + \dfrac{4}{15}\) % 
% \sframeop{} \(\dfrac{\tonda{3 x + 2} \tonda{5 x + 2}}{15}\) % 
% \item \(x^{2} + \dfrac{68 x}{11} + \dfrac{12}{11}\) % 
% \sframeop{} \(\dfrac{\tonda{x + 6} \tonda{11 x + 2}}{11}\) % 
% \item \(x^{2} - \dfrac{32 x}{33} - \dfrac{16}{33}\) % 
% \sframeop{} \(\dfrac{\tonda{3 x - 4} \tonda{11 x + 4}}{33}\) % 
% \item \(x^{2} - \dfrac{13 x}{24} - \dfrac{21}{32}\) % 
% \sframeop{} \(\dfrac{\tonda{8 x - 9} \tonda{12 x + 7}}{96}\) % 
% \item \(x^{2} - \dfrac{87 x}{22} + \dfrac{35}{22}\) % 
% \sframeop{} \(\dfrac{\tonda{2 x - 7} \tonda{11 x - 5}}{22}\) % 
% \item \(x^{2} - \dfrac{17 x}{24} - \dfrac{11}{12}\) % 
% \sframeop{} \(\dfrac{\tonda{3 x + 2} \tonda{8 x - 11}}{24}\) % 
% \item \(x^{2} + \dfrac{31 x}{60} + \dfrac{1}{24}\) % 
% \sframeop{} \(\dfrac{\tonda{10 x + 1} \tonda{12 x + 5}}{120}\) % 
% \item \(x^{2} + \dfrac{125 x}{77} + \dfrac{50}{77}\) % 
% \sframeop{} \(\dfrac{\tonda{7 x + 5} \tonda{11 x + 10}}{77}\) % 
% \item \(x^{2} - \dfrac{111 x}{110} - \dfrac{1}{10}\) % 
% \sframeop{} \(\dfrac{\tonda{10 x - 11} \tonda{11 x + 1}}{110}\) % 
% \item \(x^{2} + \dfrac{140 x}{33} + \dfrac{100}{33}\) % 
% \sframeop{} \(\dfrac{\tonda{3 x + 10} \tonda{11 x + 10}}{33}\) % 
\end{enumeratea}
\end{htmulticols}
\noindent\!\sframeop{} \(\tonda{x - 4} \tonda{x + 3}\) % D
\sframeop{} \(\tonda{x - 7}^{2}\) % A
\sframeop{} \(\tonda{x - 6} \tonda{x + 5}\) % F
\sframeop{} \(\tonda{x - 9} \tonda{x + 10}\) \\[.2em] % G
\sframeop{} \(\tonda{x + 3} \tonda{x + 6}\) % L
\sframeop{} \(\tonda{x - 7} \tonda{x + 7}\) % B
\sframeop{} \(\tonda{x - 1} \tonda{x + 4}\) % J
\sframeop{} \(\tonda{x + 1} \tonda{x + 4}\) \\[.2em] % I
\sframeop{} \(\tonda{x - 8} \tonda{x + 7}\) % E
\sframeop{} \(\tonda{x - 3} \tonda{x + 9}\) % N
\sframeop{} \(\tonda{x - 7} \tonda{x + 4}\) % M
\sframeop{} \(\tonda{x - 2} \tonda{x + 3}\) \\[.2em] % C
\sframeop{} \(\tonda{x - 1} \tonda{x + 9}\) % H
\sframeop{} \(\tonda{x - 4} \tonda{x - 1}\) % K
\sframeop{} \(\tonda{x - 10} \tonda{x + 4}\) % U
\sframeop{} \(\tonda{x - 4} \tonda{x + 12}\) \\[.2em] % Q
\sframeop{} \(\tonda{x - 3} \tonda{x + 7}\) % P
\sframeop{} \(\tonda{x - 12} \tonda{x + 6}\) % S
\sframeop{} \(\tonda{x - 12} \tonda{x - 11}\) % Z
\sframeop{} \(\tonda{x + 3} \tonda{x + 4}\) \\[.2em] % O
\sframeop{} \(\tonda{x - 11} \tonda{x - 1}\) % X
\sframeop{} \(\tonda{x + 1} \tonda{x + 11}\) % V
\sframeop{} \(\tonda{x - 1} \tonda{x + 11}\) % W
\sframeop{} \(\tonda{x - 10} \tonda{x + 5}\) \\[.2em] % T
\sframeop{} \(\tonda{x - 12} \tonda{x + 6}\) % R
\sframeop{} \(\tonda{x - 12} \tonda{x - 9}\) % Y
\end{esercizio}

%\subsubsection*{11.4 - Cubo di un binomio}
\subsubsection*{\numnameref{subsec:prodnot_cubo}}

\begin{esercizio}
\label{ese:11.24}
Riconosci quali dei seguenti polinomi sono cubi di binomi.
\TabPositions{5cm}
\begin{enumeratea}
\item \(-a^{3}-3a^{2}b+3{ab}^{2}+b^{3}\) \hfill\sino
\item \(a^{9}-6a^{4}b-12a^{2}b^{2}-8b^{3}\) \hfill\sino
\item \(8a^{9}-b^{3}-6b^{2}a^{3}+12a^{6}b\) \hfill\sino
\item \(\dfrac{1}{27}a^{6}-8b^{3}+4a^{2}b^{2}-\dfrac{2}{3}a^{4}b\) 
\hfill\sino
\end{enumeratea}
\end{esercizio}

\begin{esercizio}
\label{ese:11.25}
Sviluppa i seguenti cubi di binomio.

\begin{enumeratea}
\item \( 
\tonda{2a+b^{2}}^{3}=\tonda{2a}^{3}+3\cdot\tonda{2a}^{2}\
cdot 
b^{2}+3\tonda{2a}\cdot\tonda{b^{2}}^{2}+\tonda{b^{2}}^{3}
=\dotfill~\)
\item \(\tonda{x-2y}^{3}=x^{\ldots }-6x^{\ldots 
}y+12{xy}^{\ldots}-\ldots 
y^{\ldots }\)
\item \((a+b)^{2}+(a+b)(a-b)+(a+b)^{3}-a^{3}-b^{3}-a^{2}-b^{2}-{ab}\)
\end{enumeratea}
\end{esercizio}

% \begin{esercizio}
%  \label{ese:11.26}
%  Sviluppa i seguenti cubi di binomio.
% \begin{htmulticols}{4}
%  \begin{enumeratea}
%  \item \(\tonda{x+y}^{3}\)
%  \item \(\tonda{x-y}^{3}\)
%  \item \(\tonda{-x+y}^{3}\)
%  \item \(\tonda{\dfrac{1}{2}a+b}^{3}\)
%  \item \(\tonda{a+2}^{3}\)
%  \item \(\tonda{a+1}^{3}\)
%  \item \(\tonda{a-1}^{3}\)
%  \item \(\tonda{a-\dfrac{2}{3}b}^{3}\)
%  \item \(\tonda{x+2y}^{3}\)
%  \item \(\tonda{y-2x}^{3}\)
%  \item \(\tonda{2x+y}^{3}\)
%  \item \(\tonda{-\dfrac{1}{3}xy-3}^{3}\)
%  \item \(\tonda{x^{2}y-3}^{3}\)
%  \item \(\tonda{xy-1}^{3}\)
%  \item \(\tonda{x^{2}-2y}^{3}\)
%  \item \(\tonda{\dfrac{1}{2}a-\dfrac{2}{3}b}^{3}\)
%  \end{enumeratea}
% \end{htmulticols}
% \end{esercizio}

\begin{esercizio}
\label{ese:11.26}
Sviluppa i seguenti cubi di binomio e associali al corrispondente 
risultato.
\begin{htmulticols}{4}
\begin{enumeratea}
\spazielenx
\item \(\tonda{x - 1}^{3}\) % A
\item \(\tonda{x + 1}^{3}\) % B
\item \(\tonda{x + 2}^{3}\) % C
\item \(\tonda{x - 4}^{3}\) % D
\item \(\tonda{2 x - 1}^{3}\) % E
\item \(\tonda{4 x + 1}^{3}\) % F
\item \(\tonda{3 x + 2}^{3}\) % G
\item \(\tonda{2 x - 4}^{3}\) % H
\item \(\tonda{- 2 x + 1}^{3}\) % I
\item \(\tonda{4 x + 3}^{3}\) % J
\item \(\tonda{4 x - 3}^{3}\) % K
\item \(\tonda{- 3 x - 1}^{3}\) % L
\item \(\tonda{- 2 x - 3}^{3}\) % M
\item \(\tonda{- 4 x + 2}^{3}\) % N
\item \(\tonda{- 2 x - 4}^{3}\) % Y
% \sframeop{} \(- 8 x^{3} - 48 x^{2} - 96 x - 64\) % 
\item \(\tonda{- 4 x - 2}^{3}\) % Z
% \sframeop{} \(- 64 x^{3} - 96 x^{2} - 48 x - 8\) % 
% \item \(\tonda{- 4 x - 4}^{3}\) % 
% \sframeop{} \(- 64 x^{3} - 192 x^{2} - 192 x - 64\) % 
% \item \(\tonda{- 4 x - 3}^{3}\) % 
% \sframeop{} \(- 64 x^{3} - 144 x^{2} - 108 x - 27\) % 
% \item \(\tonda{- 4 x + 4}^{3}\) % 
% \sframeop{} \(- 64 x^{3} + 192 x^{2} - 192 x + 64\) % 
% \item \(\tonda{- 3 x + 4}^{3}\) % 
% \sframeop{} \(- 27 x^{3} + 108 x^{2} - 144 x + 64\) % 
% \item \(\tonda{- 4 x + 4}^{3}\) % 
% \sframeop{} \(- 64 x^{3} + 192 x^{2} - 192 x + 64\) % 
% \item \(\tonda{- 3 x + 4}^{3}\) % 
% \sframeop{} \(- 27 x^{3} + 108 x^{2} - 144 x + 64\) % 
% \item \(\tonda{x - 1}^{3}\) % 
% \sframeop{} \(x^{3} - 3 x^{2} + 3 x - 1\) % 
% \item \(\tonda{x + 1}^{3}\) % 
% \sframeop{} \(x^{3} + 3 x^{2} + 3 x + 1\) % 
% \item \(\tonda{x - 1}^{3}\) % 
% \sframeop{} \(x^{3} - 3 x^{2} + 3 x - 1\) % 
% \item \(\tonda{- x - 1}^{3}\) % 
% \sframeop{} \(- x^{3} - 3 x^{2} - 3 x - 1\) % 
% \item \(\tonda{- x - 1}^{3}\) % 
% \sframeop{} \(- x^{3} - 3 x^{2} - 3 x - 1\) % 
% \item \(\tonda{- x + 3}^{3}\) % 
% \sframeop{} \(- x^{3} + 9 x^{2} - 27 x + 27\) % 
% \item \(\tonda{4 x + 2}^{3}\) % 
% \sframeop{} \(64 x^{3} + 96 x^{2} + 48 x + 8\) % 
% \item \(\tonda{4 x + 1}^{3}\) % 
% \sframeop{} \(64 x^{3} + 48 x^{2} + 12 x + 1\) % 
\item \(\tonda{\dfrac{3 x}{2} - 4}^{3}\) % O
\item \(\tonda{\dfrac{x}{4} - 2}^{3}\) % P
\item \(\tonda{- x - \dfrac{2}{3}}^{3}\) % Q
\item \(\tonda{\dfrac{4 x}{3} - 1}^{3}\) % R
\item \(\tonda{x + \dfrac{1}{2}}^{3}\) % S
\item \(\tonda{x + \dfrac{1}{4}}^{3}\) % T
\item \(\tonda{x + \dfrac{3}{2}}^{3}\) % U
\item \(\tonda{- \dfrac{x}{2} - 1}^{3}\) % V
\item \(\tonda{2 x - \dfrac{3}{2}}^{3}\) % W
\item \(\tonda{- x + \dfrac{1}{2}}^{3}\) % X
% \item \(\tonda{\dfrac{x}{2} + \dfrac{4}{3}}^{3}\) % 
% \sframeop{} \(\dfrac{x^{3}}{8} + x^{2} + \dfrac{8 x}{3} + \dfrac{64}{27}\) 
% 
% \item \(\tonda{- 3 x - \dfrac{1}{2}}^{3}\) % 
% \sframeop{} \(- 27 x^{3} - \dfrac{27 x^{2}}{2} - \dfrac{9 x}{4} - 
% \dfrac{1}{8}\) % 
% \item \(\tonda{- 3 x - \dfrac{1}{4}}^{3}\) % 
% \sframeop{} \(- 27 x^{3} - \dfrac{27 x^{2}}{4} - \dfrac{9 x}{16} - 
% \dfrac{1}{64}\) % 
% \item \(\tonda{- \dfrac{3 x}{4} - 1}^{3}\) % 
% \sframeop{} \(- \dfrac{27 x^{3}}{64} - \dfrac{27 x^{2}}{16} - 
% \dfrac{9 x}{4} - 1\) % 
% \item \(\tonda{\dfrac{2 x}{3} + \dfrac{3}{4}}^{3}\) % 
% \sframeop{} \(\dfrac{8 x^{3}}{27} + x^{2} + \dfrac{9 x}{8} + 
% \dfrac{27}{64}\) % 
% \item \(\tonda{\dfrac{3 x}{4} + \dfrac{4}{3}}^{3}\) % 
% \sframeop{} \(\dfrac{27 x^{3}}{64} + \dfrac{9 x^{2}}{4} + 4 x + 
% \dfrac{64}{27}\) % 
% \item \(\tonda{- \dfrac{x}{4} + \dfrac{4}{3}}^{3}\) % 
% \sframeop{} \(- \dfrac{x^{3}}{64} + \dfrac{x^{2}}{4} - \dfrac{4 x}{3} + 
% \dfrac{64}{27}\) % 
% \item \(\tonda{\dfrac{x}{4} - \dfrac{3}{2}}^{3}\) % 
% \sframeop{} \(\dfrac{x^{3}}{64} - \dfrac{9 x^{2}}{32} + \dfrac{27 x}{16} - 
% \dfrac{27}{8}\) % 
\end{enumeratea}
\end{htmulticols}
\noindent\!\sframeop{j} \(64 x^{3} + 144 x^{2} + 108 x + 27\) % J
\sframeop{l} \(- 27 x^{3} - 27 x^{2} - 9 x - 1\) % L
\sframeop{m} \(- 8 x^{3} - 36 x^{2} - 54 x - 27\) % M
\sframeop{f} \(64 x^{3} + 48 x^{2} + 12 x + 1\) % F
\sframeop{a} \(x^{3} - 3 x^{2} + 3 x - 1\) % A
\sframeop{n} \(- 64 x^{3} + 96 x^{2} - 48 x + 8\) % N
\sframeop{c} \(x^{3} + 6 x^{2} + 12 x + 8\) % C
\sframeop{h} \(8 x^{3} - 48 x^{2} + 96 x - 64\) % H
\sframeop{i} \(- 8 x^{3} + 12 x^{2} - 6 x + 1\)\\ % I
\sframeop{e} \(8 x^{3} - 12 x^{2} + 6 x - 1\) % E
\sframeop{k} \(64 x^{3} - 144 x^{2} + 108 x - 27\) % K
\sframeop{d} \(x^{3} - 12 x^{2} + 48 x - 64\) % D
\sframeop{g} \(27 x^{3} + 54 x^{2} + 36 x + 8\) % G
\sframeop{b} \(x^{3} + 3 x^{2} + 3 x + 1\) % B
\sframeop{y} \(- 8 x^{3} - 48 x^{2} - 96 x - 64\) % Y
\sframeop{z} \(- 64 x^{3} - 96 x^{2} - 48 x - 8\) % Z
\sframeop{} \(- x^{3} - 2 x^{2} - \dfrac{4 x}{3} - \dfrac{8}{27}\) \\ % Q
\sframeop{x} \(- x^{3}+\dfrac{3 x^{2}}{2}-\dfrac{3 x}{4}+\dfrac{1}{8}\)%X
\sframeop{u} \(x^{3}+\dfrac{9 x^{2}}{2}+\dfrac{27 x}{4}+\dfrac{27}{8}\)%U
\sframeop{v} \(- \dfrac{x^{3}}{8}-\dfrac{3 x^{2}}{4}-\dfrac{3 x}{2}-1\)\\%V
\sframeop{r} \(\dfrac{64 x^{3}}{27} - \dfrac{16 x^{2}}{3} + 4 x - 1\) % R
\sframeop{} \(\dfrac{x^{3}}{64} - \dfrac{3 x^{2}}{8} + 3 x - 8\) % P
\sframeop{o} \(\dfrac{27 x^{3}}{8} - 27 x^{2} + 72 x - 64\) % O
\sframeop{t} \(x^{3}+\dfrac{3 x^{2}}{4}+\dfrac{3 x}{16}+\dfrac{1}{64}\)%T
\sframeop{w} \(8 x^{3} - 18 x^{2} + \dfrac{27 x}{2} - \dfrac{27}{8}\) % W
\sframeop{s} \(x^{3}+\dfrac{3 x^{2}}{2}+\dfrac{3 x}{4}+\dfrac{1}{8}\) % S
\end{esercizio}

\pagebreak %---------------------------------------------

% \begin{esercizio}
%  \label{ese:11.27}
%  Sviluppa i seguenti cubi di binomio.
% \begin{htmulticols}{3}
%  \begin{enumeratea}
%  \spazielenx
%  \item \((a-3)^{3}\)
%  \item \(\tonda{\dfrac{1}{2}a^{2}-\dfrac{3}{2}a}^{3}\)
%  \item \(\tonda{\dfrac{2}{3}x-1}^{3}\)
%  \item \(\tonda{x-\dfrac{1}{3}}^{3}\)
%  \item \(\tonda{\dfrac{1}{2}{xy}-2x}^{3}\)
%  \item \((x+3)^{3}\)
%  \item \(\tonda{\dfrac{2}{5}x^{2}y-5yx^{2}a}^{3}\)
%  \item \(\tonda{\dfrac{1}{2}x^{2}+1}^{3}\)
%  \item 
% \(\tonda{\dfrac{3}{4}a^{2}b^{3}c^{2}-\dfrac{1}{3}a^{2}{bc}^{2}}^{3}\)
%  \item \(\tonda{-{\dfrac{1}{2}}+\dfrac{1}{4}{xy}^{2}z^{3}}^{3}\)
%  \item \(\tonda{x^{2}-y^{2}}^{3}\)
%  \item \(\tonda{-3xy^{2}+\dfrac{3}{2}zx^{2}}^{3}\)
%  \item \(\tonda{2x^{2}z+\dfrac{2}{3}y^{2}z^{3}x}^{3}\)
%  \item \(-\tonda{\dfrac{1}{2}x^{2}-1}^{3}\)
%  \item \(\tonda{\dfrac{1}{4}ab^{2}c-4a^{2}b}^{3}\)
%  \end{enumeratea}
% \end{htmulticols}
%  \end{esercizio}

\begin{esercizio}
\label{ese:11.11}
Risali alle moltiplicazioni partendo dai prodotti e associali al 
corrispondente risultato.

\begin{htmulticols}{2}
\begin{enumeratea}
\spazielenx
\item \(x^{3} - 12 x^{2} + 48 x - 64\) % A
\item \(8 x^{3} + 12 x^{2} + 6 x + 1\) % B
\item \(8 x^{3} + 24 x^{2} + 24 x + 8\) % C
\item \(- x^{3} + 15 x^{2} - 75 x + 125\) % D
\item \(125 x^{3} + 150 x^{2} + 60 x + 8\) % E
\item \(- 8 x^{3} - 12 x^{2} - 6 x - 1\) % F
\item \(- x^{3} - 18 x^{2} - 108 x - 216\) % G
\item \(8 x^{3} + 72 x^{2} + 216 x + 216\) % H
\item \(125 x^{3} - 150 x^{2} + 60 x - 8\) % I
\item \(27 x^{3} - 162 x^{2} + 324 x - 216\)  % J
\item \(- 27 x^{3} - 27 x^{2} - 9 x - 1\) % K
\item \(125 x^{3} + 225 x^{2} + 135 x + 27\) % L
\item \(64 x^{3} - 192 x^{2} + 192 x - 64\) % M
\item \(125 x^{3} - 225 x^{2} + 135 x - 27\) % N
\item \(216 x^{3} + 216 x^{2} + 72 x + 8\) % O
\item \(125 x^{3} + 450 x^{2} + 540 x + 216\) % P
% \item \(27 x^{3} - 162 x^{2} + 324 x - 216\) \\ % 
% \sframeop{} \(27 \tonda{x - 2}^{3}\) % 
% \item \(- 64 x^{3} - 96 x^{2} - 48 x - 8\) \\ % 
% \sframeop{} \(- 8 \tonda{2 x + 1}^{3}\) % 
% \item \(- 27 x^{3} + 81 x^{2} - 81 x + 27\) \\ % 
% \sframeop{} \(- 27 \tonda{x - 1}^{3}\) % 
% \item \(- 8 x^{3} + 72 x^{2} - 216 x + 216\) \\ % 
% \sframeop{} \(- 8 \tonda{x - 3}^{3}\) % 
% \item \(- 216 x^{3} - 216 x^{2} - 72 x - 8\) % 
% \sframeop{} \(- 8 \tonda{3 x + 1}^{3}\) % 
% \item \(- 64 x^{3} + 240 x^{2} - 300 x + 125\) \\ % 
% \sframeop{} \(- \tonda{4 x - 5}^{3}\) % 
% \item \(- 27 x^{3} + 135 x^{2} - 225 x + 125\) \\ % 
% \sframeop{} \(- \tonda{3 x - 5}^{3}\) % 
% \item \(- 27 x^{3} + 162 x^{2} - 324 x + 216\) \\ % 
% \sframeop{} \(- 27 \tonda{x - 2}^{3}\) % 

\item \(x^{3} + 18 x^{2} + 108 x + 216\) % Q
\item \(27 x^{3} + 27 x^{2} + 9 x + 1\) % R
\item \(- 8 x^{3} + 12 x^{2} - 6 x + 1\) % S
\item \(- 64 x^{3} + 48 x^{2} - 12 x + 1\) % T
\item \(\dfrac{x^{3}}{8} - \dfrac{3 x^{2}}{2} + 6 x - 8\) % U
\item \(- \dfrac{x^{3}}{27} + \dfrac{x^{2}}{3} - x + 1\) % V
\item \(\dfrac{8 x^{3}}{27} + \dfrac{16 x^{2}}{3} + 32 x + 64\) % W
\item \(\dfrac{8 x^{3}}{27} - \dfrac{20 x^{2}}{3} + 50 x - 125\) % X
\item \(125 x^{3} - 100 x^{2} + \dfrac{80 x}{3} - \dfrac{64}{27}\) % Y
\item \(x^{3} + \dfrac{x^{2}}{2} + \dfrac{x}{12} + \dfrac{1}{216}\) % Z
% \item \(\dfrac{27 x^{3}}{125} - \dfrac{27 x^{2}}{5} + 45 x - 125\) 
% \sframeop{} \(\dfrac{\tonda{3 x - 25}^{3}}{125}\)
% \item \(\dfrac{27 x^{3}}{8} + \dfrac{27 x^{2}}{4} + \dfrac{9 x}{2} + 1\) 
% \sframeop{} \(\dfrac{\tonda{3 x + 2}^{3}}{8}\)
% \item \(- x^{3} - \dfrac{9 x^{2}}{2} - \dfrac{27 x}{4} - \dfrac{27}{8}\) 
% \sframeop{} \(- \dfrac{\tonda{2 x + 3}^{3}}{8}\)
% \item \(x^{3} - \dfrac{18 x^{2}}{5} + \dfrac{108 x}{25} - 
% \dfrac{216}{125}\) 
% \sframeop{} \(\dfrac{\tonda{5 x - 6}^{3}}{125}\)
% \item \(125 x^{3} - \dfrac{75 x^{2}}{4} + \dfrac{15 x}{16} - 
% \dfrac{1}{64}\) 
% \sframeop{} \(\dfrac{\tonda{20 x - 1}^{3}}{64}\)
% \item \(- x^{3} - \dfrac{12 x^{2}}{5} - \dfrac{48 x}{25} - 
% \dfrac{64}{125}\) 
% \sframeop{} \(- \dfrac{\tonda{5 x + 4}^{3}}{125}\)
% \item \(\dfrac{27 x^{3}}{8} - \dfrac{9 x^{2}}{4} + \dfrac{x}{2} - 
% \dfrac{1}{27}\) 
% \sframeop{} \(\dfrac{\tonda{9 x - 2}^{3}}{216}\)
% \item \(- \dfrac{64 x^{3}}{125} - \dfrac{96 x^{2}}{25} - \dfrac{48 x}{5} - 
% 8\) 
% \sframeop{} \(- \dfrac{8 \tonda{2 x + 5}^{3}}{125}\)
% \item \(- \dfrac{x^{3}}{27} + \dfrac{x^{2}}{18} - \dfrac{x}{36} + 
% \dfrac{1}{216}\) 
% \sframeop{} \(- \dfrac{\tonda{2 x - 1}^{3}}{216}\)
% \item \(- \dfrac{x^{3}}{216} + \dfrac{x^{2}}{48} - \dfrac{x}{32} + 
% \dfrac{1}{64}\) 
% \sframeop{} \(- \dfrac{\tonda{2 x - 3}^{3}}{1728}\)
% \item \(\dfrac{x^{3}}{8} + \dfrac{9 x^{2}}{20} + \dfrac{27 x}{50} + 
% \dfrac{27}{125}\) 
% \sframeop{} \(\dfrac{\tonda{5 x + 6}^{3}}{1000}\)
% \item \(- \dfrac{x^{3}}{8} - \dfrac{3 x^{2}}{10} - \dfrac{6 x}{25} - 
% \dfrac{8}{125}\) 
% \sframeop{} \(- \dfrac{\tonda{5 x + 4}^{3}}{1000}\)
% \item \(\dfrac{x^{3}}{125} - \dfrac{12 x^{2}}{125} + \dfrac{48 x}{125} - 
% \dfrac{64}{125}\) 
% \sframeop{} \(\dfrac{\tonda{x - 4}^{3}}{125}\)
% \item \(- \dfrac{8 x^{3}}{27} + \dfrac{10 x^{2}}{3} - \dfrac{25 x}{2} + 
% \dfrac{125}{8}\) 
% \sframeop{} \(- \dfrac{\tonda{4 x - 15}^{3}}{216}\)
% \item \(- \dfrac{8 x^{3}}{27} + \dfrac{8 x^{2}}{5} - \dfrac{72 x}{25} + 
% \dfrac{216}{125}\) 
% \sframeop{} \(- \dfrac{8 \tonda{5 x - 9}^{3}}{3375}\)
% \item \(- \dfrac{8 x^{3}}{125} + \dfrac{36 x^{2}}{125} - 
% \dfrac{54 x}{125} + \dfrac{27}{125}\) 
% \sframeop{} \(- \dfrac{\tonda{2 x - 3}^{3}}{125}\)
\end{enumeratea}
\end{htmulticols}
\noindent\!\sframeop{d} \(- \tonda{x - 5}^{3}\) % D
\sframeop{a} \(\tonda{x - 4}^{3}\) % A
\sframeop{g} \(- \tonda{x + 6}^{3}\) % G
\sframeop{c} \(8 \tonda{x + 1}^{3}\) % C
\sframeop{l} \(\tonda{5 x + 3}^{3}\) \\ % L
\sframeop{j} \(8 \tonda{x - 3}^{3}\) % J
\sframeop{e} \(\tonda{5 x + 2}^{3}\) % E
\sframeop{p} \(\tonda{5 x + 6}^{3}\) % P
\sframeop{m} \(64 \tonda{x - 1}^{3}\) % M
\sframeop{i} \(\tonda{5 x - 2}^{3}\) \\ % I
\sframeop{n} \(\tonda{5 x - 3}^{3}\) % N
\sframeop{o} \(27 \tonda{x - 2}^{3}\) % O
\sframeop{h} \(8 \tonda{x + 3}^{3}\) % H
\sframeop{b} \(\tonda{2 x + 1}^{3}\) % B
\sframeop{f} \(- \tonda{2 x + 1}^{3}\) \\ % F
\sframeop{k} \(- \tonda{3 x + 1}^{3}\) % K
\sframeop{t} \(- \tonda{4 x - 1}^{3}\) % T
\sframeop{u} \(\tonda{\dfrac{x}{2} - 2}^{3}\) % U
\sframeop{r} \(\tonda{3 x + 1}^{3}\) % R
\sframeop{s} \(- \tonda{2 x - 1}^{3}\) \\ % S
\sframeop{x} \(\tonda{\dfrac{2}{3} x - 5}^{3}\) % X
\sframeop{y} \(\tonda{5 x - \dfrac{4}{3}}^{3}\) % Y
\sframeop{q} \(\tonda{x + 6}^{3}\) % Q
\sframeop{z} \(\tonda{x + \dfrac{1}{6}}^{3}\) % Z
\sframeop{w} \(\tonda{\dfrac{2}{3}x + 2}^{3}\) \\ % W
\sframeop{v} \(\tonda{-\dfrac{1}{3}x +1}^{3}\) % V
\end{esercizio}

%\subsubsection*{11.5 - Potenza n-esima di un binomio}
\subsubsection*{\numnameref{subsec:prodnot_potenzabinomio}}

\begin{esercizio}
\label{ese:11.28}
Sviluppa la seguente potenza del binomio.
\[\tonda{2a-b^{2}}^{4}=\tonda{2a}^{4}+4\cdot
\tonda{2a}^{3}\cdot \tonda{-b^{2}}+6\tonda{2a}^{2}\cdot
\tonda{-b^{2}}^{2}+\ldots \tonda{2a}\cdot
\tonda{-b^{2}}^{\ldots }+\tonda{-b^{2}}^{\ldots }\]
\end{esercizio}

\begin{esercizio}
\label{ese:11.29}
Sviluppa le seguenti potenze di binomio e associale al corrispondente 
risultato.

\begin{htmulticols}{4}
\begin{enumeratea}
\spazielenx
\item \(\tonda{2 x - 2}^{4}\) % A
\item \(\tonda{- x - 1}^{5}\) % B
\item \(\tonda{x + 3}^{5}\) % C
\item \(\tonda{- x - 3}^{5}\) % D
\item \(\tonda{- 2 x + 1}^{5}\) % E
\item \(\tonda{2 x + \dfrac{1}{2}}^{4}\) % F
\item \(\tonda{2 x - \dfrac{1}{2}}^{4}\) % G
\item \(\tonda{- 3 x - 1}^{5}\) % H
\item \(\tonda{\dfrac{x}{2} - 2}^{5}\) % I
\item \(\tonda{- 3 x - \dfrac{1}{2}}^{4}\) % J
\item \(\tonda{\dfrac{x}{3} + 2}^{4}\) % K
\item \(\tonda{- \dfrac{x}{2} + \dfrac{1}{2}}^{4}\) % L
\item \(\tonda{- x - \dfrac{1}{2}}^{5}\) % M
\item \(\tonda{x + \dfrac{1}{3}}^{5}\) % N
% \item \(\tonda{\dfrac{x}{3} + 1}^{5}\) % O
% \item \(\tonda{- x - \dfrac{1}{3}}^{5}\) % P
% \item \(\tonda{- \dfrac{3 x}{2} + \dfrac{1}{2}}^{4}\) % 
% \sframeop{} \(\dfrac{81 x^{4}}{16} - \dfrac{27 x^{3}}{4} + 
% \dfrac{27 x^{2}}{8} - \dfrac{3 x}{4} + \dfrac{1}{16}}
% \item \(\tonda{\dfrac{3 x}{2} + \dfrac{3}{2}}^{4}\) % 
% \sframeop{} \(\dfrac{81 x^{4}}{16} + \dfrac{81 x^{3}}{4} + \dfrac{243 
% x^{2}}{8} + \dfrac{81 x}{4} + \dfrac{81}{16}}
% \item \(\tonda{- \dfrac{2 x}{3} + 1}^{5}\) % 
% \sframeop{} \(- \dfrac{32 x^{5}}{243} + \dfrac{80 x^{4}}{81} - \dfrac{80 
% x^{3}}{27} + \dfrac{40 x^{2}}{9} - \dfrac{10 x}{3} + 1}
% \item \(\tonda{- \dfrac{3 x}{2} - 1}^{5}\) % 
% \sframeop{} \(- \dfrac{243 x^{5}}{32} - \dfrac{405 x^{4}}{16} - \dfrac{135 
% x^{3}}{4} - \dfrac{45 x^{2}}{2} - \dfrac{15 x}{2} - 1}
% \item \(\tonda{- \dfrac{3 x}{2} + 1}^{5}\) % 
% \sframeop{} \(- \dfrac{243 x^{5}}{32} + \dfrac{405 x^{4}}{16} - \dfrac{135 
% x^{3}}{4} + \dfrac{45 x^{2}}{2} - \dfrac{15 x}{2} + 1}
% \item \(\tonda{\dfrac{x}{3} + \dfrac{1}{3}}^{5}\) % 
% \sframeop{} \(\dfrac{x^{5}}{243} + \dfrac{5 x^{4}}{243} + 
% \dfrac{10x^{3}}{243} 
% + \dfrac{10 x^{2}}{243} + \dfrac{5 x}{243} + \dfrac{1}{243}}
% \item \(\tonda{- \dfrac{x}{3} - \dfrac{2}{3}}^{5}\) % 
% \sframeop{} \(- \dfrac{x^{5}}{243} - \dfrac{10 x^{4}}{243} - \dfrac{40 
% x^{3}}{243} - \dfrac{80 x^{2}}{243} - \dfrac{80 x}{243} - 
% \dfrac{32}{243}}
% \item \(\tonda{- \dfrac{2 x}{3} + \dfrac{1}{3}}^{5}\) % 
% \sframeop{} \(- \dfrac{32 x^{5}}{243} + \dfrac{80 x^{4}}{243} - \dfrac{80 
% x^{3}}{243} + \dfrac{40 x^{2}}{243} - \dfrac{10 x}{243} + 
% \dfrac{1}{243}}
\end{enumeratea}
\end{htmulticols}
\noindent\!\sframeop{?} 
\(\dfrac{x^{4}}{16} - \dfrac{x^{3}}{4} + \dfrac{3 x^{2}}{8} - 
          \dfrac{x}{4} + \dfrac{1}{16}\)
\sframeop{?} \(81 x^{4} + 54 x^{3} + \dfrac{27 x^{2}}{2} + \dfrac{3 x}{2} + 
          \dfrac{1}{16}\) \\
\sframeop{?} \(\dfrac{x^{4}}{81} + \dfrac{8 x^{3}}{27} + \dfrac{8 x^{2}}{3} 
              + \dfrac{32 x}{3} + 16\)
% \sframeop{o} \(x^{5} + \dfrac{5 x^{4}}{3} + \dfrac{10 x^{3}}{9} + \dfrac{10 
%            x^{2}}{27} + \dfrac{5 x}{81} + \dfrac{1}{243}\)  % O
\sframeop{m} \(- x^{5} - \dfrac{5 x^{4}}{2} - \dfrac{5 x^{3}}{2} - \dfrac{5 
          x^{2}}{4} - \dfrac{5 x}{16} - \dfrac{1}{32}\) \\ % M
\sframeop{i} \(\dfrac{x^{5}}{32} - \dfrac{5 x^{4}}{8} + 5 x^{3} - 20 x^{2} + 
              40x - 32\) % I
\sframeop{h} \(- 243 x^{5} - 405 x^{4} - 270 x^{3} - 90 x^{2} - 15 x - 1\)%H
\sframeop{e} \(- 32 x^{5} + 80 x^{4} - 80 x^{3} + 40 x^{2} - 10 x + 1\) % E
\sframeop{g} \(16 x^{4} - 16 x^{3} + 6 x^{2} - x + \dfrac{1}{16}\) \\ % G
\sframeop{f} \(16 x^{4} + 16 x^{3} + 6 x^{2} + x + \dfrac{1}{16}\) % F
\sframeop{b} \(- x^{5} - 5 x^{4} - 10 x^{3} - 10 x^{2} - 5 x - 1\) \\ % B
\sframeop{c} \(x^{5} + 15 x^{4} + 90 x^{3} + 270 x^{2} + 405 x + 243\) % C
\sframeop{a} \(16 x^{4} - 64 x^{3} + 96 x^{2} - 64 x + 16\)  % A
% \sframeop{o} \(\dfrac{x^{5}}{243} + \dfrac{5 x^{4}}{81} + 
%      \dfrac{10 x^{3}}{27} + \dfrac{10 x^{2}}{9} + \dfrac{5 x}{3} + 1\) % O
% \sframeop{p} \(- x^{5} - \dfrac{5 x^{4}}{3} - \dfrac{10 x^{3}}{9} - 
%                \dfrac{10 x^{2}}{27} - \dfrac{5 x}{81} - \dfrac{1}{243}\)% P
\sframeop{d} \(- x^{5} - 15 x^{4} - 90 x^{3} - 270 x^{2} - 405 x - 243\) % D
\end{esercizio}

\begin{esercizio}
\label{ese:11.30}
Trova la regola generale per calcolare il cubo del trinomio
\((A+B+C)^{3}\)
\end{esercizio}

% \pagebreak %---------------------------------------------

\subsection{Esercizi riepilogativi}

\begin{esercizio}[*]
\label{ese:10.21}
Calcola i seguenti prodotti e associali al corrispondente risultato.
\begin{htmulticols}{3}
\begin{enumeratea}
\spazielenx
\item \(\tonda{4 x + 5}\tonda{- 10 x + 7}\) % A
\item \(\tonda{4 x + \dfrac{4}{3}}^{3}\) % B
\item \(\tonda{- 4 x + \dfrac{1}{6}}^{2}\) % C
\item \(\tonda{6 x + \dfrac{2}{5}}^{2}\) % D
\item \(\tonda{4 x - \dfrac{1}{2}}^{3}\) % E
\item \(\tonda{x - 3}\tonda{x - 1}\) % F
\item \(\tonda{2 x - \dfrac{8}{9}}\tonda{- 9 x + 7}\) % G
\item \(\tonda{- \dfrac{x^{2}}{2} - \dfrac{3 x}{2} + 1}^{2}\) % H
\item \(\tonda{- 4 x - \dfrac{6}{5}}\tonda{- 4 x + \dfrac{6}{5}}\) % I
\item \(\tonda{- 2 x - 2}^{3}\) % J
\item \(\tonda{x - 6}\tonda{x - 5}\) % K
\item \(\tonda{- x - 3}^{4}\) % L
\item \(\tonda{\dfrac{2 x^{2}}{5} + x + \dfrac{3}{2}}^{2}\) % M
\item \(\tonda{x - \dfrac{1}{3}}\tonda{x + \dfrac{1}{3}}\) % N
\item \(\tonda{x + 6}^{2}\) % O
\item \(\tonda{- \dfrac{2 x}{5} - \dfrac{2}{3}}\tonda{- \dfrac{2 x}{5} + 
      \dfrac{2}{3}}\) % P
\item \(\tonda{- \dfrac{5 x^{2}}{3} - \dfrac{4 x}{3} - 
      \dfrac{4}{3}}^{2}\) % Q
% \item \(\tonda{- 3 x - 10}\tonda{- 10 x^{2} + 5 x + 1}\) % R
% \item \(\tonda{- 3 x - 1}^{4}\) % S
% \item \(\tonda{- 4 x^{2} - 7 x - 8}\tonda{- 6 x + 5}\) % T
\end{enumeratea}
\end{htmulticols}
\noindent\!\sframeop{o} \(x^{2} + 12 x + 36\) % O
\sframeop{b} \(64 x^{3} + 64 x^{2} + \dfrac{64 x}{3} + \dfrac{64}{27}\) % B
\sframeop{a} \(- 40 x^{2} - 22 x + 35\) \\ % A
% \sframeop{r} \(30 x^{3} + 85 x^{2} - 53 x - 10\) % R
% \sframeop{s} \(81 x^{4} + 108 x^{3} + 54 x^{2} + 12 x + 1\) % S
\sframeop{f} \(x^{2} - 4 x + 3\) % F
\sframeop{p} \(\dfrac{4 x^{2}}{25} - \dfrac{4}{9}\) % P
\sframeop{e} \(64 x^{3} - 24 x^{2} + 3 x - \dfrac{1}{8}\) % E
\sframeop{q} \(\dfrac{25 x^{4}}{9} + \dfrac{40 x^{3}}{9} + 
\dfrac{56 x^{2}}{9} + \dfrac{32 x}{9} + \dfrac{16}{9}\) % Q
\sframeop{g} \(- 18 x^{2} + 22 x - \dfrac{56}{9}\) % G
\sframeop{n} \(x^{2} - \dfrac{1}{9}\) % N
\sframeop{l} \(x^{4} + 12 x^{3} + 54 x^{2} + 108 x + 81\) % L
\sframeop{m} \(\dfrac{4 x^{4}}{25} + \dfrac{4 x^{3}}{5} + 
\dfrac{11 x^{2}}{5} + 3x + \dfrac{9}{4}\) % M
\sframeop{i} \(16 x^{2} - \dfrac{36}{25}\) % I
\sframeop{j} \(- 8 x^{3} - 24 x^{2} - 24 x - 8\) \\ % J
\sframeop{d} \(36 x^{2} + \dfrac{24 x}{5} + \dfrac{4}{25}\) % D
\sframeop{h} \(\dfrac{x^{4}}{4} + \dfrac{3 x^{3}}{2} + \dfrac{5 x^{2}}{4} - 
3 x + 1\) % H
\sframeop{k} \(x^{2} - 11 x + 30\) \\ % K
\sframeop{c} \(16 x^{2} - \dfrac{4 x}{3} + \dfrac{1}{36}\) % C
% \sframeop{t} \(24 x^{3} + 22 x^{2} + 13 x - 40\) % T
\end{esercizio}

\begin{esercizio}
\label{ese:10.21}
Risolvi le seguenti espressioni con i polinomi.
\begin{enumeratea}
\spazielenx
\item \((-a-1-2a)-(-3-a+4a)\) 
\solno{=-a-1-2a+3+a-4a=}
\sol{-6a +2}
\item \(\tonda{2a^{2}-3b}-\quadra{\tonda{4b+3a^{2}}-\tonda{a^{2}-2b}}\) 
\solno{=2a^{2}-3b-4b-3a^{2}+a^{2}-2b=}
\sol{-9b}
\item \(\tonda{2a^{2}-5b}-\quadra{\tonda{2b+4a^{2}}-\tonda{2a^{2}-2b}}+9b\)
% \solno{=2a^{2}-5b-2b-4a^{2}+2a^{2}-2b+9b=} 
\sol{0}
\item 
\(3a\quadra{2(a-2ab)+3a\tonda{\dfrac{1}{2}+3b}-\dfrac{1}{2}a(3-5b)}\) 
% \solno{
% = 3a\quadra{2a-4ab+\dfrac{3}{2}a+9ab-\dfrac{3}{2}a+\dfrac{5}{2}ab}=}
\sol{\dfrac{45}{2}a^{2}b + 6a^2}
\item \(2(x-1)(3x+1)-\tonda{6x^{2}+3x+1}+2x(x-1)\) 
% \solno{=6x^2+2x-6x-2-6x^2-3x-1+2x^2-2x=}
\sol{2x^2 -9x-3}
\item \(\tonda{\dfrac{1}{3}x-1}(3x+1) - 2x\tonda{\dfrac{5}{4}x - 
\dfrac{1}{2}}(x+1) - \dfrac{1}{2}x\tonda{x-\dfrac{2}{3}}\) 
% \solno{=x^2-\dfrac{8}{3}x-1+(-\dfrac{5}{2}x^2+x)(x+1)-\dfrac{1}{2}x^2
% +\dfrac{1}{3}x=\\
% =\dfrac{1}{2}x^2-\dfrac{7}{3}x-1-
% \dfrac{5}{2}x^3+x^2-\dfrac{5}{2}x^2+x=}
\sol{-\dfrac{5}{2}x^3-x^2-\dfrac{4}{3}x-1}
%  \sol{x^{2} - 2 x x{\left(\dfrac{5 x}{4} - \dfrac{1}{2} \right)} -
%  \dfrac{8 x}{3} - \dfrac{x{\left(x - \dfrac{2}{3} \right)}}{2} - 
%  2 x{\left(\dfrac{5 x}{4} - \dfrac{1}{2} \right)} - 1}
\item \(\tonda{a-b}\tonda{a+b} + \tonda{a+b}\tonda{ab-a} - 
\tonda{ab+a}^2 + ab\tonda{ab+a}\) 
% \solno{=a^2-b^2+a^2b-a^2+ab^2-ab-a^2b^2-2a^2b-a^2+a^2b^2+a^2b=}
\sol{-a^2+ab^2-ab-b^2}
\item \(ab\tonda{a^{2} - b^{2}} + 2b\tonda{x^{2} - a^{2}}(a-b) - 
2bx^{2}(a-b)\) 
% \solno{=a^3b -ab^3+2b\tonda{ax^2-bx^2-a^3+a^2b}-2abx^2+2b^2x^2 =\\
% =a^3b -ab^3+2abx^2-2b^2x^2-2a^3b+2a^2b^2-2abx^2+2b^2x^2=} 
\sol{- a^{3} b + 2 a^{2} b^{2} - a b^{3}}
\item \(2y \tonda{\dfrac{3}{2}x^{2} - \dfrac{1}{2}x} -
      \tonda{2xy - \dfrac{1}{3}y} 4x + \tonda{5x+2}\tonda{xy-3}\) 
% \solno{=3x^2y-xy-8x^2y+\dfrac{4}{3}xy+5x^2y-15x+2xy-6=}
\sol{+\dfrac{7}{3}xy -15x -6}
\item \(\tonda{\dfrac{1}{2}a-\dfrac{1}{2}a^{2}}(1-a)\quadra{a^{2}+2a - 
\tonda{a^{2}+a+1}}\) 
% \solno{=\tonda{\dfrac{1}{2}a-\dfrac{1}{2}a^2-
%                   \dfrac{1}{2}a^2+\dfrac{1}{2}a^3}
%            \tonda{a-1}=\)\\
% \(=               \dfrac{1}{2}a^2-\dfrac{1}{2}a^3-
%                   \dfrac{1}{2}a^3+\dfrac{1}{2}a^4-
%                   \dfrac{1}{2}a+\dfrac{1}{2}a^2+
%                   \dfrac{1}{2}a^2-\dfrac{1}{2}a^3\)}
\sol{\dfrac{1}{2}a^{4} - \dfrac{3}{2}a^{3} + \dfrac{3}{2}a^{2} - 
\dfrac{1}{2}a}
\item \((1-3x)(1-3x)-(-3x)^{2}+5(x+1)-3(x+1)-7\) 
\sol{-4x -4}
\item \(3\tonda{x-\dfrac{1}{3}y}\quadra{2x+\dfrac{1}{3}y-(x-2y)} - 
2\tonda{x-\dfrac{1}{3}y}(2x+3y)\) %???????????????????????
% \solno{=\tonda{3x-y}\tonda{x+\dfrac{7}{3}y} - 
% \tonda{2x-\dfrac{2}{3}y}(2x+3y)=\)\\
% \(=3x^2+7xy-xy-\dfrac{7}{3}y^2-4x^2-6xy+\dfrac{4}{3}xy+2y^2=}
%  \sol{- 5 x^{2} - \dfrac{10 x y}{3} + \dfrac{5 y^{2}}{3}}\\
\sol{-x^2 +\dfrac{4}{3}xy -\dfrac{1}{3}y^2}
\item \(\dfrac{1}{24}(29x+7)-\dfrac{1}{2}x^{2}+\dfrac{1}{2}(x-3)(x-3)-2 - 
\quadra{\dfrac{1}{3}-\dfrac{3}{2}\tonda{\dfrac{3}{4}x+\dfrac{2}{3}}}\) 
\sol{-\dfrac{2}{3}x + \dfrac{83}{24}} 
% \item \(-{\dfrac{1}{4}}\tonda{2 abx+2a^{2}b^{2}+3ax} 
% +a^{2}(b^{2}+x^{2})-\quadra{\tonda{\dfrac{1}{3} ax}^{2} 
% -\tonda{\dfrac{2}{3}bx}^{2}}+
% \dfrac{1}{4}ax\tonda{2b+3}\) \\ 
%  \sol{\dfrac{a^{2} b^{2}}{2} + \dfrac{8 a^{2} x^{2}}{9} + 
%  \dfrac{a b x}{2}  + \dfrac{3 a x}{4} + \dfrac{4 b^{2} x^{2}}{9}}\\
%  \sol{a^2x^2 -\dfrac{1}{2}abx - \dfrac{3}{4}ax+ \dfrac{1}{2}ab^2}
\item \(\tonda{\dfrac{1}{3}x+\dfrac{1}{2}y-
\dfrac{3}{5}}\tonda{\dfrac{1}{3}x- 
\dfrac{1}{2}y+\dfrac{3}{5}}-\quadra{\tonda{\dfrac{1}{3}x}^{2}- 
\tonda{\dfrac{1}{2}y}^{2}}\) 
\sol{\dfrac{3}{5}y - \dfrac{9}{25}}
\item \(\tonda{\dfrac{1}{2}x-1}\tonda{\dfrac{1}{4}x^{2}+\dfrac{1}{2}x+1} + 
\tonda{ -{\dfrac{1}{2}}x}^{3}+2\tonda{\dfrac{1}{2}x+1}\) 
\sol{x + 1}
\item \((3a-2)(3a+2)-(a-1)(2a-2)+a(a-1)\tonda{a^{2}+a+1}\) 
\sol{a^4 + 7 a^2 + 3 a - 6}
\item \(-4x(5-2x)+\tonda{1-4x+x^{2}}\tonda{1-4x-x^{2}}\) 
\sol{-x^4 + 24 x^2 - 28 x + 1}
\item \(-(2x-1)(2x-1)+\quadra{x^{2}-\tonda{1+x^{2}}}^{2}-\tonda{x^{2} 
-1}\tonda{x^{2}+1}\) 
\sol{- x^{6} - 2 x^{4} - 3 x^{2} + 4 x + 2}\\
\sol{-x^4 - 4 x^2 + 4 x + 1}
\item \(4(x+1)-3x(1-x)-(x+1)(x-1)-\tonda{4+2x^{2}}\) 
\sol{x + 1}
\item \(\dfrac{1}{2}(x+1)+\dfrac{1}{4}(x+1)(x-1) - \tonda{x^{2}-1}\) 
\sol{-\dfrac{3}{4}x^2 + \dfrac{1}{2}x + \dfrac{5}{4}}
\item \((3x+1)\tonda{\dfrac{5}{2}+x} - (2x-1)(2x+1)(x-2)+2x^{3}\) 
\sol{- 4 x^{3} + 19 x^{2} + \dfrac{21 x}{2} - \dfrac{3}{2}}\\
\sol{-2x^3 + 11 x^2 + \dfrac{19}{2}x + \dfrac{1}{2}}
\item \(\tonda{a-\dfrac{1}{2}b}a^{3} - 
\tonda{\dfrac{1}{3}{ab}-1}\quadra{2a^{2}(a-b) - a\tonda{a^{2}-2{ab}}}\)  
\sol{- \dfrac{a^{4} b}{3} + a^{4} - \dfrac{a^{3} b}{2} + a^{3}}\\
\sol{-\dfrac{2}{3}a^{4} b +a^4 -\dfrac{1}{2}a^3b +a^3}
\item \(\tonda{3x^2+6xy-4y^2}\tonda{\dfrac{1}{2}xy - \dfrac{2}{3}y^2}\) 
\sol{\dfrac{3}{2}x^{3}y +x^{2}y^{2} -6xy^{3} + \dfrac{8}{3}y^{4}}
\item \((2a-3b)\tonda{\dfrac{5}{4}a^{2} + \dfrac{1}{2}{ab} - 
\dfrac{1}{6}b^{2}} - \dfrac{1}{6}a\tonda{12a^{2} - 
\dfrac{18}{5}b^{2}}+\dfrac{37}{30}ab^{2} - \dfrac{1}{2}a\tonda{a^{2} 
-\dfrac{11}{2}{ab}}\) 
\sol{\dfrac{b^3}{2}}
\item \(\dfrac{1}{3}xy\quadra{\tonda{x-y^{2}}\tonda{x^{2} -\dfrac{1}{2}y} - 
3x\tonda{-{\dfrac{1}{9}xy}}\tonda{3y}} - \dfrac{1}{3}x\tonda{x^{3}y + 
\dfrac{1}{4}xy^{2}}\) 
\sol{\dfrac{1}{6}xy^4 - \dfrac{1}{4}x^2 y^2}
\end{enumeratea}
\end{esercizio}
% 
% \begin{esercizio}[*]
% \label{ese:10.21}
% Risolvi le seguenti espressioni con i polinomi.
%  \begin{enumeratea}
% \spazielenx
%  \item \((-a-1-2)-(-3-a+a)\)
%   \sol{-a}
%  \item \(\tonda{2a^{2}-3b}-\quadra{\tonda{4b+3a^{2}}-
%         \tonda{a^{2}-2b}}\)
%   \sol{-9b}
%  \item \(\tonda{2a^{2}-5b}-\quadra{\tonda{2b+4a^{2}}-
%         \tonda{2a^{2}-2b}}-9b\)
%   \sol{-18b}
%  \item \(3a\quadra{2(a-2{ab})+3a\tonda{\dfrac{1}{2}-3b}-
%         \dfrac{1}{2}a(3-5b)}\)
%   \sol{6a^{2}-\dfrac{63}{2}a^{2}b}
%  \item \(2(x-1)(3x+1)-\tonda{6x^{2}+3x+1}+2x(x-1)\)
%   \sol{2x^2-9x-3}
%  \item 
% \(\tonda{\dfrac{1}{3}x-1}(3x+1)-2x\tonda{\dfrac{5}{4}x-\dfrac{1}{2}
% }(x+1)-\dfrac{1}{2}x\tonda{x-\dfrac{2}{3}}\)
%  \item 
% \(\tonda{b^{3}-b}(x-b)+(x+b)\tonda{ab^{2}-a}+(b+a)\tonda{ab-ab^{3}
% }+2ab\tonda{b-b^{3}}\)
%  \item 
% \(ab\tonda{a^{2}-b^{2}}+2b\tonda{x^{2}-a^{2}}(a-b)-2bx^{2}(a-b)\)
%  \item 
% \(\tonda{\dfrac{3}{2}x^{2}y-\dfrac{1}{2}{xy}}\tonda{2x-\dfrac{1}{3}
% y}4x\)
%  \item 
% \(\tonda{\dfrac{1}{2}a-\dfrac{1}{2}a^{2}}(1-a)
% \quadra{a^{2}+2a-\tonda{a^{2}+a+1}}\)
% %  \end{enumeratea}
% % \end{esercizio}
% % 
% % \begin{esercizio}
% % \label{ese:10.23}
% % Risolvi le seguenti espressioni con i polinomi.
% %  \begin{enumeratea}
%  \item \((1-3x)(1-3x)-(-3x)^{2}+5(x+1)-3(x+1)-7\)
%  \item 
% \(3\tonda{x-\dfrac{1}{3}y}\quadra{2x+\dfrac{1}{3}y-(x-2y)}
% -2\tonda{x-\dfrac{1}{3}y+2}(2x+3y)\)
% %  \item 
% % 
% % \(\dfrac{1}{24}(29x+7)-\dfrac{1}{2}x^{2}+\dfrac{1}{2}(x-3)(x-3)-2-
% % \quadra{\dfrac{1}{3}-\dfrac{3}{2}\tonda{\dfrac{3}{4}x+\dfrac{2}{3}}}\)
% %  \item \(-{\dfrac{1}{4}}\tonda{2 abx+2a^{2}b^{2}+3 
% % ax}+a^{2}(b^{2}+x^{2})-\quadra{\tonda{\dfrac{1}{3} 
% % ax}^{2}-\tonda{\dfrac{2}{3}bx}^{2}}\)
% %  \item 
% % 
% % \(\tonda{\dfrac{1}{3}x+\dfrac{1}{2}y-\dfrac{3}{5}}\tonda{\dfrac{1}{3}x-\
% % dfrac
% % 
% % {1}{2}y+\dfrac{3}{5}}-\quadra{\tonda{\dfrac{1}{3}x}^{2}-\tonda{\
% % dfrac{1}
% % {2}y}^{2}}\)
% %  \item 
% % \(\tonda{\dfrac{1}{2}x-1}\tonda{\dfrac{1}{4}x^{2}+\dfrac{1}{2}
% % 
% % x+1}+\tonda{-{\dfrac{1}{2}}x}^{3}+2\tonda{\dfrac{1}{2}x+1}\)
%  \item \((3a-2)(3a+2)-(a-1)(2a-2)+a(a-1)\tonda{a^{2}+a+1}\)
%  \item \(-4x(5-2x)+\tonda{1-4x+x^{2}}\tonda{1-4x-x^{2}}\)
%  \item 
% \(-(2x-1)(2x-1)+\quadra{x^{2}-\tonda{1+x^{2}}}^{2}-\tonda{x^{2}
% -1}\tonda{x^{2}+1}\)
% %  \end{enumeratea}
% % \end{esercizio}
% % 
% % \begin{esercizio}
% % \label{ese:10.25}
% % Risolvi le seguenti espressioni con i polinomi.
% %  \begin{enumeratea}
%  \item \(4(x+1)-3x(1-x)-(x+1)(x-1)-\tonda{4+2x^{2}}\)
%  \item \(\dfrac{1}{2}(x+1)+\dfrac{1}{4}(x+1)(x-1)-\tonda{x^{2}-1}\)
%  \item \((3x+1)\tonda{\dfrac{5}{2}+x}-(2x-1)(2x+1)(x-2)+2x^{3}\)
%  \item \(\tonda{a-\dfrac{1}{2}b}a^{3}-\tonda{\dfrac{1}{3}{ab}-1}
%  \quadra{2a^{2}(a-b)-a\tonda{a^{2}-2{ab}}}\)
%   \sol{a^{4}-\dfrac{1}{2}a^{3}b-\dfrac{1}{3}a^{4}b-a^{3}}
%  \item 
% \(\tonda{3x^2+6xy-4y^2}\tonda{\dfrac{1}{2}xy-\dfrac{2}{3}y^2}\)
%   \hfill 
% \(\quadra{\dfrac{3}{2}x^{3}y+x^{2}y^{2}-6{xy}^{3}+\dfrac{8}{3}y^{4}}\)
%  \item \((2a-3b)\tonda{\dfrac{5}{4}a^{2}+\dfrac{1}{2}{ab}-
%         \dfrac{1}{6}b^{2}}-\dfrac{1}{6}a\tonda{12a^{2}-
%         \dfrac{18}{5}b^{2}}+\dfrac{37}{30}ab^{2}-
%         \dfrac{1}{2}a\tonda{a^{2}-\dfrac{11}{2}{ab}}\)
%   \sol{\dfrac{1}{2}b^{3}}
% %  \item \(\dfrac{1}{3}xy\quadra{\tonda{x-y^{2}}\tonda{x^{2}-
% %         \dfrac{1}{2}y}-3x\tonda{-{\dfrac{1}{9}xy}}
% %         \tonda{3y}}-\dfrac{1}{3}x\tonda{x^{3}y+
% %         \dfrac{1}{4}xy^{2}}\)
% %   \sol{\dfrac{1}{6}xy^{4}-\dfrac{1}{4}x^{2}y^{2}}
%  \end{enumeratea}
% \end{esercizio}

% \begin{esercizio}[*]
% \label{ese:10.27}
% Risolvi la seguente espressione con i polinomi.
% \begin{multline*}
% 
% \dfrac{1}{2}x\quadra{\tonda{x-y^{2}}\tonda{x^{2}+\dfrac{1}{2}
% 
% y}-5x\tonda{-{\dfrac{1}{10}}{xy}}(4y)}-\dfrac{1}{2}x\tonda{x
% ^{3}
% y+\dfrac{1}{2}xy^{2}}\\
% 
% 
% -\dfrac{1}{2}x^{2}\tonda{x^{2}+\dfrac{1}{2}y+{xy}^{2}}+\dfrac{1}{4}{xy}
% \tonda{y^{2}+2x^{3}+{xy}}.
% \end{multline*}
% \end{esercizio}
% 
% \begin{esercizio}[*]
% \label{ese:10.28}
% Risolvi la seguente espressione con i polinomi.
% \begin{multline*}
% 
% 
% \tonda{\dfrac{2}{3}a-2b}\tonda{\dfrac{3}{2}a+2b}\tonda{\dfrac{9}{4
% }a^{2
% 
% }+4b^{2}}-\dfrac{3}{4}\tonda{\dfrac{9}{4}a^{2}}-a^{2}\tonda{\dfrac
% {9}{
% 4}a^{2}-5b^{2}}\\
% +5{ab}\tonda{\dfrac{3}{4}a^{2}+\dfrac{4}{3}b^{2}}.
% \end{multline*}
% \end{esercizio}
% 
% \begin{esercizio}[*]
% \label{ese:10.29}
% Risolvi la seguente espressione con i polinomi.
% \begin{multline*}
% % 
% 
% \tonda{\dfrac{1}{2}x+2y}\tonda{\dfrac{1}{2}x-2y}\tonda{\dfrac{1}{4
% }x^{2
% }-4y^{2}}-\dfrac{1}{4}x\tonda{\dfrac{27}{4}x^{3}-\dfrac{61}{3}xy^{2}
% }\\
% -16\tonda{y^{4}+x^{4}}-\dfrac{37}{12}x^{2}y^{2}+\dfrac{141}{8}x^{4}.
% \end{multline*}
% \end{esercizio}
% 
% \begin{esercizio}[*]
% \label{ese:10.30}
% Risolvi la seguente espressione con i polinomi.
% \begin{multline*}
% 
% 
% x\tonda{\dfrac{2}{3}y^{2}-\dfrac{27}{8}x^{2}}-\quadra{-\tonda{\dfrac{3}{2}
% 
% x-\dfrac{2}{3}y}\tonda{\dfrac{9}{4}x^{2}+xy+\dfrac{4}{3}y^{2}}+\
% dfrac
% {2}{3}x^{2}\tonda{\dfrac{9}{4}y^{2}+\dfrac{1}{3}y}}\\
% +\dfrac{2}{9}y\tonda{x^{2}+4y^{2}-9xy}.
% \end{multline*}
% \end{esercizio}
% 
% \begin{esercizio}[*]
% \label{ese:10.31}
% Risolvi la seguente espressione con i polinomi.
% \begin{multline*}
% 
% 
% \tonda{\dfrac{1}{2}ab+\dfrac{2}{3}xy}\tonda{\dfrac{1}{2}ab-\dfrac{2}{3}
% 
% xy}-\quadra{\tonda{\dfrac{1}{2}ab}^{2}-\tonda{\dfrac{2}{3}xy}^
% {2}
% 
% }\tonda{\dfrac{1}{2}ax}+\dfrac{3}{2}ax\tonda{\dfrac{2}{3}a-\dfrac{
% 2}{3
% }y}\\
% 
% 
% -x\tonda{\dfrac{1}{2}ax+\dfrac{3}{4}xy}-\dfrac{2}{9}x^{2}y^{2}(ax-2)+\
% dfrac
% 
% {1}{4}a^{2}b^{2}\tonda{\dfrac{1}{2}ax-1}+\dfrac{3}{4}x^{2}\tonda{y+\
% dfrac{2}
% {3}a}.
% \end{multline*}
% \end{esercizio}
% 
% \begin{esercizio}[*]
% \label{ese:10.32}
% Risolvi la seguente espressione con i polinomi.
% \begin{multline*}
% 
% 
% 
% \dfrac{1}{6}ab-\dfrac{1}{3}a^{2}-\graffa{\dfrac{3}{4}ab+
% \dfrac{1}{2}a\quadra{\dfrac{3}{2}b-\tonda{\dfrac{1}{6}a-
% \dfrac{4}{5}a\cdot {\dfrac{25}{3}a}}\tonda{-{\dfrac{2}{3}ab}}-
% \tonda{-{\dfrac{8}{3}ab}}\tonda{-{\dfrac{9}{8}b}}}}\\
% +\dfrac{1}{3}a\tonda{a-5b-9a^{3}b+\dfrac{1}{6}a^{2}b}.
% \end{multline*}
% \end{esercizio}
% 
% \begin{esercizio}[*]
% \label{ese:10.33}
% Risolvi la seguente espressione con i polinomi.
% \begin{multline*}
% 
% 
% 
% \dfrac{1}{5}x^{2}+\left\{\quadra{2x-\tonda{\dfrac{3}{2}x^{2}y-
% \dfrac{7}{4}xy+\
% dfrac
% {1}{8}y^{3}}:\tonda{-{\dfrac{1}{2}y}}} 
% 2x-\dfrac{7}{10}xy\right\}\tonda{-{\dfrac{1}{6}x^{2}}}\\
% 
% 
% +x^{2}y-\dfrac{1}{3}x\tonda{\dfrac{3}{5}x}-x^{2}\tonda{y-x^{3}-\dfrac{1}
% {12}
% xy^{2}}.
% \end{multline*}
% \end{esercizio}

% \paragraph{\ref{ese:10.27}} \(0\)
% \paragraph{\ref{ese:10.28}} \(-16b^{4}-\dfrac{27}{16}a^{2}\)
% \paragraph{\ref{ese:10.29}} \(0\)
% \paragraph{\ref{ese:10.30}} \(-\dfrac{3}{2}x^{2}y^{2}\)
% \paragraph{\ref{ese:10.31}} \(a^{2}x-axy\)
% \paragraph{\ref{ese:10.32}} \(-\dfrac{7}{9}a^{4}b+\dfrac{3}{2}a^2b^2-3ab\)
% \paragraph{\ref{ese:10.33}} \(\dfrac{1}{2}x^{4}+\dfrac{7}{60}x^{3}y\)
% \newpage

\begin{esercizio}
\label{ese:10.34}
Se \(A=x-1\), \(B=2x+2\), \(C=x^2-1\) determina
\begin{htmulticols}{3}
\begin{enumeratea}
\item \(A+B+C\)
\item \(A\cdot B-C\)
\item \(A+B\cdot C\)
\item \(A\cdot B\cdot C\)
\item \(2AC-2BC\)
\item \((A+B)\cdot C\)
\end{enumeratea}
\end{htmulticols}
\end{esercizio}

\begin{esercizio}[*]
\label{ese:10.35}
Operazioni tra polinomi con esponenti letterali.

\begin{enumeratea}
\spazielenx
\item \(\tonda{a^{n+1}-a^{n+2}+a^{n+3}}:\tonda{a^{1+n}}\)
\sol{1-a+a^{2}}
\item \(\tonda{1+a^{n+1}}\tonda{1-a^{n-1}}\)
\sol{1-a^{n-1}+a^{n+1}-a^{2}n}
\item \(\tonda{16a^{n+1}b^{n+2}-2a^{2n}b^{n+3}+5a^{n+2}b^{n+1}}:
      \tonda{2a^{n}b^{n}}\)
\sol{8ab^2-a^nb^3+\dfrac{5}{2}a^2b}
\item \(\tonda{a^{n+1}-a^{n+2}+a^{n+3}}\tonda{a^{n+1}-a^{n}}\)
\sol{a^{2n+4}-2a^{2n+3}+2a^{2n+2}-a^{2n+1}}
\item \(\tonda{a^{n}-a^{n+1}+a^{n+2}}\tonda{a^{n+1}-a^{n-1}}\)
\sol{a^{2n+3}-a^{2n+2}-a^{2n-1}+a^{2n}}
\item \(\tonda{a^{n}+a^{n+1}+a^{n+2}}\tonda{a^{n+1}-a^{n}}\)
\sol{-a^{2}n+a^{2n+3}}
\item \(\tonda{a^{n+2}+a^{n+1}}\tonda{a^{n+1}+a^{n+2}}\)
\sol{a^{2n+4}+2a^{2n+3}+a^{2n+2}}
\item \(\tonda{1+a^{n+1}}\tonda{a^{n+1}-2}\)
\sol{a^{2n+2}-a^{n+1}-2}
\item \(\tonda{a^{n+1}-a^{n}}\tonda{a^{n+1}+a^{n}}
      \tonda{a^{2n+2}+a^{2n}}\)
\sol{a^{4n+4}-a^{4n}}
% \item \(\tonda{\dfrac{1}{2}x^{n}-\dfrac{3}{2}x^{2n}}
%        \tonda{\dfrac{1}{3}x^{n}-\dfrac{1}{2}}-
%        \tonda{\dfrac{1}{3}x^{n}-1}\tonda{x^{n}+x}\)
%   \sol{\dfrac{7}{12}x^{2n}+\dfrac{3}{4}x^{n}-
% \dfrac{1}{2}x^{3n}-\dfrac{1}{3}x^{n+1}+x}
\end{enumeratea}
\end{esercizio}

\begin{htmulticols}{2}
\begin{esercizio}
\label{ese:10.36}
Se si raddoppiano i lati di un rettangolo, come varia il suo
perimetro?
\end{esercizio}

\begin{esercizio}
\label{ese:10.37}
Se si raddoppiano i lati di un triangolo rettangolo, come varia la sua
area?
\end{esercizio}

\begin{esercizio}
\label{ese:10.38}
Se si raddoppiano gli spigoli~\(a\), \(b\), e~\(c\) di un parallelepipedo, 
come varia il suo volume?
\end{esercizio}

\begin{esercizio}
\label{ese:10.39}
Come varia l'area di un cerchio se si triplica il suo raggio?
\end{esercizio}

\begin{esercizio}
\label{ese:10.40}
Determinare l'area di un rettangolo avente come
dimensioni~\(\dfrac{1}{2}a\) e~\(\dfrac{3}{4}a^{2}b\)
\end{esercizio}

\begin{esercizio}
\label{ese:10.41}
Determinare la superficie laterale di un cilindro avente raggio di
base~\(x^{2}y\) e altezza~\(\dfrac{1}{5}{xy}^{2}\)
\end{esercizio}
\end{htmulticols}

\begin{esercizio}[*]
\label{ese:11.31}
Risolvi utilizzando i prodotti notevoli.
\begin{enumeratea}
\spazielenx
\item 
\(\quadra{a+2\tonda{b-c}}\quadra{a-2\tonda{b-c}}+4b(b-2c)\)
\sol{a^{2}-4c^{2}}
\item 
\(\quadra{\tonda{a-2b}^{2}-a^{3}}\quadra{-a^{3}-\tonda{a-2b}^2}
+a^{2}(a^{2}-8{ab}+24b^{2}-a^{4})\)
\sol{+32ab^{3}-16b^{4}}
\item \(x(x-1)^{2}+(x+1)(x-1)-x(x+1)(x-3)-(x+2)^{2}\)
\sol{-5}
\item \((x+1)^{2}-(x-1)^{2}\)
\sol{4x}
\item \((x+1)^{3}-(x-1)^{3}-6x^2\)
\sol{2}
\item \((x+1)^{2}+(x-2)^{2}-(x-1)^{2}-(x+1)(x-1)\)
\sol{5}
\item \((x+2)(x-2)+(x+2)^{2}\)
\sol{2x^{2}-4x}
\item \((x+1)^{3}-(x-1)\tonda{x^{2}+x+1}+3x(x-1)\)
\sol{6x{2}+2}
\item \((x+1)(x-1)+(x+1)^{2}+(x-1)^{2}\)
\sol{3x^{2}+1}
\item \((x+y+1)(x+y-1)+(x+y)^{2}-2(x+y)(x-y)-(2y-1)(2y+1)\)
\sol{4xy}
%  \end{enumeratea}
% \end{esercizio}
% 
% \begin{esercizio}[*]
%  \label{ese:11.33}
% Risolvi utilizzando i prodotti notevoli.
%  \begin{enumeratea}
%  \item 
% 
% \(\tonda{\dfrac{1}{2}a+\dfrac{2}{3}-3b+\dfrac{1}{3}{ab}}\tonda{\dfrac{1}
% {2}
% 
% a-\dfrac{2}{3}-3b-\dfrac{1}{3}{ab}}+\dfrac{1}{9}{ab}(31+{ab})-\tonda{\
% dfrac
% {1}{2}a-\dfrac{2}{3}}\tonda{\dfrac{1}{2}a+\dfrac{2}{3}}\)
%   \sol{9b^2}
\item \((x-y)^{2}+(x+y)(y-x)\)
\sol{2y^{2}-2xy}
\item \((x+y-z)^2+(x-y+z)^2-2(x-y-z)^2\)
\sol{4xy+4xz-8yz}
\item \((a-3b)^{2}+(2a+3b)(2a-3b)-(a+2b)(b-2a)\)
\sol{7a^2-3ab-2b^2}
\item 
\(\quadra{3x^{2}-(x+2y)(x-2y)}^{2}-2x\tonda{\dfrac{1}{2}x-\dfrac{3}{2}
y}^{2}-3{xy}\tonda{x+\dfrac{3}{2}y}-\tonda{2x^{2}+4y^{2}}^{2}\)\\
\sol{-\dfrac{1}{2}x^{3}-9xy^{2}}
\item 
\(\quadra{\tonda{x^{2}+2y}^{2}-\tonda{x^{2}-2y}^{2}}\quadra{
\tonda{x^{2}+2y}^{2}+\tonda{x^{2}-2y}^{2}}\)
\sol{16 x^{6} y + 64 x^{2} y^{3}}
%  \item 
% \(8a \tonda{-a-b}\tonda{-a+b}-\tonda{2a-b}^{3}+ b^2 \tonda{6a -b}\)
%   \sol{12 a^{2} b - 8 a b^{2}}
%  \item
%  \(\tonda{x^2 -2y^2}^2 - \tonda{x^2+3y^2}\tonda{x^2-3y^2} + 
%    2y^2\tonda{2x^2-4y^2}\)
%   \sol{5y^4}
%  \item 
% \(\tonda{x^{2}+yx+\dfrac{2}{3}}^{2}-
%   \tonda{3b^{2}+\dfrac{1}{2}a^{4}+2a^{3}+\dfrac{1}{3}a^{2}}^{2}\)
%  \item 
% \(\tonda{3x^{2}-4{xy}+\dfrac{2}{5}-y^{2}x+\dfrac{1}{2}y^{3}}^{2}+
% \tonda{2x^{2}y^{2}+\dfrac{3}{2}y^{2}}\tonda{2x^{2}y^{2}-
%        \dfrac{3}{2}y^{2}}\)
%  \item 
% \(\tonda{3x^{2}-4{xy}-\dfrac{3}{2}y^{2}}^{2}+
% \tonda{2x^{2}y^{2}+\dfrac{3}{2}y^{2}}\tonda{2x^{2}y^{2}-\dfrac{3}{2}y^{2}}\)
\end{enumeratea}
\end{esercizio}

\begin{esercizio}[*]
\label{ese:11.35}
Risolvi utilizzando i prodotti notevoli.
\begin{enumeratea}
\spazielenx
\item 
\(-2x(x-1)^{2}+2x\tonda{x-\dfrac{1}{3}}^{2}-
\dfrac{4}{3}x\tonda{2x-\dfrac{4}{3}}\)
\sol{0}
\item \((a-2b)^{4}-b(2a-b)^{3}-a^{2}(a+6b)^{2}\)
\sol{17b^{4}-38{ab}^{3}-28a^{3}b}
\item \([(x-1)^2-2]^2-\tonda{x^2+x-1}^2+6x(x-1)(x+1)\)
\sol{3x^{2}}
\item \((x+1)^{4}-(x+1)^{2}(x-1)^{2}-4x(x+1)^{2}\)
\sol{0}
\item \(\dfrac{(x-2)(x+2)}{4}+\dfrac{(x-2)^{2}}{(-2)^{2}}+x\)
\sol{\dfrac{1}{2}x^{2}}
\item \(\tonda{2x-\dfrac{1}{3}}^{3}+4\tonda{x+\dfrac{1}{2}}^{2}\)
\sol{8x^{3}+\dfrac{14}{3}x+\dfrac{26}{27}}
\item \((x+1)^{3}-3(x-1)(-1-x)+(x-4)(x+1)\)
\sol{x^{3}+7x^{2}-6}
\item 
\(\tonda{x-\dfrac{1}{3}}^{2}+\tonda{x+\dfrac{1}{3}}^{2}-(x+1)^{2}
-\tonda{x-\dfrac{4}{3}}\tonda{x+\dfrac{4}{3}}\)
\sol{1-2x}
\item \((x-3)^{3}-x^{2}(x-9)-9(x-3)-9\)
\sol{18x-9}
\item \(x(x-1)^{2}(x+1)+(x-1)^{2}-x(x-1)^{3}\)
\sol{2x^3-3x^2+1}
%  \end{enumeratea}
% \end{esercizio}
% 
% \begin{esercizio}[*]
%  \label{ese:11.37}
% Risolvi utilizzando i prodotti notevoli.
%  \begin{enumeratea}
%  \item 
% \(-\tonda{x+\dfrac{3}{4}}\tonda{x+\dfrac{1}{2}}x-
%   \tonda{\tonda{x^2+\dfrac{1}{2}x+x-\dfrac{1}{2}}
%   \tonda{9x^2+3x+\dfrac{1}{4}}}+
%   \dfrac{5}{8}x + \dfrac{1}{8}\)
%   \sol{- 9 x^{4} - \dfrac{35}{2} x^{3} - \dfrac{3}{2} x^{2} + 
%            \dfrac{11}{8} x + \dfrac{1}{4}}
\item
\(-{\dfrac{1}{2}x}\tonda{x+\dfrac{3}{4}}(2x+1)-
\quadra{x+1\tonda{x-\dfrac{1}{2}}\tonda{3x+\dfrac{1}{2}}^{2}}+
\dfrac{1}{8}(5x+1)\)
\sol{8x^{3}-\dfrac{11}{4}x^{2}}
\item 
\(\dfrac{1}{9}(x-4)(x+4)+\dfrac{1}{3}(x-1)^{2}-\dfrac{1}{9}x(x-2)+
\tonda{x-\dfrac{5}{2}}\tonda{x+\dfrac{1}{3}}+\dfrac{41}{18}\)
\sol{\dfrac{4}{3}x^{2}-\dfrac{47}{18}x}
\item 
\(\tonda{\dfrac{1}{2}x^{2}+1}^{3}+\dfrac{1}{6}x^{2}-
\tonda{\dfrac{1}{2}x^{2}-1}^{3}-\dfrac{1}{6}(x+1)^{3}-
\dfrac{3}{2}x^{4}+\dfrac{1}{6}\tonda{x^3-11}\)
\sol{-{\dfrac{1}{2}}x-\dfrac{1}{3}x^{2}}
\item 
\(-x^{2}\tonda{x^{2}-1}+\tonda{x^{2}-4x+2}^{2}+4(x-1)^{2}+8(x-1)^{3}\)
\sol{x^2}
%  \item 
% \(x\tonda{2x^{2}+3x}^{2}-2x^{3}\tonda{2x-\dfrac{1}{2}}^{2}+
%   x^{3}(x-2)^{3}-x^{2}\tonda{x^{3}+2x^{2}}(x-12)\)
%   \sol{52x^4+\dfrac{1}{2}x^3}
%  \item 
% \(\tonda{\dfrac{2}{5}zx^{3}-3x^{2}y}\tonda{\dfrac{2}{5}zx^{3}+3x^{2}
% y}+\tonda{2x^{2}y^{2}z^{3}+\dfrac{1}{2}z^{2}x^{2}y}^{3}\)
%  \item 
% \(-2t(t -x) -3t^{2} +x(x +t)(t -x) + (x -t)^{2} -x\tonda{t^2 +x}\)
%   \sol{-4t^2-x^3}
%  \item 
% \(\dfrac{1}{9}(x-4)(x+4)+\dfrac{1}{3}(x-1)^{2}-\dfrac{1}{9}x(x-2)^{2}
% -x\tonda{x-\dfrac{5}{2}}\tonda{\dfrac{5}{2}-x}+\dfrac{5}{2}
% \tonda{\dfrac{1}{2}x-\dfrac{1}{3}}^{2}\)\\
%   \sol{\dfrac{8}{9}x^3-\dfrac{251}{72}x^2+\dfrac{155}{36}x-76}
\end{enumeratea}
\end{esercizio}

\subsubsection*{\numnameref{sec:poli_funzione}}

\begin{esercizio}
\label{ese:10.11}
Calcola il valore numerico dei polinomi per i valori a fianco indicati.

\begin{htmulticols}{2}
\begin{enumeratea}
\spazielenx
\item \(x^2+x\) \quad per \quad \(x=-1\)
\item \(2x^2-3x+1\) \quad per \quad \(x=0\)
\item \(3x^2-2x-1\) \quad per \quad \(x=2\)
\item \(3x^3-2x+x\) \quad per \quad \(x=-2\)
\item \(\dfrac{3}{4}a+\dfrac{1}{2}b-\dfrac{1}{6}ab\) \quad per \quad 
\(a=-\dfrac{1}{2}\), \hfill \(b=3\)
\item \(4x-6y+\dfrac{1}{5}x^2\) \quad per \quad \(x=-5\), 
\hfill \(y=\dfrac{1}{2}\)
\end{enumeratea}
\end{htmulticols}
\end{esercizio}

\begin{esercizio}
\label{ese:8.13}
Consideriamo la funzione~\(E(a) = -3 \cdot a +2 \cdot (-a +1)\)\\
Osserva che è funzione in una sola variabile e calcola il valore della 
funzione per i seguenti valori di \(a\):
\begin{htmulticols}{2}
\begin{enumeratea}
\spazielenx
\item \(E(-2) =-3\cdot (-2)+2\cdot (-(-2)+1) =\)%6+2\cdot (2+1) =6+6 =12\)
\item \(E(+1) =-3\cdot (1)+2\cdot(-(1)+1) =\)
\item \(E(-1) =-3\cdot (\ldots)+2\cdot (\ldots +1) =\)
\item \(E(0) =\)
\item \(E\tonda{\dfrac{4}{5}} =\)
\item \(E\tonda{-\dfrac{7}{5}} =\)
\item \(E(-10) =\)
\end{enumeratea}
\end{htmulticols}
\end{esercizio}

\pagebreak %----------------------------------------------

\begin{esercizio}\label{ese:stufun.2g}
Trova la corrispondenza tra le figure e le funzioni polinomiali seguenti.

\begin{htmulticols}{2}
\begin{enumerate} [left=0pt, label=\alph*)]
%   \item \mbox{\grafesea}
%   \item \parbox{\textwidth}{\grafesea}
\item \myp 
\funzpolese{.25*\x+2}{-7}{+7} %1A 
\item \myp 
\funzpolese{-.25*\x*\x -2*\x +2}{-7}{+3} %1B
\item \myp 
\funzpolese{.25*\x*\x*\x*\x-2*\x*\x+2}{-3}{+3} %1C % Non funziona \x**4!!!
\item \myp 
\funzpolese{.2*\x*\x*\x-1}{-3}{+3} %1D
\item \myp 
\funzpolese{.2*\x*\x*\x*\x*\x-1}{-2}{+2} %1E
\item \myp 
\funzpolese{-.2*\x*\x*\x+.2*\x*\x+2*\x-1}{-4}{+4} %1F
\end{enumerate}
\end{htmulticols}
\begin{htmulticols}{2}
\begin{itemize} [label=\emptybox~~]
\item \(y = \dfrac{1}{4}x+2\) \\[-1em] %2A
\item \(y = -\dfrac{1}{4}x^2-2x+2\) \\[-1em] %2B
\item \(y = \dfrac{1}{4}x^4-2x^2+2\) \\[-1em] %2C
\item \(y = -\dfrac{1}{5}x^3-1\) \\[-1em] %2D
\item \(y = \dfrac{1}{5}x^5-1\) \\[-1em] %2E
\item \(y = -\dfrac{1}{5}x^3+\dfrac{1}{5}x^2+2x-1\) \\[-1em] %2F
\end{itemize}
\end{htmulticols}
\end{esercizio}

% \begin{esercizio}[*]
%  \label{ese:11.38}
% Risolvi utilizzando i prodotti notevoli.
% \begin{multline*}
% (x-y)^{3}-(y-x)^{3}+2{xy}(x+y)(x-y)-7(x-y)\tonda{x^{2}+{xy}+y^{2}}\\
% +5\tonda{x^{3}-y^{3}}-2{xy}(x+y)(x-y+3).
% \end{multline*}
% \end{esercizio}
% 
% \begin{esercizio}[*]
%  \label{ese:11.39}
% Risolvi utilizzando i prodotti notevoli.
% \begin{multline*}
% 
% \tonda{3ab-\dfrac{1}{2}a}^{2}+\dfrac{1}{2}a+2b\tonda{\dfrac{1}{2}
% a-b}\tonda{\dfrac{1}{2}a+b}-\tonda{1-\dfrac{3}{2}a}^{3}\\
% 
% 
% -9a^{2}\tonda{\dfrac{3}{8}a+b^{2}-\dfrac{13}{18}}+5a\tonda{\dfrac{1}{2}{
% ab}
% -1}.
% \end{multline*}
% \end{esercizio}
% 
% \begin{esercizio}[*]
%  \label{ese:11.40}
% Risolvi utilizzando i prodotti notevoli.
% \begin{multline*}
% 
% 
% \dfrac{1}{3}x\left\{x^{2}-1-\quadra{3x\tonda{x-\dfrac{1}{3}}^{2}-\dfrac{2}
% {3}
% 
% x\tonda{x-\dfrac{2}{3}}^{3}}\right\}-\dfrac{2}{9}x\tonda{x-3x^{2}
% }\tonda{x+3x^{2}}\\
% 
% -\dfrac{1}{9}x^{2}\tonda{20x^{3}-13x^{2}+\dfrac{29}{3}x-\dfrac{43}{27}}.
% \end{multline*}
% \end{esercizio}
% 
% \begin{esercizio}[*]
%  \label{ese:11.41}
% Risolvi utilizzando i prodotti notevoli.
% \begin{multline*}
% 
% \tonda{x-\dfrac{1}{2}y}^{2}-\tonda{2x+\dfrac{1}{2}y^{2}}^{2}
% +\tonda{x+\dfrac{1}{2}y}\tonda{-x+\dfrac{1}{2}y}+(x-y)^{3}+x^{2}
% (3y+4)\\
% +xy(1-y)+\dfrac{1}{2}y^{2}(y-1)(y+1).
% \end{multline*}
% \end{esercizio}

% \paragraph{\ref{ese:11.38}} \(-12x^{2}y\)
% \paragraph{\ref{ese:11.39}} \(2b^3-3\)
% \paragraph{\ref{ese:11.40}} \(-\dfrac{1}{3}x\)
% \paragraph{\ref{ese:11.41}} \(x^3-y^3+\dfrac{1}{4}y^4\)

% \begin{esercizio}[*]
%  \label{ese:11.43}
% Risolvi utilizzando i prodotti notevoli.
%  \begin{enumeratea}
%  \item 
% 
% \(\quadra{\tonda{\dfrac{1}{3}x+\dfrac{2}{3}y}^{2}-\tonda{\dfrac{1}{3}x\
% right)^{
% 
% 2}}:\tonda{\dfrac{1}{3}y}+\tonda{\dfrac{1}{3}y-1}^{3}+\dfrac
% {1}{
% 3}(y-8)(y-7)+\dfrac{1}{3}(1+8y)\)
%   \sol{\dfrac{4}{3}x+\dfrac{y^{3}}{27}+18}\)
%  \item 
% 
% \(-\tonda{\dfrac{1}{4}x+1}^{2}-\dfrac{1}{16}(2x-1)^{2}-\dfrac{1}{2}(3-x)
% ^{2}
% -\dfrac{3}{16}x^{2}+5+\tonda{x+\dfrac{3}{4}}^{2}\)
%   \sol{\dfrac{17}{4}x}
%  \item 
% \(\tonda{x-\dfrac{1}{2}}\tonda{x^{2}+\dfrac{1}{4}+\dfrac{1}{2}
% 
% x}-\tonda{x+\dfrac{1}{2}}\tonda{x-\dfrac{1}{2}}-\tonda{x+\
% dfrac{1
% 
% }{2}}^{3}-\dfrac{1}{2}\tonda{7x^{2}-\dfrac{3}{4}}+\dfrac{3}{8}(2x-
% 1)\)
%   \sol{-6x^2}
% %  \item 
% % 
% \(\tonda{1-x^{n}}^{2}-\tonda{2x^{n}-1}^{2}-\tonda{2x^{n+1}}^
% {2}
% % +\tonda{x^{2n}-1}\tonda{x^{2n}+1}\)
% %   \sol{-1+2x^{n}-3x^{2n}-4x^{2n+2}+x^{4n}}
%  \end{enumeratea}
% \end{esercizio}

% \begin{esercizio}[*]
%  \label{ese:11.44}
% Risolvi utilizzando i prodotti notevoli.
% \begin{multline*}
% 
% 
% \tonda{\dfrac{1}{3}{ab}-\dfrac{2}{5}{xy}}\tonda{-{\dfrac{1}{3}}{ab}-\
% dfrac{2
% 
% }{5}{xy}}-4x^{2}\tonda{\dfrac{1}{5}y-\dfrac{3}{2}}^{2}-\tonda{x-\
% dfrac
% {1}{3}{ab}}\tonda{x+\dfrac{1}{3}{ab}}\\
% +10x^{2}\tonda{1-\dfrac{6}{25}y}.
% \end{multline*}
% \end{esercizio}
% 
% \begin{esercizio}[*]
%  \label{ese:11.45}
% Risolvi utilizzando i prodotti notevoli.
% \begin{multline*}
% 
% 
% \tonda{x+\dfrac{1}{2}}^{2}+2\tonda{x-\dfrac{1}{2}}^{3}-2\tonda{x+\
% dfrac
% 
% {1}{2}}\tonda{x-\dfrac{1}{2}}-x\quadra{(x+1)(x+2)+(x+1)^{2}+\dfrac{1
% }{2
% }x}\\
% +\dfrac{1}{2}\tonda{x^{2}+x-1}.
% \end{multline*}
% \end{esercizio}
% 
% \begin{esercizio}[*]
%  \label{ese:11.46}
% Risolvi utilizzando i prodotti notevoli.
% \begin{multline*}
% 
% 
% \tonda{\dfrac{3}{2}x-2y}\tonda{\dfrac{3}{2}x+2y}\tonda{\dfrac{9}{4
% }x^{2
% }+4y^{2}}+x\tonda{\dfrac{1}{2}x-2y}^{2}+\tonda{\dfrac{3}{2}
% x+2y}^{3}\\
% 
% 
% -\dfrac{3}{4}x\tonda{x-\dfrac{2}{3}y}\tonda{x+\dfrac{2}{3}y}+\
% tonda{4y^
% {2}-\dfrac{9}{4}x^{2}}\tonda{4y^{2}+\dfrac{9}{4}x^{2}}\\
% 
% 
% +\dfrac{1}{2}{xy}\tonda{y-\dfrac{1}{6}x}-\tonda{\dfrac{5}{2}x+2y}^
% {3}
% +\dfrac{51}{4}x^{3}.
% \end{multline*}
% \end{esercizio}
% 
% \begin{esercizio}[*]
%  \label{ese:11.47}
% Risolvi utilizzando i prodotti notevoli.
% \begin{multline*}
% 
% \tonda{x+\dfrac{1}{3}y}\tonda{x-\dfrac{1}{3}y}:\dfrac{1}{3}
% 
% -\tonda{x+\dfrac{1}{2}{xy}}^{2}:\tonda{-{\dfrac{1}{2}}x^2}+\dfrac{
% 1}{3
% }(-3x+y)(3x+y)\\
% -\dfrac{1}{2}\tonda{y^{2}+4y+4}.
% \end{multline*}
% \end{esercizio}
% 
% \begin{esercizio}[*]
%  \label{ese:11.48}
% Risolvi utilizzando i prodotti notevoli.
% \begin{multline*}
% 
% \dfrac{1}{4}(x+1)^{4}+\dfrac{1}{2}(x+1)^{2}+\dfrac{1}{8}\tonda{x^{2}
% +1}(x+1)(x-1)-\tonda{2x^{2}-2x+1}^{2}\\
% 
% +9x^{3}\tonda{\dfrac{3}{8}x-1}+\dfrac{1}{4}x^{2}\tonda{x^{2}
% +16}+6x-\dfrac{3}{8}.
% \end{multline*}
% \end{esercizio}
% 
% \begin{esercizio}[*]
%  \label{ese:11.49}
% Risolvi utilizzando i prodotti notevoli.
% \begin{multline*}
% 
% \quadra{2\tonda{a-\dfrac{1}{2}b}\tonda{a+\dfrac{1}{2}b}}^{2}
% -\tonda{2a^{2}-b}\tonda{2a^{2}+b}-6a^{2}(a-2b)(2b-a)\\
% -b^{2}\tonda{22a^{2}+\dfrac{1}{4}b^{2}+1}-6a^{3}(a-4b).
% \end{multline*}
% \end{esercizio}

% \subsection{Risposte}

% \paragraph{\ref{ese:11.44}} \(0\)
% \paragraph{\ref{ese:11.45}} \(-9x^2\)
% \paragraph{\ref{ese:11.46}} \(-\dfrac{43}{6}xy^{2}-
% \dfrac{313}{12}x^{2}y.\)
% \paragraph{\ref{ese:11.47}} \(0\)
% \paragraph{\ref{ese:11.48}} \(-2x^{2}+12x-\dfrac{3}{4}\)
% \paragraph{\ref{ese:11.49}} \(0\)
