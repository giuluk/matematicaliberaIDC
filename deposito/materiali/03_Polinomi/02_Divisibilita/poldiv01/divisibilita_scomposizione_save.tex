
\begin{esempio}{}{}
Eseguire la divisione tra i polinomi~\(A(x)=3x^4+5x-4x^3-1\) 
e~\(B(x)=3x^2-1\).

Prima di eseguire l'algoritmo dobbiamo sempre controllare che:
\begin{itemize}[nosep]
\item i polinomi siano ordinati secondo le potenze decrescenti della 
variabile, in questo caso la~\(x\) poiché ciò
non è vero, riscriviamo~\(A(x)\) ordinato:~\(A(x)=3x^4-4x^3+5x-1\)
\item dividendo e divisore siano in forma completa, cioè abbiano i termini 
con tutti i gradi; nel nostro esempio, i due polinomi non sono in
forma completa, quindi inseriamo i termini mancanti ponendo~0 come 
coefficiente delle potenze mancanti:
\[A(x)=3x^4-4x^3+0x^2+5x-1; \quad B(x)=3x^2+0x-1\]
\end{itemize}

% \newpage %-----------------------------------------

I passi da eseguire sono i seguenti:

\def \primo{.49}
\def \secondo{.49}

\begin{enumerate}[nosep]
\item ~

\vspace{-1em}
\affiancati{\primo}{\secondo}{
Disponiamo i polinomi secondo il seguente schema, del tutto simile a 
quello usato per la divisione tra numeri.
}{
\immagine[.8]{}{\divpola}
}

% \newpage %-----------------------------------------

\item ~

\vspace{-1.6em}
\affiancati{\primo}{\secondo}{
Dividiamo il primo termine del dividendo per il primo termine del divisore, 
otteniamo~\(x^2\) che è il primo termine del quoziente;
esso va riportato nello spazio dedicato al quoziente.
}{
\immagine[.8]{}{\divpolb}
}
\item ~

\vspace{-.85em}
\affiancati{\primo}{\secondo}{
Moltiplichiamo il primo termine ottenuto 
per tutti i termini del divisore e 
trascriviamo il risultato del prodotto sotto il dividendo,
avendo cura, per essere facilitati nel calcolo, di:
\begin{itemize}[nosep]
\item incolonnare i termini con lo stesso grado, ossia scrivere i risultati 
del prodotto in ordine da sinistra verso destra;
\item cambiare tutti i segni ottenuti, in questo modo risulta più pratico 
eseguire la somma algebrica dei polinomi invece della sottrazione.\\
\end{itemize}
}{
\immagine[.8]{}{\divpolc}
}
\item ~

\vspace{-.8em}
\affiancati{\primo}{\secondo}{
Sommiamo il dividendo con il polinomio sottostante e riportiamo il risultato 
in un'altra riga. Questo polinomio si chiama primo resto parziale.
Notiamo che ha grado~\(3\), maggiore del grado~\(2\) del divisore, pertanto 
la  divisione va continuata.\\
}{
\immagine[.8]{}{\divpold}
}
\item ~

\vspace{-.8em}
\affiancati{\primo}{\secondo}{
Ripetiamo il procedimento tra il resto parziale ottenuto, 
\(-4x^3+x^2+5x-1\) e il divisore~\(3x^2+0x-1\). Dividiamo il primo 
termine  del resto che è~\(-4x^3\) per il primo termine del divisore che 
è~\(3x^2\). 
Otteniamo~\(-{\dfrac{4}{3}}x\) che è il secondo termine del quoziente.
\\[18mm]
}{
\immagine[.8]{}{\divpole}
}
\item ~

\vspace{-4em}
\affiancati{\primo}{\secondo}{
Proseguiamo moltiplicando~\(-{\frac{4}{3}}x\) per~\(B(x)\), riportiamo il 
risultato del prodotto, con segno opposto, sotto i
termini del primo resto parziale e addizioniamo i due polinomi.
}{
\immagine[.8]{}{\divpolf}
}

\newpage %-----------------------------------------------

\item ~

\vspace*{-4em}
\affiancati{\primo}{\secondo}{
Possiamo ripetere per l'ultima volta il procedimento precedente tra il 
resto parziale~\(R_{p}(x)=x^2+\dfrac{11}{3}x-1\) e
il divisore~\(B(x)\) in quanto hanno lo stesso grado. Dividendo il 
termine di grado maggiore di~\(R_{p}(x)\), che è~\(x^2\),
per il termine di grado maggiore di~\(B(x)\) che è~\(3x^2\) si 
ottiene~\(\dfrac{1}{3}\) che è il terzo termine del polinomio quoziente.
}{
\vspace*{1.3em}
\immagine[.8]{}{\divpolg}
}
\end{enumerate}

Non possiamo più ripetere l'algoritmo poiché il resto ottenuto ha grado 
minore del grado del divisore.

In conclusione~\(A(x):B(x)\) ha 
quoziente~\(Q(x)=x^2-\dfrac{4}{3}x+\dfrac{1}{3}\) e 
resto~\(R(x)=+{\dfrac{11}{3}}x-\dfrac{2}{3}\).

\paragraph{Verifica}
Verifichiamo se abbiamo svolto correttamente i calcoli; dovrebbe risultare, 
come detto sopra: \(Q(x)\cdot B(x)+R(x) = A(x)\).
\begin{equation*}
\begin{split}
Q(x)\cdot B(x)+R(x) &= 
\tonda{3x^2-1}\tonda{x^2-\dfrac{4}{3}x+\dfrac{1}{3}}+
\dfrac{11}{3}x-\dfrac{2}{3} =\\ 
&=3x^4-4x^3+x^2-x^2+\dfrac{4}{3}x-
  \dfrac{1}{3}+\dfrac{11}{3}x-\dfrac{2}{3}=\\
&= 3x^4-4x^3+\dfrac{15}{3}x-\dfrac{3}{3}
= 3x^4-4x^3+5x-1 = A(x).
\end{split}
\end{equation*}
I polinomi~\(Q(x)\) e~\(R(x)\) soddisfano quindi le nostre richieste. 
Ma sono unici? È sempre possibile trovarli? 
A queste domande risponde il seguente teorema.
\end{esempio}
