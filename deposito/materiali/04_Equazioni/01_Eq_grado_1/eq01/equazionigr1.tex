% (c) 2012 - 2014 Dimitrios Vrettos - d.vrettos@gmail.com
% (c) 2014 Claudio Carboncini - claudio.carboncini@gmail.com
% (c) 2014 Daniele Zambelli - daniele.zambelli@gmail.com

\input{\folder equazionigr1_grafici.tex}

\chapter{Identità, equazioni}

\inicapitolo{
\begin{itemize}[left=0mm, nosep]
\item le identità;
\item le equazioni;
\item i metodi di soluzione delle equazioni;
\item le equazioni che non hanno soluzione;
\item la soluzione di problemi con le equazioni.
\end{itemize}
}

\section{Identità ed equazioni}
\label{sec:eq1_definizioni}

Analizziamo le proposizioni:

\begin{enumeratea}
\item ``cinque è uguale alla differenza tra sette e due'';
\item ``la somma di quattro e due è uguale a otto'';
\item ``il doppio di un numero naturale è uguale alla differenza tra nove e il 
numero stesso'';
\item ``la somma di due numeri interi è uguale a dieci''.
\end{enumeratea}

Notiamo che sono tutte costruite con il predicato
``essere uguale a''. Riscriviamo in formula ciascuna di esse:
\begin{htmulticols}{4}
\begin{enumeratea}
\item \(5=7-2\)
\item \(4+2=8\)
\item \(2x=9-x\)
\item \(x+y=10\).
\end{enumeratea}
\end{htmulticols}
Notiamo che le prime due contengono solamente numeri, le seconde
contengono anche variabili.

Le frasi del primo tipo si dicono \emph{chiuse} \ind{frasi chiuse} e
di esse si può subito stabilire se sono vere o false; così in~\(\N\) la
formula~\(5 = 7 - 2\) è vera, mentre~\(4 + 2 = 8\) è falsa.

\begin{definizione}{}{}\ind{uguaglianze} \ind{identità}
Le \emph{frasi chiuse} costruite con il predicato
<<essere uguale>> si chiamano \textbf{uguaglianze} o \textbf{identità}.
\end{definizione}

Le identità non contengono variabili libere e si può stabilire se sono 
\textbf{vere} o \textbf{false}.

% \begin{definizione}{}{}
% Le \emph{formule chiuse} costruite con il predicato
% <<essere uguale>> si chiamano \emph{uguaglianze};
% stabilito l'ambiente in cui vengono enunciate si può
% immediatamente stabilire il loro valore di verità.
% \end{definizione}
% 
% % \begin{exrig}
%  \begin{esempio}{}{}
%  La formula chiusa~\(1 - 6 = -5\) è un'uguaglianza
% vera se la consideriamo nell'insieme~\(\Z\) degli interi
% relativi, è falsa se la vediamo come sottrazione tra numeri naturali.
%  \end{esempio}

Le frasi c) e d) che contengono variabili libere si dicono \emph{aperte}, 
\ind{frasi aperte}
in tali frasi il valore di verità dipende dal valore delle variabili.
Se alle variabili sostituiamo un numero, queste frasi da aperte si 
trasformano in chiuse e allora possiamo stabilirne il valore di verità.

Il problema che affronteremo sarà quello di trovare per quali valori della 
variabile la frase aperta diventa vera.
In questi problemi le variabili libere vengono chiamate \emph{incognite}.

\begin{esempio}{}{}
Il valore di verità della frase \(2x = 6 - x\) dipende dal valore della 

\affiancati{.49}{.49}
{
variabile~\(x\). 
Possiamo provare con diversi valori: 

\vspace{.5em}
Abbiamo trovato un valore, \(+2\), che sostituito alla variabile rende vera 
la frase.
Ma ce ne saranno altri?
}{
\begin{center}
\begin{tabular}{ccc}
\(x\) & espressione & verità\\
\hline
\(-1\) & \(-2 = 6 +1\) & Falso\\
\(~~0\) & \(0 = 6~~~~ \) & Falso\\
\(+1\) & \(+2 = 6 - 1\) & Falso\\
\(+2\) & \(+4 = 6 - 2\) & Vero\\
\(+3\) & \(+6 = 6 - 3\) & Falso\\
\end{tabular}
\end{center}
}
\end{esempio}

\begin{esempio}{}{}
Il valore di verità della frase \(x + 2y = 10\) dipende dal valore 

\affiancati{.44}{.54}
{
delle variabili~\(x\) e \(y\). Possiamo provare con diversi valori: 

\vspace{.5em}
Abbiamo trovato una coppia di valori, \(+6;~ +4\), che, 
sostituiti alle variabili rende vera la frase.
Ma ce ne saranno altre?
}{
\begin{center}
\begin{tabular}{cccc}
\(x\) & \(y\) & espressione & verità\\
\hline
\(+4\) & \(+2\) & \(+4 +4 = +10\) & Falso\\
\(+5\) & \(+2\)& \(+5 +4 = +10\) & Falso\\
\(+6\) & \(+2\)& \(+6 +4 = +10\) & Vero\\
\(+7\) & \(+2\)& \(+7 +4 = +10\) & Falso\\
\(+8\) & \(+2\)& \(+8 +4 = +10\) & Falso\\
\end{tabular}
\end{center}
}

Non serve tanta fantasia per trovarne molte altre\dots
Quante?
\end{esempio}

% \begin{esempio}{}{}
% Nella formula~\(2x = 9 - x\) sostituiamo alla variabile~\(x\) il valore 
% \(0\) 
% otteniamo:~\(2\cdot 0=9-0 \Rightarrow~0=9\), falsa.
% 
% Sostituiamo ora alla
% variabile~\(x\) il valore \(3\) otteniamo~\(2\cdot 3=9-3 \Rightarrow~6=6\), 
% vera.
% \end{esempio}

% \begin{esempio}{}{}
% Nella formula~\(x + y = 10\) sostituiamo
% alle variabili coppie di numeri interi come~\(x = 2\) e~\(y = 5\) 
% otteniamo~\(2+5=10\Rightarrow~7= 10\), falsa. Se
% sostituiamo~\(x = 4\) e~\(y = 6\) ci rendiamo subito conto che
% l'uguaglianza ottenuta è \emph{vera}. Esistono
% molte altre coppie di numeri interi rendono vera
% l'uguaglianza.
% \end{esempio}

\begin{definizione}{}{}
\ind{equazione} \ind{primo membro} \ind{secondo membro} \ind{soluzioni} 
\ind{insieme soluzione} 
Chiamiamo \textbf{equazioni} le frasi aperte costruite con il predicato 
\emph{essere uguale}. 
L'espressione che si trova a sinistra del segno di uguaglianza si chiama 
\textbf{primo membro} e quella che si trova a destra si chiama 
\textbf{secondo membro}.

I valori che sostituiti alle incognite trasformano l'espressione aperta 
in un'uguaglianza \emph{Vera} si chiamano \textbf{soluzioni} 
e tutte assieme costituiscono l'\textbf{insieme soluzione} (\(\IS\)) 
dell'equazione.
\end{definizione}

% \begin{definizione}{}{}
% Le frasi aperte costruite con il predicato essere uguale si chiamano
% \emph{equazioni}; le due espressioni che compaiono a sinistra e a
% destra del segno di uguaglianza si chiamano rispettivamente
% \emph{primo membro} e \emph{secondo membro}.
% 
% L'insieme dei valori che sostituiti alle incognite trasformano l'equazione in
% un'uguaglianza vera costituisce
% l'\emph{insieme soluzione} (\(\IS\)) o più semplicemente la \emph{soluzione} 
% dell'equazione.
% \end{definizione}

Ci sono tanti tipi di equazioni a seconda di quali insiemi numerici sono 
coinvolti e da quali operazioni sono applicate alle variabili libere.

In questo capitolo affronteremo equazioni a \emph{una sola incognita}, 
\emph{polinomiali di primo grado}. 
Se non indicato diversamente, consideriamo variabili che appartengono 
al più generale insieme numerico che conosciamo (\(\Q\) o \(\R\)).

% Affronteremo per ora equazioni in
% \emph{una sola incognita} che, dopo aver svolto eventuali calcoli nei due 
% membri, comparirà a
% \emph{grado} 1 e i cui \emph{coefficienti} sono \emph{numeri razionali}.
% Cercheremo la sua soluzione
% nell'insieme~\(\Q\) dei numeri razionali, salvo esplicita indicazione 
% differente.

\begin{esempio}{}{}
Cercare le soluzioni nell'insieme indicato.
\begin{enumeratea}
\item \(x^{2} = 1 \text{ con }x\in\N.\)
Risulta vera solo se a~\(x\) sostituiamo il valore~1; infatti~1 è
l'unico numero naturale il cui quadrato è~1.
L'insieme soluzione è~\(\{1\}\).
\item \(x^{2} = 1 \text{ con }x\in\Z\).
Risulta vera se a
\(x\) sostituiamo il valore \(+1\) oppure il valore~\(-1\) infatti sia 
\(-1\) sia \(+1\) elevati al quadrato danno~1. 
L'insieme soluzione è\(\{-1;~ +1\}\).
\item \(x^{2} = 1\text{ con }x\in\R\).
L'espressione a sinistra è elevata al quadrato e quindi non può essere 
uguale a un numero negativo pertanto nessun numero reale, messo al posto 
della \(x\) rende vera l'uguaglianza (per risolvere equazioni di questo 
tipo sono stati inventati dei nuovi numeri).
\item \(2x+3=(3+x)+x \text{ con } x\in\Q\).
Eseguendo il semplice calcolo al secondo membro, ci rendiamo conto che 
l'uguaglianza risulta vera qualunque sia il valore che viene sostituito 
all'incognita. 
L'insieme soluzione è~\(\Q\).
\end{enumeratea}
\end{esempio}

In generale un'equazione in una incognita può essere:

\begin{enumeratea}
\item \emph{determinata}, quando l'insieme soluzione è un sottoinsieme 
proprio dell'insieme numerico considerato;
\item \emph{impossibile}, quando l'insieme soluzione è l'insieme 
vuoto~\(\emptyset\); 
\item \emph{indeterminata} o \emph{identità}, quando l'insieme soluzione 
coincide con l'insieme considerato.
\end{enumeratea}

\begin{esempio}{}{}
Analizziamo le equazioni:
\begin{htmulticols}{3}
\begin{enumeratea}
\item \(3\cdot x=0\)
\item \(0\cdot x=5\)
\item \(0\cdot x=0\).
\end{enumeratea}
\end{htmulticols}

Tutte e tre hanno la stessa struttura: il primo membro è il prodotto
di un coefficiente numerico per un valore incognito, il secondo membro
è un numero.

\paragraph{a)} Per trovare l'insieme soluzione della prima equazione 
cerchiamo 
in~\(\Q\) il
numero che moltiplicato per~3 dà come prodotto~0. L'unico numero che rende 
vera
l'uguaglianza è zero. Quindi l'insieme delle soluzioni è~\(\{0\}\). 
L'equazione 
è
determinata.

\paragraph{b)} Per trovare l'insieme soluzione della seconda equazione 
cerchiamo 
in~\(\Q\) il
numero che moltiplicato per~0 dà come prodotto~5. Per la proprietà
della moltiplicazione quando moltiplichiamo per~0 il prodotto è~0,
non otterremo mai~5. Quindi l'insieme soluzione è
l'insieme vuoto. L'equazione è
impossibile.

\paragraph{c)} Per trovare l'insieme soluzione della terza equazione 
cerchiamo 
in~\(\Q\) il
numero che moltiplicato per zero dà come prodotto zero. Per la
proprietà della moltiplicazione quando moltiplichiamo per~0 il
prodotto è~0 qualunque sia l'altro fattore. Quindi
l'insieme delle soluzioni è~\(\Q\). L'equazione è
indeterminata.
\end{esempio}


\subsection{Ricerca dell'insieme soluzione}
In alcuni casi la soluzione di un'equazione si può
trovare applicando semplicemente le proprietà delle operazioni.

\begin{esempio}{}{}
Analizziamo lo schema operativo dell'equazione~\(3x-1=17 \text{ con } 
x\in\N\).

Si opera sul valore incognito~\(x\) per ottenere~17:

\[\text{\emph{entra} } x,\text{ si moltiplica per tre}\to~3\cdot x%
\text{ si sottrae } 1\to~3\cdot x-1 \text{ si ottiene } 17.\]

Qual è il valore in ingresso?

Per determinare il valore in ingresso basterà percorrere alla rovescia 
lo schema effettuando le operazioni inverse:
\[\text{\emph{da} } 17 \text{ aggiungi } 1\to~18 \text{ dividi per tre 
}\to~18:3 
\to x.\]

La soluzione dell'equazione è~\(x = 6\) e~\(\IS\) (insieme
soluzione) è~\(\{6\}\).
\end{esempio}

% \ovalbox{\risolvi \ref{ese:13.1}}\vspace{1.10ex}

Per risolvere un'equazione più complessa come
\(\left(\dfrac{1}{2}x+3\right)\cdot (-5+x)=12x+\dfrac{1}{2}x^{2}\) con
\(x\in~\Q\), non possiamo applicare il procedimento precedente; potremmo
procedere per tentativi, sostituendo all'incognita
alcuni valori scelti a caso e verificando se il valore assunto dal
primo membro risulta uguale a quello assunto dal secondo membro. È
evidente però che questo procedimento raramente porterà a trovare
tutte le soluzioni di un'equazione.


\section{Principi di equivalenza}
\label{sec:eq1_principi} \ind{principi di equivalenza}

Due principi di equivalenza ci permettono di risolvere qualunque equazione 
in modo meccanico.
% \begin{osservazione}{}{}
% Per risolvere un'equazione, cioè per determinare tutte
% le eventuali soluzioni, si procede applicando i principi
% d'equivalenza.
% \end{osservazione}

\begin{definizione}{}{}
Due equazioni sono \emph{equivalenti} se hanno lo stesso insieme soluzione.
\end{definizione}

\begin{teorema}{Primo principio di equivalenza}{}
Aggiungendo o togliendo a entrambi i membri di un'equazione 
una stessa quantità, 
si ottiene un'equazione equivalente a quella data.
\end{teorema}

\begin{teorema}{Secondo principio di equivalenza}{}
Moltiplicando o dividendo entrambi i membri di un'equazione 
per una stessa quantità \textbf{diversa da zero}, 
si ottiene un'equazione equivalente.
\end{teorema}

I principi sopra enunciati permettono di trasformare qualunque equazione
nella forma canonica che ha lo stesso insieme soluzione di quella
assegnata.

\subsection{Risoluzione di equazioni numeriche intere di primo grado}

\begin{definizione}{}{}
Risolvere un'equazione significa determinare il suo Insieme Soluzione.
\end{definizione}

In questo paragrafo vedremo come usare i principi
d'equivalenza prima enunciati per condurre
un'equazione alla forma canonica e dunque determinarne
la soluzione.

La forma più semplice di un'equazione di primo grado in
un'incognita è del tipo:
\[x = \text{numero}\]

L'insieme soluzione di una
equazione di questo tipo è semplicemente:
\[\IS=\{\text{numero}\}.\]

Per esempio, l'insieme delle soluzioni dell'equazione
\(x = -3\) è: \(\IS =\{-3\}\).

Dato che l'espressione \(\IS=\{\text{numero}\}\) non dà alcuna 
informazione in più, 
spesso ci si ferma alla prima: \(x = \text{numero}\).

Il primo principio di equivalenza permette di ottenere equazioni che 
contengono a primo membro solo monomi con la variabile e a secondo membro 
solo numeri. 

\begin{esempio}{}{}
Risolvi l'equazione \(2x -5 = x +3\).

\vspace{.5em}
\begin{itemize}[left=0mm, nosep]
\item aggiungiamo \(+5\) a entrambi i membri:
\(2x -5 +5 = 3 +5 \sRarrow 2x = x +8\);
\item aggiungiamo \(-x\) a entrambi i membri:
\(2x-x=x +8 -x \sRarrow x=8\);
\end{itemize}

\(\IS =\{8\}\).
\end{esempio}

% \ovalbox{\risolvii \ref{ese:13.2}, \ref{ese:13.3}, \ref{ese:13.4}, 
% \ref{ese:13.5}}

Il secondo principio di equivalenza permette di trovare la soluzione di 
un'equazione quando si conosce il valore di un multiplo (o di un 
sottomultiplo) della variabile.

\begin{esempio}{}{}
Risolvi l'equazione: \quad \(3x = 12\).

\begin{itemize}[left=0mm, nosep]
\item dividiamo entrambi i membri per~3: \quad 
\(\dfrac{3}{3}x = \dfrac{12}{3} \sRarrow x=4\)
\end{itemize}

\(\IS =\{4\}\)
\end{esempio}

\begin{esempio}{}{}
Risolvi l'equazione: \quad \(\dfrac{3}{5}x = 9\).

\begin{itemize}[left=0mm, nosep]
\item dividiamo entrambi i membri per \(\dfrac{3}{5}\): \quad 
\(\dfrac{3}{5}x \cdot \dfrac{5}{3}=9 \cdot \dfrac{5}{3} \sRarrow x=15\)
\end{itemize}

\(\IS =\{15\}\)
\end{esempio}

% \ovalbox{\risolvii \ref{ese:13.6}, \ref{ese:13.7}, \ref{ese:13.8}}

Combinando i due metodi esposti sopra possiamo risolvere qualunque 
equazione di primo grado.

\begin{esempio}{}{}
Risolvi l'equazione: \quad \(-2x+1=3x-5\).

\vspace{.5em}
\begin{itemize}[left=0mm, nosep]
\item Sottraiamo~1 a entrambi i membri: \\
\(-2x+1-1=3x-5-1 \sRarrow -2x = 3x  -6\)
\item Sottraiamo~\(3x\) a entrambi i membri: \\
\(-2x-3x = 3x-6 -3x \sRarrow -5x = -6\)
\item Dividiamo entrambi i membri per \(-5\): \\
\(\dfrac{-5}{-5}x = \dfrac{-6}{-5} \sRarrow x = +\dfrac{6}{5}\)
\end{itemize}

\(\IS =\graffa{+\dfrac{6}{5}}\)
\end{esempio}

% \ovalbox{\risolvii \ref{ese:13.9}, \ref{ese:13.10}, \ref{ese:13.11}}

Non è detto che l'equazione si trovi già nella forma dell'uguaglianza 
di due polinomi, potrebbe essere necessario eseguire dei calcoli e 
ridurre le espressioni in forma polinomiale.

\begin{esempio}{}{}
Risolvi l'equazione: \quad \((x +1) +3(4 +x) = 12x -3\)

\vspace{.5em}
\begin{enumerate}
\item svolgiamo i calcoli al primo e al secondo membro:\\
\(x+1+12+3x=12x-3 \sRarrow 4x+13=12x-3\)
\item dobbiamo togliere \(12x\) dal secondo membro e \(13\) dal primo 
membro quindi sommiamo ad ambo i membri i monomi \(-12x -13\)
applicando il primo principio:\\
\(4x\cancel{+13}-12x\cancel{-13}=\cancel{12x}-3\cancel{-12x}-13 \sRarrow 
-8x=-16\)
\item dividiamo ambo i membri per~\(-8\),
applicando il secondo principio:\\ 
\(\dfrac{-8}{-8}x = \dfrac{-16}{-8} \sRarrow x = +2\).
\end{enumerate}

L'equazione assegnata~\((x +1) +3(4 +x) = 12x -3\)
risulta equivalente all'ultima trovata \(x = +2\), pertanto il
suo insieme soluzione è~\(\IS = \{+2\}\).
\end{esempio}

% \vspace{1.10ex}% \ovalbox{\risolvii \ref{ese:13.12}, \ref{ese:13.13}}

\begin{osservazione}{}{} \ind{forma canonica}
La forma canonica di un'equazione polinomiale è del tipo: \(ax +b = 0\)
ma spesso è più comodo collocare il termine con l'incognita a sinistra 
dell'uguale e il termine numerico a destra.
\end{osservazione}

Enunciamo alcune \emph{regole pratiche} \ind{regole pratiche} 
che ci possono aiutare nella procedura risolutiva e che discendono 
direttamente dai principi d'equivalenza.

\paragraph{a)} 
Si può spostare una funzione da una parte all'altra dell'uguale
trasformandola nella sua opposta.

\begin{esempio}{}{}
\qquad \(2x-3=2\)

\(-3\) equivale alla funzione \emph{diminuisci di 3} 
posso spostarla dall'altra parte dell'uguale e diventa 
\emph{aumenta di 3}:
\(2x-3=2 \sLRarrow 2x=2+3\)
\end{esempio}

\begin{esempio}{}{}
\qquad \(2x=5\)

Il numero \(2\) moltiplica il primo membro 
posso spostarlo dall'altra parte dell'uguale e
\emph{divide} il secondo membro
\(2x=5 \sLRarrow 2x=\dfrac{5}{2}\)
\end{esempio}

\paragraph{b)}
Gli addendi uguali che si trovano nei due membri opposti possono essere 
tolti.

\begin{esempio}{}{}
\qquad \(2x-3-5x=2-5x\)

Poiché \(-5x\) è presente e nel primo membro e anche nel secondo, 
possiamo eliminarli e l'equazione diventa:
\(2x-3\cancel{-5x}=2-\cancel{5x} \sLRarrow 2x-3=2\)
\end{esempio}

\paragraph{c)} 
Possiamo cambiare i segni dei due membri di un'equazione. 
Questo ci torna comodo quando il coefficiente dell'incognita è negativo.

\begin{esempio}{}{}
\qquad \(-3x=7\)

Posso cambiare i segni ad entrambi i membri:\\ 
\(-3x=7 \sLRarrow +3x=-7 \sLRarrow x=\dfrac{-7}{+3}=-\dfrac{7}{3}\)\\
si ottiene lo stesso risultato spostando \(-3\) che moltiplica il primo 
membro, dall'altra parte dell'uguale dove dividerà il secondo membro:\\ 
\(-3x=7 \sLRarrow x=\dfrac{+7}{-3}=-\dfrac{7}{3}\)
\end{esempio}

\begin{esempio}{}{}
Risolvi la seguente equazione applicando queste regole pratiche: \qquad
\(5x+2\cdot (3-x)+1=-(4x-1)+2\cdot (6-x)\)

\vspace{.5em}
\begin{enumerate}
\item svolgiamo i calcoli:\\
\central{\(5x+6-2x+1=-4x+1+12-2x\)}
\item eliminiamo i termini uguali che compaiono nei due membri:\\
\central{\(5x+6 \cancel{-2x}\cancel{+1}=-4x \cancel{+1}+12\cancel{-2x} 
\ssRarrow 5x+6=-4x+12\)}
\item la funzione ``aggiungi \(6\)'', spostata al secondo membro diventa 
``togli \(6\)'' e il monomio~\(4x\) del secondo membro lo spostiamo al 
primo membro, di là era tolto, di qua verrà aggiunto.\\
\central{\(5x +4x = -6 +12 \ssRarrow 9x = +6\)}
\item Dopo aver sommato i termini simili nei due membri, il 
\(9\) che moltiplica il primo membro lo sposto dall'altra parte 
dell'uguale e divide il secondo membro:\\
\central{\(x=\dfrac{6}{9} = \dfrac{2}{3} \ssrarrow 
\IS =\graffa{\dfrac{2}{3}}\)}
\end{enumerate}
\end{esempio}
% \end{soluzione}
% \ovalbox{\risolvii \ref{ese:13.14}, \ref{ese:13.15}, \ref{ese:13.16}, 
% \ref{ese:13.17}, \ref{ese:13.18}}

\section{Alcune variazioni}
\label{sec:eq1_varie}

\subsection{Equazioni a coefficienti frazionari}
\label{sec:13_coefffraz} \ind{Eq. a coefficienti frazionari}

Si può procedere come al solito portando tutti i termini con l'incognita a 
primo membro e quelli senza a secondo membro lavorando con le frazioni.

\begin{esempio}{}{}
\(\dfrac{2}{3}x+4-\dfrac{1}{2}+2x=\dfrac{x+2}{3}-\dfrac{5}{2}x+1\)

\vspace{.5em}
\begin{enumerate}%[nosep]
\item Separiamo le frazioni con le incognite: \\[1mm]
\(\dfrac{2}{3}x+4-\dfrac{1}{2}+2x=
\dfrac{x}{3}+\dfrac{2}{3}-\dfrac{5}{2}x+1\) 
\item Portiamo a primo membro i termini con le incognite e a secondo quelli 
senza:\\
\(\dfrac{2}{3}x+2x-\dfrac{x}{3}+\dfrac{5}{2}x=
+\dfrac{2}{3}+1+\dfrac{1}{2}-4\) 
\item Sommiamo i termini simili:\\[1mm]
\(\dfrac{4x+12x-2x+15x}{6} = \dfrac{+4+6+3-24}{6} \sRarrow
\dfrac{29x}{6} = -\dfrac{11}{6}\)
\item A questo punto moltiplicando entrambi i membri per 
\(\dfrac{6}{29}\) otteniamo:\\ 
\(\dfrac{\cancel{29}x}{\cancel{6}} \cdot \dfrac{\cancel{6}}{\cancel{29}} 
= 
\dfrac{-11}{\cancel{6}} \cdot \dfrac{\cancel{6}}{29} \sRarrow
x = -\dfrac{11}{29}
\)
\end{enumerate}
\end{esempio}

Oppure si possono eliminare tutti i denominatori e procedere poi con 
coefficienti interi.
Sappiamo infatti che il secondo principio d'equivalenza ci permette di 
moltiplicare ambo i membri per uno stesso numero diverso da zero.

\begin{esempio}{}{}
\(\dfrac{2}{3}x+4-\dfrac{1}{2}+2x=\dfrac{x+2}{3}-\dfrac{5}{2}x+1\)

\vspace{.5em}
% \affiancati{.49}{.49}{
\begin{enumerate}%[nosep]
\item Calcoliamo il~\(\mcm\) tra i denominatori: in questo
caso: \(\mcm\coppia{2}{3} = 6\)
\item Moltiplichiamo per~6 entrambi i membri
dell'equazione: \\[1mm]
\(6\left(\dfrac{2}{3}x+4-\dfrac{1}{2}+2x\right)=6\left(\dfrac{x+2}{3}-
\dfrac{5}{2}x+1\right)\)
\item Eseguiamo i calcoli: \quad \(4x +24 -3 +12x = 2x +4 -15x +6\)
\item Separiamo, nei due membri, i termini con e senza variabile e 
eseguiamo i calcoli: \quad
\(4x +12x -2x +15x = +4 +6 -24 +3 \sRarrow 29x = -11\)
\item Dividendo per \(29\): \quad 
\(\dfrac{29x}{29} = \dfrac{-11}{29} \sRarrow x = -\dfrac{11}{29}\)
\end{enumerate}
\end{esempio}

% \ovalbox{\risolvi \ref{ese:13.19}}

\subsection{Equazioni in cui l'incognita compare con grado maggiore di~1}
\ind{Eq. apparentemente di 2° grado}
A volte l'equazione appare di secondo grado, ma, svolgendo i calcoli, 
i termini di secondo grado si annullano e diventa una equazione di primo 
grado che sappiamo risolvere.

\begin{esempio}{}{}

\((2x+1)\cdot (x-2)=2\cdot (x+1)^{2}-5x\).

\vspace{.5em}
Apparentemente l'equazione è di secondo grado. 

\begin{enumerate}
\item svolgiamo i calcoli e otteniamo:\\[1mm]
\(2x^{2}-4x+x-2=2x^{2}+4x+2-5x\Rightarrow~2x^{2}-3x-2=2x^{2}-x+2\)
\item applichiamo le regole pratiche eliminando i monomi
uguali che si trovano nelle parti opposte dell'uguale:
\(-3x+x=+2+2\)

Abbiamo ottenuto un'equazione di primo grado.
\item \(-2x = +4 \sRarrow x = \dfrac{+4}{-2} = -2\)
\end{enumerate}

\end{esempio}


\subsection{Equazioni in cui l'incognita scompare}
\ind{Eq. impossibili} \ind{Eq. indeterminate} \ind{Eq. determinate}

A volte, eseguendo i calcoli l'incognita dell'equazione scompare. 
Come possiamo interpretare queste equazioni?

Prima di procedere, vediamo due esempi.

\begin{esempio}{}{}
\(\dfrac{4}{5}-\dfrac{x}{2}=\dfrac{2-5x}{10}\).

\begin{enumerate} [nosep]
\item Calcoliamo il~\(\mcm\) tra i denominatori: in questo
caso~\(\mcm(5, 2, 10) = 10\).
\item Moltiplichiamo per~10 ambo i membri
dell'equazione: \\[1mm]
\(10\tonda{\dfrac{4}{5} -\dfrac{x}{2}} = 10\tonda{\dfrac{2-5x}{10}}\)
\item Eseguiamo i calcoli: \quad \(8 -5x = 2 -5x\).
\item Applichiamo la regola pratica:
\(-5x +5x = 2 -8\) i monomi in \(x\) si annullano!
\item Sommando i monomi simili si ottiene: \quad \(0 \cdot x=-6\).
\end{enumerate}

Il coefficiente dell'incognita è zero; non possiamo
applicare il secondo principio e dividere ambo i membri per zero.
D'altra parte non esiste nessun numero che moltiplicato
per zero dia come prodotto~\(-6\). 

Quindi~\(\IS =\emptyset \), l'equazione risulta \textbf{impossibile}.
\end{esempio}

\begin{esempio}{}{}
\(\dfrac{x}{6}-\dfrac{2x}{3}=-{\dfrac{x}{2}}\).

\begin{enumerate} [nosep]
\item Calcoliamo il~\(\mcm\) tra i denominatori: in questo
caso~\(\mcm(6, 3, 2) = 6\).
\item Moltiplichiamo per~6 ambo i membri
dell'equazione: \\[1mm]
\(6\left(\dfrac{x}{6}-\dfrac{2x}{3}\right)=6\left(-{\dfrac{x}{2}}\right)\)
\item Eseguiamo i calcoli: \quad \(x-4x=-3x\).
\item Applicando il primo principio si ottiene~\(0 \cdot x=0\).
\end{enumerate}

Il coefficiente dell'incognita è zero; non possiamo
applicare il secondo principio e dividere ambo i membri per zero.
D'altra parte per la proprietà della moltiplicazione
qualunque numero moltiplicato per zero dà come prodotto zero. 
Quindi \(\IS = \Q\), l'equazione è \textbf{indeterminata}.
\end{esempio}

\subsubsection{Riassumiamo i 3 casi che possiamo incontrare}

Qualunque sia la situazione di partenza, possiamo sempre scrivere un'equazione 
di primo grado nella forma: \quad 
\(A \cdot x = B\) \quad dove \(A\) e \(B\) sono due numeri.

Possono presentarsi i seguenti casi:
\begin{itemize} [nosep]
\item se~\(A\neq~0\) possiamo applicare il secondo principio
d'equivalenza dividendo ambo i membri per~\(A\) quindi
\(\IS =\left\{\dfrac{B}{A}\right\}\). L'equazione è \textbf{determinata}.
\item se~\(A=0\) non possiamo applicare il secondo principio
d'equivalenza e dividere ambo i membri per~\(A\) e si
presentano due casi:
\begin{itemize} [nosep]
\item \(B\neq~0\) allora~\(\IS =\emptyset \). 
L'equazione è \textbf{impossibile}.
\item \(B=0\) allora~\(\IS =\Q\). 
L'equazione è \textbf{indeterminata}.
\end{itemize}
\end{itemize}

Lo schema precedente si può rappresentare anche con un grafo ad albero:
\immagine{Albero che illustra lo schema precedente}
{\alberoequazioni}

% \ovalbox{\risolvii \ref{ese:13.20}, \ref{ese:13.21}, \ref{ese:13.22}, 
% \ref{ese:13.23}, \ref{ese:13.24}, \ref{ese:13.25}, \ref{ese:13.26}, 
% \ref{ese:13.27}, \ref{ese:13.28}, \ref{ese:13.29},
% \ref{ese:13.30}}

% \ovalbox{\ref{ese:13.31}, \ref{ese:13.32}, \ref{ese:13.33}, 
% \ref{ese:13.34}, 
% \ref{ese:13.35}, \ref{ese:13.36}, \ref{ese:13.37}, \ref{ese:13.38}, 
% \ref{ese:13.39},
% \ref{ese:13.40}, \ref{ese:13.41}, \ref{ese:13.42}, \ref{ese:13.43}, 
% \ref{ese:13.44}}

% \ovalbox{\ref{ese:13.45}, \ref{ese:13.46}, \ref{ese:13.47}, 
% \ref{ese:13.48}, 
% \ref{ese:13.49},\ref{ese:13.50}}


\section{Problemi di I° grado in un'incognita}
\label{sec:eq1_problemi}
\ind{problemi risolvibili con equazioni}

\subsection{Un po' di storia e qualche aneddoto}
\label{subsec:equazioni_problemi_storia}

Sin dall'antichità l'uomo si è trovato di fronte a difficoltà pratiche, 
legate alla vita quotidiana e ha perciò messo a punto strategie per 
superarle.

Sembra che nell'antico Egitto le periodiche piene del
Nilo abbiano spinto l'uomo a sviluppare la capacità
di tracciare rette parallele, rette perpendicolari, di misurare il
perimetro e l'area di particolari figure geometriche o
viceversa di calcolare le misure dei lati di poligoni di dato perimetro
o data area per poter ridefinire i confini degli appezzamenti di
terreno.

Il \emph{papiro di Rhind} \ind{papiro di Rhind}
\footnote{Dal nome dell'inglese A.~H.~Rhind che lo comprò a Luxor nel~1858.}, 
testo egizio scritto in
ieratico, risalente al~1700~\aC, si autodefinisce
``istruzioni per conoscere tutte le cose
oscure'' contiene più di~85 problemi con relativi
metodi di soluzione riguardanti il calcolo della capacità di
recipienti e di magazzini, la ricerca dell'area di
appezzamenti di terreno e altre questioni aritmetiche.

Nel problema~24 del papiro, ad esempio, viene calcolato il mucchio
quando esso ed il suo settimo sono uguali a~19. Mucchio è
l'incognita del problema, indicata con il termine
\emph{aha} il cui segno è
% \input{\folder lbr/giero}
\begin{inaccessibleblock}[Geroglifico]
\includegraphics[scale=0.28]{\folder img/giero.png}.
\end{inaccessibleblock}
                                                              
Noi traduciamo la richiesta nell'equazione~\(x+\dfrac{1}{7}x=19\).

\ind{Leonardo Pisano} \ind{Fibonacci}
Nel~1202 Leonardo Pisano, conosciuto col nome paterno di
`filius Bonacci' o Fibonacci, pubblicò il \emph{Liber Abaci} 
\ind{Liber Abaci}
in cui presenta vari metodi algebrici per la risoluzione di problemi di 
matematica applicata che Leonardo aveva imparato 
dai musulmani seguendo il padre nei suoi viaggi commerciali. 
I nuovi ``algoritmi'' presentati da Fibonacci, intendevano facilitare la 
risoluzione dei problemi di calcolo evitando l'uso dell'abaco. 
Nel~1223 a Pisa, l'imperatore Federico~II di Svevia, assistette a
un singolare torneo tra matematici dell'epoca; il problema proposto era 
il seguente:

\ind{conigli}
<<Quante coppie di conigli si ottengono in un anno (salvo i
casi di morte) supponendo che ogni coppia dia alla luce
un'altra coppia ogni mese e che le coppie più
giovani siano in grado di riprodursi già al secondo mese di
vita?>>.

Fibonacci vinse la gara dando al quesito una risposta così rapida da
far persino sospettare che il torneo fosse truccato. La soluzione fu
trovata tramite l'individuazione di una particolare
successione di numeri, nota come successione di Fibonacci.

Secondo la leggenda, il grande matematico Carl Fiedrich Gauss già
\ind{Gauss}
all'età di tre anni avrebbe corretto un errore di
suo padre nel calcolo delle sue finanze. 
All'età di 10 anni fu autorizzato a seguire le lezioni di aritmetica 
di un certo Buttner. 
Un giorno, agli studenti particolarmente turbolenti, Buttner
diede come compito di punizione il calcolo della somma dei primi~100
numeri, da~1 a~100. Poco dopo, sorprendendo tutti, il giovanissimo Carl
diede la risposta esatta, ``5050''.
Si era accorto che mettendo in riga tutti i numeri da~1 a~100 e nella
riga sottostante i numeri da~100 a~1, ogni colonna dava come somma~101;
fece dunque il prodotto~\(100\times~101\) e divise per~2, ottenendo 
facilmente il risultato: Buttner rimase sgomento.

\subsubsection{Risoluzione dei problemi}

% \epigraph{La risoluzione dei problemi \ldots serve ad acuire
% l'ingegno e a dargli la facoltà di penetrare
% l'intera ragione di tutte le cose.}{{\scshape{R. Descartes}}}

I problemi che possono presentarsi nel corso degli studi o
nell'attività lavorativa sono di diversa natura: di
tipo economico, scientifico, sociale, possono riguardare insiemi
numerici o figure geometriche. La matematica ci può aiutare a
risolvere i problemi quando essi possono essere tradotti in
``forma matematica'', quando cioè
è possibile trascrivere in simboli le relazioni che intercorrono
tra le grandezze del problema.

Analizzeremo problemi di tipo algebrico o geometrico, che potranno
essere formalizzati attraverso equazioni di primo grado in una sola
incognita. Prima di buttarci alla risoluzione del problema, procediamo a:

\begin{enumeratea}
\item una lettura ``attenta'' del
testo al fine di individuare l'ambiente del problema,
le parole chiave, i dati e le informazioni implicite,
l'obiettivo;
\item la scelta della grandezza incognita e la descrizione
dell'insieme in cui si ricerca il suo valore,
ragionando sull'obiettivo del problema (condizioni sull'incognita);
\item la traduzione in ``forma matematica'' delle relazioni che intercorrono 
tra i dati e l'obiettivo, cioè l'individuazione dell'equazione risolvente;
\item la risoluzione dell'equazione trovata;
\item il confronto tra la soluzione trovata e le condizioni poste su di essa.
\end{enumeratea}

\begin{problema}{}{}
Un mattone pesa un chilo più mezzo mattone. Quanto pesa un mattone?
\end{problema}

\begin{soluzione}{}{}
\emph{Dati}: peso di un mattone~\(=\) peso di mezzo 
mattone~\(+ 1\munit{kg}.\)

\emph{Obiettivo}: peso del mattone.

Creiamoci un'immagine della situazione.
Ora, possiamo pensare di risolvere il problema spezzando, con la nostra 
immaginazione, il mattone intero in due metà.

\triaffiancati{.34}{.30}{.34}{
\immagine[.6]{Bilancia a due piatti, su uno c'è un mattone sull'altro 
mezzo mattone più un kg.}
{\bilanciaa}
% \vspace{1mm}
}{
Poi togliamo mezzo mattone dal piattello a sinistra e il mezzo mattone dal 
piattello di destra: se era in equilibrio prima, lo sarà anche dopo.
}{
\immagine[.6]{Bilancia a due piatti, su uno c'è un mattone sull'altro 
mezzo mattone più un kg.}
{\bilanciab}
% \vspace{1mm}
}

Ora raddoppiamo il contenuto dei due piattelli: a sinistra torna ad essere 
\emph{un} mattone, a destra abbiamo \(2\munit{Kg}\).

Non sempre l'immaginazione ci aiuta, in questo caso possiamo tradurre il 
problema in un'equazione e risolverla.\\
Come incognita scegliamo il peso del mattone: lo indichiamo con~\(p\).\\
Traduciamo il problema in un'equazione: \quad 
\(p=\dfrac{1}{2}p+1\munit{Kg}\)\\
Risolviamo l'equazione: \quad 
\(p-\dfrac{1}{2}p=1\munit{Kg} \sRarrow \) 
\(\dfrac{1}{2}p=1\munit{Kg} \sRarrow p=2\munit{Kg}\)\\
La soluzione ottenuta è accettabile.
\end{soluzione}

\begin{problema}{}{}
Aggiungendo a un numero naturale i suoi tre quarti, si ottiene il suo
doppio aumentato di~10. Qual è il numero?
\end{problema}

\begin{soluzione}{}{}
L'ambiente del problema è numerico: si cerca un numero naturale. 
Indichiamo con~\(n\) l'incognita cerchiamo quindi~\(n\in\N\). 
La lettura attenta del testo mette in luce le operazioni che dobbiamo 
eseguire sull'incognita e che traduciamo nei dati:

\emph{Dati}:~\(n+\dfrac{3}{4}n=2n+10\).

\emph{Obiettivo}:~\(n\in\N\).

\emph{Procedura risolutiva}:

L'equazione risolvente è già indicata nei dati~\(n+\dfrac{3}{4}n=2n+10\).

Per risolverla moltiplichiamo ambo i membri per~4, otteniamo:
\[4n+3n-8n=40\Rightarrow -n=40\Rightarrow n=-40.\]

La soluzione non è accettabile per le condizioni poste; il problema
non ha soluzione.
\end{soluzione}

\begin{problema}{}{}
Il~1{\textdegree} gennaio~1990 Anna aveva il doppio
dell'età di Piero; il~1{\textdegree} gennaio~2000
Anna aveva vent'anni più di Piero. Quale sarà
l'età di Anna il~1{\textdegree} gennaio~2010?
\end{problema}

\begin{soluzione}{}{}
Leggendo attentamente il problema notiamo che le incognite sono due:
l'età di Anna e l'età di Piero.
Indichiamo perciò con~\(a\) l'età di
Anna al~1990 e con~\(p\) quella di Piero.

Nel~2000 la loro età sarà aumentata di~10 anni. Naturalmente la
soluzione del problema sarà nell'insieme dei numeri
naturali. Scriviamo dati e obiettivo usando il formalismo matematico:

\emph{Dati}: nel~1990:~\(a = 2p\), nel~2000:~\(a +10 = (p +10) +20\).

\emph{Obiettivo}: L'età di Anna nel~2010.

\emph{Procedura risolutiva}:
Osserviamo che una volta determinata l'età di Anna
nel~1990, basterà aggiungere a questa~20 per ottenere la soluzione,
pertanto l'età di Anna nel~2010 è~\(a+20\).
Trasformiamo la seconda relazione riportata nei dati sostituendo
l'informazione relativa al~1990,
si ottiene~\(2p+10=p+10+20\Rightarrow~2p-p=20\Rightarrow p=20.\)
L'età di Piero nel~1990 era~20, quindi~\(a=40\).
Infine, l'età di Anna nel~2010 è~\(40+20=60\).
La soluzione è accettabile; il problema è determinato.
\end{soluzione}

\begin{problema}{}{}
Calcolare l'area di un rettangolo in cui
l'altezza supera \(\dfrac{1}{3}\) della base di~8m e il
perimetro è~\(\dfrac{20}{7}\) della base stessa.
\end{problema}

\begin{soluzione}{}{}
Il problema è di tipo geometrico e riguarda un rettangolo. Facendo riferimento 
alla figura abbiamo:
\begin{htmulticols}{2}
\emph{Dati}:~\(AD=\dfrac{1}{3}AB+8\), \quad \(2p=\dfrac{20}{7}AB\)

\emph{Obiettivo}: L'\(\area{ABCD}\)

\emph{Procedura risolutiva}:

\hspace{10mm}
\(\area{ABCD} = \overline{AB} \cdot \overline{AD}\).

% %  \input{\folder lbr/fig002_ret.pgf}
% \rettangolo
\immagine[1]{Un rettangolo.}
{\disegno{\rettangolo{7}{4}{blue!50!black}{blue!5}}}
\end{htmulticols}

Dobbiamo dunque determinare queste due misure. I dati del problema
indicano che la misura dell'altezza dipende da quella
della base; una volta trovata questa misura basta farne un terzo e
aggiungere~8 per avere quella dell'altezza; questo
ragionamento ci fa scegliere come incognita~\(\overline{AB}=x\)
con~\(x\) numero reale positivo.

Traduciamo con formalismo matematico la prima e la seconda relazione
contenuta nei dati:
\(\overline{AD}=\dfrac{1}{3}x+8\sstext{e} 2p=\dfrac{20}{7}x\).

Sappiamo che il perimetro di un rettangolo è il doppio della somma
della base con l'altezza. 
Riscriviamo con linguaggio matematico anche questa relazione:\\
\(2\cdot \left(x+\dfrac{1}{3}x+8\right)=\dfrac{20}{7}x\)\\
che risulta l'equazione risolvente.

Svolgiamo i calcoli e otteniamo: \quad 
\(4x=21\cdot 16\Rightarrow x=84\Rightarrow\overline{AB}=84\) \\
e quindi: \quad \(\overline{AD}=36\).

Ottenute le misure della base e dell'altezza calcoliamo 
\(\area{ABCD} = 36\cdot 84=3024\munit{{m}^{2}}\).
\end{soluzione}

\begin{problema}{}{}
In un triangolo rettangolo il perimetro è~\(120\munit{cm}\) e un cateto 
è~\(3/5\) dell'ipotenusa. Determinare l'area del triangolo.
\end{problema}

\begin{soluzione}{}{}
Rappresentiamo il triangolo:

\affiancati{.54}{.44}{

\emph{Dati}:~\(C\hat{{A}}B=90\grado\), \(2p= 120\), \(AC=\dfrac{3}{5}CB\).

\emph{Obiettivo}: L'\(\area{ABC}\).

\emph{Procedura risolutiva}:

\hspace{10mm}
\(\area{ABC} =\dfrac{1}{2}\overline{AB} \cdot \overline{AC}\)
}{
\immagine[1]{triangolo rettangolo d ipotenusa BC.}{\triangolo}
}

Il dato \(AC=\dfrac{3}{5}CB\) può essere scritto come: \(AC=3x \wedge CB=5x\).
L'altro cateto si può calcolare con il teorema di Pitagora:
\(AB=\sqrt{(5x)^2-(3x)^2}=\sqrt{25x^2-9x^2}=\sqrt{16x2}=4x\).

Il perimetro è: \(2p=4x+5x+3x=12x=120\) e da questa equazione ricaviamo 
\(x=10\) da cui: \(AB=40, BC=50, CA=30\). 

Da cui si ricava facilmente l'area: 
\(\area = AB \cdot AC \cdot \dfrac{1}{2} = 
40 \cdot 30 \cdot \dfrac{1}{2} =600\)
\end{soluzione}
