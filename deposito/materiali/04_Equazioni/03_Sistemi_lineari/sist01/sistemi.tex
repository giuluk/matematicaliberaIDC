%(c) 2012 - 2014 - Dimitrios Vrettos d.vrettos@gmail.com
% (c) 2014 Daniele Zambelli - daniele.zambelli@gmail.com

\input{\folder sistemi_grafici}

\chapter{Sistemi di equazioni}

\section{Equazione lineare in due incognite}
\label{sec:sist_eqdue}

\begin{definizione}{}{}
Una equazione di primo grado si chiama \emph{equazione lineare}.
\end{definizione}

In questo capitolo affronteremo equazioni lineari in più incognite. 
Partiamo da un semplice problema la cui soluzione si può modellizzare 
con un'equazione lineare in due incognite.

\begin{problema}{}{}
Determinare due numeri naturali la cui somma sia~16.
\end{problema}

\begin{soluzione}
Il problema si può formalizzare con l'equazione~\(x+y=16\) che è 
un'equazione in due incognite, di primo grado (lineare), 
indicando con~\(x\) e~\(y\) i due numeri richiesti.

Iniziamo restringendo l'ambiente del problema all'insieme~\(\N\) dei numeri 
naturali.

% Determiniamo l'Insieme Soluzione del problema proposto.
L'obiettivo è trovare~\(x\in\N\) e~\(y\in\N\) tali che~\(x+y=16\) 
oppure~\(\coppia{x}{y}\in\N\times\N\) tali che~\(x+y=16\).
Le coppie di numeri naturali che sono soluzioni
dell'equazione sono facilmente determinabili e sono
tutte quelle riportate nella tabella seguente.

\begin{tabular}{cccccccccccccccccccc}
\toprule
\(x\) & 0 & 1 & 2 & 3 & 4 & 5 & 6 & 7 & 8 & 9 & 10 & 11 & 12 & 13 & 
14 & 15 & 16\\
\(y\) & 16 & 15 & 14 & 13 & 12 & 11 & 10 & 9 & 8 & 7 & 6 & 5 & 4 & 3 & 
2 & 1 & 0\\
\bottomrule
\end{tabular}

L'Insieme Soluzione del problema posto è dunque
formato dalle~17 coppie di numeri naturali sopra elencate.

Riformuliamo il problema cercando coppie di numeri razionali la cui
somma sia~16.
In simboli scriviamo~\(x\in\Q\) e~\(y\in\Q\) tali che~\(x+y=16\) 
oppure~\(\coppia{x}{y}\in\Q\times\Q\) tali che~\(x+y=16\).

Possiamo subito dire che tutte le coppie precedenti sono soluzione del
problema, ma ce ne sono infinite altre, ad esempio la 
coppia~\(\coppia{-7}{+23}\) è 
soluzione del problema perché sostituendo a~\(x\) il
valore~\(-7\) e a~\(y\) il valore~\(+23\) si ha~\((-7)+(+23)=16\).
% Dal procedimento si capisce che anche la coppia~\(\coppia{+23}{-7}\) è
% soluzione del problema perché~\((+23)+(-7)=16\).

Se attribuiamo un valore arbitrario \(x_0\) alla variabile \(x\), l'altro
elemento della coppia soluzione si può ottenere sottraendo da~16 il
valore~\(x_0\):~\(y_0=16-x_0\).

Completa tu:

\begin{itemize} [nosep]
\item se~\(x=-3\) allora~\(y=16-(-3)=\ldots\ldots\) e la coppia 
\(\coppia{\ldots}{\ldots}\) è soluzione dell'equazione;
\item se~\(x=\dfrac{3}{2}\) allora~\(y =\dotfill\), la coppia 
\(\coppia{\ldots\ldots}{\ldots\ldots}\) è soluzione dell'equazione;
\item se~\(x =\dotfill\) allora~\(y=\) \dotfill, la coppia 
\(\coppia{\ldots\ldots}{\ldots\ldots}\) è soluzione dell'equazione.
\end{itemize}

Quindi, se l'ambiente del problema è
l'insieme~\(\Q\), troviamo infinite coppie di
numeri razionali che soddisfano il problema.

Lo stesso vale se formuliamo il problema nell'insieme dei
numeri reali~\(\R\).

Se~\(x=\sqrt{2}\Rightarrow y=16-\sqrt{2}\), la 
coppia~\(\coppia{\sqrt{2}}{16-\sqrt{2}}\) è soluzione
dell'equazione. 

% \newpage %------------------------------------------------------

Completa:

\begin{itemize} [nosep]
\item se~\(x=-2\sqrt{3}+1\) allora~\(y=\dotfill\)
\item se~\(x=16+\dfrac{3\sqrt{5}}{2}\) allora~\(y=\dotfill\)
\end{itemize}
\end{soluzione}

\begin{definizione}{}{}
Si chiama \emph{Insieme Soluzione} \((\IS)\) di un'equazione di primo
grado in due incognite, l'\emph{insieme delle coppie
ordinate} di \emph{numeri reali} che sostituiti rispettivamente a~\(x\) e
a~\(y\) rendono vera l'uguaglianza.
\end{definizione}

% \ovalbox{\risolvii \ref{ese:22.1}, \ref{ese:22.2}, \ref{ese:22.3}}

\subsection{Rappresentazione di un'equazione lineare sul piano cartesiano}

% \begin{exrig}\vspace{1.10ex}
 \begin{esempio}{}{}
Determinare l'insieme soluzione dell'equazione~\(3y-x+1=0\) con~\(x\in\R\) 
e~\(y\in\R\).
Osserviamo che l'equazione assegnata ha due incognite ed
è di primo grado; l'insieme soluzione sarà formato
dalle infinite coppie ordinate~\(\coppia{x}{y}\) di numeri tali 
che~\(3y-x+1=0\).

Possiamo verificare che la coppia~\(\coppia{1}{0}\) è soluzione
dell'equazione, ma come facciamo a determinare tutte le
coppie che soddisfano quella equazione?

Fissiamo l'attenzione sull'incognita~\(y\),
pensiamo l'equazione come un'equazione
nella sola~\(y\), ricaviamo~\(y\) come abbiamo fatto nelle equazioni di 
primo grado a una sola incognita, applicando i principi di equivalenza 
delle equazioni:

\begin{equation*}
3y-x+1=0\Rightarrow~3y=x-1\Rightarrow\frac{3y}{3}=\frac{x-1}{3}\Rightarrow 
y=\frac{1}{3}x-\frac{1}{3}
\end{equation*}
Al variare di~\(x\) in~\(\R\), si ottengono tutte le infinite
soluzioni dell'equazione assegnata.
Prova a determinarne alcune e a riportarle nel piano cartesiano:

\begin{minipage}{.48\textwidth}
\begin{center}
 \begin{tabular}{ccc}
\toprule
x & y & coppia\\
\midrule
\(-3\) & \ldots\ldots & \(\coppia{-3}{\ldots\ldots}\)\\
\(-2\) & \ldots\ldots & \(\coppia{-2}{\ldots\ldots}\)\\
\(-1\) & \ldots\ldots & \(\coppia{-1}{\ldots\ldots}\)\\
\(0\) & \ldots\ldots & \(\coppia{0}{\ldots\ldots}\)\\
\(+1\) & \ldots\ldots & \(\coppia{+1}{\ldots\ldots}\)\\
\(+2\) & \ldots\ldots & \(\coppia{+2}{\ldots\ldots}\)\\
\(+3\) & \ldots\ldots & \(\coppia{+3}{\ldots\ldots}\)\\
\bottomrule
\end{tabular}
\end{center}
\end{minipage}
\hfill
\begin{minipage}{.48\textwidth}
% \begin{wrapfigure}{o}{0pt}
%  \input{\folder lbr/fig001_car.pgf}
\disegno{\rcom{-6}{+6}{-4}{+4}{gray!50, very thin, step=1}}
% \end{wrapfigure}
\end{minipage}

In verità non possiamo trovare tutte le infinite coppie che risolvono
quella equazione, ma possiamo darne una rappresentazione grafica.

La formula \[y=\frac{1}{3}x-\frac{1}{3}\] rappresenta una funzione
lineare; riportiamo le coppie trovate in un riferimento cartesiano
ortogonale e tracciamo la retta che rappresenta la funzione.

Una qualunque equazione lineare~\(ax+by+c=0\) ammette infinite
soluzioni, costituite da coppie ordinate di numeri reali; esse sono le
coordinate cartesiane dei punti della retta grafico della 
funzione~\(y=-{\dfrac{a}{b}}x-\dfrac{c}{b}\).
La formula~\(y=-{\dfrac{a}{b}}x-\dfrac{c}{b}\) si chiama \emph{equazione 
esplicita 
della retta}.
\end{esempio}

\newpage %-------------------------------------------

\begin{esempio}{}{}
 Risolvi graficamente l'equazione~\(y+\dfrac{2}{3}x-2=0\text{, con } 
x\in\R\text{ e }y\in\R\).
\end{esempio}

% \begin{wrapfloat}{figure}{r}{0pt}
\rettaconpunti{-2./3*x+2}{-2./3*\x+2}
                    {-3, -2.5, ..., 5}
% \input{\folder lbr/fig002_car.pgf}
% \end{wrapfloat}

L'equazione assegnata è in due incognite, di primo
grado, è cioè una equazione lineare. Nel riferimento cartesiano
ortogonale essa rappresenta una retta.

Troviamo l'equazione esplicita della retta:
\[y+\frac{2}{3}x-2=0\Rightarrow y=-{\frac{2}{3}}x+2.\]

% Individuiamo l'ordinata del punto di intersezione della
% retta con l'asse~\(y\):~\(q=2\), quindi~\(P\coppia{0}{2}\) è un
% punto della retta.
% 
% Troviamo un altro punto appartenente alla retta: \\
% se~\(x=3\) allora~\(y=0\),
% quindi~\(A\coppia{3}{0}\) è un punto della retta.

Disegniamo la retta nel piano cartesiano: le infinite coppie
\(\coppia{x}{y}\), 
coordinate dei punti della retta tracciata, sono soluzioni
dell'equazione assegnata.
% \vspace{1.10ex}
% \end{exrig}

% \ovalbox{\risolvii \ref{ese:22.4}, \ref{ese:22.5}, \ref{ese:22.6}}

\section{Definizione di sistema di equazioni}
\label{sec:sist_definizione}

\begin{problema}{}{}
\label{pr:22.1}
Nel rettangolo~\(ABCD\), la somma del doppio dell'altezza con la metà 
della base è di~\(98\munit{m}\) aumentando l'altezza di~\(3\munit{m}\) e 
la base di~\(2\munit{m}\) il perimetro del rettangolo diventa 
di~\(180\munit{m}\). 
Determinare l'area in~\(\munit{m}^{2}\) del rettangolo.
\end{problema}
\begin{htmulticols}{2}
\emph{Dati}:
\begin{align*}
&2\overline{AD}+\frac{1}{2}\overline{AB}=98\munit{m},\\
&2(\overline{AD}+3m+\overline{AB}+2m)=180\munit{m}.
\end{align*}

\emph{Obiettivo}: Area

\immagine[.8]{Un rettangolo.}
{\disegno{\rettangolo{5}{3}{blue!50!black}{blue!5}}}
\end{htmulticols}

Per determinare l'area del rettangolo dobbiamo moltiplicare le misure 
delle sue dimensioni~\(\oarea=\overline{AB}\cdot\overline{BC}\)
che però non conosciamo; il problema ha quindi due incognite.

Analizzando i dati possiamo osservare che ci sono fornite due
informazioni che legano le grandezze incognite. 
Se poniamo~\(\overline{AD}=x\) e~\(\overline{AB}=y\)
otteniamo le due equazioni:
\[2x+\frac{1}{2}y=98;\quad~2(x+3+y+2)=180\]
che dovranno risultare soddisfatte per una stessa coppia di numeri
reali.

\begin{osservazione}{}{}
Dobbiamo tenere presente che questa sostituzione produce delle equazioni 
che potrebbero avere come soluzione dei numeri negativi, mentre nel 
problema di partenza non ha senso un rettangolo con lati di lunghezza 
negativa.
Quindi:~\(x \geqslant 0 \text{ e } y \geqslant 0\)
\end{osservazione}

\begin{definizione}{}{}
Si definisce \emph{sistema di equazioni} l'insieme di più equazioni, 
in due o più incognite, che devono essere verificate contemporaneamente. 
La scrittura formale si ottiene raggruppando le equazioni mediante una 
parentesi graffa.
\end{definizione}

Il problema \ref{pr:22.1} si formalizza dunque con il sistema:
\[\sistema{
 2x+\dfrac{1}{2}y=98 \\
 2(x+3+y+2)=180
}\]

\begin{definizione}{}{}
L'Insieme Soluzione (\(\IS\)) di un sistema di equazioni in
due incognite è formato dalle coppie di numeri reali
che rendono vere tutte le equazioni contemporaneamente.
\end{definizione}

\begin{definizione}{}{}
Si chiama \emph{grado di un sistema} il prodotto dei gradi delle
equazioni che lo compongono. 
In particolare, se le equazioni che lo compongono sono di primo grado, 
il sistema si chiama \emph{sistema lineare}.
\end{definizione}

\begin{definizione}{}{}
Un sistema si dice scritto in \emph{forma normale} o \emph{canonica} 
quando in ogni equazione tutti i monomi che contengono le incognite sono 
scritti nel primo membro e in ordine:
\[\sistema{
 a_{0}x+b_{0}y=c_{0} \\
 a_{1}x+b_{1}y=c_{1} 
}
\text{ con }x \text{, e } y \text{ variabili} \text{, e } 
a_{0},~b_{0},~c_{0},~a_{1},~b_{1},~c_{1}\text{ numeri}\]
\end{definizione}

Il sistema a cui si giunge risolvendo il problema posto all'inizio è
composto da due equazioni in due incognite, è di primo grado 
e quindi è un sistema lineare. 
La sua forma canonica si ottiene sviluppando i calcoli:

\[\sistema{
4x+y=196\\
x+y=85
}\]

Nel seguito vedremo come risolvere sistemi lineari.

\section{Metodi di soluzione di sistemi di equazioni}
\label{sec:sist_soluzione}

Per risolvere un sistema conviene sempre, come primo passo scriverlo in 
forma normale.

\subsection{Procedimento per ottenere la forma canonica di un sistema}
Per ottenere la \emph{forma canonica} di un sistema lineare dobbiamo:
\begin{enumerate*}
 \item spostare tutti i termini che contengono incognite a primo membro;
 \item spostare i termini senza incognite a secondo membro;
 \item ordinare i monomi con le incognite.
\end{enumerate*}

% \begin{exrig}
 \begin{esempio}{}{}
 Scrivere in forma canonica il sistema:
\[\sistema{4x^{2}-(y+2x)^{2}=x+1-y(4x+y-1)\\
\dfrac{x-2}{2}+\dfrac{y+3}{3}=0
}\]

Eseguiamo i calcoli nella prima equazione e riduciamo allo stesso
denominatore la seconda equazione:

\[\sistema{
 4x^{2}-y^{2}-4x^{2}-4xy=x+1-4xy-y^{2}+y\\
 \dfrac{3x-6+2y+6}{6}=\dfrac{0}{6}
 }\]

Usando il secondo principio di equivalenza delle equazioni eliminiamo i 
denominatori.

Usando il primo principio di equivalenza delle equazioni portiamo le
incognite al primo membro e i termini senza incognita a secondo membro.

Otteniamo così un sistema equivalente a quello di parteza, scritto in forma 
normale:
 \[\sistema{
   x+y=-1\\
   3x+2y=0
}\]
 \end{esempio}
% \end{exrig}

\subsection{Sistemi indeterminati o impossibili}

Prima di utilizzare uno dei metodi di soluzione che saranno descritti più 
avanti, conviene riflettere su alcuni casi particolari. Consideriamo il 
seguente sistema:

\[\sistema{3x +7y = 9 \\ 3x +7y = 10}\]

È abbastanza ovvio che la stessa espressione non possa valere 
contemporaneamente sia~9 sia~10. Questo sistema è \emph{impossibile}.

Osserviamo anche quest'altro caso:

\[\sistema{5x +8y = 17 \\ 5x +8y = 17}\]

Questa volta possiamo osservare che non abbiamo una sistema di \emph{due} 
equazioni, ma un sistema con una sola equazione ripetuta due volte. 
In questo caso, come già visto, l'equazione ha infinite soluzioni e si 
dirà \emph{indeterminato}.

Ora, se non ci troviamo in una di queste due situazioni, possiamo cercare 
dei modi per trovare l'unica soluzione del sistema.

\subsection{Metodo di sostituzione}
\emph{Risolvere il sistema} significa determinare tutte le coppie di
numeri reali che soddisfano contemporaneamente le due equazioni.

Analizziamo i diversi metodi che permettono di ottenere
l'Insieme Soluzione, cominciamo dal \emph{metodo di sostituzione}.

% \begin{exrig}
\begin{esempio}{}{}
\(\sistema{-3x+y=2\\5x-2y=7}\).
\end{esempio}

Il sistema si presenta già in forma canonica. Il metodo di
sostituzione si svolge nei seguenti passi:

\begin{enumerate}
 \item Scegliamo con intelligenza una delle due equazioni e una
delle due incognite da cui partire. 
La furbizia consiste nello scegliere una incognita con il coefficiente 
uguale a~\(+1\) o a~\(-1\).
Applicando i principi d'equivalenza delle equazioni, ricaviamo questa
incognita.
Nel nostro esempio, partiamo dalla prima equazione e ricaviamo
l'incognita~\(y\):
\[\sistema{
     -3x+y=2\\
     5x-2y=7
  }
\Rightarrow
  \sistema{y=2+3x \\
           5x-2y=7
  }
\]

\item Sostituiamo nell'altra equazione, al posto
dell'incognita trovata, l'espressione a cui è uguale. 
Nel nostro esempio abbiamo:
\[\sistema{
     y=2+3x\\
     5x-2y=7
  }
\Rightarrow
\sistema{y=2+3x \\
         5x-2(2+3x)=7
}\]

\item Otteniamo così una equazione che contiene una sola 
incognita, la risolviamo.
Nel nostro esempio abbiamo:
\[\sistema{y=2+3x\\
5x-4-6x=7}\Rightarrow
\sistema{
         y=2+3x\\
         5x-4-6x=7
        }\Rightarrow
 \sistema{y=2+3x\\
         -x=7+4
        }\Rightarrow
 \sistema{y=2+3x\\
         x=-11
  }\]

\item Sostituiamo nella prima equazione il valore dell'incognita 
trovata e calcoliamo il valore dell'altra incognita:
\[ \sistema{y=2+3\tonda{-11}\\
         x=-11
  }\Rightarrow
  \sistema{
         y=-31\\
         x=-11
  }\]

 \item Scriviamo l'insieme soluzione che è formato da una coppia ordinata 
di numeri~\(\IS=\{(-11;~-31)\}\). Questi due numeri sostituiti alle 
incognite rendono vere entrambe le uguaglianze.
\end{enumerate}

\begin{esempio}{}{}
\(\sistema{
  \dfrac{1}{2}(x-1)+3\left(y+\dfrac{1}{3}\right)=\dfrac{1}{6}\\
  y\left(1+\dfrac{2}{5}\right)-2=\dfrac{4}{5}-\dfrac{x-1}{5}
  }\)

\begin{enumerate}
\item Il sistema non si presenta nella forma canonica. 
Svolgiamo i calcoli e portiamo il sistema in forma canonica:
\[\sistema{3x+18y=-2\\x+7y=15}\]

\item Ricaviamo~\(x\) dalla seconda equazione:
\[\sistema{3x+18y=-2\\x=15-7y}\]

\item Abbiamo fatto questa scelta perché possiamo ottenere il valore 
di~\(x\) con facilità e senza frazioni. 
Sostituiamo nella prima equazione al posto 
di~\(x\) l'espressione trovata:
\[\sistema{
          3\cdot(15-7y)+18y=-2\\
          x=15-7y
          }
\]
\item Risolviamo la prima equazione che è di primo grado nella sola
incognita~\(y\) e sostituiamo questo valore nella seconda equazione:
% \[\sistema{
%           -3y=-47\\
%           x=15-7y
%           }
% \Rightarrow \sistema{
%           y=\dfrac{47}{3}\\
%           x=15-7y
%          }\]
% \item Sostituiamo il valore di~\(y\) nella seconda equazione:
\[\sistema{
          y=\dfrac{47}{3}\\
          x=15-7\left(\dfrac{47}{3}\right)
          }
\Rightarrow \sistema{x=-\dfrac{284}{3}\\
 y=\dfrac{47}{3}}\]
\item Scriviamo l'insieme delle soluzioni:
\[\IS=\left\{\left(-\frac{284}{3};~\frac{47}{3}\right)\right\}\]
\end{enumerate}

 \end{esempio}
% 
%  \begin{esempio}{}{}
%  \(\sistema{
%  \dfrac{1}{y}=2\left(\dfrac{x}{y}-\dfrac{1}{2}\right)\\
%  \dfrac{5x+4y+19}{x}=-2}\)
% 
%  Il sistema è fratto poiché in ciascuna equazione compare
% l'incognita al denominatore; per poter applicare il
% secondo principio di equivalenza delle equazioni eliminando i
% denominatori, dobbiamo porre le~\(\CE\) e individuare il Dominio del 
% sistema assegnato, cioè l'insieme in cui si 
% troverà~\(\CE: y\neq~0\text{ e }x\neq~0\) per
% cui~\(D=\R_{0}\times \R_{0}\).
% 
% Portiamo a forma canonica applicando i principi di equivalenza delle
% equazioni:
% 
% \begin{equation*}
% {\sistema{
% 	  \dfrac{1}{y}=2\left(\dfrac{x}{y}-\dfrac{1}{2}\right)\\
% 	  \dfrac{5x+4y+19}{x}=-2}\Rightarrow
%  \sistema{
% 	  \dfrac{1}{y}=\dfrac{2x}{y}-1\\
% 	  5x+4y+19=-2x}}\Rightarrow
%  \sistema{
% 	  2x-y=1\\
% 	  7x+4y=-19}
% \end{equation*}
% 
% Applichiamo il metodo di sostituzione:
% 
% \begin{multline*}
% \sistema{2x-y=1\\7x+4y=-19}\Rightarrow
% \sistema{y=2x-1\\7x+4y=-19}\Rightarrow
% \sistema{y=2x-1\\7x+4(2x-1)=-19}
% \\\Rightarrow
% \sistema{y=2x-1\\15x=-15}\Rightarrow
% \sistema{y=2(-1)-1\\x=-1}\Rightarrow
% \sistema{y=-3\\x=-1}
% \end{multline*}
% La soluzione è compatibile con le condizioni di esistenza.
% \end{esempio}
 % \end{exrig}

 % \ovalbox{\risolvii \ref{ese:22.7}, \ref{ese:22.8}, \ref{ese:22.9}, 
% \ref{ese:22.10}, \ref{ese:22.11}, \ref{ese:22.12}, \ref{ese:22.13}, 
% \ref{ese:22.14}, \ref{ese:22.15}}

% \subsection{Metodo del confronto}
% 
% % \begin{exrig}
% \begin{esempio}{}{}
% \(\sistema{-3x+y=2\\5x-2y=7}\)
% 
% \paragraph{Passo I} ricaviamo da entrambe le equazioni la stessa
% incognita. Nel nostro esempio ricaviamo la~\(y\) contemporaneamente da 
% entrambe le equazioni:
% \[\sistema{y=2+3x\\y=\dfrac{5x-7}{2}}\]
% 
% \paragraph{Passo II} poiché il primo membro delle
% equazioni è lo stesso, possiamo uguagliare anche i secondi membri,
% ottenendo un'equazione in una incognita. 
% Nell'esempio~\(2+3x=\frac{5x-7}{2}.\) 
% 
% \paragraph{Passo III} risolviamo l'equazione trovata e determiniamo il 
% valore di una delle due incognite.
% Nel nostro esempio~\(4+6x=5x-7\Rightarrow x=-11\).
% 
% \paragraph{Passo IV} si sostituisce il valore trovato
% dell'incognita in una delle due equazioni e ricaviamo
% l'altra incognita. Nel nostro esempio:
% \[\sistema{
%      x=-11\\
%      y=2+3x
%      }
% \Rightarrow\sistema{
%           x=-11\\
%           y=-31
%          }\]
% \paragraph{Passo V} possiamo ora scrivere l'insieme soluzione.
% Nel nostro esempio:~\(\IS=\{(-11;-31)\}\).
% 
% In conclusione, il sistema è determinato, la 
% coppia ordinata~\(\coppia{-11}{-31}\)
% verifica contemporaneamente le due equazioni del sistema.
%  \end{esempio}
% % \end{exrig}
% 
% % \ovalbox{\risolvii \ref{ese:22.16}, \ref{ese:22.17}, \ref{ese:22.18}, 
% % \ref{ese:22.19}}

\subsection{Metodo di riduzione}
Il metodo di riduzione si basa sulla seguente osservazione: se un
sistema è formato dalle equazioni~\(A=B\) e~\(C=D\) possiamo dedurre
da queste la nuova equazione~\(A+C=B+D\).
\begin{equation*}
\sistema{
A=B \\
C=D
}\Rightarrow A+C=B+D.
\end{equation*}
L'equazione ottenuta potrebbe presentarsi in una sola
incognita e quindi potrebbe essere facile trovare il valore di quella
incognita.

% \begin{exrig}
\begin{esempio}{}{}
\(\sistema{3x-5y=+1 \\2x+5y=-4}\)

Sommando membro a membro le due equazioni otteniamo:
\[\stackrel{\underline{\sistema{3x-5y=+1 \\2x+5y=-4}}}
  {\qquad \qquad \qquad ~~~~5x ~~~ // ~~ =
   -3 \quad \sRarrow x=-\frac{3}{5}}\]
Se vogliamo utilizzare lo stesso metodo per calcolare il valore dell'altra 
incognita dobbiamo utilizzare un trucco: moltiplicare ogni equazione per un 
opportuno valore (facile da trovare):
\[\stackrel{\sistema{3x-5y=+1\\2x+5y=-4}\Rightarrow
\begin{array}{l}\cdot \tonda{+2}\\ \cdot \tonda{-3}\end{array}
  \underline{\sistema{+6x-10y=+2\\-6x-15y=+12}}}
  {\qquad \qquad \qquad \qquad \qquad \qquad \qquad \qquad \qquad \qquad 
   //  -25y  =+14 \quad \sRarrow y=-\frac{14}{25}}\]
\end{esempio}

\begin{esempio}{}{}
\(\sistema{3x-5y=+1\\5x-4y=-4}\)

In questo caso sommando le equazioni membro a membro non riusciamo a 
eliminare un'incognita, dobbiamo usare il trucco visto sopra.
Osservando i coefficienti delle \(x\) vediamo che è possibile farli 
diventare opposti moltiplicando la prima equazione per~\(+5\) e la 
seconda per~\(-3\):
\[\stackrel{\sistema{3x-5y=+1\\5x-4y=-4}\Rightarrow
  \begin{array}{l}\cdot \tonda{+5}\\ \cdot \tonda{-3}\end{array}
  \underline{\sistema{+15x-25y=+5\\-15x+12y=+12}}}
  {\qquad \qquad \qquad \qquad \qquad \qquad \qquad \qquad \qquad \qquad 
   // ~~ -13y  =+17 \quad \sRarrow y=-\frac{17}{13}}\]
Analogamente, dopo aver osservato il valore dei coefficienti della \(y\):
\[\stackrel{\sistema{3x-5y=+1\\5x-4y=-4}\Rightarrow
  \begin{array}{l}\cdot \tonda{+4}\\ \cdot \tonda{-5}\end{array}
  \underline{\sistema{+12x-20y=+4\\-25x+20y=+20}}}
  {\qquad \qquad \qquad \qquad \qquad \qquad \qquad \qquad \qquad 
   ~~~~~ -13x ~~~~~ // ~~ =+24 \quad \sRarrow x=-\frac{24}{13}}\]
Abbiamo così determinato la coppia soluzione del 
sistema~\(\coppia{-\dfrac{24}{13}}{-\dfrac{17}{13}}\).

 \end{esempio}
% \end{exrig}

% \subsubsection{Generalizzazione del metodo di riduzione}
% 
% Assegnato il sistema lineare~\(\sistema{
%  a_{0}x+b_{0}y=c_{0}\\
%  a_{1}x+b_{1}y=c_{1} 
% }\)
% con~\(a_{0}, b_{0}, c_{0}, a_{1}, b_{1}, c_{1}\) numeri reali.
% 
% \paragraph{Passo I} per eliminare~\(y\) moltiplichiamo la prima 
% per~\(b_{1}\) e la seconda per~\(-b_{0}\):
% \[\sistema{
%  a_{0}b_{1}x+b_{0}b_{1}y=c_{0}b_{1}\\
%  -a_{1}b_{0}x-b_{1}b_{0}y=c_{1}-b_{0}
% }\]
% 
% \paragraph{Passo II} sommiamo le due equazioni:
% \[a_{0}b_{1}x-a_{1}b_{0}x=c_{0}b_{1}-c_{1}b_{0}\Rightarrow 
%  (a_{0}b_{1}-a_{1}b_{0})x=c_{0}b_{1}-c_{1}b_{0}\]
% 
% \paragraph{Passo III} ricaviamo l'incognita~\(x\):
%  \[x=\frac{c_{0}b_{1}-c_{1}b_{0}}{a_{0}b_{1}-a_{1}b_{0}}\text{, con }
%  a_{0}b_{1}-a_{1}b_{0}\neq~0\]
% 
% \paragraph{Passo IV} per eliminare~\(x\) moltiplichiamo la prima 
% per~\(-a_{1}\) e 
% la seconda per~\(a_{0}\):
% \[\sistema{
%  -a_{0}a_{1}x+b_{0}a_{1}y=-c_{0}a_{1}\\
%  a_{1}a_{0}x+b_{1}a_{0}y=c_{1}a_{0}
% }\]
% 
% \paragraph{Passo V} sommiamo le due equazioni
% \[-b_{0}a_{1}y+a_{0}b_{1}y=-a_{1}c_{0}+a_{0}c_{1}\Rightarrow
%  (a_{0}b_{1}-a_{1}b_{0})y=a_{0}c_{1}-a_{1}c_{0}\]
% 
% \paragraph{Passo VI} ricaviamo l'incognita~\(y\):
% \[y=\frac{a_{0}c_{1}-a_{1}c_{0}}{a_{0}b_{1}-a_{1}b_{0}} \text{, con } 
% a_{0}b_{1}-a_{1}b_{0}\neq~0\]
% 
% La soluzione è
% \[\left(\frac{c_{0}b_{1}-b_{0}c_{1}}{a_{0}b_{1}-a_{1}b_{0}}; \quad
%   \frac{a_{0}c_{1}-a_{1}c_{0}}{a_{0}b_{1}-a_{1}b_{0}}\right)
%   \text{, con }a_{0}b_{1}-a_{1}b_{0}\neq~0\]
% 
% % \ovalbox{\risolvii \ref{ese:22.20}, \ref{ese:22.21}, \ref{ese:22.22}, 
% % \ref{ese:22.23}}
% 
% \subsection{Metodo di Cramer}
% 
% \begin{definizione}{}{}
% Si chiama \emph{matrice del sistema lineare} di due equazioni in due 
% incognite 
% la tabella
% \[\left[\begin{array}{cc}
% a_{0} & b_{0}\\
% a_{1} & b_{1}
% \end{array}\right]\]
% in cui sono sistemati i coefficienti delle incognite del sistema posto in 
% forma canonica; si~chiama \emph{determinante della matrice} il numero 
% reale
% 
% \[D=\left|\begin{array}{cc}
% a_{0} & b_{0} \\ 
% a_{1} & b_{1}
% \end{array}\right| = a_{0} \cdot b_{1} - a_{1} \cdot b_{0}\]
% 
% ad essa associato.
% \end{definizione}
% 
% Dalla generalizzazione del metodo di riduzione
% 
% \begin{equation*}
% \left(
% \dfrac{c_{0}b_{1}-c_{1}b_{0}}{a_{0}b_{1}-a_{1}b_{0}};
% \dfrac{a_{0}c_{1}-a_{1}c_{0}}{a_{0}b_{1}-a_{1}b_{0}}
% \right)
% \text{, con }a_{0}b_{1}-a_{1}b_{0}\neq~0
% \end{equation*}
% possiamo dedurre che:
% 
% Un \emph{sistema lineare} è \emph{determinato}, ammette cioè una
% sola coppia soluzione \emph{se il determinante della matrice del sistema è 
% diverso da zero}.
% 
% % \vspazio\ovalbox{\risolvii \ref{ese:22.24}, \ref{ese:22.25}}\vspazio
% 
% La regola di Cramer 
% \footnote{Dal nome del matematico svizzero Gabriel Cramer (1704-1752).} 
% ci permette di stabilire la coppia soluzione di un sistema lineare di due 
% equazioni in due incognite, costruendo e calcolando tre determinanti:
% 
% \begin{enumeratea}
% \item \(D\) il determinante della matrice del sistema:
% 
%  \[D=\left|\begin{array}{cc}
%   a_{0} & b_{0}\\ 
%   a_{1} & b_{1}
%  \end{array}\right|=a_{0}\cdot b_{1}-a_{1}\cdot b_{0}\]
%   
% \item \(D_{x}\) il determinante della matrice ottenuta sostituendo 
% in~\(D\) 
% agli
% elementi della prima colonna i termini noti.
% 
%  \[D_{x}=\left|\begin{array}{cc}
%   c_{0} & b_{0}\\ 
%   c_{1} & b_{1}
%  \end{array}\right| = c_{0} \cdot b_{1}-c_{1} \cdot b_{0}\]
%  
% \item \(D_{y}\) il determinante della matrice ottenuta sostituendo in D 
% agli elementi della seconda colonna i termini noti. 
%  \[D=\left|\begin{array}{cc}
%   a_{0} & c_{0}\\ 
%   a_{1} & c_{1}
%  \end{array}\right| = a_{0} \cdot c_{1}-a_{1} \cdot c_{0}\]
%   
% \end{enumeratea}
% 
% Se~\(D\neq~0\) il sistema è determinato e la coppia
% soluzione è
% \begin{equation*}
% x=\frac{D_{x}}{D};\, y=\frac{D_{y}}{D}
% \end{equation*}
% 
% % \begin{exrig}
%  \begin{esempio}{}{}
% \(\sistema{
%  2x+3y=3 \\
%  4x-3y=5
% }\)
% 
% Calcoliamo i determinanti.
% 
% % \begin{align*}
% \[D=\left|\begin{array}{cc}
%  2 & 3 \\ 
%  4 & -3
% \end{array}\right| = 2 \cdot (-3) - 4 \cdot 3=-6-12 = -18\]
% 
% Poiché \(D\neq~0\) il sistema è determinato.
% 
% \[D_{x}=\left|\begin{array}{cc}
%  3 & 3 \\
%  5 & -3
% \end{array}\right| = 3 \cdot (-3) - 5 \cdot 3 = -9 - 15 = -24\]
% 
% \[D_{y}=\left|\begin{array}{cc}
%  2 & 3 \\ 
%  4 & 5
% \end{array}\right| = 2 \cdot 5 - 4 \cdot 3 = 10 - 12 = -2\]
% 
% e quindi le soluzioni sono:
% 
% \[
%  x=\frac{D_{x}}{D}=\frac{-24}{-18}=\frac{4}{3} \quad
%  y=\frac{D_{y}}{D}=\frac{-2}{-18}=\frac{1}{9}
% \]
% % \end{align*}
%  \end{esempio}
% % \end{exrig}
% 
% % \ovalbox{\risolvii \ref{ese:22.26}, \ref{ese:22.27}, \ref{ese:22.28}, 
% % \ref{ese:22.29}, \ref{ese:22.30}}

% \subsection{Classificazione dei sistemi rispetto alle soluzioni}
% Dato un sistema in forma canonica
% \(\sistema{
%  a_{0}x+b_{0}y=c_{0}\\
%  a_{1}x+b_{1}y=c_{1} 
% } \) ricordando
% che:
% 
%  \[D=\left|\begin{array}{cc}
%   a_{0} & b_{0}\\ 
%   a_{1} & b_{1}
%  \end{array}\right|=a_{0}\cdot b_{1}-a_{1}\cdot b_{0}\]
%  
%  \[D_{x}=\left|\begin{array}{cc}
%   c_{0} & b_{0}\\ 
%   c_{1} & b_{1}
%  \end{array}\right| = c_{0} \cdot b_{1}-c_{1} \cdot b_{0}\]
% 
%  \[D_{y}=\left|\begin{array}{cc}
%   a_{0} & c_{0}\\ 
%   a_{1} & c_{1}
%  \end{array}\right| = a_{0} \cdot c_{1}-a_{1} \cdot c_{0}\]
% 
% \begin{itemize*}
% \item se D\({\neq}\)0 il sistema è \emph{determinato}, esiste una sola 
% coppia soluzione~\(x=\frac{D_{x}}{D} \quad y=\frac{D_{y}}{D}\)
% \item se~\(D=0\) si possono verificare due casi:
%  \begin{itemize*}
% \item 1\(\grado\) caso: se~\(D_{x}=0\) e~\(D_{y}=0\) il sistema è 
%  \emph{indeterminato}, ogni coppia di numeri reali che verifica 
% un'equazione, verifica anche l'altra;
% \item 2\(\grado\) caso: se~\(D_{x}\neq~0\) e~\(D_{y} \neq~0\) il sistema è 
%  \emph{impossibile}, non esiste alcuna coppia che soddisfa entrambi le 
%  equazioni e~\(\IS=\emptyset \).
%  \end{itemize*}
% \end{itemize*}
% 
% % \begin{exrig}
%  \begin{esempio}{}{}
% \(\sistema{2x-3y=1 \\4x-3y=2 }\)
% 
% \[D=\left|\begin{array}{cc}2&-3\\4&-3\end{array}\right|=2\cdot (-3)+3\cdot 
% (4)=-6+12=6\neq~0\]
% il sistema è determinato.
%  \end{esempio}
% 
%  \begin{esempio}{}{}
% \(\sistema{8x-6y=2 \\4x-3y=1 }\)
% 
% \[D=\left|\begin{array}{cc}{8}&{-6}\\{4}&{-3}\end{array}\right|=8\cdot 
% (-3)+6\cdot (4)=-24+24=0\]
% il sistema è indeterminato o impossibile.
% 
% \[D_{x}=\left|\begin{array}{cc}
%  {2}&{-6}\\
%  {1}&{-3}
% \end{array}\right|=2\cdot(-3)-(-6)\cdot 1=-6+6=0\]
% 
% \[D_{y}=\left|\begin{array}{cc}
% {8}&{2}\\
% {4}&{1}
% \end{array}\right|=8\cdot1-2\cdot 4=8-8=0\]
% 
% Il sistema è indeterminato.
%  \end{esempio}
% 
%  \begin{esempio}{}{}
% \(\sistema{8x-6y=1 \\4x-3y=2}\)
% 
% 
% \[D=\left|\begin{array}{cc}{8}&{-6}\\{4}&{-3}\end{array}
% \right|=8\cdot(-3)-4\cdot (-6)=-24+24=0;\]
% il sistema è indeterminato o impossibile.
% 
% \begin{align*}
% & 
% D_{x}=\left|\begin{array}{cc}{1}&{-6}\\{2}&{-3}\end{array}
% \right|=1\cdot(-3)-(-6)\cdot 2=-3+12=+9\\
% & 
% 
% 
% D_{y}=\left|\begin{array}{cc}{8}&{1}\\{4}&{2}\end{array}\right|
% =8\cdot2-1\cdot 
% 4=16-4=12.
% \end{align*}
% Il sistema è impossibile.
%  \end{esempio}
% % \end{exrig}
% Osserviamo che se~\(D=0\) si ha
% \[a_{0}\cdot b_{1}-b_{0}\cdot a_{1}=0\Rightarrow
%   a_{0}\cdot b_{1}=b_{0}\cdot a_{1}\Rightarrow
%   \frac{a_{0}}{a_{1}}=\frac{b_{0}}{b_{1}}\]
% Ciò significa che, se i coefficienti delle incognite della prima equazione 
% sono 
% proporzionali
% ai coefficienti delle incognite della seconda equazione allora il
% sistema è indeterminato o impossibile.
% 
% In particolare, se poi~\(D_{x}=0\) si ha
% \[c_{0}\cdot b_{1}-b_{0}\cdot c_{1}=0\Rightarrow
%   c_{0}\cdot b_{1}=b_{0}\cdot c_{1}\Rightarrow 
%   \frac{c_{0}}{c_{1}}=\frac{b_{0}}{b_{1}}\]
% Quindi se anche i termini noti delle due equazioni sono nella stessa 
% proporzione,
% cioè se
% \[\frac{a_{0}}{a_{1}}=\frac{b_{0}}{b_{1}}=\frac{c_{0}}{c_{1}}\] il
% sistema è indeterminato.
% 
% Se invece~\(D_{x}{\neq}0\), cioè
% \[\frac{c_{0}}{c_{1}}\neq\frac{b_{0}}{b_{1}}\] il sistema è impossibile.
% 
% % \vspazio\ovalbox{\risolvii \ref{ese:22.31}, \ref{ese:22.32}, 
% \ref{ese:22.33}, 
% % \ref{ese:22.34}, \ref{ese:22.35}, \ref{ese:22.36}, \ref{ese:22.37}, 
% % \ref{ese:22.38}}

\section{Interpretazione grafica}
Il problema della ricerca dell'Insieme Soluzione di
un'equazione lineare in due incognite (vedi sezione: \ref{sec:sist_eqdue}) 
ci ha condotto a un proficuo collegamento tra concetti algebrici e 
concetti geometrici; in particolare abbiamo visto che:

\begin{center}
 \begin{tabularx}{.9\textwidth}{XX}
\toprule
 Concetto algebrico & Concetto geometrico\\
 \midrule
Coppia ordinata di numeri reali & Punto del piano dotato di riferimento 
cartesiano\\
Equazione lineare & Retta \\
Coppia soluzione dell'equazione& Punto della retta di equazione \\
\(ax+by+c=0\) & \(y=-{\frac{a}{b}}x-\frac{c}{b}\)\\
\bottomrule
 \end{tabularx}

\end{center}
Vedremo ora un'importante interpretazione geometrica dei sistemi lineari:

\begin{definizione}{}{}
Se interpretiamo le due equazioni lineari come rette, la soluzione è data 
dalle coordinate della loro intersezione.
\end{definizione}

\begin{esempio}{}{}
Trova la soluzione del sistema:
\(\sistema{+x+y=5\\-x+3y=3}\)

\vspace{1em}
\begin{minipage}{.48\textwidth}
Riscriviamo le due equazioni esplicitando la variabile \(y\):
\(\sistema{y=-x+5\\y=\dfrac{1}{3}x +1}\)

Ora (usando il metodo rapido) disegniamo le due rette e individuiamo la loro 
intersezione.

La soluzione del sistema è la coppia ordinata \(\coppia{3}{2}\) formata dalle 
coordinate dell'intersezione.

Si può verificarlo facilmente:\\ 
\(\sistema{+3+2=5\\-3+3 \cdot 2=3}\)
\end{minipage}
\hfill
\begin{minipage}{.48\textwidth}
\begin{center} \intersezionediduerette \end{center}
\end{minipage}

\end{esempio}

% \begin{problema}{}{}
% Determina due numeri reali di cui si sa che la loro somma è~6 e il
% doppio del primo aumentato della metà del secondo è ancora~6.
% \end{problema}
% 
%  \begin{soluzione}
% Indichiamo con~\(x\) e~\(y\) i due numeri incogniti; il problema si 
% formalizza con due 
% equazioni:~\(x+y=6\) \quad e \quad \(2x+\frac{1}{2}y=6\).\\
% Dobbiamo individuare una coppia di numeri reali che sia soluzione
% dell'una e dell'altra equazione.
% 
% \paragraph{Il punto di vista algebrico}
% La coppia di numeri reali~\(x\) e~\(y\) che risolve il problema è quella 
% che risolve il sistema
% \[\sistema{x+y=6\\2x+\dfrac{1}{2}y=6 
% }\]
% Applicando uno qualunque dei metodi algebrici esposti si ottiene~\(x=2\) 
% e~\(y=4\).
% 
% \paragraph{Il punto di vista geometrico}
% 
% Il problema si può spostare in ambiente geometrico: la coppia soluzione 
% rappresenta un punto che appartiene sia alla retta rappresentata 
% dalla prima equazione sia alla retta rappresentata dalla seconda 
% equazione, quindi rappresenta il punto di intersezione delle due rette.
% 
% Si rappresenta nel riferimento cartesiano ortogonale il sistema.
% La retta~\(a\) è quella di equazione~\(x+y=6\), che passa per i
% punti~\(\coppia{6}{0}\) e~\(\coppia{0}{6}\).
% 
% La retta~\(b\) è quella di equazione~\(2x+\frac{1}{2}y=6\), che
% passa per i punti~\(\coppia{3}{0}\) e~\(\coppia{0}{12}\).
% 
% Il punto~\(A\coppia{2}{4})\) è il punto di intersezione delle due rette, 
% le sue coordinate formano la coppia soluzione del sistema e di conseguenza
% sono i due numeri che stiamo cercando nel problema.
% 
% \begin{center}
% \begin{inaccessibleblock}[Due rette che si incontrano nel punto~(2; 4)]
%  \input{\folder lbr/fig005_car.pgf}
% \end{inaccessibleblock}
% \end{center}
%  \end{soluzione}

\subsection{Interpretazione geometrica e sistemi non determinati}
L'interpretazione geometrica ci permette di capire meglio quando un 
sistema è determinato, indeterminato o impossibile.

\paragraph{Sistema impossibile}

Partiamo da un esempio e disegniamo le rette associate al sistema.

 % \begin{exrig}
 \begin{esempio}{}{}
\(\sistema{2x-3y=-6 \\x+y+6=5(x-y)}\)
\vspace{1em}

Per prima cosa scriviamo il sistema in forma canonica:
\[
\sistema{2x-3y=-6 \\x+y+6=5(x-y)}\Rightarrow
\sistema{2x-3y=-6 \\x+y+6=5x-5y}\Rightarrow
\sistema{+2x-3y=-6 \\-4x+6y=-6}
\]
Ora esplicitiamo le \(y\) in modo da semplificarci il disegno delle rette e 
disegniamole:

\begin{minipage}{.48\textwidth}
\[\sistema{y=\dfrac{2}{3}x+2 \\[.5em] y=\dfrac{2}{3}x-1}\]

Le due rette sono parallele, poiché hanno lo stesso coefficiente angolare. 

Non esiste quindi l'intersezione delle rette e il sistema non ha soluzione.
\end{minipage}
\hfill
\begin{minipage}{.48\textwidth}
\begin{center}
\disegno{
  \funzionicolorate{-4}{+4}{-4}{+4}
                   {{2./3*x+2}/red!50!black, {2./3*x-1}/green!50!black}
  }
\end{center}
\end{minipage}

\end{esempio}

\paragraph{Sistema indeterminato}

Anche qui iniziamo da un esempio e disegniamo le rette associate al sistema.

 \begin{esempio}{}{}
\(\sistema{3x+4y=-6 \\9x+12y=-18}\)
\vspace{1em}

Ora esplicitiamo le \(y\) in modo da semplificarci il disegno delle rette e 
disegniamole:

\begin{minipage}{.48\textwidth}
\[\sistema{y=-\dfrac{3}{4}x-\dfrac{3}{2} \\[.5em] 
           y=-\dfrac{3}{4}x-\dfrac{3}{2}}\]

Le due rette non sono solo parallele, poiché hanno lo stesso coefficiente 
angolare, ma sono anche coincidenti. 

Ogni punto di una è anche punto dell'altra e il sistema ha quindi infinite 
soluzioni.
\end{minipage}
\hfill
\begin{minipage}{.48\textwidth}
\begin{center}
\disegno{
  \funzionicolorate{-4}{+4}{-4}{+4}{{3./4*x-3./2}/red!50!blue}
}
\end{center}
\end{minipage}

\end{esempio}

\subsection{Coefficienti e sistemi non determinati}

Consideriamo un generico sistema:
\[\sistema{
 a_{0}x+b_{0}y=c_{0} \\
 a_{1}x+b_{1}y=c_{1} 
}\]
Se i coefficienti delle incognite sono in proporzione:
\[\frac{a_{0}}{a_{1}} = \frac{b_{0}}{b_{1}} \sLRarrow
{a_{0}} \cdot {b_{1}} = \frac{b_{0}} \cdot {a_{1}}\]
Allora le rette corrispondenti sono parallele: il sistema non è determinato.

Se anche i termini noti sono nella stessa proporzione:
\[\frac{a_{0}}{a_{1}} = \frac{b_{0}}{b_{1}} = \frac{c_{0}}{c_{1}}\]
Allora il sistema è indeterminato, altrimenti è impossibile.
% 
% \[\frac{a_{0}}{a_{1}} = \frac{b_{0}}{b_{1}},~,~c_{0},~,~,~c_{1}\text{ 
% numeri}\]
% \paragraph{Sistema indeterminato}
% 
% Anche qui iniziamo da un esempio e disegniamo le rette associate al 
% sistema.
% 
%  \begin{esempio}{}{}
% \(\sistema{3x+4y=-6 \\9x+12y=-18}\)
% \vspace{1em}
% 
% Ora esplicitiamo le \(y\) in modo da semplificarci il disegno delle 
% rette e disegniamole:
% 
% \begin{minipage}{.48\textwidth}
% \[\sistema{y=-\dfrac{3}{4}x-\dfrac{3}{2} \\[.5em] 
%            y=-\dfrac{3}{4}x-\dfrac{3}{2}}\]
% 
% Le due rette non sono solo parallele, poiché hanno lo stesso coefficiente 
% angolare, ma sono anche coincidenti. 
% 
% Ogni punto di una è anche punto dell'altra e il sistema ha quindi infinite 
% soluzioni.
% \end{minipage}
% \hfill
% \begin{minipage}{.48\textwidth}
% \begin{center}
% \disegno{
%   \funzionicolorate{-4}{+4}{-4}{+4}{{3./4*x-3./2}/red!50!blue}
% }
% \end{center}
% \end{minipage}
% 
% \end{esempio}
% 
% 
% Si può notare che il sistema ha i coefficienti delle incognite in
% proporzione:
% \[\frac{a}{a_{1}}=\frac{2}{-4}=-{\frac{1}{2}}; \quad 
%   \frac{b}{b_{1}}=\frac{-{3}}{+6}=-{\frac{1}{2}}\]
% mentre i termini noti non sono nella stessa 
% proporzione~\(\frac{c}{c_{1}}=\frac{7}{-1}\)
% quindi il sistema è impossibile:~\(\IS=\emptyset \).
% 
% \paragraph{Il punto di vista geometrico}
% 
% Determiniamo le equazioni esplicite delle rette rappresentate dalle due
% equazioni lineari del sistema assegnato. Si ha:
% \[\sistema{
%  y=\dfrac{2}{3}x-\dfrac{7}{3}\\
%  y=\dfrac{2}{3}x-1
% }\]
% 
% Le due rette (figura~\ref{fig:23.1}) hanno lo stesso coefficiente 
% angolare, il coefficiente della~\(x\) e quindi hanno la stessa 
% inclinazione, pertanto sono parallele.
% Non hanno quindi nessun punto di 
% intersezione~\(r_{1}\cap r_{2}=\emptyset \), 
% il sistema è impossibile:~\(\IS=\emptyset \).
% 
% % \newpage
% 
%  \begin{esempio}{}{}
% \(\sistema{
%      2x+3y+1=0\\
%      y+\dfrac{1}{3}=-{\dfrac{2}{3}}x
% }\)
% 
%  \paragraph{Il punto di vista algebrico}
%  Scriviamo in forma canonica il 
% sistema~\(\sistema{2x+3y=-1\\2x+3y=-1}\).
% 
%  Osserviamo che sono due equazioni identiche, pertanto il rapporto tra i
% coefficienti delle incognite e il rapporto tra i termini noti è
% sempre~1. Il sistema è indeterminato. D'altra parte,
% se le due equazioni sono identiche significa che tutte le infinite
% coppie~\((x, y)\) che rendono vera la prima equazione, verificano anche la
% seconda.
% 
% \paragraph{Il punto di vista geometrico}
% Rappresentiamo nel riferimento cartesiano ortogonale (figura~\ref{fig:
% 23.2}) le due rette aventi come equazioni le equazioni del sistema. 
% È semplice rendersi conto che le due rette coincidono; tutti i punti di una
% coincidono con tutti i punti dell'altra:~\(r_{1}\cap r_{2}=r_{1}=r_{2}\).
% 
% \begin{figure}[htbp]
% \begin{minipage}{0.5\textwidth}
% \centering
% \begin{inaccessibleblock}[Esempio~23.17: due rette parallele]
% \input{\folder lbr/fig006_car.pgf}
% \caption{Esempio~23.17}\label{fig:23.1}
% \end{inaccessibleblock}
% \end{minipage}\hfill
% \begin{minipage}{0.5\textwidth}
% \centering
% \begin{inaccessibleblock}[Esempio~23.18: due rette coincidenti]
% \input{\folder lbr/fig007_car.pgf}
% \caption{Esempio~23.18}\label{fig:23.2}
% \end{inaccessibleblock}
% \end{minipage}
% \end{figure}
%  \end{esempio}
%  % \end{exrig}
% 
%  \vspace{-24pt}

% \ovalbox{\risolvii \ref{ese:22.39}, \ref{ese:22.40}, \ref{ese:22.41}, 
% \ref{ese:22.42}, \ref{ese:22.43}}

% \section{Sistemi fratti}
% \label{sec:22_fratti}
% 
% Nel seguente sistema
% 
% \(\sistema{\frac{2}{x+1}-\frac{3}{y-2}=\frac{2x-5y+4}
% {xy+y-2-2x} \\3y+2(x-y-1)=5x-8(-x-2y+1)}\)
% di due equazioni in due incognite, la prima equazione presenta le
% incognite anche al denominatore.
% 
% \begin{definizione}{}{}
% Si chiama \emph{sistema fratto o frazionario} il sistema in cui almeno in 
% una delle equazioni che lo
% compongono compare l'incognita al denominatore.
% \end{definizione}
% 
% Poiché risolvere un sistema significa determinare tutte le coppie
% ordinate che verificano entrambe le equazioni, nel sistema fratto
% dovremo innanzi tutto definire il Dominio o Insieme di Definizione nel
% quale individuare le coppie soluzioni.
% 
% \begin{definizione}{}{}
% Si chiama \emph{Dominio}~(\(D\)) o \emph{Insieme di Definizione}~(\(ID\)) 
% del sistema fratto,
% l'insieme delle coppie ordinate che rendono diverso da zero i denominatori 
% che compaiono nelle equazioni.
% \end{definizione}
% 
% % \begin{exrig}
%  \begin{esempio}{}{}
% 
% 
% 
% \(\sistema{\dfrac{2}{x+1}-\dfrac{3}{y-2} 
% =\dfrac{2x-5y+4} {
% xy+y-2-2x}\\3y+2(x-y-1)=5x-8(-x-2y+1)}\)
% 
% \paragraph{Passo I} Scomponiamo i denominatori nella prima equazione
% per determinare il~\(\mcm\).
% 
% 
% 
% \[\sistema{{\dfrac{2}{x+1}-\dfrac{3}{y-2}
% =\dfrac{2x-5y+4
% }{(x+1)(y-2)}}\\{3y+2(x-y-1)=5x-8(-x-2y+1)}}\Rightarrow\mcm=(x+1)(y-2)\]
% 
% \paragraph{Passo II} Poniamo le Condizioni di Esistenza da cui 
% determineremo il Dominio del
% sistema:
% \[\CE:\sistema{
%    x\neq -1\\y\neq~2
%    }\Rightarrow D=\IS=\left\{\coppia{x}{y}\in 
% \R\times\R\left|x\right.\neq -1\text{ e }y\neq~2\right\}.\]
% 
% \paragraph{Passo III} Riduciamo allo stesso denominatore la prima
% equazione, svolgiamo i calcoli nella seconda per ottenere la forma
% 
% canonica:~\(\sistema{{-5x+7y=11}\\{11x+15y=6}}\)
% 
% \paragraph{Passo IV} Risolviamo il sistema e otteniamo la coppia
% soluzione~\(\coppia{-\frac{123}{152}}{\frac{151}{152}}\) che è
% accettabile.
%  \end{esempio}
% 
%   \begin{esempio}{}{}
% 
% 
% 
% \(\sistema{{\dfrac{3x+y-1}{x}=3}\\{\dfrac{2x+3y}
% {y-1} =7}
% }\)
% 
% \paragraph{Passo I} Per la prima equazione si ha~\(\mcm=x\) per la 
% seconda~\(\mcm=y-1\).
% 
% \paragraph{Passo II} Poniamo le Condizioni di Esistenza da cui 
% determineremo 
% il Dominio:
% \[\CE:\sistema{
%    x\neq~0\\y\neq~1
%    }\rightarrow D=\IS=\left\{\coppia{x}{y}\in \R\times\R 
% |x\neq~0\text{ e }y\neq~1\right\}.\]
% 
% \paragraph{Passo III} Riduciamo allo stesso denominatore sia la prima che 
% la seconda equazione:
% \(\sistema{{3x+y-1=3x}\\{2x+3y=7y-7}}\)
% 
% \paragraph{Passo IV} Determiniamo la forma canonica:
% \(\sistema{{y-1=0}\\{2x-4y=-7}}\)
% 
% \paragraph{Passo V} Determiniamo con un qualunque metodo la coppia
% soluzione:~\(\coppia{-\frac{3}{2}}{1}\) che non accettabile
% poiché contraddice la~\(\CE\) e quindi non appartiene al dominio. Il
% sistema assegnato è quindi impossibile~\(\IS=\emptyset \).
%  \end{esempio}
% % \end{exrig}
% 
% % \ovalbox{\risolvii \ref{ese:22.44}, \ref{ese:22.45}, \ref{ese:22.46}, 
% \ref{ese:22.47}, \ref{ese:22.48}, \ref{ese:22.49}}
% 
% \section{Sistemi letterali}
% \label{sec:22_letterali}
% 
%  \begin{definizione}{}{}
%  Si chiama \emph{sistema letterale} il sistema in cui
% oltre alle incognite, solitamente indicate con~\(x\) e~\(y\), compaiono 
% altre lettere dette parametri.
%  \end{definizione}
% 
% 
% Distinguiamo tre casi distinti di discussione.
% 
% \subsection*{Le equazioni sono lineari e il parametro si trova solo al 
% numeratore}
% 
% % \begin{exrig}\vspace{1.10ex}
%  \begin{esempio}{}{}
%  \(\sistema{{2ax-(a-1)y=0}\\{-2x+3y=a}}\)
% 
% 
% È un sistema letterale in quanto, reso in forma
% canonica, presenta un parametro nei suoi coefficienti. Esso è
% lineare, pertanto la coppia soluzione, se esiste, dipenderà dal
% valore del parametro.
% 
% Per \emph{discussione del sistema letterale} s'intende
% l'analisi e la ricerca dei valori che attribuiti al
% parametro rendono il sistema determinato (in tal caso si determina la
% soluzione) ma anche scartare i valori del parametro per cui il sistema
% è impossibile o indeterminato.
% Per discutere il sistema usiamo il metodo di Cramer.
% 
% \paragraph{Passo I} Calcoliamo il determinante del sistema:
% \[D=\left|\begin{array}{cc}{2a}&{-(a-1)}\\
%   {-2}&{3}\end{array}\right|=4a+2\]
% 
% \paragraph{Passo II} Determiniamo il valore del parametro che
% rende~\(D\) diverso da 
% zero:~\(4a+2\neq~0\Rightarrow a\neq~0-\frac{1}{2}\). 
% Se~\(a\neq -{\frac{1}{2}}\) il sistema è
% determinato.
% 
% \paragraph{Passo III} Calcoliamo i determinanti~\(D_{x}\)
% e~\(D_{y}\) per trovare la coppia soluzione.
% \[D_{x}=\left|\begin{array}{cc}{0}&{-(a-1)}\\{a}&{3}\end{array}\right|
% =a\cdot (a-1);\quad
% D_{y}=\left|\begin{array}{cc}{2a}&{0}\\{-2}&{a}\end{array}\right|=
% 2a^{2}.\]
% Quindi~\(x=\frac{a\cdot (a-1)}{4a+2}\) e~\(y=\frac{2a^{2}}{4a+2}\).
% 
% \paragraph{Passo IV} Il determinante è nullo se~\(a=-{\frac{1}{2}}\) 
% poiché per % questo valore di~\(a\) i
% determinanti~\(D_{x}\) e~\(D_{y}\) sono diversi da zero si ha che 
% per~\(a=-{\frac{1}{2}}\) il sistema
% è impossibile.
% 
% Riassumendo si ha:
% \begin{center}
%  \begin{tabular}{lll}
% \toprule
% Condizioni sul parametro & Insieme Soluzione & Sistema\\
% \midrule
% \(a\neq -{\frac{1}{2}}\) & \(\left(\frac{a\cdot 
% (a-1)}{4a+2};\frac{2a^{2}}{4a+2}\right)\) & determinato\\
% \(a=-{\frac{1}{2}}\) & \(\emptyset \) & impossibile\\
% \bottomrule
% \end{tabular}
% \end{center}
%  \end{esempio}
% % \end{exrig}
% 
% \subsection*{Il parametro compare al denominatore in almeno una equazione 
% del sistema}
% 
% % \begin{exrig}\vspace{1.10ex}
% \begin{esempio}{}{}
%  \(\sistema{{\dfrac{y+a}{3}-\dfrac{a-x}{a-1}=a}\\
%  {\dfrac{x+2a}{a}-3=\dfrac{y}{2}-a}}\)
% 
% Il sistema non è fratto pur presentando termini frazionari nelle sue
% equazioni; la presenza del parametro al denominatore ci obbliga ad
% escludere dall'insieme~\(\R\) quei valori che annullano il
% denominatore.
% Se~\(a=1\) oppure~\(a=0\) ciascuna equazione del sistema è priva di
% significato, pertanto lo è anche il sistema.
% Con le condizioni di esistenza~\(\CE: a\neq~1\) e~\(a\neq~0\)
% possiamo ridurre allo stesso denominatore ciascuna equazione e condurre
% il sistema alla forma
% 
% 
% canonica:~\(\sistema{{3x+(a-1)y=2a^{2}+a}\\{2x-ay=2a-2a^{2}}
%                    }\)
% 
% 
% \paragraph{Passo I} Calcoliamo il determinante del sistema:
% \(D=\left|\begin{array}{cc}{3}&{a-1}\\{2}&{-a}\end{array}\right|=2-5a.\)
% 
% \paragraph{Passo II} Determiniamo il valore del parametro che
% rende~\(D\) diverso da zero:~\(2-5a\neq~0\Rightarrow a\neq \frac{2}{5}\)
% Se~\(a\neq \frac{2}{5}\) il sistema è determinato.
% 
% \paragraph{Passo III} Calcoliamo i determinanti~\(D_{x}\)
% e~\(D_{y}\) per trovare la coppia soluzione
% 
% 
% \[D_{x}=\left|\begin{array}{cc}{2a^{2}+a}&{a-1}\\{2a-2a^{2}}&{-a}
% \end{array} \right|=a\cdot (2a-5);\quad
% 
% D_{y}=\left|\begin{array}{cc}{3}&{2a^{2}+a}\\{2}&{2a-2a^{2}}\end{array}
% \right|=2a\cdot (2-5a).\]
% Quindi~\(x=\frac{a\cdot (2-5a)}{2-5a}\) 
% e~\(y=\frac{2a\cdot (2-5a)}{2-5a}\) 
% e semplificando~\((a;2a)\).
% 
% \paragraph{Passo IV} Il determinante è nullo se
% \(a=\frac{2}{5}\) poiché anche i determinanti~\(D_{x}\) e~\(D_{y}\) si 
% annullano si ha per~\(a=\frac{2}{5}\) sistema indeterminato.
% 
% Riassumendo si ha:
% 
% \begin{center}
% \begin{tabular}{lll}
% \toprule
% Condizioni sul parametro & Insieme Soluzione & Sistema\\
% \midrule
% \(a=0\vee a=1\) & \(\emptyset \) & privo di significato\\
% \(a\neq \frac{2}{5}\) e~\(a\neq~1\) e~\(a\neq~0\) & 
% \(\left\{(a;2a)\right\}\) & 
% determinato\\
% \(a=\frac{2}{5}\) & 
% \(\{\forall\coppia{x}{y}\in\R^2/3x-\frac{3}{5}y=\frac{18}{25}\}
% \) % & 
% indeterminato\\
% \bottomrule
% \end{tabular}
% \end{center}
% \end{esempio}
% % \end{exrig}
% \subsection*{Il sistema è frazionario}
% 
% % \begin{exrig}
%  \vspace{1.10ex}
%  \begin{esempio}{}{}
% 
% \(\sistema{\frac{y-a}{x}=\frac{2}{a}\\{x+y=1}}\)
% 
% Il sistema letterale è fratto e nel denominatore oltre al parametro
% compare l'incognita~\(x\). Se~\(a=0\) la prima equazione, e di conseguenza 
% tutto 
% il sistema, è
% privo di significato. Per poter procedere alla ricerca
% dell'Insieme Soluzione poniamo sul
% parametro la condizione di esistenza:
% \begin{equation}
% \label{eq:23.1}
% \CE: a\neq~0.
% \end{equation}
% 
% Essendo fratto dobbiamo anche stabilire il Dominio del sistema:
% \begin{equation}
%  \label{eq:23.2}
% D=\{\coppia{x}{y}\in \R\times \R | x\neq~0\}.
%  \end{equation}
% 
% 
% \paragraph{Passo I} Portiamo nella forma canonica:
% \(\sistema{-2x+ay=a^{2}\\x+y=1}\text{ con 
% }a\neq~0\text{ e }x\neq~0\).
% \paragraph{Passo II} Calcoliamo il determinante del sistema:
% \(D=\left|\begin{array}{cc}{-2}&{a}\\{1}&{1}\end{array}\right|=-2-a=-(2+a)
% \). 
% \paragraph{Passo III} Determiniamo il valore del parametro che
% rende~\(D\) diverso da zero:~\(-2-a\neq~0\Rightarrow a\neq -2\).
% Se~\(a\neq -2\) il sistema è determinato.
% \paragraph{Passo IV} calcoliamo i determinanti~\(D_{x}\)
% e~\(D_{y}\) per trovare la coppia soluzione
% 
% \[D_{x}=\left|\begin{array}{cc}a^{2}&{a}\\{1}&{1}\end{array}\right|=a\cdot 
% (a-1);\quad
% D_{y}=\left|\begin{array}{cc}-2&a^{2}\\1& 
% 1\end{array}\right|=-2-a^{2}=-(2+a^{2}).\]
% Quindi~\(x=-{\frac{a\cdot (a-1)}{2+a}}\) e~\(y=\frac{a^{2}+2}{2+a}\) è la 
% coppia soluzione accettabile
% se~\(x=-{\frac{a\cdot (a-1)}{2+a}}\neq~0\) per quanto stabilito 
% in~\ref{eq:23.2}; essendo~\(a\neq~0\)
% per la~\ref{eq:23.1} la coppia soluzione è accettabile se~\(a\neq~1\).
% 
% \paragraph{Passo V} il determinante~\(D\) è nullo se~\(a=-2\)
% essendo i determinanti~\(D_{x}\) e~\(D_{y}\) diversi
% da zero si ha:
% se~\(a=-2\) il sistema è impossibile.
% Riassumendo si ha:
% 
% \begin{center}
% \begin{tabularx}{.9\textwidth}{XXll}
% \toprule
% Parametro & Incognite & Insieme Soluzione & Sistema\\
% \midrule
%  & \(x\neq~0\) & & \\
%  \(a=0\) & & & privo di significato\\
%  \(a \neq2,a\neq0\) & & \(\left(-{\frac{a\cdot 
% (a-1)}{2+a}};\frac{a^{2}+2}{2+a}\right)\) & determinato\\
%  \(a\neq -2\) e~\(a\neq~0\) e~\(a\neq~1\) & & accettabile & \\
% \(a=-2\) & & & impossibile\\
% \bottomrule
% \end{tabularx}
% \end{center}
%  \end{esempio}
% % \end{exrig}
% 
% % \ovalbox{\risolvii \ref{ese:22.50}, \ref{ese:22.51}, \ref{ese:22.52}, 
% \ref{ese:22.53}, \ref{ese:22.54}, \ref{ese:22.55}, \ref{ese:22.56}}

\section{Sistemi lineari di tre equazioni in tre incognite}
\label{sec:compl1_sistemitreeq}

Abbiamo risolto i sistemi di due equazioni in due incognite combinando le 
equazioni in modo da ottenerne una con una sola incognita.
Lo stesso meccanismo può essere usato per risolvere sistemi lineari con un 
qualunque numero di equazioni.
Anche per i sistemi di più di due equazioni possiamo utilizzare i metodi 
presentati precedentemente.

\subsection{Sostituzione in un sistema di tre equazioni}

\begin{problema}{}{}
Determinare tre numeri reali~\(x, y, z\) (nell'ordine) tali
che il doppio del primo uguagli l'opposto del secondo,
il triplo del terzo sia uguale al primo aumentato~\(4\)
e che la somma del secondo con il terzo sia inferiore al primo di~\(12\) 
unità.
\end{problema}

\begin{soluzione}
Formalizziamo le condizioni espresse nel testo attraverso equazioni
lineari:

\begin{enumerate}[nosep]
\item il doppio del primo uguagli l'opposto del secondo:~\(2x=-y\)
\item il triplo del terzo sia uguale al primo aumentato~\(4\):~\(3z=x+4\)
\item la somma del secondo con il terzo sia inferiore al primo di~\(12\) 
unità:~\(y+z=x-12\).
\end{enumerate}

Le tre condizioni devono essere vere contemporaneamente, quindi i tre
numeri sono la terna soluzione del sistema di primo grado di tre 
equazioni in tre incognite:
\[\sistema{
  2x=-y\\
  3z=x+4\\
  y+z=x-12
}\]

Per prima cosa scriviamo il sistema in forma normale:
\[\sistema{
  2x+y=0\\
  -x+3z=4\\
  -x+y+z=-12
}\]

Possiamo ora ricavare la~\(y\) dalla prima equazione e sostituirla nelle 
altre 
due:
\[\sistema{
  y=-2x\\
  -x+3z=4\\
  -x -2x+z=-12
}
\Rightarrow
\sistema{
  y=-2x\\
  -x+3z=+4\\
  -3x+z=-12
}\]

In questo modo abbiamo ottenuto un sottosistema formato da due equazioni 
in due incognite:
\[\sistema{
  -x+3z=+4\\
  -3x+z=-12
}\]

Possiamo risolverlo facilmente con il metodo di riduzione:

\hspace{-20mm}
\begin{minipage}{.48\textwidth}
\[\stackrel{
  \begin{array}{l}\cdot \tonda{-3}\\ \cdot \tonda{+1}\end{array}
  \underline{\sistema{+3x-9z=-12\\-3x+~~z=-12}}}
  {\qquad \qquad \qquad \qquad \qquad 
   //  -8z  =-24 \quad \sRarrow z=3}\]
\end{minipage}
\begin{minipage}{.48\textwidth}
\[\stackrel{
  \begin{array}{l}\cdot \tonda{+1}\\ \cdot \tonda{-3}\end{array}
  \underline{\sistema{-x+3z=+4\\+9x-3z=+36}}}
  {\qquad \qquad \qquad \qquad 
   ~~~~~ +8x ~~ // ~~ =+40 \quad \sRarrow x=5}\]
\end{minipage}
\vspace{.5em}

Risolto così il sottosistema di due equazioni, possiamo risolvere il 
sistema di partenza:
\[\sistema{
  x=5\\
  y=-2x=-10\\
  z=3
}\]

\end{soluzione}

\subsection{Riduzione in un sistema di tre equazioni}

Possiamo utilizzare il metodo di riduzione anche per  passare dal sistema 
completo a un suo sottosistema. Lo vediamo con un esempio.

% \begin{exrig}
 \begin{esempio}{}{}
\(\sistema{3x+y-z=7\\x+3y+z=5\\x+y-3z=3}\)
\vspace{.5em}

\noindent Sommando le prime due equazioni ne otteniamo una senza l'incognita 
\(z\):
\[4x+4y=12\]
Moltiplicando la seconda equazione per~\(3\) e sommandola con la 
terza otteniamo:
\[3(x+3y+z)+x+y=3\cdot 5+3 \sRarrow 4x+10y=18\]
Costruiamo il sistema di queste due equazioni
nelle sole due incognite~\(x\) e~\(y\):
\[\sistema{x+y=3\\2x+5y=9}\]

\hspace{-20mm}
\begin{minipage}{.48\textwidth}
\[\stackrel{
  \begin{array}{l} \cdot \tonda{-2}\\ ~ \end{array}
  \underline{\sistema{-2x-2y=-6\\+2x+5y=+9}}}
  {\qquad \qquad \qquad \qquad 
   ~~~~~~~ // ~~ +3y  =+3 \quad \sRarrow y=1}\]
\end{minipage}
\begin{minipage}{.48\textwidth}
\[\stackrel{
  \begin{array}{l} \cdot \tonda{-5}\\ ~ \end{array}
  \underline{\sistema{-5x-5y=-15\\+2x+5y=+9}}}
  {\qquad \qquad \qquad \qquad 
   ~~~~ -3x ~~~ // ~~ =-6 \quad \sRarrow x=2}\]
\end{minipage}
\vspace{.5em}

Individuati così i valori di \(x\) e \(y\), possiamo sostituirli in una 
delle equazioni del sistema iniziale e trovare anche il valore di \(z\):
% \[\sistema{x=2\\y=1\\x+y-3z=3} \sRarrow 
% \sistema{x=2\\y=1\\2+1-3z=3} \sRarrow 
% \sistema{x=2\\y=1\\-3z=0} \sRarrow 
% \sistema{x=2\\y=1\\z=0} \]
\[x+y-3z=3 \sRarrow 2+1-3z=3 \sRarrow -3z=0 \sRarrow z=0\]
La terna soluzione del sistema assegnato è~\((2;~1;~0)\).
 \end{esempio}
% \end{exrig}

% \ovalbox{\risolvii \ref{ese:22.57}, \ref{ese:22.58}, \ref{ese:22.59}, 
% \ref{ese:22.60}, \ref{ese:22.61}, \ref{ese:22.62}, \ref{ese:22.63}}
% 
% \section{Sistemi da risolvere con sostituzioni delle variabili}
% \label{sec:22_sostituzione}
% 
% Alcuni sistemi possono essere ricondotti a sistemi lineari per mezzo di
% sostituzioni nelle variabili.
% 
% % \begin{exrig}
%  \begin{esempio}{}{}
% \(\sistema{\dfrac{1}{x}+\dfrac{2}{y}=3\\\dfrac{2}
% {x} -\dfrac{4}{y}=-1 }\)
% 
% Con la seguente sostituzione di variabili
% \begin{align}
%  \label{eq:23.3}
% \sistema{
%     u=\dfrac{1}{x}\\
%     v=\dfrac{1}{y}
% }
% \end{align}
% 
% il sistema diventa
%  \[\sistema{u+2v=3 \\2u-4v=-1 }\]
% 
% Per risolverlo possiamo moltiplicare per~2 la prima equazione:
% \[\sistema{2u+4v=6 \\2u-4v=-1
% }\]
% Sommando membro a membro abbiamo~\(4u=5\) dalla
% quale possiamo determinare~\(u=\frac{5}{4}\)
% 
% Per ricavare l'incognita~\(v\) moltiplichiamo la prima equazione per \
% (-2\), otteniamo
% \[\sistema{-2u-4v=-6 \\2u-4v=-1 }\]
% Sommando membro a membro abbiamo
% \[-8v=-7\Rightarrow v=\frac{7}{8}.\]
% 
% Avendo trovato i valori delle incognite~\(u\) e~\(v\) possiamo ricavare~\
% (x\) e~\(y\) 
% sostituendo con i valori trovati nella~\ref{eq:23.3}:
% \[\sistema{\dfrac{5}{4}=\dfrac{1}{x}\\\dfrac{7}{8}
% =\dfrac{1}{y}}\Rightarrow
% \sistema{x=\dfrac{4}{5}\\y=\dfrac{8}{7}}\]
%  \end{esempio}
% % \end{exrig}

% \ovalbox{\risolvii \ref{ese:22.64}, \ref{ese:22.65}, \ref{ese:22.66}}
